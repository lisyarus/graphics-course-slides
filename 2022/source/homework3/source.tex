% (c) Nikita Lisitsa, lisyarus@gmail.com, 2022

\documentclass{beamer}

\usepackage[T2A]{fontenc}
\usepackage[utf8]{inputenc}
\usepackage[russian]{babel}

\usepackage{graphicx}
\graphicspath{ {./images/} }

\usepackage{adjustbox}

\usepackage{tikz}

\usepackage{color}
\usepackage{soul}

\usepackage{hyperref}

\definecolor{blue}{rgb}{0,0,1}
\definecolor{red}{rgb}{1,0,0}

\hypersetup{
    colorlinks=true,
    linkcolor=blue,
    urlcolor=blue,
}

\makeatletter
\newcommand{\slideimage}[1]{
  \begin{figure}
    \begin{adjustbox}{width=\textwidth, totalheight=\textheight-2\baselineskip-2\baselineskip,keepaspectratio}
      \includegraphics{#1}
    \end{adjustbox}
  \end{figure}
}
\makeatother

\title{Компьютерная графика}
\subtitle{Домашнее задание 3: Снежный шар}
\date{2021}

\setbeamertemplate{footline}[frame number]

\begin{document}

\frame{\titlepage}

\begin{frame}[fragile]
\frametitle{Задание}
\begin{itemize}
\item Нарисовать `снежный шар', отражающий environment map по Френелю
\item Внутри шара -- какие-нибудь модели (часть из них должна быть анимированной) и падающие частицы (снежинки)
\item Движущийся направленный источник света и тени от него
\item Туман, влияющий на всё внутри шара, с объёмными тенями
\end{itemize}
\end{frame}

\begin{frame}[fragile]
\frametitle{Задание}
\slideimage{example.png}
\end{frame}

\begin{frame}[fragile]
\frametitle{Environment map}
\begin{itemize}
\item Можно взять из 10ой практики или найти самим, их легко найти (\verb|environment map download|)
\item Нужно, чтобы была возможность одновременно настраивать (динамически менять) яркость ambient освещения и environment map (можно значение environment map просто домножить на ambient освещение)
\begin{itemize}
\item Это нужно, чтобы лучше были видны объёмные тени
\end{itemize}
\end{itemize}
\end{frame}

\begin{frame}[fragile]
\frametitle{Шар}
\begin{itemize}
\item Должен отражать environment map с полупрозрачностью
\item Коэффициент полупрозрачности нужно вычислять с помощью \href{https://en.wikipedia.org/wiki/Schlick%27s_approximation}{аппроксимации Шлика} для \href{https://en.wikipedia.org/wiki/Fresnel_equations}{уравнений Френеля}:
\begin{equation}
R = R_0 + (1 - R_0) \cdot (1 - \cos \theta)^5
\end{equation}
\begin{equation}
R_0 = \left(\frac{1-n}{1+n}\right)^2
\end{equation}
\item \begin{math}\theta\end{math} -- угол между нормалью и направлением взгляда
\item \begin{math}n\end{math} -- коэффициент преломления стекла (возьмите 1.5 .. 4)
\item \begin{math}R\end{math} -- результирующее значение opacity (альфа-канала)
\item Код генерации сферы можно взять из 10ой практики
\item Имеет смысл рисовать шар последним
\end{itemize}
\end{frame}

\begin{frame}[fragile]
\frametitle{Модели}
\begin{itemize}
\item Нужна большая модель в центре, чтобы хорошо было видно объёмные тени -- попробуйте найти сами (\verb|free 3d models download|)
\item Нужна анимированная модель, можно взять волка из 13ой практики (или, опять же, найти самим)
\item Нужен `пол' для шарика, можно взять код генерации сферы и сделать, чтобы он генерировал половинку сферы
\end{itemize}
\end{frame}

\begin{frame}[fragile]
\frametitle{Тени}
\begin{itemize}
\item Подойдёт любой вариант shadow mapping
\item Сцена маленькая, артефактов не должно быть (не забудьте про shadow bias)
\item Пол можно не рисовать в shadow map (но \textit{на} пол тень должна падать!)
\end{itemize}
\end{frame}

\begin{frame}[fragile]
\frametitle{Туман}
\begin{itemize}
\item На любой объект внутри шара (можно кроме снежинок) должен влиять объёмный туман, находящийся только внутри шара
\item Придётся найти пересечение луча из камеры с границей шара (немного геометрии, сводится к квадратному уравнению; центр и радиус шара можно захардкодить в шейдерах)
\item В варианте без объёмных теней можно просто взять белый цвет и альфа-канал, учитывающий optical depth
\item В варианте с объёмными тенями нужен цикл интегрирования вдоль луча, аналогичный практике 12
\end{itemize}
\end{frame}

\begin{frame}[fragile]
\frametitle{Туман}
\begin{itemize}
\item Можно вычислять вклад тумана во фрагментном шейдере каждого объекта
\item Альтернативно можно вычислить туман шейдером пост-обработки (когда сцена нарисована в текстуру, позицию пикселя в пространстве можно вычислить через буфер глубины)
\item В `пустом' пространстве (где есть шарик, но нет никаких объектов внутри) тоже должен быть туман \begin{math}\Rightarrow\end{math} можно нарисовать тот же шарик с front face culling'ом, чтобы нарисовалась задняя стенка шара, и для неё написать отдельный шейдер, вычисляющий туман (этот объект будет полупрозрачным -- сквозь шарик должна быть видна environment map)
\end{itemize}
\end{frame}

\begin{frame}[fragile]
\frametitle{Объёмные тени}
\begin{itemize}
\item Внутренний цикл (в направлении света) не нужен, только внешний (вдоль луча из камеры)
\item Нужно брать значение из shadow map в текущей точке луча и интерпретировать как \verb|emission|, или как количество света для рассеяния
\item Поиграйте с коэффициентами, чтобы получилось красиво
\item Число шагов цикла можно сделать побольше (64, 128), иначе будут banding-артефакты
\end{itemize}
\end{frame}

\begin{frame}[fragile]
\frametitle{Снежинки}
\begin{itemize}
\item Аналогично 11ой практике
\item Пересоздаются где-нибудь внутри шара, когда достигают пола
\item Вместо текстуры можно захардкодить гауссиану (или что-нибудь поинтереснее) во фрагментном шейдере
\item На них влияют тени
\item Можно анимировать прозрачность, чтобы они плавно появлялись и исчезали
\end{itemize}
\end{frame}

\begin{frame}[fragile]
\frametitle{Баллы}
\begin{itemize}
\item 1 балл: рисуются какие-нибудь модели
\item 1 балл: движущийся источник света, диффузное освещение
\item 1 балл: двигается камера
\item 1 балл: на фоне есть environment map
\item 1 балл: можно настраивать яркость ambient + environment map
\item 2 балла: шар с отражением по Френелю
\item 2 балл: тени от моделей
\item 1 балл: туман
\item 2 балла: объёмные тени в тумане
\item 2 балла: анимированная модель
\item 1 балл: снежинки
\end{itemize}
Всего: 15 баллов

Защита заданий на практике 12 сентября
\end{frame}

\end{document}