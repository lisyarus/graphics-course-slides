% (c) Nikita Lisitsa, lisyarus@gmail.com, 2023

\documentclass[10pt]{beamer}

\usepackage[T2A]{fontenc}
\usepackage[russian]{babel}
\usepackage{minted}

\usepackage{graphicx}
\graphicspath{ {./images/} }

\usepackage{adjustbox}

\usepackage{color}
\usepackage{soul}

\usepackage{hyperref}

\usetheme{metropolis}

\definecolor{red}{rgb}{1,0,0}
\definecolor{codebg}{RGB}{29,35,49}
\setminted{fontsize=\footnotesize}

\definecolor{blue}{rgb}{0,0,1}
\definecolor{red}{rgb}{1,0,0}

\makeatletter
\newcommand{\slideimage}[1]{
  \begin{figure}
    \begin{adjustbox}{width=\textwidth, totalheight=\textheight-2\baselineskip-2\baselineskip,keepaspectratio}
      \includegraphics{#1}
    \end{adjustbox}
  \end{figure}
}
\makeatother

\title{Компьютерная графика}
\subtitle{Лекция 2: Растеризация, графический конвейер, шейдеры, интерполяция, аффинные преобразования}
\date{2023}

\setbeamertemplate{footline}[frame number]

\begin{document}

\frame{\titlepage}

% \begin{frame}
% \frametitle{Два слова о GPU}
% \pause
% \begin{itemize}
% \item SIMD -- Single Instruction, Multiple Data
% \pause
% \item Как SIMD-расширения процессоров (SSE, AVX), но на более крупном масштабе -- сотни/тысячи одновременных вычислений
% \pause
% \item Для производительности вычислительные ядра организованы в группы (wavefronts) по 32/64 элемента
% \pause
% \item Все ядра одного wavefront'а выполняют одну и ту же операцию (но над разными данными)
% \pause
% \item Если мы рисуем только один треугольник, под него всё равно выделяется целый wavefront и большинство его ядер делают бесполезную работу
% \pause
% \item Максимизация загрузки wavefront'ов -- важная часть низкоуровневой оптимизации GPU
% \end{itemize}
% \end{frame}

\begin{frame}
\frametitle{Растеризация}
\begin{itemize}
\item Растеризация -- превращение геометрического примитива (точки, линии, треугольника, прямоугольника, круга, и т.д.) в набор соответствующих ему пикселей на экране/изображении
\item Превращение векторных данных в растровые
\pause
\item За нас её делает OpenGL!
\pause
\item Некоторые современные графические движки GPU (Unreal 5 Nanite) делают растеризацию сами с помощью compute шейдеров
\end{itemize}
\end{frame}

\begin{frame}[fragile]
\frametitle{Растеризация: точка}
\begin{itemize}
\item Как растеризовать точку \begin{math}(x, y)\end{math}?
\pause
\usemintedstyle{lightbulb}
\begin{minted}[bgcolor=codebg]{cpp}
set_pixel(round(x), round(y), color);
\end{minted}
\pause
\usemintedstyle{solarized-light}
\item В OpenGL: \mintinline{cpp}|GL_POINTS|
\end{itemize}
\end{frame}

\begin{frame}
\frametitle{Растеризация: точка}
\slideimage{raster-point.png}
\end{frame}

\begin{frame}[fragile]
\frametitle{Растеризация: линия}
\begin{itemize}
\item Как растеризовать линию \begin{math}(x_1, y_1) \dots (x_2, y_2)\end{math}?
\pause
\item Алгоритм Брезенхэма
\pause
\item Есть вариация алгоритма для рисования окружностей
\pause
\usemintedstyle{solarized-light}
\item В OpenGL: \mintinline{cpp}|GL_LINES|
\end{itemize}
\end{frame}

\begin{frame}
\frametitle{Растеризация: линия}
\slideimage{raster-line.png}
\end{frame}

\begin{frame}
\frametitle{Растеризация: окружность}
\slideimage{raster-circle.png}
\end{frame}

\begin{frame}[fragile]
\frametitle{Растеризация: прямоугольник}
\begin{itemize}
\item Как растеризовать прямоугольник \begin{math}[x_1\dots x_2]\times[y_1\dots y_2]\end{math}?
\end{itemize}
\pause
\usemintedstyle{lightbulb}
\begin{minted}[bgcolor=codebg]{cpp}
for (int x = round(x_1); x <= round(x_2); ++x) {
    for (int y = round(y_1); y <= round(y_2); ++y) {
        set_pixel(x, y, color);
    }
}
\end{minted}
\end{frame}

\begin{frame}
\frametitle{Растеризация: прямоугольник}
\slideimage{raster-rect.png}
\end{frame}

\begin{frame}[fragile]
\frametitle{Растеризация: треугольник}
\begin{itemize}
\item Как растеризовать треугольник с вершинами \begin{math}(x_1, y_1), (x_2, y_2), (x_3, y_3)\end{math}?
\pause
\item Растеризуем ограничивающий прямоугольник, проверяя пиксели на вхождение в треугольник
\pause
\usemintedstyle{lightbulb}
\begin{minted}[bgcolor=codebg]{cpp}
int xmin = min(round(x_1), round(x_2), round(x_3));
int xmax = max(round(x_1), round(x_2), round(x_3));

int ymin = min(round(y_1), round(y_2), round(y_3));
int ymax = max(round(y_1), round(y_2), round(y_3));

for (int x = xmin; x <= xmax; ++x) {
    for (int y = ymin; y <= ymax; ++y) {
        if (inside_triangle(x, y, ...))
            set_pixel(x, y, color);
    }
}
\end{minted}
\pause
\usemintedstyle{solarized-light}
\item В OpenGL: \mintinline{cpp}|GL_TRIANGLES|
\end{itemize}
\end{frame}

\begin{frame}
\frametitle{Растеризация: треугольник}
\slideimage{raster-triangle.png}
\end{frame}

\begin{frame}[fragile]
\frametitle{Растеризация: круг}
\begin{itemize}
\item Как растеризовать круг с центром \begin{math}(x_0, y_0)\end{math} и радиусом \begin{math}R\end{math}?
\pause
\item Растеризуем ограничивающий прямоугольник, проверяя пиксели на вхождение в круг
\end{itemize}
\pause
\usemintedstyle{lightbulb}
\begin{minted}[bgcolor=codebg]{cpp}
int xmin = round(x_0 - R);
int xmax = round(x_0 + R);

int ymin = round(y_0 - R);
int ymax = round(y_0 + R);

for (int x = xmin; x <= xmax; ++x) {
    for (int y = ymin; y <= ymax; ++y) {
        if (sqr(x - x_0) + sqr(y - y_0) <= sqr(R))
            set_pixel(x, y, color);
    }
}
\end{minted}
\end{frame}

\begin{frame}
\frametitle{Растеризация: круг}
\slideimage{raster-disk.png}
\end{frame}

\begin{frame}<1>[fragile,label=rasterization]
\frametitle{Растеризация в OpenGL}
\begin{itemize}
\item Пиксель растеризуется, если \textit{центр пикселя} содержится в треугольнике
\pause
\item Если у двух треугольников есть общее ребро (и они не пересекаются внутренностями), то
\begin{itemize}
\item Каждый пиксель будет принадлежать ровно одному треугольнику, т.е. не будет наложения
\item Ни один пиксель общего ребра не будет пропущен, т.е. не будет ``дырок''
\end{itemize}
\pause
\item Подробнее: \href{https://en.wikibooks.org/wiki/GLSL_Programming/Rasterization}{\nolinkurl{en.wikibooks.org/wiki/GLSL\_Programming/Rasterization}}
\end{itemize}
\end{frame}

\begin{frame}
\frametitle{Растеризация в OpenGL}
\slideimage{pixel-covered.png}
\end{frame}

\againframe<1-2>{rasterization}

\begin{frame}
\frametitle{Растеризация в OpenGL}
\:
\slideimage{triangle-rasterization.png}
\end{frame}

\begin{frame}
\frametitle{Растеризация в OpenGL}
Не будет ``дырок'':
\slideimage{triangle-rasterization-hole.png}
\end{frame}

\begin{frame}
\frametitle{Растеризация в OpenGL}
Не будет наложения пикселей:
\slideimage{triangle-rasterization-overlap.png}
\end{frame}

\againframe<2-3>{rasterization}

\begin{frame}[fragile]
\frametitle{Растеризация в OpenGL}
\begin{itemize}
\usemintedstyle{solarized-light}
\item В современном OpenGL есть только три основных примитива для растеризации:
\pause
\begin{itemize}
\item \textbf{Точки}: \mintinline{cpp}|GL_POINTS|
\pause
\item \textbf{Линии}: \mintinline{cpp}|GL_LINES|, \mintinline{cpp}|GL_LINE_STRIP|, \mintinline{cpp}|GL_LINE_LOOP|
\pause
\item \textbf{Треугольники}: \mintinline{cpp}|GL_TRIANGLES|, \mintinline{cpp}|GL_TRIANGLE_STRIP|, \mintinline{cpp}|GL_TRIANGLE_FAN|
\end{itemize}
\pause
\item Есть особые примитивы для геометрических шейдеров: \mintinline{cpp}|GL_LINE_STRIP_ADJACENCY|, \mintinline{cpp}|GL_LINES_ADJACENCY|, \mintinline{cpp}|GL_TRIANGLE_STRIP_ADJACENCY|, \mintinline{cpp}|GL_TRIANGLES_ADJACENCY|
\pause
\item Точки и линии обычно используются для дебажного рендеринга, \textit{главный примитив -- \underline{треугольник}}
\end{itemize}
\end{frame}

\begin{frame}
\frametitle{О треугольниках}
\begin{itemize}
\item Почему основным примитивом рисования стал \textit{треугольник}?
\pause
\item Для более сложных геометрических фигур (круг, многоугольник, и т.д.):
\pause
\begin{itemize}
\item Сложнее алгоритм разтеризации
\pause
\item Сложнее задавать и интерполировать атрибуты (накладывать цвет и текстуру, вычислять освещение)
\end{itemize}
\end{itemize}
\end{frame}

\begin{frame}
\frametitle{О треугольниках}
Плюсы треугольника:
\pause
\begin{itemize}
\item Образ под действием перспективной проекции -- тоже треугольник
\pause
\item Есть единственный разумный способ интерполяции (линейная, с барицентрическими координатами)
\pause
\item Позволяет рисовать спрайты (прямоугольник -- два треугольника)
\pause
\item Позволяет рисовать многоугольники (посредством триангуляции)
\pause
\item Позволяет рисовать линии (превращая их в тонкие многоугольники)
\pause
\item Позволяет рисовать более сложные фигуры (аппроксимируя)
\end{itemize}
\end{frame}

\begin{frame}[fragile]
\frametitle{Графический конвейер (graphics pipeline)}
\begin{itemize}
\item Последовательность операций на GPU, превращающих входные данные в картинку на экране
\pause
\item Почти не отличается для разных графических API
\end{itemize}
\end{frame}

\begin{frame}[fragile]
\frametitle{Графический конвейер (graphics pipeline)}
\usemintedstyle{solarized-light}
Основные части конвейера:
\begin{itemize}
\item Входной поток вершин (vertex stream)
\pause
\begin{itemize}
\item Позже мы узнаем, как его задавать
\end{itemize}
\pause
\item Вершинный шейдер: обрабатывает вершины по одной
\pause
\item Сборка примитивов (primitive assembly) из отдельных вершин
\pause
\item Преобразование в оконную систему координат (viewport transform)
\pause
\item Отсечение невидимых граней (back-face culling)
\pause
\item Растеризация примитивов
\pause
\item Пиксельный (фрагментный) шейдер: обрабатывает пиксели по одному
\end{itemize}
\end{frame}

\begin{frame}[fragile]
\frametitle{Графический конвейер (graphics pipeline)}
\begin{itemize}
\item Мы пропустили много важных частей конвейера
\item Будем их по чуть-чуть добавлять в течение курса
\end{itemize}
\end{frame}

\begin{frame}[fragile]
\frametitle{Вершинный (vertex) шейдер}
\usemintedstyle{solarized-light}
\begin{itemize}
\pause
\item Входные данные:
\pause
\begin{itemize}
\item Аттрибуты вершин (мы позже узнаем, как их задавать) -- свои для каждой вершины
\pause
\item Uniform-переменные -- глобальные значения, не меняющиеся в течение одного вызова команды рисования (напр. \mintinline{cpp}|glDrawArrays|):
\mintinline{glsl}|uniform float scale;|
\end{itemize}
\pause
\item Выходные данные:
\begin{itemize}
\item \mintinline{glsl}|vec4 gl_Position;|
\pause
\item Переменные, интерполированное значение которых попадёт во фрагментный (пиксельный) шейдер: \mintinline{glsl}|out vec3 color;|
\end{itemize}
\end{itemize}
\end{frame}

\begin{frame}<1>[label=primitive_assembly]
\frametitle{Сборка примитивов (primitive assembly)}
\usemintedstyle{solarized-light}
\begin{itemize}
\item Превращает набор независимых вершин в \textit{притимивы} растеризации: точки/линии/треугольники
\item Правила сборки зависят от выбранного режима: \mintinline{cpp}|GL_POINTS|, \mintinline{cpp}|GL_LINES|, \mintinline{cpp}|GL_LINE_STRIP|, \mintinline{cpp}|GL_LINE_LOOP|, \mintinline{cpp}|GL_TRIANGLES|, \mintinline{cpp}|GL_TRIANGLE_STRIP|, \mintinline{cpp}|GL_TRIANGLE_FAN|
\pause
\item Порядок вершин в \mintinline{cpp}|GL_TRIANGLE_STRIP| и \mintinline{cpp}|GL_TRIANGLE_FAN| важен для \textit{back-face culling}
\end{itemize}
\end{frame}

\begin{frame}
\frametitle{Сборка примитивов (primitive assembly)}
\slideimage{primitives.png}
\end{frame}

\againframe<1->{primitive_assembly}

\begin{frame}[fragile]
\frametitle{Viewport transform}
\usemintedstyle{solarized-light}
\begin{itemize}
\item Преобразование в оконную систему координат
\item \begin{math}X: [-1, 1] \rightarrow [0, width]\end{math}
\item \begin{math}Y: [-1, 1] \rightarrow [height, 0]\end{math} (-1 внизу, 1 вверху)
\pause
\item Настроить: \mintinline{cpp}|glViewport(0, 0, width, height)|
\end{itemize}
\end{frame}

\begin{frame}<1-2>[fragile,label=back_face_culling]
\frametitle{Back-face culling}
\begin{itemize}
\usemintedstyle{solarized-light}
\item \textit{Обычно} в OpenGL треугольники, вершины которых оказываются на экране в порядке обхода \textit{по часовой стрелке}, \alert{\textbf{не рисуются}}, чтобы не рисовать треугольники, которые будут скрыты другими треугольниками ближе к камере
\pause
\begin{itemize}
\item Из-за этого в некоторых играх вы видите всю карту, когда проваливаетесь в текстуры
\end{itemize}
\pause
\item Работает только для сплошных 3D моделей без ``дырок'', имеющих внутренность (напр. \textit{manifold mesh})
\pause
\item Работает только при договорённости, что все грани описаны \textit{против часовой стрелки}, если смотреть на модель \textit{снаружи}
\pause
\item Порядок вершин в \mintinline{cpp}|GL_TRIANGLE_STRIP| и \mintinline{cpp}|GL_TRIANGLE_FAN| чётко определён, чтобы не сломать back-face culling
\end{itemize}
\end{frame}

\begin{frame}
\frametitle{Back-face culling}
\slideimage{back-face-culling.png}
\end{frame}

\againframe<2->{back_face_culling}

\begin{frame}[fragile]
\frametitle{Back-face culling}
\begin{itemize}
\usemintedstyle{solarized-light}
\item Выключен по умолчанию
\pause
\item Включить/выключить это поведение: \mintinline{cpp}|glEnable(GL_CULL_FACE)| или \mintinline{cpp}|glDisable(GL_CULL_FACE)|
\pause
\item Настроить, какие треугольники будут скрываться: \mintinline{cpp}|glCullFace(GL_BACK)|, \mintinline{cpp}|glCullFace(GL_FRONT)|, \mintinline{cpp}|glCullFace(GL_FRONT_AND_BACK)|
\pause
\item Настроить, какие треугольники считаются \mintinline{cpp}|FRONT|, а какие -- \mintinline{cpp}|BACK|: \mintinline{cpp}|glFrontFace(GL_CCW)|, \mintinline{cpp}|glFrontFace(GL_CW)|
\end{itemize}
\end{frame}

\begin{frame}[fragile]
\frametitle{Фрагментный (пиксельный, fragment) шейдер}
\usemintedstyle{solarized-light}
\begin{itemize}
\pause
\item Входные данные:
\pause
\begin{itemize}
\item Uniform-переменные
\pause
\item Проинтерполированные \mintinline{glsl}|out|-переменные вершинного шейдера: \mintinline{glsl}|in vec3 color;|
\pause
\item \mintinline{glsl}|gl_FragCoord| -- координаты центра пикселя, напр. \begin{math}(42.5, 239.5)\end{math}
\pause
\item И много других: \href{https://www.khronos.org/opengl/wiki/Fragment_Shader/Defined_Inputs}{\nolinkurl{khronos.org/opengl/wiki/Fragment\_Shader/Defined\_Inputs}}
\end{itemize}
\pause
\item Выходные данные:
\begin{itemize}
\item \mintinline{glsl}|layout (location = 0) out vec4 out_color;| -- выходной цвет в формате RGBA
\pause
\item Может быть несколько, об этом поговорим позже
\end{itemize}
\end{itemize}
\end{frame}

\begin{frame}[fragile]
\frametitle{Языки описания шейдеров}
\begin{itemize}
\item \textbf{OpenGL, Vulkan}: \textbf{\alert{\underline{GLSL}}} (GL Shading Language)
\pause
\item \textbf{DirectX}: \textbf{HLSL} (High-Level Shading Language)
\pause
\item \textbf{DirectX} (до 2012): \textbf{Cg} (C for Graphics), deprecated
\pause
\item \textbf{WebGPU}: \textbf{WGSL} (WebGpu Shading Language)
\pause
\item И другие: \href{https://en.wikipedia.org/wiki/Shading_language}{\nolinkurl{en.wikipedia.org/wiki/Shading\_language}}
\end{itemize}
\end{frame}

\begin{frame}[fragile]
\frametitle{Язык описания шейдеров GLSL}
\usemintedstyle{solarized-light}
\begin{itemize}
\item Похож на C
\pause
\item Типы данных:
\pause
\begin{itemize}
\item Скалярные: \mintinline{glsl}|bool|, \mintinline{glsl}|int|, \mintinline{glsl}|uint|, \mintinline{glsl}|float|
\pause
\item Векторные: \mintinline{glsl}|bvec2|, \mintinline{glsl}|bvec3|, \mintinline{glsl}|bvec4|, \mintinline{glsl}|ivec2|, ..., \mintinline{glsl}|uvec2|, ..., \mintinline{glsl}|vec2|, ...
\pause
\item Матричные: \mintinline{glsl}|mat2|, \mintinline{glsl}|mat3|, \mintinline{glsl}|mat4|, \mintinline{glsl}|mat2x4|, ...
\pause
\item В GLSL 400 или с расширением \mintinline{glsl}|GL_ARB_gpu_shader_fp64| есть \mintinline{glsl}|double|, \mintinline{glsl}|dvec2|, ..., \mintinline{glsl}|dmat2|, ...
\end{itemize}
\pause
\item Конструкторы векторов:
\pause
\begin{itemize}
\item \mintinline{glsl}|vec3(x, y, z)|
\pause
\item \mintinline{glsl}|vec3(x) == vec3(x, x, x)|
\pause
\item \mintinline{glsl}|vec3(vec2(x, y), z) == vec3(x, vec2(y, z))|
\end{itemize}
\pause
\item Подробнее: \href{https://www.khronos.org/opengl/wiki/Data_Type_(GLSL)}{\nolinkurl{khronos.org/opengl/wiki/Data\_Type\_(GLSL)}}
\end{itemize}
\end{frame}

\begin{frame}[fragile]
\frametitle{Язык описания шейдеров GLSL}
\usemintedstyle{solarized-light}
\begin{itemize}
\item Программа должна начинаться с \mintinline{glsl}|#version <версия> [ <профиль> ]|
\pause
\item Программа должна содержать функцию \mintinline{glsl}|void main()|
\end{itemize}
\end{frame}

\begin{frame}[fragile]
\frametitle{Язык описания шейдеров GLSL}
\usemintedstyle{solarized-light}
\begin{itemize}
\item Есть стандартные операции: \mintinline{glsl}|+|, \mintinline{glsl}|-|, \mintinline{glsl}|*|, \mintinline{glsl}|/|, \mintinline{glsl}|%|, \mintinline{glsl}|<|, \mintinline{glsl}|==|, ...
\pause
\begin{itemize}
\item Для векторов и матриц применяются покомпонентно
\pause
\item Исключение: умножение двух матриц это умножение в смысле алгебры
\pause
\item Исключение: умножение матрицы на вектор это умножение в смысле алгебры
\end{itemize}
\pause
\item Доступ к координатам векторов: \mintinline{glsl}|a.x = b.y|
\pause
\item Swizzling: \mintinline{glsl}|a.xxzy == vec4(a.x, a.x, a.z, a.y)|
\pause
\item Доступ к элементам матриц: \mintinline{glsl}|m[column][row]|
\pause
\item Есть полезные математические функции: \mintinline{glsl}|pow|, \mintinline{glsl}|sin|, \mintinline{glsl}|cos|, ...
\pause
\begin{itemize}
\item Для векторов и матриц применяются покомпонентно
\end{itemize}
\pause
\item Есть математические функции векторной алгебры: \mintinline{glsl}|dot|, \mintinline{glsl}|cross|, \mintinline{glsl}|length|, \mintinline{glsl}|normalize|, ...
\pause
\item Линейная и нелинейная интерполяция: \mintinline{glsl}|mix|, \mintinline{glsl}|smoothstep|
\end{itemize}
\end{frame}

\begin{frame}[fragile]
\frametitle{Язык описания шейдеров GLSL}
\usemintedstyle{solarized-light}
\begin{itemize}
\item Есть массивы: \mintinline{glsl}|float array[5];|
\pause
\begin{itemize}
\item Инициализация: \mintinline{glsl}|float array[5] = float[5](0.0, 1.0, 2.0, 3.0, 4.0);|
\end{itemize}
\pause
\item Константы (известные на момент компиляции): \mintinline{glsl}|const float PI = 3.141592;|
\end{itemize}
\end{frame}

\begin{frame}[fragile]
\frametitle{Язык описания шейдеров GLSL}
\usemintedstyle{solarized-light}
\begin{itemize}
\item Ветвление: \mintinline{glsl}|if (condition) { ... } else { ... }|
\pause
\item Тернарный if: \mintinline{glsl}|condition ? x : y|
\pause
\item Циклы: \mintinline{glsl}|for (int i = 0; i < 10; ++i) { ... }|
\begin{itemize}
\item Число итераций цикла должно быть константой, известной на момент компиляции шейдера
\item \alert{\textbf{Не может зависеть}} от uniform-переменных, аттрибутов вершин, и т.д.
\end{itemize}
\pause
\usemintedstyle{lightbulb}
\item Функции: \begin{minted}[bgcolor=codebg]{glsl}
vec3 reflect(vec3 v, vec3 n) {
    return v - 2.0 * n * dot(v, n);
}
\end{minted}
\pause
\begin{itemize}
\item Могут вызывать другие функции
\pause
\item Рекурсия \alert{\textbf{запрещена}}
\end{itemize}
\end{itemize}
\end{frame}

\begin{frame}[fragile]
\frametitle{Полезные ресурсы о шейдерах}
\begin{itemize}
\item Туториал: \href{https://learnopengl.com/Getting-started/Shaders}{\nolinkurl{learnopengl.com/Getting-started/Shaders}}
\item Туториал: \href{https://www.lighthouse3d.com/tutorials/glsl-tutorial}{\nolinkurl{lighthouse3d.com/tutorials/glsl-tutorial}}
\pause
\item Книжка-учебник с большим количеством примеров сложных шейдеров: \href{https://thebookofshaders.com/00/}{\texttt{The Book of Shaders}}
\pause
\item Онлайн-редактор шейдеров: \href{https://shadertoy.com/}{\nolinkurl{shadertoy.com}}
\end{itemize}
\end{frame}

\begin{frame}[fragile]
\frametitle{Четырёхмерные векторы?}
\usemintedstyle{solarized-light}
\begin{itemize}
\item GLSL поддерживает операции с матрицами и векторами размерностей вплоть до 4
\item \mintinline{glsl}|gl_Position| имеет 4 компоненты
\pause
\item Мы хотим рисовать двухмерные и трёхмерные объекты, зачем четвёртая координата?
\pause
\item Три причины:
\pause
\begin{itemize}
\item Для некоторых \textit{типов данных}: кватернионы, цвета (rgba)
\pause
\item Для \textit{перспективной проекции} (следующая лекция)
\pause
\item Для \textit{аффинных преобразований}
\end{itemize}
\end{itemize}
\end{frame}

\begin{frame}[fragile]
\frametitle{Аффинное пространство}
\begin{itemize}
\item Векторные (линейные) пространства -- круто: большая теория, эффективные операции
\pause
\item Не очень подходит для того, чтобы моделировать пространство
\pause
\begin{itemize}
\item Где в пространстве точка, соответствующая нулевому вектору?
\pause
\item Что такое сумма двух точек?
\end{itemize}
\pause
\item Мысль: точки \begin{math}\neq\end{math} векторы
\end{itemize}
\end{frame}

\begin{frame}[fragile]
\frametitle{Аффинное пространство}
\begin{itemize}
\item Аффинное пространство: векторное пространство, в котором ``забыли''  нулевой вектор
\pause
\item Формальнее, аффинное пространство \begin{math}\mathbb A\end{math} над векторным пространством \begin{math}\mathbb V\end{math} -- это:
\pause
\begin{itemize}
\item Множество точек \begin{math}\mathbb A\end{math}
\pause
\item Операция ``точка + вектор'': \begin{math}\oplus: \mathbb A \times \mathbb V \rightarrow \mathbb A\end{math}
\pause
\item Операция ``точка - точка'': \begin{math}\ominus: \mathbb A \times \mathbb A \rightarrow \mathbb V\end{math}
\pause
\item Аксиомы, связывающие их с операциями векторного пространства, например
\begin{math}(p\oplus v_1)\oplus v_2 = p\oplus(v_1+v_2)\end{math}
\end{itemize}
\pause
\item Любую точку \begin{math}p_0\in \mathbb A\end{math} можно принять за начало координат
\item Это даёт биекцию \begin{math}\mathbb A \rightarrow \mathbb V\end{math} по формуле \begin{math}p \mapsto p - p_0\end{math}
\end{itemize}
\end{frame}

\begin{frame}[fragile]
\frametitle{Аффинное пространство}
\begin{itemize}
\item Множество решений уравнения \begin{math}Lx=0\end{math} в векторном пространстве -- \textit{векторное подпространство} \begin{math}\ker L\end{math}
\pause
\item Множество решений уравнения \begin{math}Lx=y\end{math} в векторном пространстве -- \alert{\textbf{не}} векторное подпространство
\pause
\item Множество решений уравнения \begin{math}Lx=y\end{math} в векторном пространстве -- \textit{аффинное пространство} над \begin{math}\ker L\end{math}
\end{itemize}
\end{frame}

\begin{frame}[fragile]
\frametitle{Аффинные комбинации}
\begin{itemize}
\item Пусть есть набор точек \begin{math}p_1, \dots, p_N\end{math} и набор коэффициентов \begin{math}\lambda_1, \dots, \lambda_N\end{math}
\pause
\item Выберем начало координат \begin{math}p_0\end{math} и определим

\begin{math}\sum \lambda_i p_i = p_0 + \sum \lambda_i (p_i - p_0) \in \mathbb A\end{math}
\pause
\item Теорема: если \begin{math}\sum \lambda_i = 1\end{math}, то результат (точка) не зависит от выбора \begin{math}p_0\end{math}
\pause
\item В этом случае \begin{math}q = \sum \lambda_i p_i\end{math} называется аффинной комбинацией точек \begin{math}p_i\end{math} с коэффициентами \begin{math}\lambda_i\end{math}
\pause
\item Коэффициенты \begin{math}\lambda_i\end{math} называются барицентрическими координатами точки \begin{math}q\end{math}
\pause
\item Поиск \begin{math}\lambda_i\end{math} по известным \begin{math}q\end{math} и \begin{math}p_i\end{math} сводится к решению линейной системы уравнений (квадратная только если \begin{math}N = D + 1\end{math}, где \begin{math}D\end{math} -- размерность пространства)
\pause
\item Аффинная комбинация пустого множества точек не определена
\end{itemize}
\end{frame}

\begin{frame}[fragile]
\frametitle{Аффинные комбинации: пример}
\slideimage{lerp-base.png}
\end{frame}

\begin{frame}[fragile]
\frametitle{Аффинные комбинации: пример}
\slideimage{lerp-1.png}
\end{frame}

\begin{frame}[fragile]
\frametitle{Аффинные комбинации: пример}
\slideimage{lerp-2.png}
\end{frame}

\begin{frame}[fragile]
\frametitle{Аффинные комбинации: пример}
\slideimage{lerp-3.png}
\end{frame}

\begin{frame}[fragile]
\frametitle{Аффинные комбинации: пример}
\slideimage{lerp-4.png}
\end{frame}

\begin{frame}[fragile]
\frametitle{Векторнозначные комбинации}
\begin{itemize}
\item Пусть есть набор точек \begin{math}p_1, \dots, p_N\end{math} и набор коэффициентов \begin{math}\lambda_1, \dots, \lambda_N\end{math}
\pause
\item Выберем начало координат \begin{math}p_0\end{math} и определим

\begin{math}\sum \lambda_i p_i = \sum \lambda_i (p_i - p_0) \in \mathbb V\end{math}
\pause
\item Теорема: если \begin{math}\sum \lambda_i = 0\end{math}, то результат (вектор) не зависит от выбора \begin{math}p_0\end{math}
\end{itemize}
\end{frame}

\begin{frame}[fragile]
\frametitle{Аффинная оболочка}
\begin{itemize}
\item Пусть есть набор точек \begin{math}p_1, \dots, p_N\end{math}
\pause
\item Их \textit{аффинная оболочка} -- множество всех возможных аффинных комбинаций, т.е. множество \begin{math}\sum \lambda_i p_i\end{math} для всех возможных \begin{math}\lambda_i\end{math} таких, что \begin{math}\sum \lambda_i = 1\end{math}
\pause
\item Аффинная оболочка точки -- сама \textit{точка}
\pause
\item Аффинная оболочка двух точек -- содержащая их \textit{прямая}
\pause
\item Аффинная оболочка трёх точек -- содержащая их \textit{плоскость}
\pause
\item И т.д.
\end{itemize}
\end{frame}

\begin{frame}[fragile]
\frametitle{Линейная интерполяция}
\begin{itemize}
\item Линейная интерполяция = интерполяция, использующая барицентрические координаты как коэффициенты
\pause
\item Пусть есть точки \begin{math}p_i\end{math} и точка \begin{math}q = \sum \lambda_i p_i\end{math}
\pause
\item Пусть в точках \begin{math}p_i\end{math} задано значение некоторой величины \begin{math}f_i\end{math}
\pause
\item Линейная интерполяция величины \begin{math}f\end{math} в точке \begin{math}q\end{math} -- значение \begin{math}\sum \lambda_i f_i\end{math}
\pause
\item Для этого нужно вычисляить \begin{math}\lambda_i\end{math} по известным \begin{math}p_i\end{math} и \begin{math}q\end{math}, что сводится к системе линейных уравнений
\pause
\item \textbf{Ровно это} происходит при интерполяции значений, переданных из вершинного шейдера во фрагментный \pause (с точностью до перспективной проекции, о чём мы поговорим позже)
\end{itemize}
\end{frame}

\begin{frame}[fragile]
\frametitle{Линейная интерполяция цветов}
\slideimage{triangle.png}
\end{frame}

\begin{frame}[fragile]
\frametitle{Выпуклые оболочки}
\begin{itemize}
\item Выпуклая комбинация: аффинная комбинация, в которой все коэффициенты неотрицательны \begin{math}\lambda_i \geq 0\end{math}
\pause
\begin{itemize}
\item Вместе с \begin{math}\sum\lambda_i = 1\end{math} это означает, что \begin{math}\lambda_i \in [0, 1]\end{math}
\end{itemize}
\pause
\item Множество всех выпуклых комбинаций набора точек = \textit{выпуклая оболочка}
\pause
\item Выпуклая оболочка точки -- сама \textit{точка}
\pause
\item Выпуклая оболочка двух точек -- \textit{отрезок} между этими точками
\pause
\item Выпуклая оболочка трёх точек -- \textit{треугольник} с этими точками в качестве вершин
\pause
\item Выпуклая оболочка четырёх точек -- \textit{тетраэдр} с этими точками в качестве вершин
\pause
\item И т.д.
\end{itemize}
\end{frame}

\begin{frame}[fragile]
\frametitle{Выпуклые оболочки}
\usemintedstyle{solarized-light}
\begin{itemize}
\item При растеризации центры пикселей попадают внутрь треугольника
\pause
\item \begin{math}\Rightarrow\end{math} Пиксели лежат в \textit{выпуклой оболочке} исходных вершин
\pause
\item \begin{math}\Rightarrow\end{math} Коэффициенты интерполяции всегда в диапазоне \begin{math}\lambda_i \in [0, 1]\end{math}
\pause
\item \begin{math}\Rightarrow\end{math} Проинтерполированные значения лежат в выпуклой оболочке значений в вершинах
\begin{itemize}
\item Если \mintinline{cpp}|out|-значения величины в вершинах были \begin{math}1.5\end{math}, \begin{math}0.3\end{math} и \begin{math}5.7\end{math}, то \mintinline{cpp}|in|-значения в пикселях будут в диапазоне \begin{math}[0.3, 5.7]\end{math} \textit{(с точностью до floating-point ошибок)}
\end{itemize}
\pause
\item N.B. Это может нарушаться при \textit{мультисэмплинге}, о котором мы поговорим позднее
\end{itemize}
\end{frame}

\begin{frame}[fragile]
\frametitle{Аффинные преобразования}
\begin{itemize}
\item Линейные преобразования -- сохраняют \textit{линейные комбинации} векторов: \begin{math}S \left(\sum \lambda_i v_i\right) = \sum \lambda_i S v_i\end{math}
\pause
\item Аффинные преобразования -- сохраняют \textit{аффинные комбинации} точек: \begin{math}S \left(\sum \lambda_i p_i\right) = \sum \lambda_i S p_i\end{math}
\pause
\item Если выбрано начало координат \begin{math}p_0\end{math}, преобразование \begin{math}S: \mathbb A \rightarrow \mathbb A\end{math} можно понимать как преобразование соответствующего векторного пространства
\begin{math}S: \mathbb V \rightarrow \mathbb V\end{math} по формуле \begin{math}S(v) = S(p_0 + v) - p_0\end{math}
\pause
\item Можно показать, что в таком виде любое аффинное преобразование это линейное преобразование + сдвиг на фиксированный вектор:

\begin{math}S(v) = Av + b\end{math}
где
\begin{itemize}
\item \begin{math}A\end{math} -- линейное преобразование пространства \begin{math}\mathbb V\end{math}
\item \begin{math}b\end{math} -- вектор из пространства \begin{math}\mathbb V\end{math}
\end{itemize}
\end{itemize}
\end{frame}

\begin{frame}[fragile]
\frametitle{Аффинные преобразования}
\begin{itemize}
\item В коде аффинное преобразование пространства размерности \begin{math}N\end{math} можно представлять как \textit{пару} \begin{math}(A,b)\end{math} из матрицы \begin{math}N\times N\end{math} и вектора размерности \begin{math}N\end{math}
\pause
\item Пара \begin{math}(A,b)\end{math} действует на точку/вектор \begin{math}v\end{math} по формуле

\begin{math}(A, b) \cdot v = Av + b\end{math}
\pause
\item Композиция преобразований:

\begin{math}(A_1, b_1) \cdot (A_2, b_2) \cdot v = (A_1, b_1) \cdot (A_2 v + b_2) = A_1 (A_2 v + b_2) + b_1 = (A_1 A_2) v + (A_1 b_2 + b_1) = (A_1 A_2, A_1 b_2 + b_1) \cdot v\end{math}
\pause
\item Тождественное преобразование:
\begin{math}(\mathbb I, 0)\end{math}
\pause
\item Обратное преобразование:

\begin{math}(A, b)^{-1} = (A^{-1}, -A^{-1}b)\end{math}
\end{itemize}
\end{frame}

\begin{frame}[fragile]
\frametitle{Аффинные преобразования}
\begin{itemize}
\item Повсеместно используются для задания положения, ориентации и размера объектов
\pause
\item Векторная часть преобразования: положение 3D-объекта
\pause
\item Матричная часть преобразования: вращение и размер объекта
\pause
\item При чём тут четвёртая координата?
\end{itemize}
\end{frame}

\begin{frame}[fragile]
\frametitle{Однородные координаты}
\begin{itemize}
\item Трюк: можно вложить \textit{вектор} размерности \begin{math}N\end{math} в векторное пространство размерности \begin{math}N+1\end{math}, добавив \begin{math}0\end{math} в качестве последней координаты
\pause
\item Трюк: можно вложить \textit{точку} размерности \begin{math}N\end{math} в векторное пространство размерности \begin{math}N+1\end{math}, добавив \begin{math}1\end{math} в качестве последней координаты
\pause
\item Согласуется с операциями на векторах и точках
\pause
\item Согласуется с аффинными комбинациями
\end{itemize}
\end{frame}

\begin{frame}[fragile]
\frametitle{Однородные координаты}
\begin{itemize}
\item Позволяет представить аффинное преобразование \begin{math}N\end{math}-мерного пространства как матрицу \begin{math}(N+1)\times (N+1)\end{math}:

\begin{center}
\begin{math}
\begin{pmatrix}
A & b \\
0 & 1
\end{pmatrix}
\end{math}
\end{center}
\pause
\item Можно применять как к точкам, так и к векторам (векторы игнорируют сдвиг на \begin{math}b\end{math})
\pause
\item Композиция преобразований = умножение матриц
\pause
\item \begin{math}\Rightarrow\end{math} Позволяет удобно комбинировать преобразования (напр. масштабирование + сдвиг + поворот + другой сдвиг)
\end{itemize}
\end{frame}

\begin{frame}[fragile]
\frametitle{Точка в однородных координатах}
\begin{center}
\begin{math}
\begin{pmatrix}
p_1 \\
p_2 \\
p_3 \\
1
\end{pmatrix}
\end{math}
\end{center}
\end{frame}

\begin{frame}[fragile]
\frametitle{Вектор в однородных координатах}
\begin{center}
\begin{math}
\begin{pmatrix}
v_1 \\
v_2 \\
v_3 \\
0
\end{pmatrix}
\end{math}
\end{center}
\end{frame}

\begin{frame}[fragile]
\frametitle{Аффинное преобразование в однородных координатах}
\begin{center}
\begin{math}
\begin{pmatrix}
A_{1,1} & A_{1,2} & A_{1,3} & b_1 \\
A_{2,1} & A_{2,2} & A_{2,3} & b_2 \\
A_{3,1} & A_{3,2} & A_{3,3} & b_3 \\
0 & 0 & 0 & 1
\end{pmatrix}
\end{math}
\end{center}
\end{frame}

\begin{frame}[fragile]
\frametitle{Матрица сдвига на фиксированный вектор}
\begin{center}
\begin{math}
\begin{pmatrix}
1 & 0 & 0 & t_x \\
0 & 1 & 0 & t_y \\
0 & 0 & 1 & t_z \\
0 & 0 & 0 & 1
\end{pmatrix}
\cdot
\begin{pmatrix}
p_x \\
p_y \\
p_z \\
1
\end{pmatrix}
=
\begin{pmatrix}
p_x + t_x \\
p_y + t_y\\
p_z + t_z\\
1
\end{pmatrix}
\end{math}

\begin{math}
\begin{pmatrix}
1 & 0 & 0 & t_x \\
0 & 1 & 0 & t_y \\
0 & 0 & 1 & t_z \\
0 & 0 & 0 & 1
\end{pmatrix}
\cdot
\begin{pmatrix}
v_x \\
v_y \\
v_z \\
0
\end{pmatrix}
=
\begin{pmatrix}
v_x\\
v_y\\
v_z\\
0
\end{pmatrix}
\end{math}
\end{center}
\end{frame}

\begin{frame}[fragile]
\frametitle{Матрица масштабирования на константу}
\begin{center}
\begin{math}
\begin{pmatrix}
s & 0 & 0 & 0 \\
0 & s & 0 & 0 \\
0 & 0 & s & 0 \\
0 & 0 & 0 & 1
\end{pmatrix}
\cdot
\begin{pmatrix}
p_x \\
p_y \\
p_z \\
1
\end{pmatrix}
=
\begin{pmatrix}
s p_x\\
s p_y\\
s p_z\\
1
\end{pmatrix}
\end{math}

\begin{math}
\begin{pmatrix}
s & 0 & 0 & 0 \\
0 & s & 0 & 0 \\
0 & 0 & s & 0 \\
0 & 0 & 0 & 1
\end{pmatrix}
\cdot
\begin{pmatrix}
v_x \\
v_y \\
v_z \\
0
\end{pmatrix}
=
\begin{pmatrix}
s v_x\\
s v_y\\
s v_z\\
0
\end{pmatrix}
\end{math}
\end{center}
\end{frame}

\begin{frame}[fragile]
\frametitle{Матрица поворота на угол в плоскости XY}
\begin{center}
\begin{math}
\begin{pmatrix}
\cos \theta & - \sin \theta & 0 & 0 \\
\sin \theta & \cos \theta & 0 & 0 \\
0 & 0 & 1 & 0 \\
0 & 0 & 0 & 1
\end{pmatrix}
\cdot
\begin{pmatrix}
p_x \\
p_y \\
p_z \\
1
\end{pmatrix}
=
\begin{pmatrix}
p_x \cos \theta - p_y \sin \theta \\
p_x \sin \theta + p_y \cos \theta \\
p_z \\
1
\end{pmatrix}
\end{math}

\begin{math}
\begin{pmatrix}
\cos \theta & - \sin \theta & 0 & 0 \\
\sin \theta & \cos \theta & 0 & 0 \\
0 & 0 & 1 & 0 \\
0 & 0 & 0 & 1
\end{pmatrix}
\cdot
\begin{pmatrix}
v_x \\
v_y \\
v_z \\
0
\end{pmatrix}
=
\begin{pmatrix}
v_x \cos \theta - v_y \sin \theta \\
v_x \sin \theta + v_y \cos \theta \\
v_z \\
0
\end{pmatrix}
\end{math}
\end{center}
\end{frame}

\begin{frame}
\frametitle{Аффинные преобразования: ссылки}
\begin{itemize}
\item \href{https://learnopengl.com/Getting-started/Transformations}{\nolinkurl{learnopengl.com/Getting-started/Transformations}}
\item \href{https://open.gl/transformations}{\nolinkurl{open.gl/transformations}}
\item \href{https://en.wikipedia.org/wiki/Affine_space}{\nolinkurl{en.wikipedia.org/wiki/Affine\_space}}
\item \href{https://en.wikipedia.org/wiki/Affine_transformation}{\nolinkurl{en.wikipedia.org/wiki/Affine\_transformation}}
\end{itemize}
\end{frame}

\end{document}