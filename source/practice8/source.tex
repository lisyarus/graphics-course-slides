% (c) Nikita Lisitsa, lisyarus@gmail.com, 2023

\documentclass[10pt]{beamer}

\usepackage[T2A]{fontenc}
\usepackage[russian]{babel}
\usepackage{minted}

\usepackage{graphicx}
\graphicspath{ {./images/} }

\usetheme{metropolis}

\usepackage{adjustbox}

\usepackage{color}
\usepackage{soul}

\usepackage{hyperref}

\definecolor{blue}{rgb}{0,0,1}
\definecolor{red}{rgb}{1,0,0}
\definecolor{codebg}{RGB}{29,35,49}

\makeatletter
\newcommand{\slideimage}[1]{
  \begin{figure}
    \begin{adjustbox}{width=\textwidth, totalheight=\textheight-2\baselineskip-2\baselineskip,keepaspectratio}
      \includegraphics{#1}
    \end{adjustbox}
  \end{figure}
}
\makeatother

\title{Компьютерная графика}
\subtitle{Практика 8: Shadow mapping}
\date{2023}

\setbeamertemplate{footline}[frame number]

\begin{document}

\usemintedstyle{solarized-light}

\frame{\titlepage}

\begin{frame}[fragile]
\slideimage{0.png}
\end{frame}

\begin{frame}[fragile]
\frametitle{Задание 1}
\begin{footnotesize}
При инициализации создаём и настраиваем shadow map и framebuffer
\begin{itemize}
\item Выбираем размер shadow map: например, \mintinline{cpp}|shadow_map_size = 1024|
\item Создаём текстуру для shadow map: min/mag фильтры -- \mintinline{cpp}|GL_NEAREST|, размеры -- \mintinline{cpp}|shadow_map_size x shadow_map_size|, internal format -- \mintinline{cpp}|GL_DEPTH_COMPONENT24|, format -- \mintinline{cpp}|GL_DEPTH_COMPONENT|, type -- \mintinline{cpp}|GL_FLOAT|, в данных -- \mintinline{cpp}|nullptr|
\item Настраиваем ей параметры \mintinline{cpp}|GL_TEXTURE_WRAP_S| и \mintinline{cpp}|GL_TEXTURE_WRAP_T| в значение \mintinline{cpp}|GL_CLAMP_TO_EDGE|
\item Создаём framebuffer, присоединяем к нему нашу текстуру в качестве глубины (\mintinline{cpp}|glFramebufferTexture|, \mintinline{cpp}|GL_DEPTH_ATTACHMENT|), \mintinline{cpp}|target| лучше использовать \mintinline{cpp}|GL_DRAW_FRAMEBUFFER|
\item Проверяем, что фреймбуффер настроен правильно (\mintinline{cpp}|glCheckFramebufferStatus|)
\item N.B. Экран будет чёрный, так как мы не сделали дефолтный фреймбуфер текущим :)
\end{itemize}
\end{footnotesize}
\end{frame}

\begin{frame}[fragile]
\frametitle{Задание 1}
\slideimage{1.png}
\end{frame}

\begin{frame}[fragile]
\frametitle{Задание 2}
\begin{footnotesize}
Добавляем дебажный прямоугольник с собственной шейдерной программой, чтобы видеть содержимое нашей shadow map
\begin{itemize}
\item В начале рендеринга (перед \mintinline{cpp}|glClear|, сразу после обработки событий) делаем текущим дефолтный (ID = 0) фреймбуфер, чтобы снова увидеть сцену
\item Создаём новый вершинный шейдер: выдаёт (в \mintinline{cpp}|gl_Position| захардкоженные координаты вершин, используя \mintinline{glsl}|gl_VertexID| (как в первой практике), и передаёт (через \mintinline{glsl}|out vec2 texcoord|) во фрагментный шейдер текстурные координаты (без каких-либо матриц)
\item Должно быть 6 вершин -- два треугольника, образующих прямоугольник
\item Координаты вершин должны быть где-то в нижнем левом углу экрана (например, [-1.0 .. -0.5] по обеим осям)
\item Текстурные координаты должны быть [0.0 .. 1.0] по обеим осям, чтобы они покрыли всю текстуру, т.е. (0, 0) у левого нижнего угла, (1, 0) у правого нижнего, и т.д.
\end{itemize}
\end{footnotesize}
\end{frame}

\begin{frame}[fragile]
\frametitle{Задание 2}
\begin{footnotesize}
Добавляем дебажный прямоугольник с собственной шейдерной программой, чтобы видеть содержимое нашей shadow map
\begin{itemize}
\item Фрагментный шейдер: читает цвет из переданной текстуры (\mintinline{glsl}|uniform sampler2D|) и выводит в \mintinline{glsl}|out_color|, можно только красный канал: \mintinline{glsl}|vec4(texture(...).r)| (другие каналы содержат нули -- так себя ведёт тип пикселя \mintinline{cpp}|GL_DEPTH_COMPONENT|)
\item При инициализации создаём фиктивный VAO (без настройки атрибутов вершин)
\item После рисования основной модели, перед \mintinline{cpp}|SDL_GL_SwapBuffers| рисуем прямоугольник с помощью \mintinline{cpp}|glDrawArrays(GL_TRIANGLES, 0, 6)| (не забываем сделать текущими созданный VAO, новую шейдерную программу и текстуру shadow map, а также выключить тест глубины, чтобы прямоугольник не оказался `за' основной сценой)
\item По-хорошему для связи \mintinline{glsl}|sampler2D| и текстуры нужен texture unit; для простоты можем воспользоваться тем, что по умолчанию активный texture unit -- нулевой, и значение uniform-переменных по умолчанию -- тоже ноль
\item N.B. прямоугольник будет белым (или чёрным, зависит от драйвера), так как shadow map пока пустой
\end{itemize}
\end{footnotesize}
\end{frame}

\begin{frame}[fragile]
\frametitle{Задание 2}
Теперь весь код рисования кадра должен выглядеть как-то так:
\begin{itemize}
\item Делаем текущим дефолтный (ID = 0) фреймбуфер, настраиваем viewport, очищаем color и depth буферы, настраиваем depth test и culling
\item Включаем основную шейдерную программу, рисуем сцену
\item Включаем новую шейдерную программу, рисуем прямоугольник
\end{itemize}
\end{frame}

\begin{frame}[fragile]
\frametitle{Задание 2}
\slideimage{2.png}
\end{frame}

\begin{frame}[fragile]
\frametitle{Задание 3}
\begin{footnotesize}
Генерируем shadow map
\begin{itemize}
\item Выбираем проекцию для shadow map: для начала сгодится проекция `снизу-вверх' (как будто камера смотрит сверху)
\item При рисовании кадра вычисляем оси проекции:
\item \mintinline{cpp}|light_Z = glm::vec3(0, -1, 0)| -- направление, противоположное вгзляду камеры
\item \mintinline{cpp}|light_X = glm::vec3(1,  0, 0)|
\item \mintinline{cpp}|light_Y = glm::cross(light_X, light_Z)|
\item Матрица проекции: \mintinline{cpp}|glm::mat4(glm::transpose(glm::mat3(light_X, light_Y, light_Z)))| (пользуемся тем, что матрица из этих трёх векторов -- ортогональная; в общем случае \mintinline{cpp}|transpose| надо заменить на \mintinline{cpp}|inverse|: см. лекцию про камеры и проекции)
\end{itemize}
\end{footnotesize}
\end{frame}

\begin{frame}[fragile]
\frametitle{Задание 3}
\begin{footnotesize}
Генерируем shadow map
\begin{itemize}
\item Пишем новую шейдерную программу для вычисления shadow map:
\item Вершинный шейдер преобразует вершины \mintinline{cpp}|gl_Position = shadow_projection * model * vec4(in_position, 1.0)| (матрица view здесь не нужна -- мы не настраиваем реальную камеру, а просто вычисляем тени)
\item Фрагментный шейдер ничего не делает (пустая функция \mintinline{glsl}|main|; глубина пикселя, которая нам и нужна, пишется сама, автоматически)
\item \textbf{Перед} рисованием основной сцены и прямоугольника: используем созданный ранее фреймбуфер для рисования, настраиваем viewport (размер -- \mintinline{cpp}|shadow_map_size x shadow_map_size|), очищаем буфер глубины, включаем front-face culling (чтобы избавиться от shadow acne), включаем depth test, рисуем нашу модель созданной шейдерной программой
\item После этого не забываем вернуть back-face culling
\item Модель должна появиться в нашем дебажном прямоугольнике
\end{itemize}
\end{footnotesize}
\end{frame}

\begin{frame}[fragile]
\frametitle{Задание 3}
Теперь весь код рисования кадра должен выглядеть как-то так:
\begin{itemize}
\item Делаем текущим созданный в задании 1 фреймбуфер, настраиваем viewport, очищаем depth буфер, настраиваем depth test и front-face culling
\item Включаем шейдерную программу для рисования shadow map, рисуем сцену
\item Делаем текущим дефолтный (ID = 0) фреймбуфер, настраиваем viewport, очищаем color и depth буферы, настраиваем depth test и back-face culling
\item Включаем основную шейдерную программу, рисуем сцену
\item Включаем шейдерную программу для прямоугольника, рисуем прямоугольник
\end{itemize}
\end{frame}

\begin{frame}[fragile]
\frametitle{Задание 3}
\slideimage{3.png}
\end{frame}

\begin{frame}[fragile]
\frametitle{Задание 4}
\begin{footnotesize}
\setminted{fontsize=\footnotesize}
Используем shadow map
\begin{itemize}
\item Передаём текстуру shadow map (\mintinline{glsl}|uniform sampler2D|) и проекцию для неё (\mintinline{glsl}|uniform mat4 projection|) в основную шейдерную программу
\item Во фрагментном шейдере:
\begin{itemize}
\item Вычисляем ndc-координаты текущей точки после применения проекции: \mintinline{glsl}|vec4 ndc = shadow_projection * vec4(position, 1.0)|
\item Проверяем точку на попадание в видимую область shadow map (XY-координаты \mintinline{glsl}|ndc| должны быть в диапазоне [-1..1])
\item Если точка попала в shadow map, вычисляем её текстурные координаты для shadow map \mintinline{glsl}|shadow_texcoord = ndc.xy * 0.5 + 0.5| и глубину \mintinline{glsl}|shadow_depth = ndc.z * 0.5 + 0.5|
\item Если значение в shadow map \mintinline{glsl}|texture(shadow_map, shadow_texcoord)| меньше глубины нашей точки, она в тени (к ней не нужно применять прямое освещение, но ambient остаётся)
\end{itemize}
\item Тень будет выглядеть так, будто свет падает сверху
\end{itemize}
\end{footnotesize}
\end{frame}

\begin{frame}[fragile]
\frametitle{Задание 4}
\slideimage{4.png}
\end{frame}

\begin{frame}[fragile]
\frametitle{Задание 5}
\fontsize{10pt}{10pt}
Вычисляем настоящую проекцию
\begin{itemize}
\item \mintinline{cpp}|light_Z = -light_direction|
\item \mintinline{cpp}|light_X| -- любой вектор, ортогональный \mintinline{cpp}|light_Z|
\item \mintinline{cpp}|light_Y = glm::cross(light_X, light_Z)|
\end{itemize}
\end{frame}

\begin{frame}[fragile]
\frametitle{Задание 5}
\slideimage{5.png}
\end{frame}

\begin{frame}[fragile]
\frametitle{Задание 6}
\fontsize{10pt}{10pt}
Включаем PCF
\begin{itemize}
\item Меняем min/mag фильтры shadow map на \mintinline{cpp}|GL_LINEAR|
\item Настраиваем текстуре shadow map опции \mintinline{cpp}|GL_TEXTURE_COMPARE_MODE = GL_COMPARE_REF_TO_TEXTURE| и \mintinline{cpp}|GL_TEXTURE_COMPARE_FUNC = GL_LEQUAL| (тоже через \mintinline{cpp}|glTexParameteri|)
\item Заменяем в основном фрагментном шейдере \mintinline{glsl}|sampler2D shadow_map| на \mintinline{glsl}|sampler2DShadow|
\item Сравнение \mintinline{glsl}|texture(shadow_map, shadow_texcoord) < shadow_depth| заменяется на один вызов \mintinline{glsl}|texture(shadow_map, shadow_texcoord)| -- вернёт значение от 0 до 1 (если в тени или не в тени, соответственно)
\item N.B. дебажный прямоугольник перестанет работать :(
\end{itemize}
\end{frame}

\begin{frame}[fragile]
\frametitle{Задание 6}
\slideimage{6.png}
\end{frame}

\begin{frame}[fragile]
\frametitle{Задание 7*}
\fontsize{10pt}{10pt}
Добавляем размытие к PCF
\begin{itemize}
\item Во фрагментном шейдере, вместо однократного чтения shadow map \mintinline{glsl}|texture(shadow_map, shadow_texcoord)| читаем значения из соседних пикселей (надо будет что-то прибавить к \mintinline{glsl}|shadow_texcoord|) и усредняем по Гауссу
\item Тени должны получиться более размытыми
\end{itemize}
\end{frame}

\begin{frame}[fragile]
\frametitle{Задание 7*}
\slideimage{7.png}
\end{frame}

\end{document}
