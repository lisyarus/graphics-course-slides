% (c) Nikita Lisitsa, lisyarus@gmail.com, 2022

\documentclass{beamer}

\usepackage[T2A]{fontenc}
\usepackage[utf8]{inputenc}
\usepackage[russian]{babel}

\usepackage{graphicx}
\graphicspath{ {./images/} }

\usepackage{adjustbox}

\usepackage{color}
\usepackage{soul}

\usepackage{hyperref}

\usepackage{amsmath}

\usepackage{tikz}
\usetikzlibrary{decorations}
\usetikzlibrary{decorations.pathreplacing}
\usepackage{xifthen}

\definecolor{red}{rgb}{1,0,0}
\definecolor{green}{rgb}{0,0.5,0}
\definecolor{blue}{rgb}{0,0,1}
\definecolor{magenta}{rgb}{0.75,0,0.75}

\hypersetup{
    colorlinks=true,
    linkcolor=blue,
    urlcolor=blue,
}

\makeatletter
\newcommand{\slideimage}[1]{
  \begin{figure}
    \begin{adjustbox}{width=\textwidth, totalheight=\textheight-2\baselineskip-2\baselineskip,keepaspectratio}
      \includegraphics{#1}
    \end{adjustbox}
  \end{figure}
}
\makeatother

\title{Компьютерная графика}
\subtitle{Лекция 15: рендеринг текста, bitmap-шрифты, векторные шрифты, (M)SDF-шрифты, чем заняться дальше}
\date{2021}

\setbeamertemplate{footline}[frame number]

\begin{document}

\frame{\titlepage}

\begin{frame}[fragile]
\frametitle{Рендеринг текста}
\begin{itemize}
\item Текст -- очень сложная штука
\pause
\item Слева направо (латиница, кириллица), справа налево (арабский, иврит), сверху вниз (хангыль, старомонгольское письмо)
\pause
\item Иероглифы (кандзи), слоговое письмо (катакана, хирагана), консонантное письмо (арабский), консонантно-вокалическое письмо (латиница, кириллица)
\pause
\item Одна \textit{графема} может представлять один или несколько звуков или слогов
\pause
\item Могут быть сложные правила по соединению символов между собой (арабский, лигатуры в латинице), дополнения к символам (диакритика)
\end{itemize}
\end{frame}

\begin{frame}[fragile]
\frametitle{Алфавиты}
\slideimage{alphabet.jpg}
\end{frame}

\begin{frame}[fragile]
\frametitle{Корейская письменность}
\slideimage{korean.png}
\end{frame}

\begin{frame}[fragile]
\frametitle{Арабская письменность}
\slideimage{arabic.png}
\end{frame}

\begin{frame}[fragile]
\frametitle{Этапы рендеринга текста}
\begin{itemize}
\item Абстрактный текст
\pause
\item + кодировка \begin{math}\Rightarrow\end{math} машинное представление текста
\pause
\item + шрифт + настройки шейпинга (shaping) \begin{math}\Rightarrow\end{math} набор глифов (изображений символов) и их координат
\pause
\item + алгоритм рендеринга \begin{math}\Rightarrow\end{math} нарисованный текст
\end{itemize}
\end{frame}

\begin{frame}[fragile]
\frametitle{Кодировки}
\fontsize{10pt}{10pt}
\begin{itemize}
\item Описывают машинное представление текста, т.е. соответствие \textbf{последовательностей} символов и \textbf{последовательностей} бит
\pause
\item ASCII: 7 бит (обычно дополняется нулевым старшим битом до 8 бит), первые 32 символа - управляющие (\verb|\r|, \verb|\n|, tab, ...), остальные 96 -- буквы английского алфавита (большие и маленькие) и прочие символы (различные скобки, арифметические операции, пунктуация, пробел, ...)
\pause
\begin{itemize}
\item Многие кодировки совпадают с ASCII в диапазоне 0-127 или 32-127
\end{itemize}
\pause
\item Огромное количество в основном 8-битных кодировок для разных алфавитов и систем:
\begin{itemize}
\item ISO/IEC 8859 -- 15 разных вариантов (ISO/IEC 8859-5 для русского языка)
\item Code page XXX -- много разных кодировок для DOS (Code page 866 для русского языка)
\item Windows code pages (Windows-1251 для русского языка)
\item KOI-8 и вариации -- для русского языка
\item etc.
\end{itemize}
\pause
\item Unicode-кодировки
\end{itemize}
\end{frame}

\begin{frame}[fragile]
\frametitle{Unicode}
\begin{itemize}
\item Unicode -- стандарт, описывающий соответствие абстрактных символов целочисленным кодам (\textit{code points}) в диапазоне \verb|0..10FFFFh| исключая \verb|D800h..DFFFh| для суррогатных пар в UTF-16 (итого 1112064 code point'а), и рекомендации по их интерпретации и визуализации
\pause
\item На сегодняшний день описывает 149186 символов (в прошлом году было 144697)
\pause
\item Сам unicode -- \textbf{не кодировка}, но есть основанные на нём кодировки:
\pause
\begin{itemize}
\item UTF-8: от 1 до 4 байт на символ, совпадает с ASCII в диапазоне \verb|0..7Fh|, самая распространённая сегодня кодировка (95\% интернета)
\pause
\item UCS-2: устаревшая, 2 байта на символ, не поддерживает весь unicode
\pause
\item UTF-16: 2 или 4 байта на символ
\pause
\item UTF-32: 4 байта на символ
\pause
\item GB 18030: специальная кодировка для китайских иероглифов (но тоже поддерживает весь unicode)
\end{itemize}
\end{itemize}
\end{frame}

\begin{frame}[fragile]
\frametitle{Code points, глифы и графемы}
\begin{itemize}
\item Code point -- один unicode элемент (абстрактный символ)
\pause
\item Глиф -- одно изображение в шрифте
\pause
\item Графема -- один визуальный символ (один или несколько глифов)
\pause
\item В общем случае один code point \textbf{не соответствует} одному глифу или одной графеме
\pause
\item Примеры:
\begin{itemize}
\item Два символа \verb|ff| могут быть представлены двумя глифами или одним глифом (лигатурой) ff
\pause
\item Символ \verb|Ò| может быть одним или двумя code point'ами и одним или двумя (\verb|O + `|) глифами, но считается одной графемой
\end{itemize}
\end{itemize}
\end{frame}

\begin{frame}[fragile]
\frametitle{Шрифты}
\begin{itemize}
\item Содержит набор \textit{глифов} (изображений символов в каком-либо виде) и правил их использования
\pause
\item Виды шрифтов:
\pause
\begin{itemize}
\item Bitmap-шрифты: глиф -- готовое изображение (bitmap)
\pause
\item Векторные шрифты: глиф описывается как геометрическая фигура
\pause
\item (M)SDF-шрифты: глиф описывается с помощью \textit{signed distance field} (SDF)
\end{itemize}
\pause
\item Современные форматы шрифтов (\verb|.ttf| -- TrueType, \verb|.otf| -- OpenType) -- векторные, описывают границу глифа как набор отрезков и квадратичных кривых Безье (т.е. 2-ого порядка)
\pause
\item Bitmap и SDF шрифты часто строятся по векторным шрифтам
\pause
\item \textbf{FreeType} -- самая распространённая библиотека для чтения векторных шрифтов; умеет растеризовать в bitmap и (с версии 2.11.0, июль 2021) в SDF
\end{itemize}
\end{frame}

\begin{frame}[fragile]
\frametitle{Шейпинг (shaping)}
\begin{itemize}
\item Процесс преобразования последовательности символов в набор отпозиционированных глифов
\pause
\item Может включать в себя:
\pause
\begin{itemize}
\item Настройки: направление (слева-направо, справа-налево, сверху-вниз, снизу-вверх), размер шрифта, межбуквенное расстояние, стиль (жирный, курсив, и т.п.)
\pause
\item Hinting: сдвиг глифов, чтобы они были лучше выровнены по пиксельной сетке
\pause
\item Kerning: изменение расстояния между соседними глифами для лучшего восприятия
\pause
\item Лигатуры: последовательность несвязанных символов, представленная одним глифом (ff, fi, <=>)
\end{itemize}
\pause
\item \textbf{Не занимается} расстановкой строк и абзацев
\pause
\item Для простых моноширинных шрифтов шейпинг может сводиться к расположению глифов на равных расстояниях друг от друга
\pause
\item \textbf{harfbuzz} -- одна из самых распространённых библиотек для шейпинга текста, используется всеми на свете
\item FreeType позволяет сделать шейпинг, но хуже, чем harfbuzz
\end{itemize}
\end{frame}

\begin{frame}[fragile]
\frametitle{Hinting}
\slideimage{hinting.png}
\end{frame}

\begin{frame}[fragile]
\frametitle{Kerning}
\slideimage{kerning.png}
\end{frame}

\begin{frame}[fragile]
\frametitle{Лигатуры}
\slideimage{ligatures.png}
\end{frame}

\begin{frame}[fragile]
\frametitle{Рендеринг bitmap-шрифтов}
\begin{itemize}
\item Обычно представлены в виде texture atlas: одна текстура, содержащая все глифы шрифта
\pause
\item Содержит информацию о расположении глифов в текстуре (текстурные координаты левого верхнего и правого нижнего пикселя)
\pause
\item Глифы рисуются как текстурированные прямоугольники
\pause
\item Плохо ведёт себя при масштабировнии (как увеличении, так и уменьшении), mipmap'ы не особо помогают
\pause
\item Очень прост в реализации
\pause
\item Часто используется для дебажного текста, инди-игр, и т.п.
\end{itemize}
\end{frame}

\begin{frame}[fragile]
\frametitle{Bitmap-шрифт}
\slideimage{bitmap-font.png}
\end{frame}

\begin{frame}[fragile]
\frametitle{Bitmap-шрифт: описание в коде}
\begin{verbatim}
struct bitmap_font
{
  GLuint texture_id;

  struct glyph
  {
    vec2 top_left;
    vec2 bottom_right;
  };

  std::unordered_map<std::char32_t, glyph> glyphs;
};
\end{verbatim}
\end{frame}

\begin{frame}[fragile]
\frametitle{Bitmap-шрифт: фрагментный шейдер}
\begin{verbatim}
uniform sampler2D font_texture;
uniform vec3 text_color;

in vec2 texcoord;

layout (location = 0) out vec4 out_color;

void main()
{
  float alpha = texture(font_texture, texcoord).r;
  out_color = vec4(text_color, alpha);
}
\end{verbatim}
\end{frame}

\begin{frame}[fragile]
\frametitle{Рендеринг векторных шрифтов}
\begin{itemize}
\item Глиф описывается как набор геометрических фигур (фигура может описывать `дырку' в другой фигуре, как дырка в букве `О'), граница фигуры -- набор отрезков и квадратичных кривых Безье
\pause
\item Много разных способов рендеринга:
\pause
\begin{itemize}
\item Аппроксимация набором треугольников
\pause
\item Запаковка фигур в текстуру + шейдер, который честно вычисляет площадь пересечения фигуры и пикселя
\pause
\item Полигональная аппроксимация глифа (рисуется с использованием stencil буфера) + треугольник со специальным шейдером для каждой кривой Безье
\pause
\item Slug algorithm (запатентован)
\end{itemize}
\pause
\item Обычно легко переносит масштабирование
\pause
\item Сложен в реализации
\pause
\item Используется для текста максимально возможного качества
\end{itemize}
\end{frame}

\begin{frame}[fragile]
\frametitle{Векторный глиф}
\slideimage{vector-glyph.png}
\end{frame}

\begin{frame}[fragile]
\frametitle{Slug algorithm}
\slideimage{slug.jpg}
\end{frame}

\begin{frame}[fragile]
\frametitle{Signed distance field (SDF)}
\begin{itemize}
\item Описание двумерного или трёхмерного объекта/фигуры \textit{функцией расстояния} до границы объекта
\pause
\item Обычно положительна снаружи объекта и отрицательна внутри (поэтому \textit{signed}), \begin{math}f(p) = 0\end{math} -- граница объекта
\pause
\item SDF может быть представлена явной формулой (напр. \begin{math}f(p) = \|p - O\| - R\end{math} -- расстояние до сферы радиуса \begin{math}R\end{math} с центром в точке \begin{math}O\end{math}) или текстурой
\pause
\item SDF-сцены часто используются для экспериментального рендеринга и удобны для raymarching'а
\end{itemize}
\end{frame}

\begin{frame}[fragile]
\frametitle{Рендеринг SDF-шрифтов (Valve, 2007)}
\begin{itemize}
\item Описывается так же, как bitmap-шрифт, но текстура хранит значения SDF для глифов
\pause
\item Фрагментный шейдер читает значение SDF из текстуры шрифта: если оно меньше 0, то пиксель находится внутри глифа (e.g. чёрный пиксель), иначе -- нет (e.g. прозрачный пиксель)
\pause
\item Прост в реализации
\pause
\item Требует чуть больше места под глифы, но менее требователен к разрешению
\pause
\item Неплохо масштабируется (есть артефакты, но менее серьёзные, чем для bitmap-шрифтов)
\pause
\item Один из самых распространённых способов рендеринга шрифтов
\end{itemize}
\end{frame}

\begin{frame}[fragile]
\frametitle{SDF-шрифт}
\slideimage{sdf-font.png}
\end{frame}

\begin{frame}[fragile]
\frametitle{SDF-шрифт: артефакты при magnification}
\slideimage{sdf-artifacts.jpg}
\end{frame}

\begin{frame}[fragile]
\frametitle{SDF-текстура}
\begin{itemize}
\item Тестуры позволяют хранить значения от 0 до 1
\pause
\item Мы хотим хранить произвольные, но не очень большие по модулю числа (расстояние в пикселях)
\pause
\item \begin{math}\Rightarrow\end{math} В текстуре придётся хранить что-то в духе \begin{math}0.5 + sdf / scale\end{math}, где \verb|scale| -- максимальное представимое расстояние
\pause
\item \begin{math}\Rightarrow\end{math} При чтении из текстуры нужно применять обратное преобразование \begin{math}(sdf - 0.5) * scale\end{math}
\pause
\item Лучше включить для этой текстуры анизотропную фильтрацию, чтобы текст хорошо выглядел `сбоку'
\end{itemize}
\end{frame}

\begin{frame}[fragile]
\frametitle{Рендеринг SDF-шрифтов}
\begin{itemize}
\item Можно легко реализовать много дополнительных эффектов:
\pause
\begin{itemize}
\item Сглаживание: вместо жёсткой границы плавно меняем прозрачность пикселя в зависимости от значения SDF
\pause
\item Сглаживание с учётом масштаба и перспективы: по \verb|dFfx,dFfy| можно вычислить диапазон значений SDF, чтобы сглаживание было ровно в 1 пиксель в экранных координатах
\pause
\item Обводка текста другим цветом: рисуем цвет обводки, если \begin{math}0 \leq f(p) \leq \varepsilon\end{math}
\pause
\item Псевдотрёхмерный текст: по градиенту SDF можно восстановить нормаль к глифу
\end{itemize}
\end{itemize}
\end{frame}

\begin{frame}[fragile]
\frametitle{SDF-шрифт с эффектами}
\slideimage{sdf-effects.png}
\end{frame}

\begin{frame}[fragile]
\frametitle{SDF-шрифт: фрагментный шейдер}
\fontsize{10pt}{10pt}
\begin{verbatim}
uniform sampler2D sdfTexture;
uniform float sdfScale;
uniform vec3 textColor;

in vec2 texcoord;

layout (location = 0) out vec4 out_color;

void main()
{
  float sdfTextureValue = texture(sdfTexture, texcoord).r;
  float sdfValue = sdfScale * (sdfTextureValue - 0.5);
  // сглаживание
  float alpha = smoothstep(-0.5, 0.5, sdfValue);
  out_color = vec4(textColor, alpha);
}
\end{verbatim}
\end{frame}

\begin{frame}[fragile]
\frametitle{MSDF (Chlumský, 2015)}
\begin{itemize}
\item У SDF-текста есть типичные артефакты: острые углы сглаживаются, из-за чего приходится брать SDF-текстуру большого разрешения
\pause
\item Идея: билинейная интерполяция не портит прямые линии \begin{math}\Rightarrow\end{math} представим глиф \textit{пересечением} нескольких объектов, чтобы острые углы, представленные пересечением нескольких прямых, не сглаживались
\pause
\item В текстуре есть 4 канала (RGBA) \begin{math}\Rightarrow\end{math} можем сохранить сразу 4 разных SDF в одной текстуре!
\end{itemize}
\end{frame}

\begin{frame}[fragile]
\frametitle{MSDF}
\begin{itemize}
\item \begin{math}\Rightarrow\end{math} MSDF: Multi-channel signed distance field
\pause
\item На практике оказывается, что хватает 3 каналов, т.е. трёх различных SDF
\item Вместо пересечения (т.е. минимума из трёх значений SDF) лучше брать \textit{медиану} (т.е. значение посередине между двумя другими)
\pause
\item Есть инструменты (программа, библиотека, сайт) для генерации таких текстур по шрифту
\end{itemize}
\end{frame}

\begin{frame}[fragile]
\frametitle{MSDF-шрифт}
\slideimage{msdf-font.png}
\end{frame}

\begin{frame}[fragile]
\frametitle{MSDF: сравнение с SDF}
\slideimage{msdf.png}
\end{frame}

\begin{frame}[fragile]
\frametitle{MSDF: пример кода}
\fontsize{10pt}{10pt}
\begin{verbatim}
uniform sampler2D sdfTexture;
uniform float sdfScale;
uniform vec3 textColor;

in vec2 texcoord;

layout (location = 0) out vec4 out_color;

float median(vec3 v) {
    return max(min(v.r, v.g), min(max(v.r, v.g), v.b));
}

void main()
{
  float sdfTextureValue =
    median(texture(sdfTexture, texcoord).rgb);
  float sdfValue = sdfScale * (sdfTextureValue - 0.5);
  // сглаживание
  float alpha = smoothstep(-0.5, 0.5, sdfValue);
  out_color = vec4(textColor, alpha);
}
\end{verbatim}
\end{frame}

\begin{frame}[fragile]
\frametitle{Шрифты и шейпинг: ссылки}
\begin{itemize}
\item \href{https://freetype.org}{FreeType}
\item \href{https://harfbuzz.github.io}{harfbuzz}
\end{itemize}
\end{frame}

\begin{frame}[fragile]
\frametitle{Векторные и bitmap-шрифты: ссылки}
\begin{itemize}
\item \href{https://learnopengl.com/In-Practice/Text-Rendering}{Туториал по рендерингу bitmap-шрифтов}
\item \href{https://wdobbie.com/post/gpu-text-rendering-with-vector-textures}{Один способ рендеринга векторных шрифтов}
\item \href{https://medium.com/@evanwallace/easy-scalable-text-rendering-on-the-gpu-c3f4d782c5ac}{Другой способ рендеринга векторных шрифтов}
\item \href{https://jcgt.org/published/0006/02/02}{Slug algorithm}
\item \href{https://sluglibrary.com}{Slug library}
\end{itemize}
\end{frame}

\begin{frame}[fragile]
\frametitle{SDF и MSDF-шрифты: ссылки}
\begin{itemize}
\item \href{https://steamcdn-a.akamaihd.net/apps/valve/2007/SIGGRAPH2007_AlphaTestedMagnification.pdf}{Статья от Valve про SDF-текст}
\item \href{https://blog.mapbox.com/drawing-text-with-signed-distance-fields-in-mapbox-gl-b0933af6f817}{Туториал по рендерингу SDF-шрифтов}
\item \href{https://evanw.github.io/font-texture-generator/}{SDF font generator}
\item \href{https://github.com/Chlumsky/msdfgen}{Репозиторий с генератором MSDF-шрифтов от автора этой техники}
\item \href{https://github.com/Chlumsky/msdfgen/files/3050967/thesis.pdf}{Shape Decomposition for Multi-channel Distance Fields (Chlumský, 2015)}
\item \href{https://msdf-bmfont.donmccurdy.com/}{MSDF font generator}
\end{itemize}
\end{frame}

\begin{frame}[fragile]
\frametitle{Чем заняться дальше?}
\begin{itemize}
\item Очень много источников света
\item Очень много объектов
\item Очень большая сцена
\item Сложные объекты
\item Другие API
\item Raytracing
\item Статьи и конференции
\end{itemize}
\end{frame}

\begin{frame}[fragile]
\frametitle{Очень много источников света}
\begin{itemize}
\item Deferred shading
\item Tiled/Clustered shading
\item Compute shaders (OpenGL 4.3 или расширение)
\item \href{https://learnopengl.com/Guest-Articles/2022/Area-Lights}{Площадные источники света}
\end{itemize}
\end{frame}

\begin{frame}[fragile]
\frametitle{Очень много объектов}
\begin{itemize}
\item Batching
\item Frustum culling
\item Occlusion culling
\item LOD
\end{itemize}
\end{frame}

\begin{frame}[fragile]
\frametitle{Очень большая сцена (e.g. планета)}
\begin{itemize}
\item LOD (e.g. для ландшафта)
\item Проблемы с точностью (\verb|float| не хватает; нужно рисовать относительно некой anchor-точки в \verb|double|)
\item Проблемы с буфером глубины (\href{https://nlguillemot.wordpress.com/2016/12/07/reversed-z-in-opengl/}{reversed z})
\end{itemize}
\end{frame}

\begin{frame}[fragile]
\frametitle{Сложные объекты}
\begin{itemize}
\item Вода: отражение + преломление (по Френелю) + `туман' в плотности воды + \href{https://en.wikipedia.org/wiki/Trochoidal_wave}{волны Герстнера}
\item Растительность (vegetation): трава, кусты, деревья
\item Облака + небо (volume rendering)
\item Сложные BRDF
\end{itemize}
\end{frame}

\begin{frame}[fragile]
\frametitle{Другие API}
\begin{itemize}
\item OpenGL 4.0+: compute shaders, indirect rendering, direct state access
\item Vulkan: \href{https://vulkan-tutorial.com}{\nolinkurl{vulkan-tutorial.com}}
\end{itemize}
\end{frame}

\begin{frame}[fragile]
\frametitle{Raytracing + real-time GI}
\begin{itemize}
\item \href{https://raytracing.github.io/books/RayTracingInOneWeekend.html}{Ray Tracing in One Weekend} (\textbf{не} real-time)
\item \href{https://developer.nvidia.com/rtx/raytracing/vkray}{Vulkan raytracing tutorial}
\item \href{https://research.nvidia.com/sites/default/files/publications/GIVoxels-pg2011-authors.pdf}{Voxel cone tracing}
\item \href{https://www.ea.com/seed/news/siggraph21-global-illumination-surfels}{Surflet-based GI}
\item \href{https://research.nvidia.com/publication/2021-06_restir-gi-path-resampling-real-time-path-tracing}{ReSTIR GI}
\end{itemize}
\end{frame}

\begin{frame}[fragile]
\frametitle{Статьи и конференции}
\begin{itemize}
\item \href{https://developer.nvidia.com/gpugems}{GPU Gems 1, 2, 3}
\item SIGGRAPH (e.g. \href{https://kesen.realtimerendering.com/sig2022.html}{2022})
\item \href{GDC}{https://www.gdcvault.com/free/}
\item \href{https://www.jendrikillner.com/tags/weekly/}{Graphics Programming Weekly}
\end{itemize}
\end{frame}

\begin{frame}[fragile]
\frametitle{В заключение}
\begin{itemize}
\item Область real-time рендеринга очень активно развивается
\item О рисовании любого объекта/эффекта можно найти десятки статей и даже PhD
\item Есть тысячи туториалов по всему на свете
\item Не бойтесь гуглить и читать непонятные статьи, со временем станет понятнее
\item Не бойтесь писать мне :)
\end{itemize}
\end{frame}

\end{document}