% (c) Nikita Lisitsa, lisyarus@gmail.com, 2021

\documentclass{beamer}

\usepackage[T2A]{fontenc}
\usepackage[utf8]{inputenc}
\usepackage[russian]{babel}

\usepackage{graphicx}
\graphicspath{ {./images/} }

\usepackage{adjustbox}

\usepackage{color}
\usepackage{soul}

\usepackage{hyperref}

\usepackage{amsmath}

\usepackage{tikz}
\usetikzlibrary{decorations}
\usetikzlibrary{decorations.pathreplacing}
\usepackage{xifthen}

\definecolor{red}{rgb}{1,0,0}
\definecolor{green}{rgb}{0,0.5,0}
\definecolor{blue}{rgb}{0,0,1}
\definecolor{magenta}{rgb}{0.75,0,0.75}

\makeatletter
\newcommand{\slideimage}[1]{
  \begin{figure}
    \begin{adjustbox}{width=\textwidth, totalheight=\textheight-2\baselineskip-2\baselineskip,keepaspectratio}
      \includegraphics{#1}
    \end{adjustbox}
  \end{figure}
}
\makeatother

\title{Компьютерная графика}
\subtitle{Лекция 13: состояние OpenGL (напоминание), матрицы проекций (напоминание), рендеринг в cubemap, дистрибуция приложений на OpenGL}
\date{2021}

\setbeamertemplate{footline}[frame number]

\begin{document}

\frame{\titlepage}

\begin{frame}[fragile]
\frametitle{Настройка графического конвейера}
\begin{itemize}
\item Графический конвейер (pipeline) - набор всех операций, происходящих с данными от момента вызова \verb|glDraw*| до появления пикселей на экране (или текстуре/рендербуфере)
\pause
\item Графический конвейер = programmable pipeline + fixed-function pipeline
\pause
\item Настройка programmable pipeline: шейдеры (шейдерная программа)
\pause
\item Настройка fixed-function pipeline: включение/выключение (\verb|glEnable/glDisable|) конкретных операций и их специфическая настройка
\end{itemize}
\end{frame}

\begin{frame}[fragile]
\frametitle{Настройка fixed-function pipeline}
\begin{itemize}
\item Depth clamp
\begin{itemize}
\item По умолчанию, все примитивы обрезаются по уравнению \begin{math}z \leq |w|\end{math}
\item Можно заменить обрезание clamping'ом через \verb|glEnable(GL_DEPTH_CLAMP)|
\end{itemize}
\pause
\item Culling
\begin{itemize}
\item Можно не рисовать back-facing или front-facing полигоны
\item Включить: \verb|glEnable(GL_CULL_FACE)|
\item Настроить, что \textbf{не} рисуется: \verb|glCullFace|
\item Настроить, что считается back-facing, а что front-facing: \verb|glFrontFace|
\end{itemize}
\pause
\item Viewport
\begin{itemize}
\item Настроить перевод из NDC (normalized device coordinates, [-1..1]) в пиксельные координаты: \verb|glViewport|
\item Обычно нужно делать каждый раз при изменении размеров окна или при переключении фреймбуферов
\end{itemize}
\end{itemize}
\end{frame}

\begin{frame}[fragile]
\frametitle{Настройка fixed-function pipeline}
\begin{itemize}
\item Depth test
\begin{itemize}
\item Можно не рисовать пиксели, находящиеся сзади уже нарисованных пикселей
\item Включить: \verb|glEnable(GL_DEPTH_TEST)|
\item Настроить: \verb|glDepthFunc|
\item Настроить преобразование из NDC в [0, 1]: \verb|glDepthRangef|
\item Включить/выключить запись значений глубины: \verb|glDepthMask|
\end{itemize}
\pause
\item Stencil test
\begin{itemize}
\item Включить: \verb|glEnable(GL_STENCIL_TEST)|
\item Настроить: \verb|glStencilFunc|, \verb|glStencilOp|, \verb|glStencilMask|
\end{itemize}
\pause
\item Scissor test
\begin{itemize}
\item Можно не рисовать пиксели, находящиеся вне некоторого прямоугольника
\item Включить: \verb|glEnable(GL_SCISSOR_TEST)|
\item Настроить: \verb|glScissor|
\end{itemize}
\end{itemize}
\end{frame}

\begin{frame}[fragile]
\frametitle{Настройка fixed-function pipeline}
\begin{itemize}
\item Color mask
\begin{itemize}
\item Настроить запись в конкретные цветовые каналы: \verb|glColorMask|
\end{itemize}
\pause
\item Blending
\begin{itemize}
\item Можно записывать значение некоторой функции от входного цвета и уже записанного цвета
\item Включить: \verb|glEnable(GL_BLEND)|
\item Настроить: \verb|glBlendFunc|/\verb|glBlendFuncSeparate|, \verb|glBlendEquation|, \verb|glBlendColor|
\end{itemize}
\pause
\item Color logical operation
\begin{itemize}
\item Можно записывать результат некоторой побитовой операции от входного цвета и уже записанного цвета
\item Выключает blending
\item Включить: \verb|glEnable(GL_COLOR_LOGIC_OP)|
\item Настроить: \verb|glLogicOp|
\end{itemize}
\end{itemize}
\end{frame}

\begin{frame}[fragile]
\frametitle{Проекции}
\begin{itemize}
\item Ничто не мешает делать с вершинами абсолютно любые преобразования в вершинном шейдере
\pause
\item Обычно используют ортографическую или перспективную проекцию, так как они выражаются матрицами (с учётом perspective divide)
\pause
\item Ортографическая проекция: \textbf{не использует} perspective divide, последняя строка равна (0 0 0 1)
\pause
\item Перспективная проекция: \textbf{использует} perspective divide, последняя строка содержит зависимость от X, Y или Z
\end{itemize}
\end{frame}

\begin{frame}[fragile]
\frametitle{Ортографическая проекция}
\fontsize{10pt}{10pt}
\begin{itemize}
\item Проецирует параллельно некому вектору: точки пространства, отличающиеся на вектор, параллельный направлению взгляда, проецируются в одну точку экрана
\pause
\item Видимая область пространства - параллелепипед
\pause
\item Видимый размер объектов \textbf{не зависит} от расстояния до них
\pause
\item Удобно описать центром видимой области \begin{math}C\end{math} и осями \begin{math}X, Y, Z\end{math}
\pause
\begin{itemize}
\item Центр \begin{math}C\end{math} переходит в центр экрана ((0, 0, 0) в NDC)
\pause
\item Точки, отличающиеся на вектор параллельный \begin{math}X\end{math}, после проекции отличаются только X-координатой
\pause
\item Точки, отличающиеся на вектор параллельный \begin{math}Y\end{math}, после проекции отличаются только Y-координатой
\pause
\item Точки, отличающиеся на вектор параллельный \begin{math}Z\end{math}, после проекции попадают в один пиксель (и отличаются только Z-координатой)
\end{itemize}
\pause
\item N.B.: можно понимать её как композицию аффинного преобразования, двигающего камеру в точку C с осями XYZ, и стандартной ортографической проекции (с единичной матрицей)
\pause
\item Матрица проекции: см. лекцию 4
\end{itemize}
\end{frame}

\begin{frame}[fragile]
\frametitle{Перспективная проекция}
\begin{itemize}
\item Проецирует из некой точки: точки пространства, лежащие на одной прямой с центром проекции, проецируются в одну точку экрана
\pause
\item Видимая область пространства - усечённая пирамида
\pause
\item Видимый размер объектов \textbf{зависит} от расстояния до них: чем объект дальше, тем он мельче
\pause
\item Удобно описать параметрами проекции near, far, fovx, fovy и аффинными преобразованием, двигающим камеру (как с ортографической проекцией)
\pause
\item Обычно за направление взгляда берут ось -Z, чтобы получить левую систему координат
\pause
\begin{itemize}
\item near: всё ближе этого значения (по Z-координате) будет отсекаться
\pause
\item far: всё дальше этого значения (по Z-координате) будет отсекаться
\pause
\item fovx, fovy - угол обзора по X и Y
\end{itemize}
\pause
\item Матрица проекции: см. лекцию 4
\end{itemize}
\end{frame}

\end{document}