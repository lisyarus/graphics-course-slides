% (c) Nikita Lisitsa, lisyarus@gmail.com, 2023

\documentclass{beamer}

\usepackage[T2A]{fontenc}
\usepackage[russian]{babel}
\usepackage{minted}

\usepackage{graphicx}
\graphicspath{ {./images/} }

\usepackage{adjustbox}

\usetheme{metropolis}

\usepackage{color}
\usepackage{soul}

\usepackage{hyperref}

\definecolor{blue}{rgb}{0,0,1}
\definecolor{red}{rgb}{1,0,0}

\makeatletter
\newcommand{\slideimage}[1]{
  \begin{figure}
    \begin{adjustbox}{width=\textwidth, totalheight=\textheight-2\baselineskip-2\baselineskip,keepaspectratio}
      \includegraphics{#1}
    \end{adjustbox}
  \end{figure}
}
\makeatother

\title{Компьютерная графика}
\subtitle{Практика 13: Скелетная анимация}
\date{2023}

\usemintedstyle{solarized-light}

\setbeamertemplate{footline}[frame number]

\begin{document}

\frame{\titlepage}

\begin{frame}[fragile]
\slideimage{0.png}
\end{frame}

\begin{frame}[fragile]
\frametitle{Задание 1}
Проверяем, что нормально заданы веса вершин
\begin{itemize}
\item В вершинном шейдере добавляем два новых атрибута вершин:
\begin{itemize}
\item Номера костей для вершины: \mintinline{glsl}|ivec4 in_joints|, \mintinline{glsl}|location = 3|
\item Веса костей для вершины: \mintinline{glsl}|vec4 in_weights|, \mintinline{glsl}|location = 4|
\item (в VAO эти атрибуты уже настроены)
\end{itemize}
\item Передаём \mintinline{glsl}|in_weights| во фрагментный шейдер и используем в качестве цвета
\end{itemize}
\end{frame}

\begin{frame}[fragile]
\frametitle{Задание 1}
\slideimage{1.png}
\end{frame}

\begin{frame}[fragile]
\frametitle{Задание 2}
\fontsize{8pt}{8pt}
\selectfont
Применяем преобразования костей
\begin{itemize}
\item Возвращаем нормальное вычисление цвета во фрагментном шейдере
\item В вершинном шейдере: заводим uniform-массив для матриц костей \mintinline{glsl}|uniform mat4x3 bones[100]|
\begin{itemize}
\fontsize{8pt}{8pt}
\selectfont
\item Location получаем как \mintinline{cpp}|glGetUniformLocation(program, "bones")|
\end{itemize}
\item Вычисляем взвешенное среднее \mintinline{glsl}|mat4x3 average| матриц для вершины (\mintinline{glsl}|in_joints| -- индексы четырёх матриц, \mintinline{glsl}|in_weights| -- их веса)
\item Перед применением матрицы \mintinline{glsl}|model| применяем к входной вершине матрицу \mintinline{glsl}|mat4(average)|
\item Перед применением матрицы \mintinline{glsl}|mat3(model)| применяем к входной нормали матрицу \mintinline{glsl}|mat3(average)|
\item В цикле рендеринга заводим переменную \mintinline{cpp}|scale| как-то меняющуюся со временем (например, \mintinline{cpp}|0.75 + cos(time) * 0.25|)
\item В цикле рендеринга заводим массив матриц \mintinline{cpp}|std::vector<glm::mat4x3>| размера \mintinline{cpp}|input_model.bones.size()| и заполняем значением \mintinline{cpp}|glm::mat4x3(scale)|
\item Передаём матрицы в uniform-массив одним вызовом \mintinline{cpp}|glUniformMatrix4x3fv|
\end{itemize}
\end{frame}

\begin{frame}[fragile]
\frametitle{Задание 2}
\slideimage{2.png}
\end{frame}

\begin{frame}[fragile]
\frametitle{Задание 3}
\fontsize{8pt}{8pt}
\selectfont
Вычисляем преобразования костей
\begin{itemize}
\item Достаём из входной модели анимацию с названием \mintinline{cpp}|"hip-hop"|
\item Для каждой кости \mintinline{cpp}|i| вычисляем её преобразование: \mintinline{cpp}|glm::mat4 transform = translation * rotation * scale|
\begin{itemize}
\fontsize{8pt}{8pt}
\selectfont
\item \mintinline{cpp}|translation| берётся из \mintinline{cpp}|animation.bones[i].translation(0.f)| (0 -- время кадра, пока про него не думаем), остальные компоненты аналогично
\item Получить по \mintinline{cpp}|translation| матрицу можно через \mintinline{cpp}|glm::translate(glm::mat4(1.f), translation)|, аналогично \mintinline{cpp}|glm::scale| для \mintinline{cpp}|scale|
\item Получить по кватерниону \mintinline{cpp}|rotation| матрицу можно через \mintinline{cpp}|glm::toMat4|
\item Если у кости есть родитель \mintinline{cpp}|input_model.bones[i].parent != -1|, домножаем матрицу \mintinline{cpp}|transform| на матрицу родителя: \mintinline{cpp}|parent_transform * transform|
\item Записываем \mintinline{cpp}|transform| в \mintinline{cpp}|bones[i]| (в этот момент \mintinline{cpp}|mat4| конвертируется в \mintinline{cpp}|mat4x3|)
\end{itemize}
\item После этого отдельным циклом домножаем каждую кость на её inverse-bind матрицу: \mintinline{cpp}|bones[i] = bones[i] * input_model.bones[i].inverse_bind_matrix|
\item Модель будет очень большая (так сделана исходная анимация), отмасштабируйте её с помощью матрицы \mintinline{cpp}|model|
\end{itemize}
\end{frame}

\begin{frame}[fragile]
\frametitle{Задание 3}
\slideimage{3.png}
\end{frame}

\begin{frame}[fragile]
\frametitle{Задание 4}
Анимируем модель
\begin{itemize}
\item Вместо значения 0, передаваемого в \mintinline{cpp}|animation.bones[i].translation(...)| и т.п., передаём значение \mintinline{cpp}|std::fmod(time, animation.max_time)|
\end{itemize}
\end{frame}

\begin{frame}[fragile]
\frametitle{Задание 4}
\slideimage{4.png}
\end{frame}

\begin{frame}[fragile]
\frametitle{Задание 5*}
\begin{footnotesize}
Переключаемся между тремя анимациями
\begin{itemize}
\item В модели есть три анимации: \mintinline{cpp}|"hip-hop"|, \mintinline{cpp}|"rumba"| и \mintinline{cpp}|"flair"|
\item Модель должна \textbf{плавно} переключаться между ними при нажатии на клавиши \mintinline{cpp}|SDLK_1|, \mintinline{cpp}|SDLK_2| и \mintinline{cpp}|SDLK_3|, соответственно
\item Для плавного переключения придётся запомнить имена старой и новой анимации, а так же время, прошедшее со старта переключения -- оно будет параметром интерполяции между двумя анимациями
\begin{itemize}
\begin{footnotesize}
\item Для \mintinline{cpp}|translation| каждой кости интерполируем (\mintinline{cpp}|glm::lerp|) между \mintinline{cpp}|translation| двух анимаций
\item Аналогично для \mintinline{cpp}|rotation| и \mintinline{cpp}|scale| (для вращения лучше использовать \mintinline{cpp}|glm::slerp|)
\end{footnotesize}
\end{itemize}
\item Обратите внимание, что \mintinline{cpp}|max_time| у этих анимаций разный!
\end{itemize}
\end{footnotesize}
\end{frame}

\begin{frame}[fragile]
\frametitle{Задание 5*}
\slideimage{5.png}
\end{frame}

\end{document}