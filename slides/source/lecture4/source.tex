% (c) Nikita Lisitsa, lisyarus@gmail.com, 2021

\documentclass{beamer}

\usepackage[T2A]{fontenc}
\usepackage[utf8]{inputenc}
\usepackage[russian]{babel}

\usepackage{graphicx}
\graphicspath{ {./images/} }

\usepackage{adjustbox}

\usepackage{color}
\usepackage{soul}

\usepackage{hyperref}

\usepackage{tikz}
\usetikzlibrary{decorations}
\usetikzlibrary{decorations.pathreplacing}
\usepackage{xifthen}

\definecolor{red}{rgb}{1,0,0}
\definecolor{green}{rgb}{0,0.5,0}
\definecolor{blue}{rgb}{0,0,1}
\definecolor{magenta}{rgb}{0.75,0,0.75}

\makeatletter
\newcommand{\slideimage}[1]{
  \begin{figure}
    \begin{adjustbox}{width=\textwidth, totalheight=\textheight-2\baselineskip-2\baselineskip,keepaspectratio}
      \includegraphics{#1}
    \end{adjustbox}
  \end{figure}
}
\makeatother

\title{Компьютерная графика}
\subtitle{Лекция 4: Камера, проекции, буфер глубины}
\date{2021}

\setbeamertemplate{footline}[frame number]

\begin{document}

\frame{\titlepage}

\begin{frame}[fragile]
\frametitle{Проекции}
\begin{itemize}
\item Мир - двухмерный/трёхмерный, произвольного размера, с произвольного ракурса
\pause
\item Экран - двухмерный, \begin{math}[-1, 1] \times [-1, 1]\end{math}
\pause
\begin{itemize}
\item Координата Z тоже \begin{math}[-1, 1]\end{math}
\item Про неё поговорим подробнее чуть позже
\end{itemize}
\pause
\item За преобразование отвечает проекция
\end{itemize}
\end{frame}

\begin{frame}[fragile]
\frametitle{Ортографическая проекция}
\begin{itemize}
\item Самый простой способ: \pause игнорировать третью координату
\begin{center}Концептуально: \begin{math}(x, y, z) \mapsto (x, y)\end{math}\end{center}
\begin{center}
\begin{math}
\begin{pmatrix}
1 & 0 & 0 & 0 \\
0 & 1 & 0 & 0 \\
0 & 0 & 1 & 0 \\
0 & 0 & 0 & 1
\end{pmatrix}
\end{math}
\end{center}
\pause
\item Именно это делает OpenGL (если \verb|gl_Position.w = 1|)
\pause
\item X и Y всё ещё \begin{math}[-1, 1]\end{math}
\pause
\item Как сделать \begin{math}X \in [-W, W]\end{math} и \begin{math}Y \in [-H, H]\end{math}?
\pause
\begin{center}
\begin{math}
\begin{pmatrix}
\frac{1}{W} & 0 & 0 & 0 \\
0 & \frac{1}{H} & 0 & 0 \\
0 & 0 & 1 & 0 \\
0 & 0 & 0 & 1
\end{pmatrix}
\end{math}
\end{center}
\end{itemize}
\end{frame}

\begin{frame}[fragile]
\frametitle{Ортографическая проекция}
\begin{itemize}
\item Как сделать \begin{math}X \in [X_0 - W, X_0 + W]\end{math} и \begin{math}Y \in [Y_0 - H, Y_0 + H]\end{math}?
\pause
\begin{itemize}
\item Сдвинуть на \begin{math}(-X_0, -Y_0)\end{math}, а затем применить масштабирование:
\end{itemize}
\begin{center}
\begin{math}
\begin{pmatrix}
\frac{1}{W} & 0 & 0 & 0 \\
0 & \frac{1}{H} & 0 & 0 \\
0 & 0 & 1 & 0 \\
0 & 0 & 0 & 1
\end{pmatrix}
\cdot
\begin{pmatrix}
1 & 0 & 0 & -X_0 \\
0 & 1 & 0 & -Y_0 \\
0 & 0 & 1 & 0 \\
0 & 0 & 0 & 1
\end{pmatrix}
\end{math}
\end{center}
\pause
\item Если размер области \begin{math}W\end{math}, нужно разделить на \begin{math}W\end{math}
\item Если центр в \begin{math}X_0\end{math}, нужно сдвинуть на \begin{math}-X_0\end{math}
\pause
\item Общая идея: если камера получена каким-то преобразованием, нужно применить обратное преобразование
\end{itemize}
\end{frame}

\begin{frame}[fragile]
\frametitle{Ортографическая проекция: обратное преобразование}
\begin{itemize}
\item Можно считать, что по умолчанию OpenGL делает ортографическую проекцию на квадрат \begin{math}[-1, 1]^2\end{math} в плоскости XY параллельно оси Z
\item Камера находится в начале координат
\pause
\item Если ко всему виртуальному миру (включая камеру!) применить аффинное преобразование, изображение не изменится
\pause
\item \begin{math}\Rightarrow\end{math} Если наша камера получена аффинным преобразованием из стандартной камеры OpenGL, к объектам мира нужно применить обратное преобразование
\end{itemize}
\end{frame}

\begin{frame}[fragile]
\frametitle{Ортографическая проекция}
\slideimage{orthographic.png}
\end{frame}

\begin{frame}[fragile]
\frametitle{Ортографическая проекция: общий случай}
\begin{itemize}
\item В общем случае ортографическую камеру можно задать
\pause
\begin{itemize}
\item Положением камеры \begin{math}C = (C_x, C_y, C_z)\end{math}
\pause
\item Осями координат камеры \begin{math}X = (X_x, X_y, X_z), Y = (Y_x, Y_y, Y_z), Z = (Z_x, Z_y, Z_z)\end{math}
\end{itemize}
\pause
\item Делает проекцию параллельно вектору \begin{math}Z\end{math} на параллелограмм \begin{math}C \pm X \pm Y\end{math}
\item Параллелограмм отождествляется с экраном
\pause
\item Обычно \begin{math}X, Y, Z\end{math} взаимно ортогональны
\pause
\item Как выразить эту проекцию матрицей?
\pause
\item Преобразование из стандартной системы координат OpenGL \begin{math}[-1, 1]^3\end{math} в координаты области, видимой через эту камеру: смена системы координат + параллельный перенос
\begin{center}
\begin{math}
\begin{pmatrix}
X_x & Y_x & Z_x & C_x \\
X_y & Y_y & Z_y & C_y \\
X_z & Y_z & Z_z & C_z \\
0 & 0 & 0 & 1
\end{pmatrix}
\end{math}
\end{center}
\pause
\item Обратное преобразование (проекция) - обратная матрица
\end{itemize}
\end{frame}

\begin{frame}[fragile]
\frametitle{Ортографическая проекция: общий случай}
\begin{itemize}
\item Можно представить \begin{math}X, Y, Z\end{math} как произведение длины и нормированного вектора:
\begin{center}
\begin{math}X = W \cdot \hat X \quad Y = H \cdot \hat Y \quad Z = D \cdot \hat Z\end{math}
\end{center}
\pause
\item Матрицу можно разбить на перенос, масштабирование и поворот
\begin{center}
\begin{math}
\begin{pmatrix}
X_x & Y_x & Z_x & C_x \\
X_y & Y_y & Z_y & C_y \\
X_z & Y_z & Z_z & C_z \\
0 & 0 & 0 & 1
\end{pmatrix}
=
\begin{pmatrix}
1 & 0 & 0 & C_x \\
0 & 1 & 0 & C_y \\
0 & 0 & 1 & C_z \\
0 & 0 & 0 & 1
\end{pmatrix}
\cdot
\begin{pmatrix}
\hat X_x & \hat Y_x & \hat Z_x & 0 \\
\hat X_y & \hat Y_y & \hat Z_y & 0 \\
\hat X_z & \hat Y_z & \hat Z_z & 0 \\
0 & 0 & 0 & 1
\end{pmatrix}
\cdot
\begin{pmatrix}
W & 0 & 0 & 0 \\
0 & H & 0 & 0 \\
0 & 0 & D & 0 \\
0 & 0 & 0 & 1
\end{pmatrix}
\end{math}
\end{center}
\end{itemize}
\end{frame}

\begin{frame}[fragile]
\frametitle{Ортографическая проекция: общий случай}
\begin{itemize}
\item Тогда обратная матрица (матрица проекции):
\begin{center}
\begin{math}
\begin{pmatrix}
X_x & Y_x & Z_x & C_x \\
X_y & Y_y & Z_y & C_y \\
X_z & Y_z & Z_z & C_z \\
0 & 0 & 0 & 1
\end{pmatrix}^{-1}
=
\begin{pmatrix}
\frac{1}{W} & 0 & 0 & 0 \\
0 & \frac{1}{H} & 0 & 0 \\
0 & 0 & \frac{1}{D} & 0 \\
0 & 0 & 0 & 1
\end{pmatrix}
\cdot
\begin{pmatrix}
\hat X_x & \hat Y_x & \hat Z_x & 0 \\
\hat X_y & \hat Y_y & \hat Z_y & 0 \\
\hat X_z & \hat Y_z & \hat Z_z & 0 \\
0 & 0 & 0 & 1
\end{pmatrix}^{-1}
\cdot
\begin{pmatrix}
1 & 0 & 0 & -C_x \\
0 & 1 & 0 & -C_y \\
0 & 0 & 1 & -C_z \\
0 & 0 & 0 & 1
\end{pmatrix}
\end{math}
\end{center}
\end{itemize}
\end{frame}

\begin{frame}[fragile]
\frametitle{Ортографическая проекция: ортогональный случай}
\begin{itemize}
\item Если \begin{math}X, Y, Z\end{math} ортогональны, то 
\begin{center}
\begin{math}
\begin{pmatrix}
\hat X_x & \hat Y_x & \hat Z_x & 0 \\
\hat X_y & \hat Y_y & \hat Z_y & 0 \\
\hat X_z & \hat Y_z & \hat Z_z & 0 \\
0 & 0 & 0 & 1
\end{pmatrix}^{-1}
=
\begin{pmatrix}
\hat X_x & \hat Y_x & \hat Z_x & 0 \\
\hat X_y & \hat Y_y & \hat Z_y & 0 \\
\hat X_z & \hat Y_z & \hat Z_z & 0 \\
0 & 0 & 0 & 1
\end{pmatrix}^T
=
\begin{pmatrix}
\hat X_x & \hat X_y & \hat X_z & 0 \\
\hat Y_x & \hat Y_y & \hat Y_z & 0 \\
\hat Z_x & \hat Z_y & \hat Z_z & 0 \\
0 & 0 & 0 & 1
\end{pmatrix}
\end{math}
\end{center}
\end{itemize}
\end{frame}

\begin{frame}[fragile]
\frametitle{Ортографическая проекция: применение}
\begin{itemize}
\pause
\item 2D рендеринг: 2D игры, UI, карты
\pause
\item Проектировние моделей, зданий, деталей (вид сверху, вид сбоку, вид спереди)
\pause
\item Стилизация: 3D мир с ортографической проекцией
\end{itemize}
\end{frame}

\begin{frame}[fragile]
\frametitle{Ортографическая проекция: проблемы}
\begin{itemize}
\pause
\item Объекты на разном расстоянии от камеры выглядят одинаково
\pause
\item Нельзя оценить расстояние до объекта по его изображению
\pause
\item Реальные камеры и глаза работают не так
\end{itemize}
\end{frame}

\begin{frame}[fragile]
\frametitle{Перспективная проекция}
\slideimage{perspective.png}
\end{frame}

\begin{frame}[fragile]
\frametitle{Перспективная проекция}
\begin{itemize}
\item Есть {\color{blue}центр проекции} и {\color{blue}плоскость проекции}
\item {\color{red}Проекция} {\color{green}точки} - пересечение {\color{magenta}прямой}, проходящей через эту {\color{green}точку} и {\color{blue}центр проекции}, с {\color{blue}плоскостью проекции}
\end{itemize}
\pause
\begin{center}
\begin{tikzpicture}
\draw[blue,thick] (2.0, -2.0) -- (2.0, 2.0);
\draw[magenta,thick,dashed] (0.0, 0.0) -- (6.0, 1.5);
\node at (0.0, 0.0) {\color{blue}\textbullet};
\node at (2.0, 0.5) {\color{red}\textbullet};
\node at (6.0, 1.5) {\color{green}\textbullet};
\end{tikzpicture}
\end{center}
\end{frame}

\begin{frame}[fragile]
\frametitle{Перспективная проекция}
\begin{center}
\begin{tikzpicture}
\draw[blue,thick] (2.0, -2.0) -- (2.0, 2.0);
\draw[magenta,thick,dashed] (0.0, 0.0) -- (6.0, 1.5);
\draw[black,dashed,-stealth] (-2.0, 0.0) -- (8.0, 0.0);
\draw[black,dashed,-stealth] (0.0, -3.0) -- (0.0, 3.0);
\draw[black,dashed] (6.0, 1.5) -- (6.0, -1.0);
\node at (0.0, 0.0) {\color{blue}\textbullet};
\node at (2.0, 0.5) {\color{red}\textbullet};
\node at (6.0, 1.5) {\color{green}\textbullet};

\node at (8.0, -0.25) {Z};
\node at (-0.25, 3.0) {Y};

\draw[thick,decorate,decoration={brace,amplitude=10pt,mirror}] (0.125, -0.125) -- (1.875, -0.125);
\draw[thick,decorate,decoration={brace,amplitude=10pt,mirror}] (0.125, -1.125) -- (5.875, -1.125);

\node at (1.0, -0.75) {\begin{math}z_P\end{math}};
\node at (3.0, -1.75) {\begin{math}z\end{math}};

\draw[thick,decorate,decoration={brace,amplitude=5pt,mirror}] (6.125, 0.125) -- (6.125, 1.375);
\draw[thick,decorate,decoration={brace,amplitude=2pt,mirror}] (2.125, 0.125) -- (2.125, 0.375);

\node at (2.5, 0.25) {\begin{math}y_P\end{math}};
\node at (6.5, 0.75) {\begin{math}y\end{math}};

\node at (3.0, 3.0) {\begin{math}y_P = y\frac{z_P}{z} = z_P\frac{y}{z}\end{math}};
\end{tikzpicture}
\end{center}
\end{frame}

\begin{frame}[fragile]
\frametitle{Перспективная проекция}
\begin{itemize}
\item Чтобы вычислить перспективную проекцию с центром в начале координат и плоскостью проеции \begin{math}Z=z_P\end{math}, нужно разделить на Z-координату:
\begin{center}
\begin{math}
(x,y,z) \mapsto \left(z_P\frac{x}{z}, z_P\frac{y}{z}\right)
\end{math}
\end{center}
\pause
\item Как выразить эту проекцию матрицей? \pause Никак: матрицы не умеют делить одни координаты на другие
\end{itemize}
\end{frame}

\begin{frame}[fragile]
\frametitle{Perspective divide}
\begin{itemize}
\item Чтобы поддержать перспективную проекцию, в графическом конвейере (после вершинного шейдера, перед переводом в экранные координаты) есть специальный обязательный шаг: perspective divide
\pause
\item \verb|gl_Position| делится на последнюю координату \verb|gl_Position.w|
\begin{center}
\begin{math}
(x, y, z, w) \mapsto \left(\frac{x}{w}, \frac{y}{w}, \frac{z}{w}\right)
\end{math}
\end{center}
\pause
\item Если \begin{math}w=1\end{math}, ничего не меняется (ортографическая проекция)
\pause
\item Если \begin{math}w=\end{math} расстояние до камеры, получается перспективная проекция
\end{itemize}
\end{frame}

\begin{frame}[fragile]
\frametitle{Перспективная проекция}
\begin{itemize}
\item Как выразить перспективную проекцию матрицей с последующим perspective divide?
\pause
\begin{center}
\begin{math}
\begin{pmatrix}
1 & 0 & 0 & 0 \\
0 & 1 & 0 & 0 \\
0 & 0 & 1 & 0 \\
0 & 0 & 1 & 0 \\
\end{pmatrix}
\cdot
\begin{pmatrix}
x \\ y \\ z \\ 1
\end{pmatrix}
=
\begin{pmatrix}
x \\ y \\ z \\ z
\end{pmatrix}
\mapsto
\begin{pmatrix}
x/z \\ y/z \\ 1
\end{pmatrix}
\end{math}
\end{center}
\pause
\item Обычно матрица перспективной проекции выглядит сложнее
\end{itemize}
\end{frame}

\end{document}