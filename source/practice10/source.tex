% (c) Nikita Lisitsa, lisyarus@gmail.com, 2025

\documentclass[10pt]{beamer}

\usepackage[T2A]{fontenc}
\usepackage[russian]{babel}
\usepackage{minted}

\usepackage{graphicx}
\graphicspath{ {./images/} }


\usetheme{metropolis}

\usepackage{adjustbox}

\usepackage{color}
\usepackage{soul}

\usepackage{hyperref}

\definecolor{blue}{rgb}{0,0,1}
\definecolor{red}{rgb}{1,0,0}
\definecolor{codebg}{RGB}{29,35,49}

\makeatletter
\newcommand{\slideimage}[1]{
  \begin{figure}
    \begin{adjustbox}{width=\textwidth, totalheight=\textheight-2\baselineskip-2\baselineskip,keepaspectratio}
      \includegraphics{#1}
    \end{adjustbox}
  \end{figure}
}
\makeatother

\title{Компьютерная графика}
\subtitle{Практика 10: Normal mapping, environment mapping}
\date{2025}

\setbeamertemplate{footline}[frame number]

\usemintedstyle{solarized-light}

\begin{document}

\frame{\titlepage}

\begin{frame}[fragile]
\slideimage{0.png}
\end{frame}

\begin{frame}[fragile]
\frametitle{Задание 1}
Готовимся к normal mapping
\begin{itemize}
\item Выводим нормаль к поверхности вместо цвета: \mintinline{glsl}|albedo = normal * 0.5 + vec3(0.5)|
\end{itemize}
\end{frame}

\begin{frame}[fragile]
\slideimage{1.png}
\end{frame}

\begin{frame}[fragile]
\frametitle{Задание 2}
Читаем normal map
\begin{itemize}
\item Загружаем текстуру \mintinline{glsl}|textures/brick_normal.jpg| (в коде есть функция \mintinline{glsl}|load_texture|)
\item Заводим во фрагментном шейдере uniform-переменную \mintinline{glsl}|sampler2D normal_texture|, перед рисованием устанавливаем её значение в 1 (аналогично \mintinline{glsl}|albedo_texture|)
\item Перед рисованием делаем эту текстуру текущей для первого texture unit'а (\mintinline{glsl}|GL_TEXTURE1|)
\item Во фрагментном шейдере читаем эту текстуру в качестве значения альбедо \mintinline{glsl}|albedo = texture(normal_texture, texcoord).rgb|
\end{itemize}
\end{frame}

\begin{frame}[fragile]
\slideimage{2.png}
\end{frame}

\begin{frame}[fragile]
\frametitle{Задание 3}
Вычисляем настоящие нормали
\begin{itemize}
\item Во фрагментном шейдере вычисляем bitangent вектор: \mintinline{glsl}|vec3 bitangent = cross(normal, tangent)|
\item Вычисляем матрицу преобразования из normal map в мировые координаты: \mintinline{glsl}|mat3 tbn = mat3(tangent, bitangent, normal)|
\item Прочитанное из normal map значение переводим из [0, 1] в [-1, 1] и применяем матрицу: \mintinline{glsl}|vec3 real_normal = tbn * (texture(...).xyz * 2.0 -| \mintinline{glsl}|- vec3(1.0))|
\item В качестве цвета используем новую нормаль: \mintinline{glsl}|albedo = real_normal * 0.5 + vec3(0.5)|
\end{itemize}
\end{frame}

\begin{frame}[fragile]
\slideimage{3.png}
\end{frame}

\begin{frame}[fragile]
\frametitle{Задание 4}
Используем настоящие нормали для освещения
\begin{itemize}
\item Во фрагментном шейдере возвращаем старое \mintinline{glsl}|albedo| (бралось из \mintinline{glsl}|albedo_texture|)
\item В вычислении освещения используем \mintinline{glsl}|real_normal| вместо \mintinline{glsl}|normal|
\end{itemize}
\end{frame}

\begin{frame}[fragile]
\slideimage{4.png}
\end{frame}

\begin{frame}[fragile]
\frametitle{Задание 5}
\begin{footnotesize}
Добавляем reflection mapping
\begin{itemize}
\item Читаем текстуру \mintinline{glsl}|textures/environment_map.jpg|, добавляем для неё uniform-переменную, устанавливаем её в значение 2, текстуру делаем текущей для 2ого texture unit'а (\mintinline{glsl}|GL_TEXTURE2|)
\item Во фрагментном шейдере по положению камеры и положению пикселя (\mintinline{glsl}|position|) вычисляем направление на камеру
\item С помощью нормали (\mintinline{glsl}|real_normal|) вычисляем направление отражённого луча
\item Вычисляем текстурную координату для environment map:
\mintinline{glsl}|x = atan(dir.z, dir.x) / PI * 0.5 + 0.5;|
\mintinline{glsl}|y = -atan(dir.y, length(dir.xz)) / PI + 0.5;|
\usemintedstyle{solarized-light}
\item В качестве значения итогого цвета берём среднее между старым цветом (\mintinline{glsl}|albedo * lightness|) и значением из environment map
\end{itemize}
\end{footnotesize}
\end{frame}

\begin{frame}[fragile]
\frametitle{Задание 5}
Добавляем reflection mapping
\begin{itemize}
\item N.B.: если отражения слишком сильно искажаются картой нормалей, можно слегка подвинуть нормаль: \mintinline{glsl}|real_normal =| \mintinline{glsl}|= normalize(mix(normal, real_normal, 0.5))|
\end{itemize}
\end{frame}

\begin{frame}[fragile]
\slideimage{5.png}
\end{frame}

\begin{frame}[fragile]
\frametitle{Задание 6*}
\begin{footnotesize}
Добавляем environment mapping
\begin{itemize}
\item Добавляем новую шейдерную программу, в которой захардкожен полноэкранный прямоугольник (как в первой практике), и фиктивный VAO
\item В вершинном шейдере вычисляется соответствующая координата в пространстве и передаётся во фрагментный шейдер:
\mintinline{glsl}|vec4 ndc = vec4(vertex, 0.0, 1.0);|
\mintinline{glsl}|vec4 clip_space = view_projection_inverse * ndc;|
\mintinline{glsl}|position = clip_space.xyz / clip_space.w;|
\item \mintinline{glsl}|view_projection_inverse| -- матрица \mintinline{glsl}|inverse(projection * view)|
\item Во фрагментном шейдере вычисляется направление из камеры в этот пиксель
\item Это направление используется для чтения из environment map (так же, как в предыдущем задании)
\item N.B. нужно подумать, какие тут нужны uniform-переменные и текстуры, и правильно передать их
\end{itemize}
\end{footnotesize}
\end{frame}

\begin{frame}[fragile]
\slideimage{6.png}
\end{frame}

\end{document}
