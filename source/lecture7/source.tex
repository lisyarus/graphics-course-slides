% (c) Nikita Lisitsa, lisyarus@gmail.com, 2023

\documentclass{beamer}

\usepackage[T2A]{fontenc}
\usepackage[russian]{babel}
\usepackage{minted}

\usepackage{graphicx}
\graphicspath{ {./images/} }

\usepackage{adjustbox}

\usepackage{color}
\usepackage{soul}

\usepackage{hyperref}

\usepackage{amsmath}

\usepackage{tikz}
\usetikzlibrary{decorations}
\usetikzlibrary{decorations.pathreplacing}
\usepackage{xifthen}

\usetheme{metropolis}
\setminted{fontsize=\footnotesize}

\definecolor{red}{rgb}{1,0,0}
\definecolor{green}{rgb}{0,0.5,0}
\definecolor{blue}{rgb}{0,0,1}
\definecolor{magenta}{rgb}{0.75,0,0.75}

\definecolor{codebg}{RGB}{29,35,49}

\makeatletter
\newcommand{\slideimage}[1]{
  \begin{figure}
    \begin{adjustbox}{width=\textwidth, totalheight=\textheight-2\baselineskip-2\baselineskip,keepaspectratio}
      \includegraphics{#1}
    \end{adjustbox}
  \end{figure}
}
\makeatother

\title{Компьютерная графика}
\subtitle{Лекция 7: Распространение света, уравнение рендеринга, модели освещения}
\date{2023}

\setbeamertemplate{footline}[frame number]

\begin{document}

\frame{\titlepage}

\begin{frame}[fragile]
\frametitle{Почему важно освещение}
\begin{itemize}
\item \begin{math}\approx\end{math}90\% информации человек воспринимает через зрение
\pause
\item Зрение воспринимает то, как \textit{освещены} окружающие объекты
\pause
\item Человек с рождения учится воспринимать мир освещённым
\end{itemize}
\end{frame}

\begin{frame}<1>[fragile,label=why]
\frametitle{Почему важно освещение}
\begin{itemize}
\item Помогает понять форму объектов
\pause
\item Помогает понять соотношение между объектами в пространстве
\pause
\item Помогает понять материал объектов
\pause
\item Помогает создать атмосферу
\pause
\item Помогает сделать объекты более привычными человеку
\pause
\item \begin{math}\Longrightarrow\end{math} Какой бы графикой вы ни занимались, полезно добавить освещение
\end{itemize}
\end{frame}

\begin{frame}[fragile]
\frametitle{Почему важно освещение}
\slideimage{suzanne.png}
\end{frame}

\begin{frame}[fragile]
\frametitle{Почему важно освещение}
\slideimage{suzanne-lit.png}
\end{frame}

\againframe<1-2>{why}

\begin{frame}[fragile]
\frametitle{Почему важно освещение}
\slideimage{suzanne-box.png}
\end{frame}

\begin{frame}[fragile]
\frametitle{Почему важно освещение}
\slideimage{suzanne-box-lit.png}
\end{frame}

\againframe<2-3>{why}

\begin{frame}[fragile]
\frametitle{Почему важно освещение}
\slideimage{suzanne-material.png}
\end{frame}

\againframe<3-4>{why}

\begin{frame}[fragile]
\frametitle{Почему важно освещение}
\slideimage{zelda.png}
\end{frame}

\againframe<4-5>{why}

\begin{frame}[fragile]
\frametitle{Почему важно освещение}
\slideimage{win31.png}
\end{frame}

\begin{frame}[fragile]
\frametitle{Почему важно освещение}
\slideimage{zelda2.jpg}
\end{frame}

\againframe<5-6>{why}

\begin{frame}[fragile]
\frametitle{Что такое свет}
\begin{itemize}
\item Свет -- электромагнитная волна
\pause
\item Пара векторных полей (электрическое \begin{math}\mathbf{E}\end{math} и магнитное \begin{math}\mathbf{B}\end{math}), описываемых уравнениями Максвелла
\begin{center}
\begin{math}
\begin{matrix}
\nabla \cdot \mathbf{E} = 4 \pi \rho \\
\nabla \cdot \mathbf{B} = 0 \\
\nabla \times \mathbf{E} = -\frac{1}{c} \frac{\partial \mathbf B}{\partial t} \\
\nabla \times \mathbf{B} = \frac{1}{c} \left(4 \pi \mathbf J + \frac{\partial \mathbf E}{\partial t} \right)
\end{matrix}
\end{math}
\end{center}
\pause
\item Линейное волновое уравнение \begin{math}\Longrightarrow\end{math} раскладывается в сумму волн фиксированной длины волны \begin{math}\lambda\end{math} (монохроматические волны)
\pause
\item Много нетривиальных эффектов: интерференция, дифракция, поляризация, сложное взаимодействие с веществом
\end{itemize}
\end{frame}

\begin{frame}[fragile]
\frametitle{Свет}
\begin{itemize}
\item В графике обычно достаточно геометрической оптики -- предела при \begin{math}\lambda\rightarrow 0\end{math}
\pause
\item Свет распространяется прямыми лучами (исключение -- граница раздела сред или неоднородные среды)
\pause
\item Свет распространяется бесконечно быстро (\begin{math}c \rightarrow \infty\end{math})
\pause
\item Луч света разбивается в сумму монохроматических лучей, каждая имеет свою амплитуду (количество света)
\pause
\item Интересуют только волны \textit{видимого спектра}
\end{itemize}
\end{frame}

\begin{frame}[fragile]
\frametitle{Видимый спектр}
\slideimage{visible-spectrum.jpg}
\end{frame}

\begin{frame}[fragile]
\frametitle{Фиолетовый цвет}
\slideimage{blue-red-spectrum.png}
\end{frame}

\begin{frame}<1-3>[fragile,label=materials]
\frametitle{Столкновение света с поверхностью}
\begin{itemize}
\item Что происходит, когда луч света сталкивается с некой поверхностью?
\pause
\item Зависит от её \textit{материала}:
\pause
\begin{itemize}
\item Абсолютно чёрное тело: поглощает весь свет
\pause
\item Зеркало: отражает свет в строго определённом направлении
\pause
\item Старое, плохо отполированное зеркало / лакированная поверхность: отражает свет примерно в определённом направлении
\pause
\item Диффузная (ламбертова) поверхность: отражает свет во все стороны равномерно
\pause
\item Вода, стекло: частично отражает, частично преломляет (закон Френеля)
\pause
\item Воск: частично отражает, частично преломляет, но свет заходит неглубоко (subsurface scattering)
\end{itemize}
\end{itemize}
\end{frame}

\begin{frame}[fragile]
\frametitle{Абсолютно чёрное тело}
\slideimage{blackbody.png}
\end{frame}

\againframe<3-4>{materials}

\begin{frame}[fragile]
\frametitle{Зеркало}
\slideimage{specular.png}
\end{frame}

\againframe<4-5>{materials}

\begin{frame}[fragile]
\frametitle{Старое зеркало}
\slideimage{specular_rough.png}
\end{frame}

\againframe<5-6>{materials}

\begin{frame}[fragile]
\frametitle{Диффузная поверхность}
\slideimage{diffuse.png}
\end{frame}

\againframe<6-7>{materials}

\begin{frame}[fragile]
\frametitle{Стекло}
\slideimage{glass.png}
\end{frame}

\againframe<7-8>{materials}

\begin{frame}[fragile]
\frametitle{Воск}
\slideimage{wax.png}
\end{frame}

\begin{frame}[fragile]
\frametitle{Столкновение света с поверхностью}
\begin{itemize}
\item Что происходит, когда луч света сталкивается с некой поверхностью?
\pause
\item \begin{math}\Longrightarrow\end{math} Свет, выходящий в каком-то направлении, складывается из света, пришедшего из \textit{всех направлений}
\end{itemize}
\end{frame}

\begin{frame}[fragile]
\frametitle{Абсолютно чёрное тело}
\begin{center}
\begin{tikzpicture}
\draw[black,thick] (0.0, 0.0) -- (10.0, 0.0);
\draw[blue,thick,-stealth] (8.0, 3.0) -- (5.5, 0.5);

\node[] at (5.0, 0.0) {\begin{math}\bullet\end{math}};
\node[] at (5.0, -0.5) {\begin{math}p\end{math}};
\node[blue] at (8.5, 3.5) {\begin{math}\vec\omega_{in}\end{math}};
\end{tikzpicture}
\end{center}
\end{frame}

\begin{frame}[fragile]
\frametitle{Зеркало}
\begin{center}
\begin{tikzpicture}
\draw[black,thick] (0.0, 0.0) -- (10.0, 0.0);
\draw[blue,thick,-stealth] (8.0, 3.0) -- (5.5, 0.5);
\draw[magenta,thick,-stealth] (4.5, 0.5) -- (2.0, 3.0);

\node[] at (5.0, 0.0) {\begin{math}\bullet\end{math}};
\node[] at (5.0, -0.5) {\begin{math}p\end{math}};
\node[blue] at (8.5, 3.5) {\begin{math}\vec\omega_{in}\end{math}};
\node[magenta] at (1.5, 3.5) {\begin{math}\vec\omega_{out}\end{math}};
\end{tikzpicture}
\end{center}
\end{frame}

\begin{frame}[fragile]
\frametitle{Старое зеркало / лакированная поверхность}
\begin{center}
\begin{tikzpicture}
\draw[black,thick] (0.0, 0.0) -- (10.0, 0.0);
\draw[blue,thick,-stealth] (8.0, 3.0) -- (5.5, 0.5);
\draw[magenta,thick,-stealth] (4.5, 0.5) -- (2.0, 3.0);
\draw[magenta,thick,-stealth] (4.64, 0.6) -- (3.5, 2.5);
\draw[magenta,thick,-stealth] (4.78, 0.67) -- (4.5, 1.5);
\draw[magenta,thick,-stealth] (4.4, 0.36) -- (2.5, 1.5);
\draw[magenta,thick,-stealth] (4.33, 0.22) -- (3.5, 0.5);

\node[] at (5.0, 0.0) {\begin{math}\bullet\end{math}};
\node[] at (5.0, -0.5) {\begin{math}p\end{math}};
\node[blue] at (8.5, 3.5) {\begin{math}\vec\omega_{in}\end{math}};
\node[magenta] at (1.5, 3.5) {\begin{math}\vec\omega_{out}\end{math}};
\end{tikzpicture}
\end{center}
\end{frame}

\begin{frame}[fragile]
\frametitle{Диффузная поверхность}
\begin{center}
\begin{tikzpicture}
\draw[black,thick] (0.0, 0.0) -- (10.0, 0.0);
\draw[blue,thick,-stealth] (8.0, 3.0) -- (5.5, 0.5);
\draw[magenta,thick,-stealth] (4.387627564304205, 0.35355339059327373) -- (1.3257653858252327, 2.121320343559642);
\draw[magenta,thick,-stealth] (4.646446609406726, 0.6123724356957946) -- (2.878679656440357, 3.674234614174767);
\draw[magenta,thick,-stealth] (5.0, 0.7071067811865476) -- (5.0, 4.242640687119285);
\draw[magenta,thick,-stealth] (5.353553390593274, 0.6123724356957946) -- (7.121320343559642, 3.6742346141747673);
\draw[magenta,thick,-stealth] (5.612372435695795, 0.35355339059327373) -- (8.674234614174768, 2.121320343559642);

\node[] at (5.0, 0.0) {\begin{math}\bullet\end{math}};
\node[] at (5.0, -0.5) {\begin{math}p\end{math}};
\node[blue] at (8.5, 3.5) {\begin{math}\vec\omega_{in}\end{math}};
\node[magenta] at (1.5, 3.5) {\begin{math}\vec\omega_{out}\end{math}};
\end{tikzpicture}
\end{center}
\end{frame}

\begin{frame}[fragile]
\frametitle{Стекло}
\begin{center}
\begin{tikzpicture}
\draw[black,thick] (0.0, 0.0) -- (10.0, 0.0);
\draw[blue,thick,-stealth] (8.0, 3.0) -- (5.5, 0.5);
\draw[magenta,thick,-stealth] (4.646446609406726, -0.6123724356957944) -- (2.878679656440356, -3.6742346141747655);

\node[] at (5.0, 0.0) {\begin{math}\bullet\end{math}};
\node[] at (5.0, -0.5) {\begin{math}p\end{math}};
\node[blue] at (8.5, 3.5) {\begin{math}\vec\omega_{in}\end{math}};
\node[magenta] at (2.5251262658470814, -4.28660704987056) {\begin{math}\vec\omega_{out}\end{math}};
\end{tikzpicture}
\end{center}
\end{frame}

\begin{frame}[fragile]
\frametitle{Уравнение рендеринга (Kajiya, 1986)}
\begin{equation*}
\hspace*{-0.5cm}
{\color{magenta}I_{out}}(p, {\color{magenta}\vec\omega_{out}}, \lambda) = I_{e}(p, {\color{magenta}\vec\omega_{out}}, \lambda)+\int\limits_{\mathbb S^2} {\color{blue}I_{in}}(p, {\color{blue}\vec\omega_{in}}, \lambda) \cdot {\color{green}f}(p, {\color{blue}\vec\omega_{in}}, {\color{magenta}\vec\omega_{out}}, \lambda) \cdot ({\color{blue}\vec\omega_{in}} \cdot \vec{n}) d{\color{blue}\vec\omega_{in}}
\end{equation*}
\hspace*{-0.5cm}
\pause
\begin{itemize}
\item \begin{math}{\color{magenta}I_{out}}\end{math} -- количество света, выходящего из точки \begin{math}p\end{math} в направлении \begin{math}{\color{magenta}\vec\omega_{out}}\end{math} с длиной волны \begin{math}\lambda\end{math}
\pause
\item \begin{math}I_{e}\end{math} -- количество света, излучаемого поверхностью из точки \begin{math}p\end{math} в направлении \begin{math}{\color{magenta}\vec\omega_{out}}\end{math} с длиной волны \begin{math}\lambda\end{math}
\pause
\item \begin{math}{\color{blue}I_{in}}\end{math} -- количество света, приходящего в точку \begin{math}p\end{math} из направления \begin{math}{\color{blue}\vec\omega_{in}}\end{math} с длиной волны \begin{math}\lambda\end{math}
\pause
\item \begin{math}{\color{green}f}\end{math} -- функция, определяющая, сколько света с длиной волны \begin{math}\lambda\end{math}, пришедшего из направления \begin{math}{\color{blue}\vec\omega_{in}}\end{math}, отразится в направлении \begin{math}{\color{magenta}\vec\omega_{out}}\end{math}
\pause
\item \begin{math}\vec n\end{math} -- нормаль (перпендикуляр) к поверхности в точке \begin{math}p\end{math}
\end{itemize}
\end{frame}

\begin{frame}[fragile]
\frametitle{Косинус угла падения}
\begin{itemize}
\item Откуда взялся \begin{math}{\color{blue}\vec\omega_{in}} \cdot \vec{n}\end{math}?
\pause
\item На поверхность площадью \begin{math}A\end{math} падает световой поток с площадью поперечного сечения \begin{math}{\color{blue} \Phi}\end{math} под углом \begin{math}\color{red}\theta\end{math}
\item Тогда \begin{math}{\color{blue} \Phi} = A \cdot {\color{red} \cos \theta}\end{math}
\item \begin{math}\Rightarrow\end{math} На единицу площади приходится \begin{math}{\color{red}\cos \theta}\end{math} от общей плотности потока
\end{itemize}
\begin{center}
\begin{tikzpicture}
\draw[black,thick] (0.0, 0.0) -- (10.0, 0.0);
\draw[black,thick] (4.5, 0.0) -- (8.5, 4.0);
\draw[black,thick] (5.5, 0.0) -- (9.0, 3.5);
\draw[black,thick] (4.5, -0.25) -- (4.5, 0.25);
\draw[black,thick] (5.5, -0.25) -- (5.5, 0.25);
\draw[blue,thick] (7.5, 3.0) -- (8.0, 2.5);
\draw[red,thick] (5.5, 0.25) arc (90:45:0.25);

\node[black] at (5.0, -0.5) {A};
\node[blue] at (8.0, 3.0) {\begin{math}\Phi\end{math}};
\node[red] at (5.75, 0.5) {\begin{math}\theta\end{math}};

\only<3->{
  \draw[black,thick,-stealth] (5.0, 3.0) -- (5.0, 3.5);
  \draw[blue,thick,-stealth] (5.0, 3.0) -- (5.35, 3.35);

  \node[black] at (5.0, 4.0) {\begin{math}\vec{n}\end{math}};
  \node[blue] at (5.7, 3.7) {\begin{math}\vec{\omega}_{in}\end{math}};

  \node[] at (2.0, 2.0) {\begin{math}{\color{red} \cos\theta} = \vec{n} \cdot {\color{blue} \vec{\omega}_{in}}\end{math}};
}
\end{tikzpicture}
\end{center}
\end{frame}

\begin{frame}[fragile]
\frametitle{Материал}
\begin{itemize}
\item \begin{math}{\color{green}f}\end{math} определяет \textit{материал} объекта
\pause 
\item Абсолютно чёрное тело: \begin{math}{\color{green}f}(p, {\color{blue}\vec\omega_{in}}, {\color{magenta}\vec\omega_{out}}, \lambda) = 0\end{math}
\pause
\item Идеальное зеркало: \begin{math}{\color{green}f}(p, {\color{blue}\vec\omega_{in}}, {\color{magenta}\vec\omega_{out}}, \lambda) = \delta({R_{\vec n}(\color{blue}\vec\omega_{in}}) - {\color{magenta}\vec\omega_{out}})\end{math}
\begin{itemize}
\item \begin{math}R\end{math} -- отраженный вектор: \begin{math}R_{\vec n}(\vec \omega) = 2\vec n \cdot (\vec n \cdot \vec \omega) - \vec \omega\end{math}
\end{itemize}
\pause
\item Старое зеркало: \begin{math}{\color{green}f}(p, {\color{blue}\vec\omega_{in}}, {\color{magenta}\vec\omega_{out}}, \lambda) = \left(R_{\vec n}({\color{blue}\vec\omega_{in}})\cdot{\color{magenta}\vec\omega_{out}}\right)^7\end{math}
\pause
\item Диффузная поверхность: \begin{math}{\color{green}f}(p, {\color{blue}\vec\omega_{in}}, {\color{magenta}\vec\omega_{out}}, \lambda) = 1\end{math}
\end{itemize}
\end{frame}

\begin{frame}[fragile]
\frametitle{Цвет}
\begin{itemize}
\item Зависимость \begin{math}{\color{green}f}\end{math} от \begin{math}\lambda\end{math} обеспечивает цвет объектов
\pause
\item Мы видим свет, отражённый объектом
\pause
\item Объект синего цвета не отражает красный цвет (кажется чёрным при освещении красным светом)
\end{itemize}
\end{frame}

\begin{frame}[fragile]
\frametitle{BRDF, BTDF, BSDF}
\begin{itemize}
\item Если \begin{math}{\color{green}f}\end{math} только отражает свет, её называют \textbf{BRDF}: Bidirectional Reflectance Distribution Function
\pause 
\item Если \begin{math}{\color{green}f}\end{math} только преломляет свет, её называют \textbf{BTDF}: Bidirectional Transmittance Distribution Function
\pause 
\item В общем случае её называют \textbf{BSDF}: Bidirectional Scattering Distribution Function
\end{itemize}
\end{frame}

\begin{frame}[fragile]
\frametitle{BSDF}
\begin{itemize}
\item BSDF неотрицательна: \begin{math}{\color{green}f}(p, {\color{blue}\vec\omega_{in}}, {\color{magenta}\vec\omega_{out}}, \lambda)\geq 0\end{math}
\pause
\item Обычно BSDF предполагается нормированной: тело не может отразить больше света, чем пришло
\begin{equation*}
\int\limits_{\mathbb S^2} {\color{green}f}(p, {\color{blue}\vec\omega_{in}}, {\color{magenta}\vec\omega_{out}}, \lambda)({\color{blue}\vec\omega_{in}} \cdot \vec{n}) d{\color{blue}\vec\omega_{in}} \leq 1
\end{equation*}
\pause
\item Helmholtz reciprocity:
\begin{equation*}
{\color{green}f}(p, {\color{blue}\vec\omega_{in}}, {\color{magenta}\vec\omega_{out}}, \lambda) = {\color{green}f}(p, {\color{magenta}\vec\omega_{out}}, {\color{blue}\vec\omega_{in}}, \lambda)
\end{equation*}
\pause
\item Комбинация набора BSDF \begin{math}\{{\color{green}f_i}\}\end{math} -- тоже BSDF:
\begin{equation*}
{\color{green}\mu_i} \geq 0, \sum {\color{green}\mu_i} \leq 1 \Rightarrow \sum {\color{green}\mu_i f_i}
\end{equation*}
\end{itemize}
\end{frame}

\begin{frame}[fragile]
\frametitle{Уравнение рендеринга}
\begin{itemize}
\item Откуда взять \begin{math}{\color{blue}I_{in}}\end{math}? \pause Ответ: это чей-то \begin{math}{\color{magenta}I_{out}}\end{math} 
\end{itemize}
\begin{center}
\begin{tikzpicture}
\draw[black,thick] (0.0, 0.0) -- (10.0, 0.0);
\draw[black,thick] (0.0, 0.0) -- (0.0, 6.0);

\node[] at (0.0, 3.0) {\begin{math}{\color{magenta}\bullet}\end{math}};
\node[] at (6.0, 0.0) {\begin{math}{\color{blue}\bullet}\end{math}};

\node[] at (-0.5, 3.0) {\begin{math}{\color{magenta}I_{out}}\end{math}};
\node[] at (0.5, 3.5) {\begin{math}{\color{magenta}\vec\omega_{out}}\end{math}};
\node[] at (6.0, -0.5) {\begin{math}{\color{blue}I_{in}}\end{math}};
\node[] at (6.5, 0.5) {\begin{math}{\color{blue}\vec\omega_{in}}\end{math}};

\draw[magenta,thick,-stealth] (0.5, 2.75) -- (2.0, 2.0);
\draw[blue,thick,-stealth] (4.0, 1.0) -- (5.5, 0.25);
\end{tikzpicture}
\end{center}
\end{frame}

\begin{frame}[fragile]
\frametitle{Уравнение рендеринга}
\begin{itemize}
\item Интегральное уравнение для каждой точки каждой поверхности сцены
\pause
\item Включает зависимость от материала объектов в каждой точке
\pause
\item Геометрия сцены связывает уравнения для разных точек
\pause
\item Обычно вместо всех возможных значений \begin{math}\lambda\end{math} берут дискретный набор ({\color{red}красный}, {\color{green}зелёный}, {\color{blue}синий})
\pause
\item Задача решения этого уравнения называется \textit{Global Illumination} (GI)
\end{itemize}
\end{frame}

\begin{frame}[fragile]
\frametitle{Уравнение рендеринга}
\begin{itemize}
\item Как описывать геометрию?
\pause
\item Как описывать материал (BRDF)?
\pause
\item Как решать само уравнение?
\pause
\item Как это выглядит в коде?
\end{itemize}
\end{frame}

\begin{frame}[fragile]
\frametitle{Как описывать геометрию}
\begin{itemize}
\item Зависит от используемого метода решения уравнения
\pause
\item Самый частый вариант -- триангуляция поверхности
\pause
\item Параметрическая поверхность, сплайны, NURBS
\pause
\item Неявная поверхность \begin{math}f(p) = 0\end{math}, по \begin{math}\nabla f\end{math} можно получить нормаль к поверхности
\pause
\item Подвид неявных поверхностей -- \textit{signed distance function/field} (SDF) -- функция \begin{math}f(p)\end{math}, возвращающая расстояние до поверхности, удобна для некоторых алгоритмов (напр. raymarching)
\end{itemize}
\end{frame}

\begin{frame}[fragile]
\frametitle{NURBS}
\slideimage{nurbs.png}
\end{frame}

\begin{frame}[fragile]
\frametitle{Как решать уравнение рендеринга: алгоритм radiosity}
\begin{itemize}
\item Взять в качестве переменных среднюю освещённость в каждой вершине сцены, интерполировать освещённость в треугольниках, свести к системе линейных уравнений
\pause
\item Требует вычислить влияние каждого треугольника на каждый с учётом видимости
\pause
\item Обычно решается итеративными методами (Якоби, Гаусса-Зейделя)
\pause
\item Каждая итерация учитывает ещё одно возможное отражение света (bounce) перед попаданием в камеру
\pause
\item Плохо подходит для динамических и real-time сцен
\end{itemize}
\end{frame}

\begin{frame}[fragile]
\frametitle{Как решать уравнение рендеринга: raytracing + monte-carlo integration}
\begin{itemize}
\item Посылать лучи из камеры, симулируя обратное распространение света (\textit{raytracing})
\item В точке пересечения луча с объектом сцены аппроксимировать интеграл, пуская отражённые лучи в случайном направлении (\textit{monte-carlo integration})
\pause
\item Основной алгоритм, использующийся при offline рендеринге
\pause
\item Требует как-то ограничить количество отражений (bounces)
\pause
\item Требует построить некую структуру данных (BVH -- \textit{bounding volume hierarchy}) для ускорения поиска пересечений лучей со сценой (RTX делают именно это)
\end{itemize}
\end{frame}

\begin{frame}[fragile]
\frametitle{Как решать уравнение рендеринга: raytracing + monte-carlo integration}
\begin{itemize}
\item Картинка получается очень шумной (зернистой) из-за случайности, очень много исследований направлено на борьбу с шумом (\textit{importance sampling, resampling, reservoir sampling})
\pause
\item Хочется переиспользовать результаты вычислений между кадрами \begin{math}\Longrightarrow\end{math} нужно эти результаты сохранять в
\pause
\begin{itemize}
\item Пикселях
\pause
\item Точках поверхностей (\textit{surflets})
\pause
\item Точках пространства (\textit{voxels})
\end{itemize}
\pause
\item Использование для real-time -- предмет активного исследования (см. \textit{ReSTIR GI})
\end{itemize}
\end{frame}

\begin{frame}[fragile]
\frametitle{ReSTIR GI}
\slideimage{restir.png}
\end{frame}

\begin{frame}[fragile]
\frametitle{Как решать уравнение рендеринга: real-time подход}
\begin{itemize}
\item Набор эвристических аппрокцимаций
\pause
\item Прямое освещение: \begin{math}\color{blue}\vec\omega_{in}\end{math} -- направление на источник света, \begin{math}\color{magenta}\vec\omega_{out}\end{math} -- направление на камеру
\pause
\item Тени: учитывается, `виден' ли объект источником света, или находится в тени
\pause
\item Отражения (точные / размытые)
\pause
\item Непрямое освещение (многократно отражённый свет) почти всегда игнорируется или эмулируется
\pause
\begin{itemize}
\item Ambient освещение -- свет, приходящий `отовсюду' (с неба, отражённый от стен, и т.п.)
\pause
\item Ambient occlusion (AO) -- затенение частей объекта, куда доходит меньше ambient света
\end{itemize}
\end{itemize}
\end{frame}

\begin{frame}[fragile]
\frametitle{Только источник света}
\slideimage{two_balls_only_emissive.png}
\end{frame}

\begin{frame}[fragile]
\frametitle{Только прямое освещение}
\slideimage{two_balls_no_shadow.png}
\end{frame}

\begin{frame}[fragile]
\frametitle{Прямое освещение и тени}
\slideimage{two_balls_no_diffuse.png}
\end{frame}

\begin{frame}[fragile]
\frametitle{Прямое и непрямое освещение, тени}
\slideimage{two_balls_full.png}
\end{frame}

\begin{frame}[fragile]
\frametitle{Прозрачность и отражение}
\slideimage{two_balls_ref.png}
\end{frame}

\begin{frame}[fragile]
\frametitle{Откуда взять материал (BRDF)}
\begin{itemize}
\item Эвристические формулы (Phong)
\pause
\item Вывод из предположений о структуре поверхности (\textit{microfacets})
\pause
\item Прямое измерение (MERL 100)
\pause
\item Симуляция
\end{itemize}
\end{frame}

\begin{frame}[fragile]
\frametitle{Как описать материал (BRDF)}
\begin{itemize}
\item Явная формула (точная, или аппроксимация) с набором параметров материала
\pause
\item Набор предподсчитанных значений, как-то записанных в текстуру
\end{itemize}
\end{frame}

\begin{frame}[fragile]
\frametitle{Расчёт прямого освещения}
\begin{itemize}
\item Почти всегда происходит во фрагментном шейдере
\pause
\item Вычисления используют \verb|vec3| для работы с цветом в {\color{red}R}{\color{green}G}{\color{blue}B}
\pause
\item Входные данные:
\pause
\begin{itemize}
\item Нормаль к поверхности
\pause
\item Параметры материала (цвет, гладкость, и т.п.)
\pause
\item Источники света
\end{itemize}
\pause
\item Полная освещённость = ambient освещение + вклад от каждого источника света
\end{itemize}
\end{frame}

\begin{frame}[fragile]
\frametitle{Ambient освещение}
\begin{itemize}
\item Свет, приходящий `отовсюду'
\pause
\begin{itemize}
\item На улице: небо
\pause
\item В помещении: диффузное отражение от стен, пола, потолка, других объектов
\end{itemize}
\pause
\item Задаётся своим цветом
\pause
\item Обычно добавляется чтобы точки, куда не попал другой свет, не выглядели совсем чёрными
\end{itemize}
\end{frame}

\begin{frame}[fragile]
\frametitle{Источники света}
\begin{itemize}
\item Два самых часто используемых типа: направленные и точечные
\pause
\item Направленные:
\pause
\begin{itemize}
\item Задаются направлением и яркостью
\pause
\item Моделируют удалённые источники -- солнце, луна
\end{itemize}
\pause
\item Точечные:
\pause
\begin{itemize}
\item Задаются расположением и функцией зависимости яркости от расстояния
\pause
\item Моделируют (относительно) близкие источники -- костёр, свеча, лампа
\end{itemize}
\end{itemize}
\end{frame}

\begin{frame}[fragile]
\frametitle{Точечный источник света}
\begin{itemize}
\item Яркость убывает с расстоянием
\pause
\item Часто берут такую функцию затухания (\textit{attenuation}):
\end{itemize}
\begin{equation*}
\frac{1}{C_0 + C_1 r + C_2 r^2}
\end{equation*}
\end{frame}

\begin{frame}[fragile]
\frametitle{Диффузная (ламбертова) BRDF}
\begin{equation*}
{\color{green}f}(p, {\color{blue}\vec\omega_{in}}, {\color{magenta}\vec\omega_{out}}, \lambda) = \frac{C(\lambda)}{\pi}
\end{equation*}
\begin{itemize}
\item Отражает свет во всех направлениях одинаково
\item \begin{math}C(\lambda) \in [0, 1]\end{math} -- `альбедо', определяет цвет
\item \begin{math}\pi\end{math} -- нормировочный множитель, часто входит в \begin{math}C(\lambda)\end{math} (напр. в текстуру)
\end{itemize}
\end{frame}

\begin{frame}[fragile]
\frametitle{Отражающая (зеркальная, specular) BRDF}
\begin{equation*}
{\color{green}f}(p, {\color{blue}\vec\omega_{in}}, {\color{magenta}\vec\omega_{out}}, \lambda) = C(\lambda) \left(R_{\vec n}({\color{blue}\vec\omega_{in}})\cdot{\color{magenta}\vec\omega_{out}}\right)^{power}
\end{equation*}
\begin{itemize}
\item Отражает больше света в направлении отражённого луча: \begin{math}R_{\vec n}(\vec \omega) = 2\vec n \cdot (\vec n \cdot \vec \omega) - \vec \omega\end{math}
\item Параметр \begin{math}power\end{math} определяет `размытость' отражения: чем он больше, тем оно менее размытое
\begin{itemize}
\item Часто вычисляется на основе более удобного параметра \verb|roughness|, например \begin{math}power = \frac{1}{roughness^2}-1\end{math}
\end{itemize}
\item Сложно правильно нормировать
\end{itemize}
\end{frame}

\begin{frame}[fragile]
\frametitle{Модель Фонга (Phong)}
\begin{itemize}
\item Вклад источника света = diffuse + specular
\pause
\item Полная освещённость = ambient + diffuse + specular
\end{itemize}
\end{frame}

\begin{frame}[fragile]
\frametitle{Модель Фонга (Phong)}
\slideimage{phong.png}
\end{frame}

\begin{frame}[fragile]
\frametitle{Модель Фонга + направленный источник света}
\usemintedstyle{lightbulb}
\begin{minted}[bgcolor=codebg,fontsize=\tiny]{glsl}
uniform vec3 ambient_light_color;
uniform vec3 light_direction;
uniform vec3 light_color;
uniform vec3 view_direction;

in vec3 normal;
in vec3 ambient_albedo;
in vec3 diffuse_albedo;
in vec3 specular_albedo;
in float specular_power;

layout (location = 0) out vec4 out_color;

void main() {
  float cosine = dot(normal, light_direction);
  float light_factor = max(0.0, cosine);
  vec3 reflected_direction = 2.0 * normal * cosine - light_direction;

  vec3 ambient = ambient_light_color * ambient_albedo;
  vec3 diffuse = diffuse_albedo * light_color * light_factor;
  vec3 specular = specular_albedo * pow(max(0.0, dot(reflected_direction,
    view_direction)), specular_power);

  vec3 result_color = ambient + diffuse + specular;
  out_color = vec4(result_color, 1.0);
}
\end{minted}
\end{frame}

\begin{frame}[fragile]
\frametitle{Модель Гуро}
\begin{itemize}
\item То же самое, что и модель Фонга, но в вершинном шейдере
\item Результирующий цвет интерполируется между пикселями
\item Выглядит плохо, в современности почти не используется
\end{itemize}
\slideimage{gourand.png}
\end{frame}

\begin{frame}[fragile]
\frametitle{Модель Блинна-Фонга (Blinn-Phong, modified Phong)}
\begin{itemize}
\item Как модель Фонга, но по-другому вычисляется specular составляющая
\pause
\item \begin{math}h = \frac{{\color{blue}\vec\omega_{in}} + {\color{magenta}\vec\omega_{out}}}{\|{\color{blue}\vec\omega_{in}} + {\color{magenta}\vec\omega_{out}}\|}\end{math} -- т.н. \textit{halfway vector} (часто встречается в моделях освещения)
\begin{equation*}
{\color{green}f_{specular}}(p, {\color{blue}\vec\omega_{in}}, {\color{magenta}\vec\omega_{out}}, \lambda) = C(\lambda) \left(h \cdot n\right)^{power}
\end{equation*}
\end{itemize}
\end{frame}

\begin{frame}[fragile]
\frametitle{Phong vs Blinn-Phong}
\slideimage{blinn-phong.png}
\end{frame}

\begin{frame}[fragile]
\frametitle{Microfacets}
\begin{itemize}
\item Чтобы честно вывести формулу для BRDF \textit{(physically-based BRDF)}, нужно предположение о микроструктуре поверхности (\textit{microsurface}) \pause \begin{math}\Rightarrow\end{math} микрофасеты (\textit{microfacets})
\pause
\item Предполагается, что поверхность состоит из маленьких площадок -- микрофасетов
\pause
\item Задаётся распределение ориентаций (нормалей) этих площадок
\pause
\item Задаётся материал площадок
\pause
\item Выводится формула для BRDF (обычно -- приближённая)
\end{itemize}
\end{frame}

\begin{frame}[fragile]
\frametitle{Microfacets}
\slideimage{microfacet.png}
\end{frame}

\begin{frame}[fragile]
\frametitle{Microfacets}
\begin{itemize}
\item Oren-Nayar BRDF -- микрофасетная модель, в которой фасеты имеют диффузную BRDF
\item Torrance-Sparrow BRDF -- микрофасетная модель, в которой фасеты -- идеальные зеркала
\end{itemize}
\end{frame}

\begin{frame}[fragile]
\frametitle{Модель Кука-Торренса (Cook-Torrance)}
\begin{equation*}
{\color{green}f}(p, {\color{blue}\vec\omega_{in}}, {\color{magenta}\vec\omega_{out}}, \lambda) = ({\color{blue}\vec\omega_{in}} \cdot n)
\left[ (1-s) C_d(\lambda) + s C_s(\lambda) R\right]
\end{equation*}
\begin{equation*}
R = \frac{D \cdot G \cdot F}{4 \cdot ({\color{blue}\vec\omega_{in}} \cdot n) \cdot ({\color{magenta}\vec\omega_{out}} \cdot n)}
\end{equation*}
\pause
\begin{itemize}
\item \begin{math}C_d(\lambda), C_s(\lambda)\end{math} -- диффузное и отражающее альбедо (цвет)
\item \begin{math}s\end{math} -- насколько сильно поверхность отражает свет (интерполизует между диффузной и specular BRDF)
\item \begin{math}D\end{math} -- плотность распределения (distribution) нормалей микрофасетов
\item \begin{math}G\end{math} -- коэффициент затенения геометрии (geometry) самой себя
\item \begin{math}F\end{math} -- коэффициент Френеля (Fresnel)
\item Конкретный вид функций \begin{math}D, G, F\end{math} может варьироваться (см. ссылки)
\end{itemize}
\end{frame}

\begin{frame}[fragile]
\frametitle{Disney principled BRDF (2012)}
\begin{itemize}
\item Основывается на микрофасетных моделях
\item Не влезет на слайд :)
\item Даёт огромное количество параметров для тонкой настройки, и потому удобна для художников
\item Её вариации используют почти все крупные движки и программы (Unity, Unreal, Blender, ...)
\end{itemize}
\end{frame}

\begin{frame}[fragile]
\frametitle{Blender principled BRDF}
\slideimage{principled.png}
\end{frame}

\begin{frame}[fragile]
\frametitle{Ссылки}
\fontsize{8pt}{8pt}
Туториалы про освещение:
\begin{itemize}
\item \href{https://learnopengl.com/Lighting/Basic-Lighting}{\texttt{learnopengl.com/Lighting/Basic-Lighting}}
\item \href{http://www.opengl-tutorial.org/beginners-tutorials/tutorial-8-basic-shading}{\texttt{opengl-tutorial.org/beginners-tutorials/tutorial-8-basic-shading}}
\item \href{https://www.lighthouse3d.com/tutorials/glsl-tutorial/point-lights}{\texttt{lighthouse3d.com/tutorials/glsl-tutorial/point-lights}}
\end{itemize}
Сложные BRDF:
\begin{itemize}
\item \href{http://www.cemyuksel.com/research/papers/CGI2020_ConstantTimeEnergyNormalizationForThePhongSpecularBRDFs.pdf}{\texttt{Нормировка Phong BRDF}}
\item \href{https://graphicscompendium.com/references/cook-torrance}{\texttt{Описание Cook-Torrance BRDF и примеры функций D, G, F}}
\item \href{https://github.com/TwoTailsGames/Unity-Built-in-Shaders/blob/master/CGIncludes/UnityStandardBRDF.cginc}{\texttt{Стандартная BRDF в Unity}}
\item \href{https://blog.selfshadow.com/publications/s2012-shading-course/burley/s2012_pbs_disney_brdf_notes_v3.pdf}{\texttt{Описание Disney BRDF}}
\end{itemize}
Raytracing:
\begin{itemize}
\item \href{https://raytracing.github.io/books/RayTracingInOneWeekend.html}{\texttt{Книжка Raytracing in one weekend}}
\item \href{https://www.youtube.com/watch?v=gsZiJeaMO48}{\texttt{Видео про ReSTIR GI}}
\end{itemize}
\end{frame}

\end{document}