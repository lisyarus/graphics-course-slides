% (c) Nikita Lisitsa, lisyarus@gmail.com, 2021

\documentclass{beamer}

\usepackage[T2A]{fontenc}
\usepackage[utf8]{inputenc}
\usepackage[russian]{babel}

\usepackage{graphicx}
\graphicspath{ {./images/} }

\usepackage{adjustbox}

\usepackage{color}
\usepackage{soul}

\usepackage{hyperref}

\usepackage{amsmath}

\usepackage{tikz}
\usetikzlibrary{decorations}
\usetikzlibrary{decorations.pathreplacing}
\usepackage{xifthen}

\definecolor{red}{rgb}{1,0,0}
\definecolor{green}{rgb}{0,0.5,0}
\definecolor{blue}{rgb}{0,0,1}
\definecolor{magenta}{rgb}{0.75,0,0.75}

\makeatletter
\newcommand{\slideimage}[1]{
  \begin{figure}
    \begin{adjustbox}{width=\textwidth, totalheight=\textheight-2\baselineskip-2\baselineskip,keepaspectratio}
      \includegraphics{#1}
    \end{adjustbox}
  \end{figure}
}
\makeatother

\title{Компьютерная графика}
\subtitle{Лекция 10: Вычисление проекции для shadow mapping, PCF, ESM, VSM, PSM, CSM}
\date{2021}

\setbeamertemplate{footline}[frame number]

\begin{document}

\frame{\titlepage}

\begin{frame}[fragile]
\frametitle{Проекция для shadow map: точечный источник}
\begin{itemize}
\item 6 перспективных проекций для 6 граней cubemap-текстуры shadow map
\pause
\item Каждая проекция смотрит в направлении некоторой оси: \begin{math}\pm X, \pm Y, \pm Z\end{math}
\pause
\item Угол видимой области - \begin{math}90^\circ\end{math} по обеим осям (квадратная камера)
\pause
\item \verb|near| и \verb|far| - максимальное (соответственно, минимальное), при котором не обрезается ничего лишнего
\pause
\begin{itemize}
\item \verb|near| - минимальное расстояние до вершины объекта сцены
\item \verb|far| - максимальное расстояние до вершины объекта сцены
\pause
\item \verb|far| можно вычислить, зная bounding box'ы всех объектов сцены
\end{itemize}
\end{itemize}
\end{frame}

\begin{frame}[fragile]
\frametitle{Проекция для shadow map: направленный источник}
\begin{itemize}
\item Одна ортографическая проекция: задаётся центром \begin{math}C\end{math} и осями \begin{math}X, Y, Z\end{math}
\pause
\item \begin{math}Z\end{math} параллельный направлению на источник света
\pause
\item \begin{math}X, Y\end{math} перпендикулярны \begin{math}Z\end{math} (и друг другу)
\begin{itemize}
\item Можно выбрать любые, перпендикулярные \begin{math}Z\end{math}
\item Можно попробовать выбрать их так, чтобы максимизировать эффективное разрешение shadow map (т.е. чтобы одному пикселю shadow map соответствовала как можно меньшая область пространства) - сводится к задаче OBB (oriented bounding box)
\end{itemize}
\pause
\item Откуда взять \begin{math}C\end{math} и длины векторов \begin{math}X,Y,Z\end{math}?
\end{itemize}
\end{frame}

\begin{frame}[fragile]
\frametitle{Проекция для shadow map: направленный источник}
\begin{itemize}
\item Вычислим bounding box всей сцены
\pause
\item В качестве \begin{math}C\end{math} возьмём центр bounding box'а сцены
\pause
\item Мы знаем направление вектора \begin{math}X\end{math}, нужно вычислить его длину
\begin{itemize}
\item Пусть \begin{math}\hat X\end{math} - вектор направления (нормированный)
\item Мы хотим, чтобы для всех вершин \begin{math}И\end{math} сцены выполнялось \begin{math}|(V - C) \cdot \hat X| \leq |X|\end{math}
\pause
\item Пройдёмся по всем восьми вершинам bounding box'а сцены и вычислим \begin{math}\max\limits_V |(V - C) \cdot \hat X|\end{math} - это длина вектора \begin{math}X\end{math}
\end{itemize}
\pause
\item Аналогично для \begin{math}Y, Z\end{math}
\end{itemize}
\end{frame}

\begin{frame}[fragile]
\frametitle{Проблемы shadow mapping}
\begin{itemize}
\item В базовом shadow mapping хорошо видны пиксели shadow map (квадратная лесенка на границе тени)
\pause
\begin{itemize}
\item Тем хуже, чем больше сцена (пиксели становятся больше)
\item Особенно плохо, если тень движется
\end{itemize}
\pause
\item Хочется сгладить тени, убрав пиксели и сделав аппроксимацию мягких теней
\pause
\item Саму shadow map сглаживать (размывать / использовать линейную фильтрацию / mipmaps) нет смысла: получатся интерполированные значения глубин, сравнение с ними даёт видимые артефакты и ничего не улучшает
\end{itemize}
\end{frame}

\begin{frame}[fragile]
\frametitle{Пиксели shadow mapping}
\slideimage{shadow_map_nearest.png}
\end{frame}

\begin{frame}[fragile]
\frametitle{Пиксели shadow mapping (с линейной фильтрацией)}
\slideimage{shadow_map_linear.png}
\end{frame}

\begin{frame}[fragile]
\frametitle{Percentage-closer filtering (PCF)}
\begin{itemize}
\item Shadow mapping выдаёт 0 или 1 в зависимости от того, находится ли текущий пиксель в тени
\pause
\item PCF: прочитаем не один пиксель из shadow map, а несколько, и проинтерполируем результаты сравнения
\pause
\item Можно взять много соседних пикселей и усреднить результат сравнения по Гауссу
\item Можно взять четыре соседних пикселя и проинтерполировать между ними
\end{itemize}
\end{frame}

\begin{frame}[fragile]
\frametitle{Percentage-closer filtering (PCF): пример кода}
\fontsize{10pt}{10pt}
\begin{verbatim}
float shadow_factor = 0.0;
const int N = 3;
vec2 shadow_map_size = vec2(textureSize(shadow_map, 0));
float total = float((2 * N + 1) * (2 * N + 1));
for (int dx = -N; dx <= N; ++dx) {
    for (int dy = -N; dy <= N; ++dy) {
        if (texture(shadow_map, shadow_pos.xy
            + vec2(dx, dy) / shadow_map_size).r > shadow_pos.z)
            shadow_factor += 1.0 / total;
    }
}
\end{verbatim}
\end{frame}

\begin{frame}[fragile]
\frametitle{Shadow sampler}
\fontsize{10pt}{10pt}
\begin{itemize}
\item В OpenGL есть ограниченная поддержка PCF в виде shadow sampler'ов
\pause
\item Для текстуры нужно настроить параметры:
\begin{itemize}
\item \verb|GL_TEXTURE_COMPARE_MODE| в значение \verb|GL_COMPARE_REF_TO_TEXTURE| (по умолчанию - \verb|GL_NONE|)
\item \verb|GL_TEXTURE_COMPARE_FUNC| в значение \verb|GL_LESS| (теоретически могут быть \verb|GL_ALWAYS| и т.п.)
\end{itemize}
\pause
\item Текстуру с \verb|GL_COMPARE_REF_TO_TEXTURE| нельзя использовать как \verb|sampler2D|, для неё есть отдельный тип sampler'а: \verb|sampler2DShadow|
\pause
\item Функция \verb|texture(shadow_map, texcoord)| принимает трёхмерный вектор текстурных координат:
\begin{itemize}
\item Первые две - обычные X,Y текстурные координаты
\item Третья - используется для сравнения со значением в shadow map
\end{itemize}
\pause
\item Если сравнение успешно, возвращается 1, иначе 0 (функция сравнения настраивается \verb|GL_TEXTURE_COMPARE_FUNC|)
\pause
\item Для такой текстуры можно включить линейную фильтрацию - интерполироваться будут не значения глубины, а результаты сравнения
\end{itemize}
\end{frame}

\begin{frame}[fragile]
\frametitle{Shadow sampler: пример кода}
\begin{verbatim}
float shadow_factor = texture(shadow_map,
    shadow_pos.xyz);
\end{verbatim}
\end{frame}

\begin{frame}[fragile]
\frametitle{Shadow sampler с линейной фильтрацией}
\slideimage{pcf.png}
\end{frame}

\begin{frame}[fragile]
\frametitle{Shadow sampler с линейной фильтрацией и 7x7 размытием по Гауссу}
\slideimage{pcf_gauss.png}
\end{frame}

\begin{frame}[fragile]
\frametitle{Проблемы PCF}
\begin{itemize}
\item {\color{green}+} Есть аппаратная поддержка (shadow sampler)
\item {\color{red}—} Аппаратная поддержка даёт только интерполяцию между 4 соседними пикселями
\item {\color{red}—} Размытие всё равно приходится вычислять для каждого пикселя сцены уже при расчёте освещения - дорого и нельзя применить сепарабельные фильтры
\end{itemize}
\end{frame}

\begin{frame}[fragile]
\frametitle{Filtered shadow maps}
\begin{itemize}
\item Хочется, чтобы shadow map содержал значения, которые имеет смысл интерполировать
\pause
\item Тогда можно применить весь арсенал работы с текстурами: линейную фильтрацию, mipmaps, размытие, etc.
\end{itemize}
\end{frame}

\begin{frame}[fragile]
\frametitle{Filtered shadow maps}
\begin{itemize}
\item Идея: рассматривать значение в одном пикселе shadow map как описывающем некое распределение вероятности
\pause
\item График уровня освещённости в зависимости от расстояния до источника света:
\end{itemize}
\begin{center}
\begin{tikzpicture}
\draw[black,thick] (0.0, 0.0) -- (10.0, 0.0);
\draw[red,thick] (0.0, 0.0) -- (5.0, 0.0) -- (5.0, 1.0) -- (10.0, 1.0);

\node[black] at (10.5, 0.0) {0};
\node[red] at (10.5, 1.0) {1};
\node[blue] at (5.0, -0.5) {\begin{math}Z_0\end{math}};
\end{tikzpicture}
\end{center}
\end{frame}

\begin{frame}[fragile]
\frametitle{Filtered shadow maps}
\begin{itemize}
\item Распределения вероятности можно усреднять: если \begin{math}F_1(z), F_2(z)\end{math} - распределения вероятности, то \begin{math}F(z) = (1 - t) \cdot F_1(z) + t \cdot F_2(z)\end{math} - тоже распределение вероятности
\pause
\item Есть много вариантов того, как представить распределение одним пикселем shadow map
\end{itemize}
\end{frame}

\begin{frame}[fragile]
\frametitle{Exponential shadow maps (ESM)}
\begin{itemize}
\item Идея: функцию распределения можно аппроксимировать экспонентой \begin{math}F(z) = \exp(C \cdot (z_0 - z)) = \exp(-Cz)\exp(Cz_0)\end{math} (\begin{math}z\end{math} - расстояние до освещаемого пикселя, \begin{math}z_0\end{math} - расстояние до источника тени)
\pause
\item Интерполяция таких функций сводится к интерполяции \begin{math}\exp(Cz_0)\end{math}
\pause
\item Будем записывать в shadow map значение \begin{math}\exp(Cz_0)\end{math}
\pause
\item В шейдере будем умножать это значение на \begin{math}\exp(-Cz)\end{math}
\pause
\item Если результат больше единицы (\begin{math}z < z_0\end{math}), принимаем освещённость за единицу
\pause
\end{itemize}
\slideimage{exp_plot.png}
\end{frame}

\begin{frame}[fragile]
\frametitle{ESM с линейной фильтрацией}
\slideimage{exp.png}
\end{frame}

\begin{frame}[fragile]
\frametitle{ESM с линейной фильтрацией и 7x7 размытием по Гауссу}
\slideimage{exp_gauss.png}
\end{frame}

\begin{frame}[fragile]
\frametitle{ESM}
\begin{itemize}
\item {\color{green}+} Shadow map можно фильтровать (размывать, mipmaps, etc)
\item {\color{green}+} Прост в реализации
\item {\color{red}—} Требует floating-point буфер глубины или рендеринг в floating-point текстуру
\item {\color{red}—} При фильтрации возникают артефакты из-за того, что результирующая освещённость может быть больше 1
\end{itemize}
\end{frame}

\begin{frame}[fragile]
\frametitle{Variance shadow maps (VSM)}
\begin{itemize}
\item Идея: запишем в shadow map мат.ожидание глубины и её квадрата
\pause
\item Их можно усреднять:
\begin{equation}
\int z \left[(1-t)p_1(z) + tp_2(z)\right] dz = (1-t)\int z p_1(z) dz + t\int z p_2(z) dz
\end{equation}
\item (аналогично для \begin{math}z^2\end{math})
\pause
\item Моделируем пиксель как дискретное распределение с единственным значением \begin{math}\Rightarrow\end{math} мат.ожидание глубины совпадает с её значением (и аналогично для квадрата глубины)
\end{itemize}
\end{frame}

\begin{frame}[fragile]
\frametitle{Variance shadow maps (VSM)}
\begin{itemize}
\item По мат.ожиданию глубины (\begin{math}\mu\end{math}) и её квадрата можно вычислить дисперсию
\begin{equation}
\sigma^2 = \mathbb{E}[z^2] - \mathbb{E}[z]^2
\end{equation}
\pause
\item По мат.ожиданию глубины и дисперсии можно вычислить освещённость через неравенство Чебышёва:
\begin{equation}
P(z) = \frac{\sigma^2}{\sigma^2 + (z - \mu)^2}
\end{equation}
\end{itemize}
\end{frame}

\begin{frame}[fragile]
\frametitle{Variance shadow maps (VSM): алгоритм}
\begin{itemize}
\item В shadow map записываем глубину и её квадрат (нужна текстура с 2 компонентами)
\pause
\item По желанию с результирующей текстурой shadow map производим фильтрацию (e.g. размытие)
\pause
\item В шейдере читаем shadow map, вычисляем \begin{math}\mu\end{math} и \begin{math}\sigma^2\end{math}
\pause
\item Если глубина нашего пикселя меньше \begin{math}\mu\end{math}, освещённость равна 1
\pause
\item Иначе, вычисляем освещённость через неравенство Чебышёва
\end{itemize}
\end{frame}

\begin{frame}[fragile]
\frametitle{Variance shadow maps (VSM): shadow bias}
\begin{itemize}
\item VSM позволяет элегантно включить shadow bias
\pause
\item Моделируем пиксель как параллелограмм с линейно меняющейся глубиной
\pause
\item Градиент глубины можно аппроксимировать с помощью \verb|dFdx| и \verb|dFdy| в шейдере
\pause
\item Итоговый квадрат глубины вычисляется как
\begin{equation}
z^2 + \frac{1}{4}\left[\left(\frac{\partial z}{\partial x}\right)^2 + \left(\frac{\partial z}{\partial y}\right)^2\right]
\end{equation}
\pause
\item Убирает \textit{почти} все артефакты shadow acne
\end{itemize}
\end{frame}

\begin{frame}[fragile]
\frametitle{VSM с линейной фильтрацией}
\slideimage{vsm.png}
\end{frame}

\begin{frame}[fragile]
\frametitle{VSM с линейной фильтрацией и 7x7 размытием по Гауссу}
\slideimage{vsm_gauss.png}
\end{frame}

\begin{frame}[fragile]
\frametitle{VSM}
\begin{itemize}
\item {\color{green}+} Shadow map можно фильтровать (размывать, mipmaps, etc)
\item {\color{green}+} Достаточно прост в реализации
\item {\color{red}—} Требует floating-point буфер глубины или рендеринг в floating-point текстуру
\item {\color{red}—} Возникает артефакт light bleeding
\end{itemize}
\end{frame}

\begin{frame}[fragile]
\frametitle{Light bleeding}
\slideimage{vsm_light_bleeding.png}
\end{frame}

\begin{frame}[fragile]
\frametitle{Light bleeding}
\begin{itemize}
\item Обычно возникает, когда \begin{math}\sigma^2\end{math} оказывается слишком большим
\pause
\item Можно исправить, трактуя маленькие значения коэффициента освещённости как нулевые, и преобразовав остальные значения в диапазон \begin{math}[0, 1]\end{math}
\pause
\item Конкретные значения нужно подбирать в зависимости от сцены
\end{itemize}
\end{frame}

\begin{frame}[fragile]
\frametitle{Light bleeding}
\slideimage{light_bleeding_scheme.jpg}
\end{frame}

\begin{frame}[fragile]
\frametitle{Convolution shadow maps (CSM)}
\begin{itemize}
\item Записывает в shadow map коэффициенты преобразования Фурье от функции распределения
\pause
\item {\color{green}+} Можно фильтровать (размывать, mipmaps, etc)
\item {\color{red}—} Сложнее в реализации
\item {\color{red}—} Требует многокомпонентных буферов (часто до 16 компонент на пиксель)
\item {\color{red}—} Возникают ringing-артефакты (см. \href{https://en.wikipedia.org/wiki/Gibbs_phenomenon}{Gibbs phenomenon})
\end{itemize}
\end{frame}

\begin{frame}[fragile]
\frametitle{Ссылки}
\begin{itemize}
\item \href{https://ogldev.org/www/tutorial42/tutorial42.html}{ogldev.org/www/tutorial42/tutorial42.html}
\item \href{https://learnopengl.com/Advanced-Lighting/Shadows/Shadow-Mapping}{learnopengl.com/Advanced-Lighting/Shadows/Shadow-Mapping}
\item \href{https://docs.microsoft.com/en-us/windows/win32/dxtecharts/common-techniques-to-improve-shadow-depth-maps}{docs.microsoft.com/en-us/windows/win32/dxtecharts/common-techniques-to-improve-shadow-depth-maps}
\item \href{https://jankautz.com/publications/esm_gi08.pdf}{Оригинальная статья про ESM}
\item \href{https://www.intel.com/content/dam/develop/external/us/en/documents/vsm-paper-182631.pdf}{Оригинальная статья про VSM}
\item \href{https://developer.nvidia.com/gpugems/gpugems3/part-ii-light-and-shadows/chapter-8-summed-area-variance-shadow-maps}{Улучшения VSM, включая более реалистичные мягкие тени}
\end{itemize}
\end{frame}

\begin{frame}[fragile]
\frametitle{Тени на больших сценах}
\begin{itemize}
\item Если сцена очень большая, даже огромных (8k) shadow map текстур с хорошим сглаживанием может не хватить
\pause
\item Классически есть два решения:
\begin{itemize}
\item Использовать неравномерное соответствие пикселей shadow map и объектов в мире
\item Использовать более одной shadow map
\end{itemize}
\end{itemize}
\end{frame}

\begin{frame}[fragile]
\frametitle{Perspective shadow maps (PSM)}
\begin{itemize}
\item При генерации shadow map помимо обычного преобразования вершин применяет ещё перспективное преобразование так, чтобы больше пикселей shadow map приходилось на близкие к камере объекты
\end{itemize}
\slideimage{psm.jpeg}
\end{frame}

\begin{frame}[fragile]
\frametitle{Perspective shadow maps (PSM)}
\begin{itemize}
\item {\color{green}+} Часто решает проблему точности shadow map
\item {\color{red}—} Сложен в реализации
\item {\color{red}—} Очень много худших случаев (например, камера смотрит на источник света), в которых алгоритм почти не работает
\end{itemize}
\end{frame}

\begin{frame}[fragile]
\frametitle{Cascaded shadow maps (CSM)}
\begin{itemize}
\item Будем использовать несколько shadow maps (cascades): одну - для ближайших объектов, другую - для объектов на среднем расстоянии, третью - для далёких объектов, и т.п.
\end{itemize}
\slideimage{csm.jpg}
\end{frame}

\begin{frame}[fragile]
\frametitle{Cascaded shadow maps (CSM)}
\begin{itemize}
\item {\color{green}+} Прост в реализации
\item {\color{green}+} Можно комбинировать с VSM/ESM/etc
\item {\color{red}—} Большой расход памяти
\item {\color{red}—} Нужно аккуратно обработать переход между разными каскадами
\item Самый часто использующийся сегодня алгоритм
\end{itemize}
\end{frame}

\begin{frame}[fragile]
\frametitle{Ссылки}
\begin{itemize}
\item \href{https://www-sop.inria.fr/reves/Basilic/2002/SD02/PerspectiveShadowMaps.pdf}{Оригинальная статья про PSM}
\item \href{https://developer.download.nvidia.com/books/HTML/gpugems/gpugems_ch14.html}{PSM + улучшения}
\item \href{https://developer.download.nvidia.com/SDK/10.5/opengl/src/cascaded_shadow_maps/doc/cascaded_shadow_maps.pdf}{Статья про CSM}
\item \href{https://ogldev.org/www/tutorial49/tutorial49.html}{ogldev.org/www/tutorial49/tutorial49.html}
\item \href{https://learnopengl.com/Guest-Articles/2021/CSM}{learnopengl.com/Guest-Articles/2021/CSM}
\end{itemize}
\end{frame}

\end{document}