% (c) Nikita Lisitsa, lisyarus@gmail.com, 2021

\documentclass{beamer}

\usepackage[T2A]{fontenc}
\usepackage[utf8]{inputenc}
\usepackage[russian]{babel}

\usepackage{graphicx}
\graphicspath{ {./images/} }

\usepackage{adjustbox}

\usepackage{color}
\usepackage{soul}

\usepackage{hyperref}

\usepackage{amsmath}

\usepackage{tikz}
\usetikzlibrary{decorations}
\usetikzlibrary{decorations.pathreplacing}
\usepackage{xifthen}

\definecolor{red}{rgb}{1,0,0}
\definecolor{green}{rgb}{0,0.5,0}
\definecolor{blue}{rgb}{0,0,1}
\definecolor{magenta}{rgb}{0.75,0,0.75}

\hypersetup{
    colorlinks=true,
    linkcolor=blue,
    urlcolor=blue,
}

\makeatletter
\newcommand{\slideimage}[1]{
  \begin{figure}
    \begin{adjustbox}{width=\textwidth, totalheight=\textheight-2\baselineskip-2\baselineskip,keepaspectratio}
      \includegraphics{#1}
    \end{adjustbox}
  \end{figure}
}
\makeatother

\title{Компьютерная графика}
\subtitle{Лекция 11: HDR, tone mapping, gamma correction, sRGB, color banding, dithering}
\date{2021}

\setbeamertemplate{footline}[frame number]

\begin{document}

\frame{\titlepage}

\begin{frame}[fragile]
\frametitle{Артефакт из 10ой практики}
\slideimage{practice10_bug.png}
\end{frame}

\begin{frame}[fragile]
\frametitle{Артефакт из 10ой практики}
\begin{itemize}
\item Возникает при зацикливании текстурных координат при расчёте mipmap-уровня
\pause
\item Mipmap-уровень, который будет взят для текстуры, вычисляется в группах \begin{math}2\times 2\end{math} пикселей на основе разности текстурных координат в соседних пикселях
\pause
\item В одном пикселе группы вычисляется текстурная координата, близкая к 0, а в соседнем -- близкая к 1
\pause
\item \begin{math}\Rightarrow\end{math} Выбирается слишком большой mipmap-уровень (в данном случае он выглядит просто серым)
\end{itemize}
\end{frame}

\begin{frame}[fragile]
\frametitle{Артефакт из 10ой практики}
\begin{itemize}
\item Может возникнуть в любой ситуации, когда текстура накладывается цилиндрически
\pause
\item Как решать:
\pause
\begin{itemize}
\item Первый способ: разрезать в этом месте геометрию (например, для цилиндра дублировать вершины `нулевого меридиана', у одной копии текстурная координата равна 0, у другой -- 1)
\item Второй способ: хитрее вычислять текстурные координаты и выбирать вариант с меньшим расхождением в соседних пикселях
\end{itemize}
\pause
\item \href{https://bgolus.medium.com/distinctive-derivative-differences-cce38d36797b}{Distinctive Derivative Differences} -- подробная статья про проблему и решения
\end{itemize}
\end{frame}

\begin{frame}[fragile]
\frametitle{HDR}
\begin{itemize}
\item Проблема: после вычисления освещённости мы можем получить значения компонент цвета большие 1, и они будут обрезаны до 1
\pause
\item В реальном мире интенсивность света (например, между солнечным днём и тёмной ночью) может меняться на 15 порядков
\pause
\item При рендеринге объект может освещаться тусклым ambient освещением, а может сразу десятком ярких источников света
\pause
\item High Dynamic Range (HDR) -- термин, означающий большой (не ограниченный [0, 1]) диапазон интенсивностей
\pause
\item HDR текстура (например, environment map) -- содержит значения, выходящие за диапазон [0, 1] (обычно 32-bit floating-point)
\pause
\item HDR рендеринг -- позволяет отобразить HDR-освещение
\end{itemize}
\end{frame}

\begin{frame}[fragile]
\frametitle{Tone mapping}
\begin{itemize}
\item Нужно превратить диапазон освещённостей \begin{math}[0, \infty)\end{math} в диапазон \begin{math}[0, 1]\end{math}
\pause
\item В принципе, подойдёт любая монотонно возрастающая функция \begin{math}[0, \infty)\rightarrow [0, 1]\end{math}
\begin{itemize}
\item \begin{math}x \mapsto \frac{x}{1+x}\end{math} -- Reinhard operator
\item \begin{math}x \mapsto \arctan(x)\end{math}
\item \begin{math}x \mapsto \frac{2}{1+e^{-x}}-1\end{math}
\end{itemize}
\pause
\item Иногда используют более сложные функции, взятые из кинематографа (filmic tone mapping)
\pause
\item Эти функции часто позволяют динамически настраивать среднюю освещённость сцены (exposure, выдержка)
\end{itemize}
\end{frame}

\begin{frame}[fragile]
\frametitle{Reinhard operator}
\slideimage{reinhard.png}
\end{frame}

\begin{frame}[fragile]
\frametitle{Filmic tone mapping curve (ACES)}
\slideimage{aces.png}
\end{frame}

\begin{frame}[fragile]
\frametitle{HDR \& tone mapping: ссылки}
\begin{itemize}
\item \href{https://learnopengl.com/Advanced-Lighting/HDR}{\nolinkurl{learnopengl.com/Advanced-Lighting/HDR}}
\item \href{https://skylum.com/blog/what-is-tone-mapping}{\nolinkurl{skylum.com/blog/what-is-tone-mapping}}
\item \href{https://www.veneratech.com/what-is-hdr-tone-mapping}{\nolinkurl{veneratech.com/what-is-hdr-tone-mapping}}
\end{itemize}
\end{frame}

\begin{frame}<1>[fragile,label=gamma]
\frametitle{Гамма}
\begin{itemize}
\item Обычно, интенсивность света \begin{math}I\end{math}, излучаемого монитором, нелинейно зависит от значения \begin{math}V\end{math}, записанного в пикселе
\pause
\item Это лучше соответствует восприятию света человеком и даёт больше точности тёмным цветам, которые человек различает лучше
\pause
\item Почти всегда используется показательная функция:
\begin{equation}I \sim V^\gamma\end{equation}
\pause
\item \begin{math}\gamma\end{math} обычно равна 2.2 (некоторые компьютеры Macintosh использовали 1.8)
\end{itemize}
\end{frame}

\begin{frame}
\frametitle{Линейное значение пикселя vs линейная интенсивность излучения}
\begin{figure}
\slideimage{gamma-scale.png}
\end{figure}
\end{frame}

\againframe<2->{gamma}

\begin{frame}<1>[fragile,label=gamma2]
\frametitle{Проблемы гаммы}
\begin{itemize}
\item Картинка может издалека выглядеть ярче, чем её усреднённый вариант (e.g. mipmap)
\pause
\item Искажается восприятие относительных яркостей, особенно при реалистичном рендеринге (e.g. объект в два раза ярче не будет выглядеть в два раза ярче)
\pause
\item Неправильно выглядит освещение, наложение источников света, и т.д.
\end{itemize}
\end{frame}

\begin{frame}
\frametitle{Серый (цвет=0.5) квадрат и квадрат с мелкой шахматной раскраской}
\begin{figure}
\slideimage{gamma-checkers.png}
\end{figure}
\end{frame}

\againframe<2->{gamma2}

\begin{frame}[fragile]
\frametitle{Коррекция гаммы (gamma-correction)}
\begin{itemize}
\item Коррекция гаммы -- общий термин для применения любых нелинейных преобразований над интенсивностью пикселя
\pause
\item В рендеринге под гамма-коррекцией обычно подразумевают применение обратного к \begin{math}V^{2.2}\end{math} преобразования, чтобы получить линейную зависимость выходящего излучения от значения пикселя
\end{itemize}
\end{frame}

\begin{frame}[fragile]
\frametitle{Коррекция гаммы (gamma-correction)}
\begin{verbatim}
// Вычислили цвет пикселя
// с учётом освещения
vec3 color = ...;

color = pow(color, vec3(1.0 / 2.2));
\end{verbatim}
\end{frame}

\begin{frame}[fragile]
\frametitle{Эффекты коррекции гаммы}
\slideimage{gamma-ex1.png}
\end{frame}

\begin{frame}[fragile]
\frametitle{Эффекты коррекции гаммы}
\slideimage{gamma-ex2.png}
\end{frame}

\begin{frame}[fragile]
\frametitle{Эффекты коррекции гаммы}
\slideimage{gamma-ex3.png}
\end{frame}

\begin{frame}[fragile]
\frametitle{Эффекты коррекции гаммы}
\begin{itemize}
\item Типичные эффекты гамма-коррекции:
\pause
\begin{itemize}
\item Картинка становится ярче (чем темнее цвет, тем больше прирост яркости)
\pause
\item Картинка становится менее контрастной
\pause
\item Освещение выглядит правильнее
\pause
\item Переход от освещённой к неосвещённой области объекта выглядит более резким (как в реальности)
\end{itemize}
\end{itemize}
\end{frame}

\begin{frame}[fragile]
\frametitle{Коррекция гаммы: ссылки}
\begin{itemize}
\item \href{https://en.wikipedia.org/wiki/Gamma_correction}{\nolinkurl{en.wikipedia.org/wiki/Gamma\_correction}}
\item \href{http://blog.johnnovak.net/2016/09/21/what-every-coder-should-know-about-gamma}{\nolinkurl{What every coder should know about gamma}}
\item \href{http://filmicworlds.com/blog/linear-space-lighting-i-e-gamma}{\nolinkurl{Linear-space lighting (i.e. gamma)}}
\item \href{https://learnopengl.com/Advanced-Lighting/Gamma-Correction}{\nolinkurl{Learnopengl.com tutorial}}
\end{itemize}
\end{frame}

\begin{frame}[fragile]
\frametitle{sRGB}
\begin{itemize}
\item По тем же причинам, по которым существует нелинейная гамма у мониторов (отдать больше бит под тёмные цвета), хочется хранить текстуры не в сыром формате, а в гамма-преобразованном: значение пикселя \begin{math}V\end{math} зависит от желаемой интенсивности \begin{math}I\end{math} как \begin{math}V \sim I^\frac{1}{\gamma}\end{math}
\pause
\item Такой формат хранения цвета называется sRGB
\begin{itemize}
\item N.B. Точная формула преобразования sRGB отличается от \begin{math}I^\frac{1}{2.2}\end{math}, но очень близка к ней
\end{itemize}
\pause
\item Обычно изображения хранятся именно в таком формате
\pause
\item Проверить формат можно любым редактором изображений или программой \verb|identify| пакета \verb|imagemagick|
\end{itemize}
\end{frame}

\begin{frame}[fragile]
\frametitle{sRGB}
\begin{itemize}
\item При чтении из sRGB-текстуры нужно возводить прочитанное значение в степень 2.2
\pause
\item При записи в sRGB-текстуру нужно возводить записываемое значение в степень 1/2.2
\pause
\item В OpenGL есть поддержка sRGB для текстур и фреймбуферов
\end{itemize}
\end{frame}

\begin{frame}[fragile]
\frametitle{sRGB-текстуры}
\begin{itemize}
\item Специальное значение internal format для текстуры (\verb|GL_SRGB8| или \verb|GL_SRGB8_ALPHA8|) означает, что текстура хранит sRGB-значения -- они будут \textit{автоматически} переведены в линейные значения (т.е. возведены в степень 2.2) при чтении из шейдера
\pause
\item \verb|glEnable(GL_FRAMEBUFFER_SRGB)| включит автоматическое обратное преобразование (возведение в степень 1/2.2) при рисовании в sRGB-текстуру
\pause
\item {\color{red}N.B.:} \verb|glGenerateMipmap| \textit{не обязан} правильно обрабатывать sRGB-текстуры!
\end{itemize}
\end{frame}

\begin{frame}[fragile]
\frametitle{sRGB-correct mipmap}
\slideimage{srgb-mipmap.png}
\end{frame}

\begin{frame}[fragile]
\frametitle{sRGB: ссылки}
\begin{itemize}
\item \href{https://en.wikipedia.org/wiki/SRGB}{\nolinkurl{en.wikipedia.org/wiki/SRGB}}
\item \href{https://stackoverflow.com/a/12894053/2315602}{\nolinkurl{Stackoverflow post on sRGB}}
\item \href{https://www.khronos.org/opengl/wiki/Framebuffer#Colorspace}{\nolinkurl{sRGB framebuffers}}
\end{itemize}
\end{frame}

\begin{frame}[fragile]
\frametitle{Color banding}
\slideimage{banding.jpg}
\end{frame}

\begin{frame}[fragile]
\frametitle{Color banding}
\begin{itemize}
\item Артефакт, когда чётко видна граница между областями, заполненными одним цветом
\pause
\item Обычно не сильно заметен, особенно после наложения текстур и пост-обработки
\pause
\item Хорошо заметен на монотонных поверхностях (напр. небо)
\pause
\item Сильнее заметен с тёмными цветами (из-за особенностей человеческого зрения)
\pause
\item Может усиливаться при неаккуратной работе с форматами хранения и нелинейными преобразованиями
\end{itemize}
\end{frame}

\begin{frame}[fragile]
\frametitle{Color banding: пример возникновения}
\begin{itemize}
\item Рисуем сцену, записываем линейную интенсивность в текстуру в формате \verb|GL_RGB8|
\pause
\item Выводим сцену на экран с гамма-коррекцией: значения из текстуры возводятся в степень 1/2.2
\pause
\item Значение текстуры 0 переходит в значение 0 на экране
\pause
\item Значение текстуры 1 переходит в значение \verb|255 * (1/255)^(1/2.2) = 21| на экране
\pause
\item Колоссальная потеря точности: значения от 1 до 20 не используются!
\pause
\item Для избавления можно увеличить битность промежуточной текстуры до \verb|GL_RGB16|
\end{itemize}
\end{frame}

\begin{frame}[fragile]
\frametitle{Color banding}
\begin{itemize}
\item Banding, связанный с потерей точности, обычно решают увеличением точности
\pause
\item Banding, связанный с ограниченной точностью самого экрана, обычно решают с помощью \textit{дизеринга}
\end{itemize}
\end{frame}

\begin{frame}[fragile]
\frametitle{Dithering}
\slideimage{dither.png}
\end{frame}

\begin{frame}[fragile]
\frametitle{Dithering}
\begin{itemize}
\item Способ рисования градиентов (переходов между цветами) за счёт варьирования количества пикселей двух цветов
\pause
\item Работает только для больших (по размеру) градиентов
\pause
\item Не зависит от точности
\end{itemize}
\end{frame}

\begin{frame}[fragile]
\frametitle{Dithering}
\slideimage{dither2.png}
\end{frame}

\begin{frame}<1>[fragile,label=dither]
\frametitle{Dithering: реализация}
\begin{itemize}
\item Допустим, мы рисуем в однобитное (чёрно-белое) изображение, и хотим добиться серого цвета с помощью дизеринга
\pause
\item Нам нужен способ определить, какой из пикселей будет чёрным, а какой -- белым
\pause
\item При одноразовой обработке изображения можно оптимизировать паттерн дизеринга под конкретное изображение
\pause
\item В real-time графике это обычно делается с помощью т.н. dither mask / dither matrix (ordered dithering)
\end{itemize}
\end{frame}

\begin{frame}[fragile]
\frametitle{Dithering}
\slideimage{dither3.png}
\end{frame}

\againframe<2->{dither}

\begin{frame}[fragile]
\frametitle{Dithering: реализация}
\begin{itemize}
\item Dither mask -- одноканальная текстура, хранящая для каждого пикселя пороговое значение, начиная с которого его надо рисовать белым
\pause
\item Например, для серого цвета с яркостью 0.25 мы нарисуем белыми те пиксели, для которых значение в dither map \begin{math}\leq 0.25\end{math}
\pause
\item Например, для серого цвета с яркостью 0.5 мы нарисуем белыми те пиксели, для которых значение в dither map \begin{math}\leq 0.5\end{math}
\pause
\item Например, для серого цвета с яркостью 0.75 мы нарисуем белыми те пиксели, для которых значение в dither map \begin{math}\leq 0.75\end{math}
\end{itemize}
\end{frame}

\begin{frame}[fragile]
\frametitle{Dither mask}
\slideimage{dither-mask.png}
\end{frame}

\begin{frame}[fragile]
\frametitle{Dithering: реализация}
\begin{itemize}
\item Для дизеринга при рисовании в 8-битный буфер:
\begin{itemize}
\item Хотим записать значение \begin{math}v \in [0, 1]\end{math}
\item Вычисляем \begin{math}v' = 255 \cdot v\end{math}
\item Вычисляем \begin{math}v_0 = floor(v')\end{math}
\item Вычисляем \begin{math}v_1 = v_0 + 1\end{math}
\item Вычисляем \begin{math}dv = v' - v0\end{math}
\item Если значение в dither mask меньше \begin{math}dv\end{math}, рисуем значение \begin{math}v_1 / 255\end{math}, иначе \begin{math}v_0 / 255\end{math}
\end{itemize}
\pause
\item Дизеринг нужно делать для каждого канала (RGB) по отдельности
\pause
\item Обычно dither mask имеет маленький размер (напр. 16x16) и циклически (\verb|GL_REPEAT|) повторяется на весь экран; её пиксели пробегают все возможные значения от 0 до 255
\end{itemize}
\end{frame}

\begin{frame}[fragile]
\fontsize{10pt}{10pt}
\frametitle{Dithering: реализация}
\begin{verbatim}
vec3 color = ...;

vec3 c = color * 255.0;
vec3 c0 = floor(c);
vec3 c1 = c0 + vec3(1.0);
vec3 dc = c - c0;

float threshold = texture(dither_mask, pixel_coord);

color = c0 / 255.0;
if (dc.r > threshold) color.r = c1.r / 255.0;
if (dc.g > threshold) color.g = c1.g / 255.0;
if (dc.b > threshold) color.b = c1.b / 255.0;
\end{verbatim}
\end{frame}

\begin{frame}[fragile]
\frametitle{Генерация dither mask}
\begin{itemize}
\item Bayer dithering -- очень быстро генерируется, могут быть заметны артефакты из-за регулярности паттерна
\pause
\item Void-and-cluster -- использует медленный blue noise для генерации маски, лучше распределение пикселей
\end{itemize}
\end{frame}

\begin{frame}[fragile]
\frametitle{Сравнение bayer и void-and-cluster}
\slideimage{dither-mask-compare.png}
\end{frame}

\begin{frame}[fragile]
\frametitle{Screen-door transparency}
\slideimage{dither-transparency.png}
\end{frame}

\begin{frame}[fragile]
\frametitle{Screen-door transparency}
\begin{itemize}
\item Дизеринг часто применяют как дешёвый аналог прозрачности
\pause
\item Если значение в альфа-канале больше значения из dither mask, то рисуем пиксель, иначе -- не рисуем (\verb|discard|)
\pause
\item Не требует сортировки объектов по расстоянию, работает с deferred shading'ом
\pause
\item Выглядит заметно хуже, часто применяется для временно прозрачных объектов (появляющихся/исчезающих), далёких объектов (деревья, кусты)
\end{itemize}
\end{frame}

\begin{frame}[fragile]
\frametitle{Return of the Obra Dinn (2018)}
\slideimage{obra-dinn.png}
\end{frame}

\begin{frame}[fragile]
\frametitle{Dithering: ссылки}
\begin{itemize}
\item \href{https://en.wikipedia.org/wiki/Dither}{\nolinkurl{en.wikipedia.org/wiki/Dither}}
\item \href{https://surma.dev/things/ditherpunk}{\nolinkurl{surma.dev/things/ditherpunk}}
\item \href{http://alex-charlton.com/posts/Dithering_on_the_GPU}{\nolinkurl{alex-charlton.com/posts/Dithering_on_the_GPU}}
\item \href{https://forums.tigsource.com/index.php?topic=40832.msg1363742}{Dithering in Return of the Obra Dinn}
\end{itemize}
\end{frame}

\end{document}