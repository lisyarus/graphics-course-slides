% (c) Nikita Lisitsa, lisyarus@gmail.com, 2021

\documentclass{beamer}

\usepackage[T2A]{fontenc}
\usepackage[utf8]{inputenc}
\usepackage[russian]{babel}

\usepackage{graphicx}
\graphicspath{ {./images/} }

\usepackage{adjustbox}

\usepackage{color}
\usepackage{soul}

\usepackage{hyperref}

\usepackage{amsmath}

\usepackage{tikz}
\usetikzlibrary{decorations}
\usetikzlibrary{decorations.pathreplacing}
\usepackage{xifthen}

\definecolor{red}{rgb}{1,0,0}
\definecolor{green}{rgb}{0,0.5,0}
\definecolor{blue}{rgb}{0,0,1}
\definecolor{magenta}{rgb}{0.75,0,0.75}

\makeatletter
\newcommand{\slideimage}[1]{
  \begin{figure}
    \begin{adjustbox}{width=\textwidth, totalheight=\textheight-2\baselineskip-2\baselineskip,keepaspectratio}
      \includegraphics{#1}
    \end{adjustbox}
  \end{figure}
}
\makeatother

\title{Компьютерная графика}
\subtitle{Лекция 9: Геометрические шейдеры, shadow volumes, shadow mapping и его разновидности}
\date{2021}

\setbeamertemplate{footline}[frame number]

\begin{document}

\frame{\titlepage}

\begin{frame}[fragile]
\frametitle{Геометрические шейдеры}
\begin{itemize}
\item Тип шейдера, наравне в вершинным и фрагментным
\pause
\item Создаётся как \verb|glCreateShader(GL_GEOMETRY_SHADER)|
\pause
\item Встраивается после вершинного шейдера, до perspective divide
\pause
\item Оперирует целыми примитивами (точками/линиями/треугольниками), т.е. наборами вершин
\pause
\item Может менять тип примитива и количество вершин
\pause
\item Может варьировать количество вершин на выходе
\end{itemize}
\end{frame}

\begin{frame}<1>[fragile,label=geometry_shader_examples]
\frametitle{Геометрические шейдеры: примеры использования}
\fontsize{10pt}{10pt}
\begin{itemize}
\item Расчёт нормалей для flat shading'а
\pause
\begin{itemize}
\item Вход: треугольник (тройка вершин)
\item Выход: те же вершины с посчитанной нормалью
\end{itemize}
\pause
\item Визуализация нормалей
\pause
\begin{itemize}
\item Вход: вершина с нормалью
\item Выход: линия из двух вершин - исходная вершина, исходная вершина + нормаль
\end{itemize}
\pause
\item Shadow volumes
\pause
\item Billboards - плоские фигуры, всегда смотрящие в сторону камеры
\pause
\begin{itemize}
\item Вход: одна вершина (точка)
\item Выход: набор треугольников
\pause
\item Системы частиц
\pause
\item Облака
\pause
\item Деревья
\pause
\item Трава
\end{itemize}
\pause
\item Изолинии
\pause
\begin{itemize}
\item Вход: треугольник
\item Выход: изолиния, проходящая через этот треугольник (0 или 2 вершины)
\end{itemize}
\end{itemize}
\end{frame}

\begin{frame}[fragile]
\frametitle{Flat shading}
\slideimage{flat-shading.png}
\end{frame}

\againframe<2-4>{geometry_shader_examples}

\begin{frame}[fragile]
\frametitle{Визуализация нормалей}
\slideimage{normals-viz.png}
\end{frame}

\againframe<5-6>{geometry_shader_examples}

\begin{frame}[fragile]
\frametitle{Billboards}
\slideimage{billboards.jpg}
\end{frame}

\againframe<7-8>{geometry_shader_examples}

\begin{frame}[fragile]
\frametitle{Billboards: система частиц (дым)}
\slideimage{smoke.jpg}
\end{frame}

\againframe<8-9>{geometry_shader_examples}

\begin{frame}[fragile]
\frametitle{Billboards: облака}
\slideimage{clouds.jpg}
\end{frame}

\againframe<9-10>{geometry_shader_examples}

\begin{frame}[fragile]
\frametitle{Billboards: деревья}
\slideimage{trees.jpg}
\end{frame}

\againframe<10-11>{geometry_shader_examples}

\begin{frame}[fragile]
\frametitle{Billboards: трава}
\slideimage{grass.jpg}
\end{frame}

\begin{frame}[fragile]
\frametitle{Геометрические шейдеры: пример}
\fontsize{10pt}{10pt}
\begin{verbatim}
#version 330 core
uniform mat4 transform;
// Входные примитивы - точки
layout (points) in;
// Выходные примитивы - линии, в сумме не больше 2х вершин
layout (line_strip, max_vertices = 2) out;
// Данные из вершинного шейдера
in vec3 normal[];
  
void main() {    
    gl_Position = transform * gl_in[0].gl_Position;
    EmitVertex();

    gl_Position = transform * (gl_in[0].gl_Position
        + vec4(normal[0], 0));
    EmitVertex();

    EndPrimitive();
} 
\end{verbatim}
\end{frame}

\begin{frame}[fragile]
\frametitle{Геометрические шейдеры: пример}
\fontsize{8pt}{8pt}
\begin{verbatim}
#version 330 core
// Входные примитивы - линии
layout (lines) in;
// Выходные примитивы - треугольники, в сумме не больше 4х вершин
layout (triangle_strip, max_vertices = 4) out;
// Данные для фрагментного шейдера
out vec4 color;
  
void main() {
    gl_Position = gl_in[0].gl_Position + vec4(-1.0, -1.0, 0.0, 0.0);
    color = vec4(1.0, 0.0, 0.0, 1.0);
    EmitVertex();

    gl_Position = gl_in[0].gl_Position + vec4( 1.0, -1.0, 0.0, 0.0);
    color = vec4(0.0, 1.0, 0.0, 1.0);
    EmitVertex();

    gl_Position = gl_in[1].gl_Position + vec4(-1.0,  1.0, 0.0, 0.0);
    color = vec4(0.0, 0.0, 1.0, 1.0);
    EmitVertex();

    gl_Position = gl_in[1].gl_Position + vec4( 1.0,  1.0, 0.0, 0.0);
    color = vec4(1.0, 1.0, 1.0, 1.0);
    EmitVertex();

    EndPrimitive();
} 
\end{verbatim}
\end{frame}

\begin{frame}[fragile]
\frametitle{Геометрические шейдеры: ссылки}
\begin{itemize}
\item \href{https://www.khronos.org/opengl/wiki/Geometry_Shader}{khronos.org/opengl/wiki/Geometry\_Shader}
\item \href{https://learnopengl.com/Advanced-OpenGL/Geometry-Shader}{learnopengl.com/Advanced-OpenGL/Geometry-Shader}
\item \href{https://open.gl/geometry}{open.gl/geometry}
\item \href{https://www.lighthouse3d.com/tutorials/glsl-tutorial/geometry-shader}{lighthouse3d.com/tutorials/glsl-tutorial/geometry-shader}
\item \href{https://developer.download.nvidia.com/books/HTML/gpugems/gpugems_ch07.html}{GPU Gems, Chapter 7. Rendering Countless Blades of Waving Grass}
\end{itemize}
\end{frame}

\begin{frame}[fragile]
\frametitle{Тени: теория}
\begin{itemize}
\item Точка сцены, в которую не попадает (заблокирован чем-то) прямой свет из конкретного источника света
\item Свойство точки по отношению к конкретному источнику света
\end{itemize}
\end{frame}

\begin{frame}[fragile]
\frametitle{Тени}
\slideimage{shadows1.png}
\end{frame}

\begin{frame}[fragile]
\frametitle{Тени: теория}
\begin{itemize}
\item Если источник света точечный (или бесконечно удалённый), тень - бинарное свойство: луч из точки сцены в источник света или пересекает что-то (точка в тени), или нет (точка не в тени) \begin{math}\Rightarrow\end{math} жёсткие тени (hard shadows)
\pause
\item Если источник света объёмный, точка сцены может находиться в тени относительно части источника света, и не находиться в тени относительно другой его части \begin{math}\Rightarrow\end{math} мягкие тени (soft shadows)
\pause
\begin{itemize}
\item Точки, полностью находящиеся в тени - \textit{umbra}
\item Точки, частично находящиеся в тени - \textit{penumbra}
\end{itemize}
\end{itemize}
\end{frame}

\begin{frame}[fragile]
\frametitle{Тени: мягкие vs жёсткие}
\slideimage{shadow-scheme1.png}
\end{frame}

\begin{frame}[fragile]
\frametitle{Мягкие тени}
\slideimage{shadows2.png}
\end{frame}

\begin{frame}[fragile]
\frametitle{Солнечное затмение}
\slideimage{eclipse.png}
\end{frame}

\begin{frame}[fragile]
\frametitle{Тени: теория}
\begin{itemize}
\item Реальные источники света - объёмные
\pause
\item Размер и форма penumbra зависит от размеров и формы источника света, а также от расстояния до него (чем дальше, тем меньше penumbra)
\pause
\begin{itemize}
\item В пределе расстояния \begin{math}\rightarrow\infty\end{math} мягкая тень вырождается в жёсткую
\end{itemize}
\pause
\item В real-time графике получить правильные тени (как жёсткие, так и мягкие) довольно сложно
\end{itemize}
\end{frame}

\begin{frame}[fragile]
\frametitle{Тени: алгоритмы}
\begin{itemize}
\item Shadow volumes
\pause
\begin{itemize}
\item {\color{green}+} Идеальные жёсткие тени
\item {\color{red}---} Алиасинг
\item {\color{red}---} Сложно сделать мягкие тени
\item {\color{red}---} Очень большой fill rate (количество обрабатываемых пикселей), плохо предсказумая производительность
\end{itemize}
\pause
\item Shadow mapping
\pause
\begin{itemize}
\item {\color{green}+} Производительность растёт +/- линейно с ростом сложности сцены
\item {\color{red}---} Крупный алиасинг ("пиксельные" тени)
\item {\color{green}+} Много вариаций, улучшающих качество и позволяющих делать как жёсткие, так и мягкие тени
\end{itemize}
\end{itemize}
\end{frame}

\end{document}