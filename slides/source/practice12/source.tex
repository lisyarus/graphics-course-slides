% (c) Nikita Lisitsa, lisyarus@gmail.com, 2021

\documentclass{beamer}

\usepackage[T2A]{fontenc}
\usepackage[utf8]{inputenc}
\usepackage[russian]{babel}

\usepackage{graphicx}
\graphicspath{ {./images/} }

\usepackage{adjustbox}

\usepackage{color}
\usepackage{soul}

\usepackage{hyperref}

\definecolor{blue}{rgb}{0,0,1}
\definecolor{red}{rgb}{1,0,0}

\makeatletter
\newcommand{\slideimage}[1]{
  \begin{figure}
    \begin{adjustbox}{width=\textwidth, totalheight=\textheight-2\baselineskip-2\baselineskip,keepaspectratio}
      \includegraphics{#1}
    \end{adjustbox}
  \end{figure}
}
\makeatother

\title{Компьютерная графика}
\subtitle{Практика 12: Объёмный рендеринг}
\date{2021}

\setbeamertemplate{footline}[frame number]

\begin{document}

\frame{\titlepage}

\begin{frame}[fragile]
\frametitle{Задание 1}
Находим пересечение с кубом
\begin{itemize}
\item Во фрагментном шейдере заводим координаты крайних точек куба (\verb|vec3 box_min(-1.0), box_max(1.0)|) - можно как uniform'ы, можно как const переменные
\pause
\item Вычисляем \textbf{нормированный} вектор направления из камеры в текущую точку поверхности куба - это вектор направления луча
\pause
\item Вычисляем пересечение этого луча с кубом: интервал \begin{math}[t_{min}, t_{max}]\end{math} для которых \begin{math}p + t \cdot d\end{math} пересекает куб
\pause
\item Делаем \verb|tmin >= 0|, чтобы не включать часть пересечения сзади камеры
\pause
\item В качестве цвета пикселя выводим \verb|tmax - tmin| (это значение часто будет больше единицы, так что можно его нормировать, разделив на 3.5)
\end{itemize}
\end{frame}

\begin{frame}[fragile]
\frametitle{Задание 2}
Загружаем 3D текстуру
\begin{itemize}
\item Создаём текстуру типа \verb|GL_TEXTURE_3D|
\pause
\item Устанавливаем min фильтр в \verb|GL_LINEAR|, mag в \verb|GL_LINEAR_MIPMAP_NEAREST|
\pause
\item Устанавливаем wrap\_r, wrap\_s, wrap\_t в \verb|GL_CLAMP_TO_EDGE|
\pause
\item Считываем данные из файла \verb|house256| (256x256x256, 4 байта на пиксель) и загружаем в текстуру (\verb|glTexImage3D|)
\begin{itemize}
\item Если GPU не потянет, можно взять файл поменьше - \verb|house128| или \verb|house64| (128x128x128 и 64x64x64 соответственно)
\end{itemize}
\pause
\item Добавляем текстуру в шейдер (\verb|sampler3D|), выводим в качестве цвета значение из текстуры в точке \verb|position|
\begin{itemize}
\item Нужно перевести пространственные координаты в текстурные: \verb|(position - box_min) / box_size|
\item В качестве альфа-канала возьмём 1, иначе ничего не увидим
\end{itemize}
\end{itemize}
\end{frame}

\begin{frame}[fragile]
\frametitle{Задание 3}
Вычисляем среднее значение цвета вдоль луча
\begin{itemize}
\item В вершинном шейдере делаем цикл (например, в 64 шага); шаг цикла соответствует 1/64 части отрезка \begin{math}[t_{min}, t_{max}]\end{math}, т.е. \begin{math}[t_{min} + \frac{\Delta t \cdot i}{64}, t_{min} + \frac{\Delta t \cdot (i + 1)}{64}]\end{math}, где \begin{math}\Delta t = t_{max} - t_{min}\end{math}
\begin{itemize}
\item Каждому значению \begin{math}t\in[t_{min}, t_{max}]\end{math} соответствует точка \begin{math}p + t \cdot d\end{math}
\item Вместо 64 можно взять любое другое число; чем больше, тем красивее и медленнее
\end{itemize}
\pause
\item Вычисляем цвет из текстуры в текущей точке (лучше взять середину отрезка, т.е. точку соответствующую \begin{math}\frac{i + 0.5}{64}\end{math}), и переводим в premultiplied формат (RGB-часть нужно домножить на альфа-канал)
\begin{itemize}
\item Лучше читать 0-ой mipmap-уровень (\verb|textureLod(sampler, texcoord, 0.0)|)
\end{itemize}
\pause
\item Суммируем усреднённое значение цвета по всем точкам (цвет в точке нужно домножить на длину одного отрезка, т.е. на \begin{math}\frac{\Delta t}{64}\end{math})
\pause
\item Усреднённый цвет выводим в качестве цвета
\end{itemize}
\end{frame}

\begin{frame}[fragile]
\frametitle{Задание 4}
Вычисляем настоящую интенсивность цвета используя front-to-back проход
\begin{itemize}
\item Заводим коэффициент интенсивности \verb|C = 16.0| (чем больше, тем менее прозрачной будет сцена) - он используется для перевода значений альфа-канала в плотность среды
\pause
\item Вычисляем реальную плотность вдоль \verb|i|-ого отрезка как \begin{math}1 - \exp(- C \frac{\Delta t}{64} A)\end{math}, где \begin{math}A\end{math} - значение альфа-канала, и перезаписываем в альфа-канал текущего (прочитанного из текстуры) цвета
\pause
\item Переводим цвет в premultiplied формат (домножаем RGB-часть на альфа-канал, который мы перезаписали в предыдущем пункте)
\pause
\item Прибавляем к результату текущий цвет, умноженный на \verb|1 - result.a| (чтобы учесть прозрачность уже посчитанного цвета)
\pause
\item После суммирования применяем к RGB-части результирующего пикселя гамма-коррекцию
\end{itemize}
\end{frame}

\begin{frame}[fragile]
\frametitle{Задание 5}
Добавляем тени
\begin{itemize}
\item Для каждого цвета, прочитанного из текстуры (т.е. на каждой итерации цикла) вычисляем его освещённость (число в диапазоне \begin{math}[0, 1]\end{math})
\pause
\item Чтобы вычислить освещённость, пускаем луч из текущей точки в направлении источника света и пересекаем с кубом (делаем \verb|tmin >= 0|, иначе тени будут неверными)
\pause
\item Вдоль этого луча запускаем внутренний цикл, аналогичный внешнему, но нас интересует только альфа-канал результата; освещённость - это \verb|1 - A|
\begin{itemize}
\item Число итераций лучше взять не очень большим, например, 8 или 16
\item Можно упростить вычисления, просто просуммировав \begin{math}C \cdot A \cdot \frac{\Delta t}{16}\end{math}, и посчитав освещённость как \verb|exp(-sum)|
\item Так как итераций мало, лучше читать какой-нибудь низкодетализированный mipmap текстуры (например, 4ый)
\end{itemize}
\pause
\item Домножаем RGB-часть цвета текущей точки (во внешнем цикле) на значение освещённости
\end{itemize}
\end{frame}

\end{document}