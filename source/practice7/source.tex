% (c) Nikita Lisitsa, lisyarus@gmail.com, 2025

\documentclass[10pt]{beamer}

\usepackage[T2A]{fontenc}
\usepackage[russian]{babel}
\usepackage{minted}

\usepackage{graphicx}
\graphicspath{ {./images/} }

\usetheme{metropolis}

\usepackage{adjustbox}

\usepackage{color}
\usepackage{soul}

\usepackage{hyperref}

\definecolor{blue}{rgb}{0,0,1}
\definecolor{red}{rgb}{1,0,0}
\definecolor{codebg}{RGB}{29,35,49}

\makeatletter
\newcommand{\slideimage}[1]{
  \begin{figure}
    \begin{adjustbox}{width=\textwidth, totalheight=\textheight-2\baselineskip-2\baselineskip,keepaspectratio}
      \includegraphics{#1}
    \end{adjustbox}
  \end{figure}
}
\makeatother

\title{Компьютерная графика}
\subtitle{Практика 7: Освещение}
\date{2025}

\usemintedstyle{solarized-light}

\setbeamertemplate{footline}[frame number]

\begin{document}

\frame{\titlepage}

\begin{frame}[fragile]
\slideimage{0.png}
\end{frame}

\begin{frame}[fragile]
\frametitle{Задание 1}
Добавляем ambient освещение
\begin{itemize}
\item Во фрагментном шейдере заводим две новые uniform-переменные: \mintinline{glsl}|vec3 albedo| и \mintinline{glsl}|vec3 ambient_light|
\item Перед рисованием устанавливаем их значения:
\begin{itemize}
\item Ambient light -- что-нибудь маленькое, например \mintinline{glsl}|(0.2, 0.2, 0.2)|
\item Albedo -- цвет объета, выберите сами
\end{itemize}
\item Во фрагментном шейдере делаем \mintinline{glsl}|color = albedo * ambient_light|
\end{itemize}
\end{frame}

\begin{frame}[fragile]
\slideimage{1.png}
\end{frame}

\begin{frame}[fragile]
\frametitle{Задание 2}
Добавляем диффузное освещение от направленного источника
\begin{itemize}
\item Во фрагментном шейдере заводим две новые uniform-переменные: \mintinline{glsl}|vec3 sun_direction| и \mintinline{glsl}|vec3 sun_color|
\item Перед рисованием устанавливаем их значения:
\begin{itemize}
\item Sun direction -- какой-нибудь единичный вектор, направленный примерно вверх и вперёд
\item Sun color -- цвет солнца, например \mintinline{glsl}|(1.0, 0.9, 0.8)|
\end{itemize}
\item Во фрагментном шейдере заводим функцию \mintinline{glsl}|vec3 diffuse(vec3 direction)|, которая вычисляет силу диффузного освещения от источника света в данном направлении:
\usemintedstyle{lightbulb}
\begin{minted}[bgcolor=codebg]{glsl}
return albedo * max(0.0, dot(normal, direction));
\end{minted}
\usemintedstyle{solarized-light}
\item Используем её для вычисления освещения \mintinline{glsl}|diffuse(sun_direction) * sun_color| и прибавляем это к результирующему цвету
\end{itemize}
\end{frame}

\begin{frame}[fragile]
\slideimage{2.png}
\end{frame}

\begin{frame}[fragile]
\frametitle{Задание 3}
Добавляем точечный источник света
\begin{itemize}
\item Во фрагментном шейдере заводим три новые uniform-переменные: \mintinline{glsl}|vec3 point_light_position|, \mintinline{glsl}|vec3 point_light_color| и \mintinline{glsl}|vec3 point_light_attenuation|
\item Перед рисованием устанавливаем их значения:
\begin{itemize}
\item Position -- позиция, меняющаяся со временем
\item Color -- цвет источника, выберите сами
\item Attenuation -- коэффициенты затухания \begin{math}C_0, C_1, C_2\end{math}, например \mintinline{glsl}|(1.0, 0.0, 0.01)|
\end{itemize}
\item Во фрагментном шейдере находим расстояние до источника света и вектор направления на него
\item Используем функцию \mintinline{glsl}|diffuse| для вычисления освещения аналогично солнцу, домножаем на коэффициент затухания и прибавляем это к результирующему цвету
\item N.B. к этому моменту \mintinline{glsl}|color| будет суммой трёх составляющих: ambient, солнце и точечный источник
\end{itemize}
\end{frame}

\begin{frame}[fragile]
\slideimage{3.png}
\end{frame}

\begin{frame}[fragile]
\frametitle{Задание 4}
Добавляем specular освещение
\begin{itemize}
\item Во фрагментном шейдере заводим две новые uniform-переменные: \mintinline{glsl}|float glossiness| и \mintinline{glsl}|float roughness|
\item Перед рисованием устанавливаем их значения:
\begin{itemize}
\item Glossiness -- сила отражения, например \mintinline{glsl}|5.0|
\item Roughness -- `шершавость' поверхности, например \mintinline{glsl}|0.1| (гладкая поверхность)
\end{itemize}
\item Во фрагментном шейдере добавляем функцию \mintinline{glsl}|vec3 specular(vec3 direction)|, которая вычисляет specular составляющую (см. слайды лекции)
\usemintedstyle{lightbulb}
\begin{minted}[bgcolor=codebg]{glsl}
return glossiness * albedo * pow(max(0.0,
    dot(reflected, view_direction)), power);
\end{minted}
\usemintedstyle{solarized-light}
\item Прибавляем specular составляющую к вкладам обоих источников света, т.е. заменяем \mintinline{glsl}|diffuse(direction)| на \mintinline{glsl}|diffuse(direction) + specular(direction)|
\end{itemize}
\end{frame}

\begin{frame}[fragile]
\slideimage{4.png}
\end{frame}

\begin{frame}[fragile]
\frametitle{Задание 5}
Играем с прозрачностью
\begin{itemize}
\item Если переменная \mintinline{glsl}|transparent| (она уже есть в коде, значение меняется по пробелу) имеет значение \mintinline{glsl}|true|, включаем прозрачность:
\begin{itemize}
\item \mintinline{cpp}|glEnable(GL_BLEND)|
\item \mintinline{cpp}|glBlendEquation|
\item \mintinline{cpp}|glBlendFunc|
\end{itemize}
\item Иначе выключаем прозрачность \mintinline{cpp}|glDisable(GL_BLEND)|
\item Не забудем записать в альфа-канал во фрагментном шейдере значение, отличное от 1 (например, 0.5)
\end{itemize}
\end{frame}

\begin{frame}[fragile]
\slideimage{5.png}
\end{frame}

\begin{frame}[fragile]
\frametitle{Задание 6*}
Рисуем несколько моделей с разными параметрами материала
\begin{itemize}
\item Рисуем модель девять раз квадратом 3x3 (в плоскости XY)
\item У моделей должны отличаться параметры glossiness и roughness:
\begin{itemize}
\item Roughness принимает значения примерно от 0.1 до 0.5
\item Значения glossiness подберите сами
\end{itemize}
\end{itemize}
\end{frame}

\begin{frame}[fragile]
\slideimage{6.png}
\end{frame}

\end{document}