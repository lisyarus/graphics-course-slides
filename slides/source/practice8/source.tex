% (c) Nikita Lisitsa, lisyarus@gmail.com, 2021

\documentclass{beamer}

\usepackage[T2A]{fontenc}
\usepackage[utf8]{inputenc}
\usepackage[russian]{babel}

\usepackage{graphicx}
\graphicspath{ {./images/} }

\usepackage{adjustbox}

\usepackage{color}
\usepackage{soul}

\usepackage{hyperref}

\definecolor{blue}{rgb}{0,0,1}
\definecolor{red}{rgb}{1,0,0}

\makeatletter
\newcommand{\slideimage}[1]{
  \begin{figure}
    \begin{adjustbox}{width=\textwidth, totalheight=\textheight-2\baselineskip-2\baselineskip,keepaspectratio}
      \includegraphics{#1}
    \end{adjustbox}
  \end{figure}
}
\makeatother

\title{Компьютерная графика}
\subtitle{Практика 8: Shadow mapping}
\date{2021}

\setbeamertemplate{footline}[frame number]

\begin{document}

\frame{\titlepage}

\begin{frame}[fragile]
\frametitle{Задание 1}
\fontsize{10pt}{10pt}
Создаём и настраиваем shadow map и framebuffer
\begin{itemize}
\item Выбираем размер shadow map: например, \verb|shadow_map_res = 1024|
\pause
\item Создаём текстуру для shadow map: min/mag фильтры - \verb|GL_NEAREST|, размеры - \verb|shadow_map_res| на \verb|shadow_map_res|, internal format - \verb|GL_DEPTH_COMPONENT24|, format - \verb|GL_DEPTH_COMPONENT|, type - \verb|GL_FLOAT|, в данных - \verb|nullptr|
\item Настраиваем ей параметры \verb|GL_TEXTURE_WRAP_S| и \verb|GL_TEXTURE_WRAP_T| в значение \verb|GL_CLAMP_TO_EDGE|
\pause
\item Создаём framebuffer
\pause
\item Присоединяем к нему нашу текстуру в качестве глубины (\verb|GL_DEPTH_ATTACHMENT|)
\pause
\item Проверяем, что фреймбуффер настроен правильно (\verb|glCheckFramebufferStatus|)
\end{itemize}
\end{frame}

\begin{frame}[fragile]
\frametitle{Задание 2}
\fontsize{10pt}{10pt}
Добавляем дебажную шейдерную программу, чтобы видеть содержимое нашей shadow map
\begin{itemize}
\item Вершинный шейдер: выдаёт захардкоженные координаты вершин, используя \verb|gl_VertexID|, и передаёт во фрагментный шейдер текстурные координаты
\begin{itemize}
\item Должно быть 6 вершин - два треугольника, образующих прямоугольник
\item Координаты вершин должны быть где-то в нижнем левом углу экрана (например, [-1.0 .. -0.75] по обеим осям)
\item Текстурные координаты должны быть [0.0 .. 1.0]
\end{itemize}
\pause
\item Фрагментный шейдер: читает цвет из переданной текстуры и выводит в \verb|out_color|
\pause
\item Создаём фиктивный VAO (без настройки атрибутов вершин)
\pause
\item Рисуем с помощью \verb|glDrawArrays(GL_TRIANGLES, 0, 6)| (не забываем сделать текущими VAO, программу и текстуру shadow map)
\end{itemize}
\end{frame}

\begin{frame}[fragile]
\frametitle{Задание 3}
\fontsize{10pt}{10pt}
Генерируем shadow map
\begin{itemize}
\item Выбираем проекцию для shadow map: для начала сгодится проекция "снизу-вверх" (как будто камера смотрит сверху)
\pause
\begin{itemize}
\item \verb|light_Z = glm::vec3(0, -1, 0)|
\item \verb|light_X = glm::vec3(1,  0, 0)|
\item \verb|light_Y = glm::cross(light_X, light_Z)|
\pause
\item \verb|light_X, light_Y, light_Z| - строки матрицы проекции (N.B. в GLM \verb|m[i][j]| это \verb|i|-ый \textbf{столбец} и \verb|j|-ая \textbf{строка})
\end{itemize}
\pause
\item Пишем шейдерную программу:
\pause
\begin{itemize}
\item Вершинный шейдер преобразует вершины (\verb|gl_Position = shadow_transform * model * ...|)
\item Фрагментный шейдер ничего не делает (пустая функция \verb|main|)
\end{itemize}
\pause
\item Перед рисованием основного кадра: используем shadow framebuffer для рисования, настраиваем \verb|glViewport|, очищаем буфер глубины, включаем front-face culling, рисуем нашу модель созданной шейдерной программой
\pause
\item Не забываем вернуть дефолтный framebuffer для рисования и вернуть \verb|glViewport|
\pause
\item Модель должна появиться в нашем дебажном прямоугольнике
\end{itemize}
\end{frame}

\begin{frame}[fragile]
\frametitle{Задание 4}
\fontsize{10pt}{10pt}
Используем shadow map
\begin{itemize}
\item Передаём текстуру shadow map и проекцию для неё в основную шейдерную программу
\pause
\item Во фрагментном шейдере:
\begin{itemize}
\item Проверяем текущий пиксель на попадание в видимую область shadow map (все координаты \verb|shadow_pos| после всех преобразований должны быть в диапазоне [0..1])
\item Сравниваем значение из shadow map с Z-координатой \verb|shadow_pos|
\item Если пиксель внутри видимой области и его Z-координата больше считанной из shadow map, он в тени (к нему не нужно применять прямое освещение, но ambient остаётся)
\end{itemize}
\end{itemize}
\end{frame}

\begin{frame}[fragile]
\frametitle{Задание 5}
\fontsize{10pt}{10pt}
Вычисляем настоящую проекцию
\begin{itemize}
\item \verb|light_Z = -light_direction|
\item \verb|light_X| - любой вектор, ортогональный \verb|light_Z|
\item \verb|light_Y = glm::cross(light_X, light_Z)|
\end{itemize}
\end{frame}

\end{document}