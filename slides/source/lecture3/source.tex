% (c) Nikita Lisitsa, lisyarus@gmail.com, 2021

\documentclass{beamer}

\usepackage[T2A]{fontenc}
\usepackage[utf8]{inputenc}
\usepackage[russian]{babel}

\usepackage{graphicx}
\graphicspath{ {./images/} }

\usepackage{adjustbox}

\usepackage{color}
\usepackage{soul}

\usepackage{hyperref}

\definecolor{blue}{rgb}{0,0,1}
\definecolor{red}{rgb}{1,0,0}

\makeatletter
\newcommand{\slideimage}[1]{
  \begin{figure}
    \begin{adjustbox}{width=\textwidth, totalheight=\textheight-2\baselineskip-2\baselineskip,keepaspectratio}
      \includegraphics{#1}
    \end{adjustbox}
  \end{figure}
}
\makeatother

\title{Компьютерная графика}
\subtitle{Лекция 3: Объекты OpenGL, буферы, аттрибуты вершин, индексированный рендеринг}
\date{2021}

\setbeamertemplate{footline}[frame number]

\begin{document}

\frame{\titlepage}

\begin{frame}[fragile]
\frametitle{Объекты OpenGL}
\begin{itemize}
\item Шейдеры, шейдерные программы - программируемая часть конвейера
\pause
\item Буферы - хранят данные на GPU
\pause
\item Vertex Array - описывают аттрибуты вершин
\pause
\item Текстуры - изображения, которые можно читать из шейдера, и в которые можно рисовать
\pause
\item Renderbuffer - буферы, в которые можно рисовать
\pause
\item Framebuffer - содержат настройки рисования в текстуры и renderbuffer'ы
\end{itemize}
\end{frame}

\begin{frame}[fragile]
\frametitle{Создание/удаление объектов OpenGL}
\begin{itemize}
\item Объект представляется идентификатором типа \verb|GLuint|
\begin{itemize}
\item Id уникален среди объектов одного типа (шейдеры, программы, ...)
\end{itemize}
\pause
\item Шейдеры и программы:
\begin{itemize}
\item \verb|glCreateShader()|/\verb|glDeleteShader(shader)|
\item \verb|glCreateProgram()|/\verb|glDeleteProgram(program)|
\end{itemize}
\pause
\item Остальные объекты:
\begin{verbatim}
glGenBuffers(count, ptr)/glDeleteBuffers(count, ptr)
glGenVertexArrays/glDeleteVertexArrays
glGenTextures/glDeleteTextures
glGenRenderbuffers/glDeleteRenderbuffers
glGenFramebuffers/glDeleteFramebuffers
\end{verbatim}
\pause
\begin{itemize}
\item Можно создать/удалить один объект:
\begin{verbatim}
GLuint texture;
glGenTextures(1, &texture);
...
glDeleteTexture(1, &texture);
\end{verbatim}
\end{itemize}
\end{itemize}
\end{frame}

\begin{frame}[fragile]
\frametitle{Объекты OpenGL}
\begin{itemize}
\item Подразумевается, что объекты переиспользуются по максимуму
\item Не нужно создавать новую текстуру каждый кадр - создайте один раз и переиспользуйте её
\item Не нужно создавать новый буфер при каждом обновлении данных - создайте один раз и переиспользуйте его
\end{itemize}
\end{frame}

\begin{frame}[fragile]
\frametitle{Объекты OpenGL с нулевым id}
\begin{itemize}
\item Как правило, объект с нулевым id считается несуществующим (как нулевой указатель)
\pause
\item Исключения:
\begin{itemize}
\item Framebuffer с нулевым id - default framebuffer, рисует в окно, привязанное к контексту OpenGL, его нельзя менять
\pause
\item В OpenGL ES: Vertex Array с нулевым id - ничем не отличается от других vertex array'ев, но существует (создан) по умолчанию
\end{itemize}
\pause
\item Функции создания объектов никогда не возвращают нулевой id
\item Функции удаления объектов игнорируют нулевой id
\end{itemize}
\end{frame}

\begin{frame}[fragile]
\frametitle{Работа с объектами OpenGL}
\begin{itemize}
\item Почти всегда чтобы работать с объектом, нужно сделать его "текущим"
\begin{itemize}
\item Текущий объект запоминается в контексте OpenGL
\item Если вы работаете с одним контекстом, можно считать, что id текущего объекта - глобальная константа
\end{itemize}
\pause
\item Некоторые функции не требуют выставления объекта текущим: \verb|glShaderSource|, \verb|glCompileShader|, \verb|glLinkProgram|, ...
\pause
\item Некоторые объекты нельзя сделать текущими - шейдеры
\end{itemize}
\end{frame}

\begin{frame}[fragile]
\frametitle{Работа с объектами OpenGL}
\begin{itemize}
\item Сделать программу текущей: \verb|glUseProgram(program)|
\begin{itemize}
\item Функции \verb|glGetUniformLocation|, \verb|glUniform1f|, ... работают с текущей программой
\end{itemize}
\pause
\item Сделать vertex array текущим: \verb|glBindVertexArray(vao)|
\begin{itemize}
\item Функции работы с vertex array используют текущий vertex array
\end{itemize}
\pause
\item Функции рисования (\verb|glDrawArrays|) используют текущую шейдерную программу и текущий vertex array
\end{itemize}
\end{frame}

\begin{frame}[fragile]
\frametitle{Работа с объектами OpenGL}
\begin{itemize}
\item Для буферов, текстур, framebuffer'ов и renderbuffer'ов нет одного текущего объекта, но есть текущий объект для конкретного target'а
\pause
\item Можно считать, что есть словарь \verb|Target -> Id| текущих объектов
\pause
\item Для каждого вида объектов (буферы, текстуры, ...) есть отдельный словарь и отдельный набор возможных значений Target
\pause
\item Смысл и особенности разных значений Target зависят от вида объекта
\pause
\item \verb|glBindBuffer(target, id)|
\item \verb|glBindTexture(target, id)|
\item \verb|glBindRenderbuffer(target, id)|
\item \verb|glBindFramebuffer(target, id)|
\end{itemize}
\end{frame}

\begin{frame}[fragile]
\frametitle{Буферы}
\begin{itemize}
\item Могут хранить произвольные данные на GPU
\pause
\item \verb|glGenBuffers|/\verb|glDeleteBuffers|
\pause
\item Возможные значения target для \verb|glBindBuffer|:
\begin{itemize}
\item \verb|GL_ARRAY_BUFFER| (VBO) - массив вершин
\pause
\item \verb|GL_ELEMENT_ARRAY_BUFFER| (EBO) - массив индексов вершин
\pause
\item \verb|GL_UNIFORM_BUFFER| (UBO) - массив значений uniform-переменных
\pause
\item ...и другие
\end{itemize}
\pause
\item Текущий \verb|GL_ELEMENT_ARRAY_BUFFER| хранится не глобально, а в текущем VAO
\end{itemize}
\end{frame}

\begin{frame}[fragile]
\frametitle{Буферы: запись данных}
\begin{itemize}
\item Выделить память на GPU и загрузить данные:
\begin{verbatim}
glBufferData(GLenum target, GLsizeiptr size,
    const GLvoid * data, GLenum usage)
\end{verbatim}
\pause
\begin{itemize}
\item \verb|target| - \verb|GL_ARRAY_BUFFER| и т.п.
\item \verb|size| - размер данных в байтах
\item \verb|data| - указатель на данные
\item \verb|usage| - подсказка драйверу о том, как данные будут использоваться
\end{itemize}
\pause
\item Если \verb|data = nullptr|, буфер будет выделен, но данные будут не инициализированы
\pause
\item Если буфер уже содержал данные, они заменяются новыми (и происходит реаллокация памяти)
\pause
\item После вызова \verb|glBufferData| с данными по указателю \verb|data| можно делать всё, что угодно (в т.ч. удалить)
\item Копирование данных в память GPU тоже происходит асинхронно
\pause
\begin{itemize}
\item \begin{math}\Rightarrow\end{math} Драйвер, скорее всего, сначала копирует данные в собственную память
\end{itemize}
\end{itemize}
\end{frame}

\begin{frame}[fragile]
\frametitle{Буферы: параметр usage}
\begin{table}
\centering
\begin{tabular}{ c c c }
 \verb|GL_STATIC_DRAW| & \verb|GL_STATIC_READ| & \verb|GL_STATIC_COPY| \\ 
 \verb|GL_DYNAMIC_DRAW| & \verb|GL_DYNAMIC_READ| & \verb|GL_DYNAMIC_COPY| \\
 \verb|GL_STREAM_DRAW| & \verb|GL_STREAM_READ| & \verb|GL_STREAM_COPY|
\end{tabular}
\end{table}
\pause
\begin{itemize}
\item Данные будут обновляться
\begin{itemize}
\item \verb|STATIC| - один раз
\item \verb|DYNAMIC| - иногда
\item \verb|STREAM| - почти каждый кадр
\end{itemize}
\pause
\item Буфер будет использоваться для:
\begin{itemize}
\item \verb|DRAW| - записи данных в него
\item \verb|READ| - чтения данных из него
\item \verb|COPY| - и записи, и чтения
\end{itemize}
\pause
\item Это только подсказка драйверу и не влияет на корректность работы
\end{itemize}
\end{frame}

\begin{frame}[fragile]
\frametitle{Буферы: запись и чтение данных}
\begin{itemize}
\item Загрузить данные в часть буфера:
\begin{verbatim}
glBufferSubData(GLenum target, GLintptr offset,
    GLsizeiptr size, const GLvoid * data)
\end{verbatim}
\pause
\begin{itemize}
\item Гарантированно не реаллоцирует память GPU
\item Память уже должна быть выделена (\verb|glBufferData|)
\end{itemize}
\pause
\item Прочитать данные из буфера:
\begin{verbatim}
glGetBufferSubData(GLenum target, GLintptr offset,
    GLsizeiptr size, GLvoid * data)
\end{verbatim}
\pause
\begin{itemize}
\item К моменту выхода из этой функции данные уже прочитаны
\item \begin{math}\Rightarrow\end{math} Синхронная функция, блокирующая исполнение
\end{itemize}
\end{itemize}
\end{frame}

\begin{frame}[fragile]
\frametitle{Mapped buffer}
\begin{itemize}
\item Можно получить виртуальный указатель на буфер или его часть и пользоваться им для чтения/записи
\pause
\item \verb|glMapBuffer(target, access)| - возвращает mapped указатель
\item \verb|access| может принимать значения
\begin{itemize}
\item \verb|GL_READ_ONLY| - по указателю можно читать
\item \verb|GL_WRITE_ONLY| - по указателю можно писать
\item \verb|GL_READ_WRITE| - по указателю можно читать и писать
\end{itemize}
\pause
\item \verb|glUnmapBuffer(target)|
\pause
\item Между \verb|glMapBuffer| и \verb|glUnmapBuffer| работать с буфером (загружать данные другими методами, использовать данные для рисования) нельзя
\pause
\item После \verb|glUnmapBuffer| mapped указатель использовать нельзя
\end{itemize}
\end{frame}

\begin{frame}[fragile]
\frametitle{Буферы: типичный пример использования}
\begin{verbatim}
GLuint vbo;
glGenBuffers(1, &vbo);

std::vector<vertex> vertices;
...

glBindBuffer(GL_ARRAY_BUFFER, vbo);
glBufferData(GL_ARRAY_BUFFER,
    vertices.size() * sizeof(vertices[0]),
    vertices.data(), GL_STATIC_DRAW);
\end{verbatim}
\end{frame}

\begin{frame}[fragile]
\frametitle{Буферы: ссылки}
\begin{itemize}
\item \href{https://www.khronos.org/opengl/wiki/Buffer_Object}{khronos.org/opengl/wiki/Buffer\_Object}
\item \href{https://www.songho.ca/opengl/gl_vbo.html}{songho.ca/opengl/gl\_vbo.html}
\end{itemize}
\end{frame}

\begin{frame}[fragile]
\frametitle{Аттрибуты вершин: шейдер}
\begin{verbatim}
#version 330 core

layout (location = 0) in vec3 position;
layout (location = 1) in vec3 normal;
layout (location = 2) in vec4 color;
layout (location = 3) in int materialID;

void main() {
    ...
}
\end{verbatim}
\end{frame}

\begin{frame}[fragile]
\frametitle{Аттрибуты вершин}
\begin{itemize}
\item Индекс: \verb|0, 1, ... glGet(GL_MAX_VERTEX_ATTRIBS)|
\begin{itemize}
\item \verb|glGet(GL_MAX_VERTEX_ATTRIBS)| \begin{math}\geq\end{math} 16
\end{itemize}
\pause
\item Включен/выключен
\begin{itemize}
\item \verb|glEnableVertexAttribArray(index)| - включить аттрибут
\item Выключены по умолчанию
\item Хранится в vertex array
\end{itemize}
\end{itemize}
\end{frame}

\begin{frame}[fragile]
\frametitle{Аттрибуты вершин: хранение}
Параметры хранения данных аттрибута:
\begin{itemize}
\pause
\item Формат:
\begin{itemize}
\item Количество компонент - 1, 2, 3, 4
\pause
\item Тип - \verb|GL_BYTE|, \verb|GL_UNSIGNED_BYTE|, \verb|GL_SHORT|, \verb|GL_UNSIGNED_SHORT|, \verb|GL_INT|, \verb|GL_UNSIGNED_INT|, \verb|GL_HALF_FLOAT|, \verb|GL_FLOAT|, \verb|GL_DOUBLE|, \verb|GL_INT_2_10_10_10_REV|, \verb|GL_UNSIGNED_INT_2_10_10_10_REV|
\pause
\item Нормированный / не нормированный
\end{itemize}
\pause
\item Расположение данных:
\begin{itemize}
\item Адрес начала данных (на GPU)
\item Расстояние (в байтах) между значениями этого аттрибута, соответствующими соседним вершинам
\end{itemize}
\pause
\item Всё это запоминается в vertex array
\end{itemize}
\end{frame}

\begin{frame}[fragile]
\frametitle{Аттрибуты вершин: нормированность}
\begin{itemize}
\item Как интерпретировать значение аттрибута, который в шейдере объявлен как floating-point (\verb|float|, \verb|vec3|, и т.п.), но в данных он хранится как целочисленный?
\pause
\item 2 варианта:
\begin{itemize}
\item Не нормированный: передавать, как есть, делая преобразование \verb|int -> float|
\pause
\item Нормированный: преобразовывать диапазон целочисленных значений в \begin{math}[-1, 1]\end{math} (для знаковых) или \begin{math}[0, 1]\end{math} (для беззнаковых)
\verb|unsigned short: x -> x / 65535|
\verb|short: x -> (2 x + 1) / 65535|
\end{itemize}
\pause
\item Полезно для передачи цвета: \verb|0 .. 255| превращаются в стандартные для OpenGL \verb|0.0 .. 1.0|
\pause
\item Полезно для компактного хранения аттрибутов
\pause
\item Имеет смысл только для floating-point аттрибутов
\end{itemize}
\end{frame}

\begin{frame}[fragile]
\frametitle{Аттрибуты вершин: floating-point}
\begin{verbatim}
glVertexAttribPointer(GLuint index, GLint size,
    GLenum type, GLboolean normalized,
    GLsizei stride, const GLvoid * pointer)
\end{verbatim}
\pause
\begin{itemize}
\item \verb|index| - индекс аттрибута
\pause
\item \verb|size| - размерность (число компонент)
\pause
\item \verb|type| - тип компонент
\pause
\item \verb|normalized| - нормированный или нет
\pause
\item \verb|stride| - расстояние между соседними значениями
\pause
\item \verb|pointer| - указатель на данные
\pause
\begin{itemize}
\item На самом деле, \verb|pointer| - сдвиг относительно начала памяти текущего \verb|GL_ARRAY_BUFFER|
\item Например, если данные этого аттрибута начинаются на 12ом байте текущего \verb|GL_ARRAY_BUFFER|, нужно передать \verb|(const void *)(12)|
\pause
\item Разные аттрибуты могут лежать в разных буферах
\end{itemize}
\end{itemize}
\end{frame}

\begin{frame}[fragile]
\frametitle{Аттрибуты вершин: целочисленные}
\begin{verbatim}
glVertexAttribIPointer(GLuint index, GLint size,
    GLenum type,
    GLsizei stride, const GLvoid * pointer)
\end{verbatim}
\begin{itemize}
\item Всё то же самое, но 
\begin{itemize}
\item Нет параметра \verb|normalized|
\item \verb|type| может быть только целочисленным
\end{itemize}
\end{itemize}
\end{frame}

\begin{frame}[fragile]
\frametitle{Аттрибуты вершин: stride}
\begin{itemize}
\item Пусть \verb|start| - смещение, указанное в параметре \verb|pointer| функции \verb|glVertexAttribPointer|
\item Тогда соответстующий аттрибут вершины с номером \verb|i| будет взят по смещению \verb|start + stride * i|
\pause
\begin{itemize}
\item Нулевая вершина: \verb|start|
\item Первая вершина: \verb|start + stride|
\item Вторая вершина: \verb|start + 2 * stride|
\item И т.д.
\end{itemize}
\pause
\item Если \verb|stride = 0|, \verb|stride| считается равным размеру аттрибута \verb|stride = size * sizeof(type)|
\begin{itemize}
\item Например, для \verb|size = 3| и \verb|type = GL_UNSIGNED_INT|, \verb|stride| будет вычислен как \verb|stride = 3 * sizeof(unsigned int) = 3 * 4 = 12|
\end{itemize}
\pause
\item Нужен для гибкости хранения аттрибутов
\end{itemize}
\end{frame}

\begin{frame}[fragile]
\frametitle{Аттрибуты вершин: пример array-of-structs, один буфер}
\fontsize{10pt}{10pt}
\begin{verbatim}
struct vertex {
    float position[3];
    float normal[3];
    unsigned char color[4];
};
vertex vertices[N];

glBindBuffer(GL_ARRAY_BUFFER, vbo);

glEnableVertexAttribArray(0);
glVertexAttribPointer(0, 3, GL_FLOAT, GL_FALSE,
    sizeof(vertex), (void*)(0));

glEnableVertexAttribArray(1);
glVertexAttribPointer(1, 3, GL_FLOAT, GL_FALSE,
    sizeof(vertex), (void*)(12));

glEnableVertexAttribArray(2);
glVertexAttribPointer(2, 4, GL_UNSIGNED_BYTE, GL_TRUE,
    sizeof(vertex), (void*)(24));
\end{verbatim}
\end{frame}

\begin{frame}[fragile]
\frametitle{Аттрибуты вершин: пример struct-of-arrays, один буфер}
\fontsize{10pt}{10pt}
\begin{verbatim}
float positions[3 * N];
float normal[3 * N];
unsigned char color[4 * N];

glBindBuffer(GL_ARRAY_BUFFER, vbo);

glEnableVertexAttribArray(0);
glVertexAttribPointer(0, 3, GL_FLOAT, GL_FALSE,
    0, (void*)(0));

glEnableVertexAttribArray(1);
glVertexAttribPointer(1, 3, GL_FLOAT, GL_FALSE,
    0, (void*)(12 * N));
    
glEnableVertexAttribArray(2);
glVertexAttribPointer(2, 4, GL_UNSIGNED_BYTE, GL_TRUE,
    0, (void*)(24 * N));
\end{verbatim}
\end{frame}

\begin{frame}[fragile]
\frametitle{Аттрибуты вершин: пример struct-of-arrays, три буфера}
\fontsize{10pt}{10pt}
\begin{verbatim}
float positions[3 * N];
float normal[3 * N];
unsigned char color[4 * N];

glBindBuffer(GL_ARRAY_BUFFER, position_vbo);
glEnableVertexAttribArray(0);
glVertexAttribPointer(0, 3, GL_FLOAT, GL_FALSE,
    0, (void*)(0));

glBindBuffer(GL_ARRAY_BUFFER, normal_vbo);
glEnableVertexAttribArray(1);
glVertexAttribPointer(1, 3, GL_FLOAT, GL_FALSE,
    0, (void*)(0));

glBindBuffer(GL_ARRAY_BUFFER, color_vbo);
glEnableVertexAttribArray(2);
glVertexAttribPointer(2, 4, GL_UNSIGNED_BYTE, GL_TRUE,
    0, (void*)(0));
\end{verbatim}
\end{frame}

\begin{frame}[fragile]
\frametitle{Аттрибуты вершин: как хранить?}
\begin{itemize}
\item Если не можете выбрать, используйте array-of-structs
\pause
\item Если часть аттрибутов постоянны, а другую часть нужно иногда обновлять - разделите эти части по двум разным VBO
\end{itemize}
\end{frame}

\begin{frame}[fragile]
\frametitle{Аттрибуты вершин: пример полностью}
\begin{verbatim}
// Инициализация:
program = createProgram()
vertices = generateVertices()
vbo = createVBO(vertices)
vao = createVAO()
setupVAO(vao, vbo)

// Рендеринг:
glUseProgram(program)
glBindVertexArray(vao)
glDrawArrays(GL_TRIANGLES, 0, count)
\end{verbatim}
\pause
\begin{itemize}
\item \href{https://www.khronos.org/opengl/wiki/Vertex_Specification}{khronos.org/opengl/wiki/Vertex\_Specification}
\end{itemize}
\end{frame}

\begin{frame}[fragile]
\frametitle{Индексированный рендеринг}
\slideimage{spheres.png}
\begin{itemize}
\item В моделях одна вершина часто входит в несколько треугольников (в среднем, 6)
\pause
\item Если использовать \verb|GL_TRIANGLES|, придётся копировать каждую вершину несколько раз
\pause
\item \verb|GL_TRIANGLE_STRIP| и т.п. только частично решают проблему
\pause
\item \begin{math}\Rightarrow\end{math} Индексированный рендеринг
\end{itemize}
\end{frame}

\begin{frame}[fragile]
\frametitle{Индексированный рендеринг}
\begin{itemize}
\item При обычном (не индексированном) рендеринге индексы входных вершин берутся напрямую из параметров \verb|glDrawArrays|: \verb|first|, \verb|first + 1|, \verb|first + 2|, \verb|...|, \verb|first + count - 1|
\pause
\item При индексированном рендеринге индексы описываются напрямую
\end{itemize}
\end{frame}

\begin{frame}[fragile]
\frametitle{Индексированный рендеринг: квадрат}
\slideimage{quad.png}
\pause
\begin{itemize}
\item Без индексов, \verb|GL_TRIANGLES|:
\begin{itemize}
\item Вершины: \verb|v0, v1, v2, v2, v1, v3|
\end{itemize}
\pause
\item Без индексов, \verb|GL_TRIANGLE_STRIP|:
\begin{itemize}
\item Вершины: \verb|v0, v1, v2, v3|
\end{itemize}
\pause
\item С индексами, \verb|GL_TRIANGLES|:
\begin{itemize}
\item Вершины: \verb|v0, v1, v2, v3|
\item Индексы: \verb|0, 1, 2, 2, 1, 3|
\end{itemize}
\end{itemize}
\end{frame}

\begin{frame}[fragile]
\frametitle{Индексированный рендеринг}
\begin{itemize}
\item Индексы хранятся в \verb|GL_ELEMENT_ARRAY_BUFFER|
\item id текущего \verb|GL_ELEMENT_ARRAY_BUFFER| хранится в текущем VAO
\pause
\begin{verbatim}
glDrawElements(GLenum mode, GLsizei count,
    GLenum type, const GLvoid * indices)
\end{verbatim}
\pause
\item \verb|mode| - как в \verb|glDrawArrays|: \verb|GL_TRIANGLES|, ...
\pause
\item \verb|count| - количество индексов
\pause
\item \verb|type| - тип индексов: \verb|GL_UNSIGNED_BYTE|, \verb|GL_UNSIGNED_SHORT|, \verb|GL_UNSIGNED_INT|
\pause
\item \verb|indices| - смещение в текущем \verb|GL_ELEMENT_ARRAY_BUFFER| до места, где начинаются индексы
\end{itemize}
\end{frame}

\begin{frame}[fragile]
\frametitle{Индексированный рендеринг: пример}
\fontsize{10pt}{10pt}
\begin{verbatim}
// Инициализация:
program = createProgram()
vertices, indices = generateMesh()
vbo = createBuffer(vertices)
ebo = createBuffer(indices)
vao = createVAO()

glBindVertexArray(vao)
glBindBuffer(GL_ARRAY_BUFFER, vbo)
// настраиваем аттрибуты
glBindBuffer(GL_ELEMENT_ARRAY_BUFFER, ebo)

// Рендеринг:
glUseProgram(program)
glBindVertexArray(vao)
glDrawElements(GL_TRIANGLES, count,
    GL_UNSIGNED_SHORT, (void*)(0))
\end{verbatim}
\end{frame}

\begin{frame}[fragile]
\frametitle{Primitive restart}
\begin{itemize}
\item При использовании \verb|GL_LINE_STRIP|, \verb|GL_LINE_LOOP|, \verb|GL_TRIANGLE_STRIP|, \verb|GL_TRIANGLE_FAN| все входные вершины образуют один line strip, line loop, triangle strip, triangle fan соответственно
\pause
\item Что, если мы хотим одну последовательность вершин разбить на несколько примитивов?
\pause
\item Специальное значение индекса вершины означает что со следующей вершины нужно начать новый примитив:
\begin{verbatim}
GL_TRIANGLE_STRIP:
[0, 1, 2, 3, -1, 4, 5, 6, 7]


[0, 1, 2, 3], [4, 5, 6, 7]
\end{verbatim}
\pause
\item \verb|glEnable(GL_PRIMITIVE_RESTART)|
\pause
\item \verb|glPrimitiveRestartIndex(index)|
\end{itemize}
\end{frame}

\begin{frame}[fragile]
\frametitle{Индексированный рендеринг: ссылки}
\begin{itemize}
\item \href{https://www.khronos.org/opengl/wiki/Vertex_Rendering#Basic_Drawing}{khronos.org/opengl/wiki/Vertex\_Rendering\#Basic\_Drawing}
\item \href{http://www.opengl-tutorial.org/intermediate-tutorials/tutorial-9-vbo-indexing}{opengl-tutorial.org/intermediate-tutorials/tutorial-9-vbo-indexing}
\end{itemize}
\end{frame}

\end{document}