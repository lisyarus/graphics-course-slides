% (c) Nikita Lisitsa, lisyarus@gmail.com, 2021

\documentclass{beamer}

\usepackage[T2A]{fontenc}
\usepackage[utf8]{inputenc}
\usepackage[russian]{babel}

\usepackage{graphicx}
\graphicspath{ {./images/} }

\usepackage{adjustbox}

\usepackage{color}
\usepackage{soul}

\usepackage{hyperref}

\definecolor{blue}{rgb}{0,0,1}
\definecolor{red}{rgb}{1,0,0}

\makeatletter
\newcommand{\slideimage}[1]{
  \begin{figure}
    \begin{adjustbox}{width=\textwidth, totalheight=\textheight-2\baselineskip-2\baselineskip,keepaspectratio}
      \includegraphics{#1}
    \end{adjustbox}
  \end{figure}
}
\makeatother

\title{Компьютерная графика}
\subtitle{Практика 9: Shadow mapping 2}
\date{2021}

\setbeamertemplate{footline}[frame number]

\begin{document}

\frame{\titlepage}

\begin{frame}[fragile]
\frametitle{Задание 1}
Подбираем shadow bias
\begin{itemize}
\item Нужно прибавить константу (bias) к значению, прочитанному из shadow map
\pause
\item Слишком большое значение приведёт к заметному peter panning (тень будет съезжать в сторону от модели)
\pause
\item Слишком маленького значения будет недостаточно, чтобы убрать shadow acne
\end{itemize}
\end{frame}

\begin{frame}[fragile]
\frametitle{Задание 2}
Вычисляем хорошую матрицу проекции для shadow map
\begin{itemize}
\item Нужно найти центр видимой области и оси \begin{math}X, Y, Z\end{math}
\pause
\item Направления осей \begin{math}\hat X, \hat Y, \hat Z\end{math} уже посчитаны, нужно только вычислить их длину
\pause
\item С помощью функции \verb|bbox| можно посчитать bounding box сцены, её центр \begin{math}C\end{math} - центр видимой области
\pause
\item Пройдясь по всем 8 вершинам \begin{math}V\end{math} bounding box'а сцены, можно посчитать скалярное произведение \begin{math}(V - C) \cdot \hat X\end{math}, максимум модуля таких произведений - длина вектора \begin{math}X\end{math}
\pause
\item Аналогично для \begin{math}Y\end{math} и \begin{math}Z\end{math}
\pause
\item Используя \begin{math}X,Y,Z,C\end{math} можно построить матрицу ортографической проекции
\end{itemize}
\end{frame}

\begin{frame}[fragile]
\frametitle{Задание 3}
Shadow sampler + PCF
\begin{itemize}
\item Меняем min/mag фильтры shadow map на \verb|GL_LINEAR|
\pause
\item Устанавливаем текстуре shadow map свойства \verb|GL_TEXTURE_COMPARE_MODE = GL_COMPARE_REF_TO_TEXTURE| и \verb|GL_TEXTURE_COMPARE_FUNC, GL_LESS|
\pause
\item Меняем тип sampler'а в шейдере на \verb|sampler2DShadow|
\pause
\item Меняем обращение к текстуре: \verb|texture(..., shadow_pos.xyz + bias)| (возвращает \verb|float|, а не \verb|vec4|!)
\end{itemize}
\end{frame}

\begin{frame}[fragile]
\frametitle{Задание 4}
\fontsize{10pt}{10pt}
Variance shadow maps
\begin{itemize}
\item Убираем свойства \verb|GL_TEXTURE_COMPARE_MODE| и \verb|GL_TEXTURE_COMPARE_FUNC|
\pause
\item Устанавливаем текстуре shadow map тип данных \verb|GL_RG32F| (format и type не принипиальны, можно \verb|GL_RGBA| и \verb|GL_FLOAT|)
\pause
\item Добавляем shadow map к фреймбуферу как \verb|GL_COLOR_ATTACHMENT0| (вместо \verb|GL_DEPTH_ATTACHMENT|)
\pause
\item Создаём текстуру или renderbuffer, которые будут использоваться для глубины, и добавляем его как \verb|GL_DEPTH_ATTACHMENT| фреймбуфера
\pause
\item Во фрагментном шейдере, рисующем shadow map, добавляем out-переменную и пишем в неё \verb|vec4(z, z * z, 0.0, 0.0)| (\verb|z| можно достать из \verb|gl_FragCoord|)
\pause
\item В основном фрагментном шейдере меняем тип sampler'а обратно на \verb|sampler2D|, читаем из него и используем неравенство Чебышёва для вычисления освещённости:
\begin{verbatim}
vec2 data = texture(shadow_map, shadow_pos.xy).rg;
float mu = data.r;
float sigma = data.g - mu * mu;
float z = shadow_pos.z;
float factor = (z < mu) ? 1.0
    : sigma / (sigma + (z - mu) * (z - mu));
\end{verbatim}
\end{itemize}
\end{frame}

\begin{frame}[fragile]
\frametitle{Задание 5}
Исправляем артефакты
\begin{itemize}
\item Добавляем наклон поверхности к среднему значению квадрата глубины:
\begin{itemize}
\item Через \verb|dFdx| и \verb|dFdy| можно получить градиент глубины по X и Y
\item К квадрату глубины добавляем \begin{math}\frac{1}{4}\left[\left(\frac{\partial Z}{\partial X}\right)^2 + \left(\frac{\partial Z}{\partial Y}\right)^2\right]\end{math}
\end{itemize}
\pause
\item Добавляем shadow bias (вычитаем константу из значения \verb|Z|, использующегося для вычисления освещённости)
\pause
\item Значение, получающееся из формулы неравенства Чебышёва, преобразуем: диапазон \begin{math}[0, \delta]\end{math} переходит в \begin{math}0\end{math}, а диапазон \begin{math}[\delta, 1]\end{math} переходит в \begin{math}[0, 1]\end{math} (\begin{math}\delta\end{math} - некое фиксированное значение, например, \verb|0.125|)
\end{itemize}
\end{frame}

\begin{frame}[fragile]
\frametitle{Задание 6}
Размываем shadow map
\begin{itemize}
\item Вместо чтения одного пикселя из shadow map, читаем набор значений из соседних пикселей (аналогично тому, как делалось размытие в задании №5 практики №7) и усредняем по Гауссу
\pause
\item Полученный двумерный вектор используем для вычисления освещённости через неравенство Чебышёва
\pause
\item {\color{red}N.B.}: по-хорошему это размытие нужно делать отдельными проходами с отдельными шейдерами и отдельным размытием по X и Y (см. \textit{separable Gaussian blur})
\end{itemize}
\end{frame}

\end{document}