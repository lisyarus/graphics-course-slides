% (c) Nikita Lisitsa, lisyarus@gmail.com, 2022

\documentclass{beamer}

\usepackage[T2A]{fontenc}
\usepackage[utf8]{inputenc}
\usepackage[russian]{babel}

\usepackage{graphicx}
\graphicspath{ {./images/} }

\usepackage{adjustbox}

\usepackage{color}
\usepackage{soul}

\usepackage{hyperref}

\definecolor{blue}{rgb}{0,0,1}
\definecolor{red}{rgb}{1,0,0}

\makeatletter
\newcommand{\slideimage}[1]{
  \begin{figure}
    \begin{adjustbox}{width=\textwidth, totalheight=\textheight-2\baselineskip-2\baselineskip,keepaspectratio}
      \includegraphics{#1}
    \end{adjustbox}
  \end{figure}
}
\makeatother

\title{Компьютерная графика}
\subtitle{Практика 2: Uniform'ы и матрицы преобразований}
\date{2021}

\setbeamertemplate{footline}[frame number]

\begin{document}

\frame{\titlepage}

\begin{frame}
\slideimage{0.png}
\end{frame}

\begin{frame}[fragile]
\frametitle{Задание 1}
Уменьшим треугольник в 2 раза, используя uniform переменную
\begin{itemize}
\item В коде вершинного шейдера:
\begin{itemize}
\item \verb|uniform float scale;|
\item Нужно где-то умножить вектор координат на \verb|scale|
\item NB: последняя координата \verb|gl_Position| должна остаться равной 1
\end{itemize}
\item После создания программы, до основного цикла:
\begin{itemize}
\item \verb|glUseProgram|
\item \verb|glGetUniformLocation| -- возвращает уникальный идентификатор, ползволяющий работать с этой uniform-переменной
\item \verb|glUniform1f| -- устанавливает значение конкретной uniform-переменной типа \verb|float|
\end{itemize}
\end{itemize}
\end{frame}

\begin{frame}
\frametitle{Задание 1}
\slideimage{1.png}
\end{frame}

\begin{frame}[fragile]
\frametitle{Задание 2}
Заставим треугольник постоянно крутиться
\begin{itemize}
\item В коде вершинного шейдера:
\begin{itemize}
\item \verb|uniform float angle;|
\item Нужно где-то повернуть вектор координат на \verb|angle|
\item Формула поворота есть в слайдах лекции (пока нужны только первые 2 координаты)
\end{itemize}
\item После создания программы, до основного цикла:
\begin{itemize}
\item \verb|glUseProgram|
\item \verb|glGetUniformLocation|
\item \verb|float time = 0.f;|
\end{itemize}
\item В теле основного цикла рендеринга, после вычисления \verb|dt|
\begin{itemize}
\item \verb|time += dt;|
\end{itemize}
\item В теле основного цикла рендеринга, после \verb|glUseProgram|, до \verb|glDrawArrays|
\begin{itemize}
\item \verb|glUniform1f|
\item В качестве значения можно использовать \verb|time|
\end{itemize}
\end{itemize}
\end{frame}

\begin{frame}
\frametitle{Задание 2}
\slideimage{2.png}
\end{frame}

\begin{frame}[fragile]
\frametitle{Задание 3}
Заменим ручное применение преобразования на матрицу
\begin{itemize}
\item В коде шейдера:
\begin{itemize}
\item Заменяем две uniform-переменные на одну: \verb|uniform mat4 transform|
\item Заменяем ручное вращение и масштабирование на умножение на матрицу: \verb|gl_Position = transform * vec4(...);|
\end{itemize}
\item Обновляем вызов \verb|glGetUniformLocation|
\item В теле основного цикла рендеринга
\begin{itemize}
\item Создаём матрицу \begin{math}4\times 4\end{math} - массив из 16 \verb|float|'ов
\begin{verbatim}
float transform[16] =
{
    ?, ?, ?, ?, // 1 строка
    ?, ?, ?, ?, // 2 строка
    ?, ?, ?, ?, // 3 строка
    ?, ?, ?, ?, // 4 строка
};
\end{verbatim}
\item Заполняем матрицу значениями, чтобы это была матрица применения поворота и масштабирования
\item \verb|glUniformMatrix4fv|, \verb|count = 1|, \verb|transpose = GL_TRUE|
\end{itemize}
\end{itemize}
\end{frame}

\begin{frame}
\frametitle{Задание 3}
\slideimage{2.png}
\end{frame}

\begin{frame}[fragile]
\frametitle{Задание 4}
Добавляем в матрицу сдвиг, зависящий от времени
\begin{itemize}
\item В теле основного цикла рендеринга
\begin{itemize}
\item Заводим переменные под сдвиг:
\begin{verbatim}
float x = ?;
float y = ?;
\end{verbatim}
\item Обновляем матрицу преобразования:
\begin{verbatim}
float transform[16] = ...;
\end{verbatim}
\end{itemize}
\end{itemize}
\end{frame}

\begin{frame}
\frametitle{Задание 4}
\slideimage{4.png}
\end{frame}

\begin{frame}[fragile]
\frametitle{Задание 5}
Добавляем учёт aspect ratio экрана
\begin{itemize}
\item В коде шейдера:
\begin{itemize}
\item Добавляем uniform-переменную \verb|view|, аналогичную переменной \verb|transform|
\item Применяем обе матрицы: \verb|gl_Position = view * transform * ...;|
\end{itemize}
\item После создания программы, до основного цикла:
\begin{itemize}
\item Добавляем \verb|glGetUniformLocation|
\end{itemize}
\item В теле основного цикла рендеринга
\begin{itemize}
\item Вычисляем \verb|aspect_ratio = width / height| (N.B.: если написать так, будет целочисленное деление; нам нужно деление во floating point)
\item Создаём новую матрицу, которая делит x-координату на \verb|aspect_ratio|
\begin{verbatim}
float view[16] = ...;
\end{verbatim}
\item Устанавливаем значение новой uniform-переменной с помозью \verb|glUniformMatrix4fv|
\end{itemize}
\end{itemize}
\end{frame}

\begin{frame}
\frametitle{Задание 5}
\slideimage{5.png}
\end{frame}

\begin{frame}[fragile]
\frametitle{Задание 6}
Выключаем VSync
\begin{itemize}
\item В теле основного цикла рендеринга
\begin{itemize}
\item Выводим в лог значение переменной \verb|dt| (время, потраченное на один кадр в секундах) - скорее всего, будет в районе \verb|0.016|
\end{itemize}
\item После создания OpenGL-контекста:
\begin{itemize}
\item \verb|SDL_GL_SetSwapInterval(0);|
\item Проверяем значение переменной \verb|dt| - должно стать значительно меньше (например, \verb|0.001|)
\end{itemize}
\item В теле основного цикла рендеринга
\begin{itemize}
\item Заменяем вычисление \verb|dt| на какую-нибудь константу, например \verb|float dt = 0.016f;|
\item Должен получиться эффект, похожий на \href{https://en.wikipedia.org/wiki/Wagon-wheel_effect}{wagon-wheel effect}
\end{itemize}
\item N.B.: может не сработать (зависит от системы, драйвера, и т.п.), ничего страшного
\end{itemize}
\end{frame}

\begin{frame}[fragile]
\frametitle{Задание 7*}
Заменяем треугольник на шестиугольник
\begin{itemize}
\item Нужно поменять количество вершин в вызове \verb|glDrawArrays|
\item Нужно правильно вычислить координаты вершин в вершинном шейдере
\end{itemize}
\end{frame}

\begin{frame}
\frametitle{Задание 7}
\slideimage{7.png}
\end{frame}

\end{document}