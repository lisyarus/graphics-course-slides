% (c) Nikita Lisitsa, lisyarus@gmail.com, 2021

\documentclass{beamer}

\usepackage[T2A]{fontenc}
\usepackage[utf8]{inputenc}
\usepackage[russian]{babel}

\usepackage{graphicx}
\graphicspath{ {./images/} }

\usepackage{adjustbox}

\usepackage{color}
\usepackage{soul}

\usepackage{hyperref}

\usepackage{amsmath}

\usepackage{tikz}
\usetikzlibrary{decorations}
\usetikzlibrary{decorations.pathreplacing}
\usepackage{xifthen}

\definecolor{red}{rgb}{1,0,0}
\definecolor{green}{rgb}{0,0.5,0}
\definecolor{blue}{rgb}{0,0,1}
\definecolor{magenta}{rgb}{0.75,0,0.75}

\makeatletter
\newcommand{\slideimage}[1]{
  \begin{figure}
    \begin{adjustbox}{width=\textwidth, totalheight=\textheight-2\baselineskip-2\baselineskip,keepaspectratio}
      \includegraphics{#1}
    \end{adjustbox}
  \end{figure}
}
\makeatother

\title{Компьютерная графика}
\subtitle{Лекция 14: рендеринг текста, bitmap-шрифты, векторные шрифты, SDF-шрифты, volume rendering, volume slicing, raymarching}
\date{2021}

\setbeamertemplate{footline}[frame number]

\begin{document}

\frame{\titlepage}

\begin{frame}[fragile]
\frametitle{Рендеринг текста}
\begin{itemize}
\item Абстрактный текст
\pause
\item + кодировка \begin{math}\Rightarrow\end{math} машинное представление текста
\pause
\item + шрифт + настройки шейпинга (shaping) \begin{math}\Rightarrow\end{math} набор глифов (изображений символов) и их координат
\pause
\item \begin{math}\Rightarrow\end{math} нарисованный текст
\end{itemize}
\end{frame}

\begin{frame}[fragile]
\frametitle{Кодировки}
\fontsize{10pt}{10pt}
\begin{itemize}
\item Описывают машинное представление текста, т.е. соответствие последовательностей символов и последовательностей бит
\pause
\item ASCII: 7 бит (обычно дополняется нулевым старшим битом до 8 бит), первые 32 символа - управляющие (\verb|\r|, \verb|\n|, tab, ...), остальные 96 - буквы английского алфавита (большие и маленькие) и прочие символы (различные скобки, арифметические операции, пунктуация, пробел, ...)
\pause
\begin{itemize}
\item Многие кодировки совпадают с ASCII в диапазоне 0-127 или 32-127
\end{itemize}
\pause
\item Огромное количество в основном 8-битных кодировок для разных алфавитов и систем:
\begin{itemize}
\item ISO/IEC 8859 - 15 разных вариантов (ISO/IEC 8859-5 для русского языка)
\item Code page XXX - много разных кодировок для DOS (Code page 866 для русского языка)
\item Windows code pages (Windows-1251 для русского языка)
\item KOI-8 и вариации - для русского языка
\item etc.
\end{itemize}
\pause
\item Unicode-кодировки
\end{itemize}
\end{frame}

\begin{frame}[fragile]
\frametitle{Unicode}
\begin{itemize}
\item Unicode - стандарт, описывающий соответствие символов целочисленным кодам в диапазоне \verb|0..10FFFFh| исключая \verb|D800h..DFFFh| (используется для суррогатных пар в UTF-16; всего 1112064 символов), и рекомендации по их интерпретации и визуализации
\pause
\item На сегодняшний день описывает 144697 символа
\pause
\item Unicode-кодировки:
\pause
\begin{itemize}
\item UTF-8: от 1 до 4 байт на символ, совпадает с ASCII в диапазоне \verb|0..7Fh|, самая распространённая сегодня кодировка (95\% интернета)
\pause
\item UCS-2: устаревшая, 2 байта на символ, не поддерживает весь unicode
\pause
\item UTF-16: 2 или 4 байта на символ
\pause
\item UTF-32: 4 байта на символ
\pause
\item GB 18030: специальная кодировка для китайских иероглифов (но тоже поддерживает весь unicode)
\end{itemize}
\pause
\item N.B.: один символ unicode \textbf{не соответствует} одному видимому символу (\textit{графеме})
\end{itemize}
\end{frame}

\begin{frame}[fragile]
\frametitle{Шрифты}
\begin{itemize}
\item Содержит набор \textit{глифов} (изображений символов в каком-либо виде) и правил их использования
\pause
\item Виды шрифтов:
\pause
\begin{itemize}
\item Bitmap-шрифты: глиф - готовое изображение (bitmap)
\pause
\item Векторные шрифты: глиф описывается как геометрическая фигура
\pause
\item SDF-шрифты: глиф описывается с помощью \textit{signed distance field} (SDF)
\end{itemize}
\pause
\item Современные форматы шрифтов (\verb|.ttf| - TrueType, \verb|.otf| - OpenType) - векторные, описывают границу глифа как набор отрезков и квадратичных кривых Безье (т.е. 2-ого порядка)
\pause
\item Bitmap и SDF шрифты часто строятся по векторным шрифтам
\pause
\item Freetype - самая распространённая библиотека для чтения векторных шрифтов; умеет растеризовать в bitmap и (с версии 2.11.0, июль 2021) в SDF
\end{itemize}
\end{frame}

\begin{frame}[fragile]
\frametitle{Шейпинг}
\begin{itemize}
\item Процесс преобразования последовательности символов в набор отпозиционированных глифов
\pause
\item Может включать в себя:
\pause
\begin{itemize}
\item Настройки шейпинга: направление (слева-направо, справа-налево, сверху-вниз, снизу-вверх), размер шрифта, межбуквенное расстояние, стиль (жирный, курсив, и т.п.)
\pause
\item Hinting: применение дополнительных преобразований к векторному глифу в зависимости от разрешения
\pause
\item Kerning: изменение расстояния между соседними глифами
\pause
\item Лигатуры: последовательность несвязанных символов, представленная одним глифом (ff, fi, <=>)
\end{itemize}
\pause
\item harfbuzz - одна из самых распространённых библиотек для шейпинга текста
\item Freetype позволяет сделать шейпинг, но хуже, чем harfbuzz
\end{itemize}
\end{frame}

\begin{frame}[fragile]
\frametitle{Hinting}
\slideimage{hinting.png}
\end{frame}

\begin{frame}[fragile]
\frametitle{Kerning}
\slideimage{kerning.png}
\end{frame}

\begin{frame}[fragile]
\frametitle{Лигатуры}
\slideimage{ligatures.png}
\end{frame}

\end{document}