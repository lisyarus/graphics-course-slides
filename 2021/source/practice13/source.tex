% (c) Nikita Lisitsa, lisyarus@gmail.com, 2021

\documentclass{beamer}

\usepackage[T2A]{fontenc}
\usepackage[utf8]{inputenc}
\usepackage[russian]{babel}

\usepackage{graphicx}
\graphicspath{ {./images/} }

\usepackage{adjustbox}

\usepackage{color}
\usepackage{soul}

\usepackage{hyperref}

\definecolor{blue}{rgb}{0,0,1}
\definecolor{red}{rgb}{1,0,0}

\makeatletter
\newcommand{\slideimage}[1]{
  \begin{figure}
    \begin{adjustbox}{width=\textwidth, totalheight=\textheight-2\baselineskip-2\baselineskip,keepaspectratio}
      \includegraphics{#1}
    \end{adjustbox}
  \end{figure}
}
\makeatother

\title{Компьютерная графика}
\subtitle{Практика 13: Timer queries, instancing, frustum culling, LOD}
\date{2021}

\setbeamertemplate{footline}[frame number]

\begin{document}

\frame{\titlepage}

\begin{frame}[fragile]
\fontsize{10pt}{10pt}
\frametitle{Задание 1}
Замеряем время рисования кадра с помощью timer queries
\begin{itemize}
\item Заводим \verb|std::vector| для ID query объектов и ещё один для запоминания, свободен объект или нет
\pause
\item В начале каждого кадра находим индекс первого свободного query объекта, или, если такого нет, создаём новый и добавляем в массив (и помечаем как свободный)
\pause
\item Помечаем выбранный query объект как занятый, и вставляем \verb|glBeginQuery(GL_TIME_ELAPSED, id)| в начало кадра (перед \verb|glClear|) и соответствующий \verb|glEndQuery| в конец кадра (перед \verb|SwapBuffers|)
\pause
\item В конце кадра проверяем каждый query объект на то, готовы ли его данные: если готовы \textendash{} достаём их и логируем (лучше разделить на 10\textsuperscript{6} или 10\textsuperscript{9} чтобы получить миллисекунды или секунды, соответственно), и помечаем как свободный
\pause
\item В дальнейших заданиях имеет смысл смотреть на это значение и сравнивать с предыдущими заданиями (всё-таки мы занимаемся оптимизацией)
\pause
\item В конце программы можно залогировать размер массива \verb|query| объектов: это примерное отставание (в кадрах) GPU от CPU
\end{itemize}
\end{frame}

\begin{frame}[fragile]
\frametitle{Задание 2}
Рисуем 1024 копии объекта
\begin{itemize}
\item В цикле рисуем копии объекта, например, сеткой, с координатами в диапазоне \verb|[-16 .. 16)| с шагом в 1 по X и Z
\pause
\item Для сдвига можно использовать uniform-переменную \verb|offset| (она уже есть в коде)
\end{itemize}
\end{frame}

\begin{frame}[fragile]
\frametitle{Задание 3}
Рисуем 1024 копии объекта, используя instancing
\begin{itemize}
\item В вершинном шейдере заменяем uniform-переменную \verb|offset| на атрибут вершины (\verb|location = 2|)
\pause
\item Заводим массив (\verb|std::vector|) сдвигов (\verb|glm::vec3|, 1024 штуки), использовавшихся в предыдущем задании, и заполняем при старте программы
\pause
\item Заводим VBO для этих сдвигов и загружаем в него данные (так же, как в обычный VBO)
\pause
\item Настраиваем атрибут с \verb|index = 2|, беря данные из нового VBO, и делаем \verb|glVertexAttribDivisor(2, 1)| (чтобы этот атрибут менялся один раз на instance, а не на вершину)
\pause
\item Вместо цикла по 1024 объектам делаем один вызов \verb|glDrawElementsIsntanced|
\end{itemize}
\end{frame}

\begin{frame}[fragile]
\fontsize{10pt}{10pt}
\frametitle{Задание 4}
Добавляем frustum culling
\begin{itemize}
\item Возвращаем цикл из задания 2 и uniform-переменную \verb|offset| (атрибут можно убрать, он нам уже не нужен)
\begin{itemize}
\item (код настройки атрибута, загрузки его данных и instanced рисования оставляем, но комментируем \textendash{} чтобы можно было потом проверить)
\end{itemize}
\pause
\item На старте, после загрузки модели вычисляем её bounding box функцией \verb|bbox| (из \verb|mesh_utils.hpp|) \textendash{} получаем координаты минимальной и максимальной точек bounding box'а
\pause
\item Каждый кадр создаём объект \verb|frustum| (из \verb|frustum.hpp|), передавая ему матрицу \verb|projection*view|
\pause
\item В цикле прохода по всем объектам, для каждого объекта создаём объект типа \verb|aabb| (из \verb|aabb.hpp|), учитывая bbox модели и offset текущего объекта
\pause
\item Если \verb|aabb| объекта не пересекает \verb|frustum| (функция \verb|intersect| из \verb|intersect.hpp|), пропускаем рисование этого объекта
\pause
\item Логируем число нарисованных объектов (если вы не меняли начальные параметры камеры, на старте будет 495 объектов)
\end{itemize}
\end{frame}

\begin{frame}[fragile]
\fontsize{10pt}{10pt}
\frametitle{Задание 5 (начало)}
Добавляем LODs
\begin{itemize}
\item Вместо одного файла с моделью загружаем все шесть LOD'ов: \verb|bunny0.obj .. bunny5.obj|
\pause
\item Храним все LOD'ы в одном наборе VAO+VBO+EBO:
\begin{itemize}
\item Настройка VAO никак не меняется
\item VBO \textendash{} последовательно записанные наборы вершин всех LOD'ов, от 0 до 5
\item EBO \textendash{} последовательно записанные наборы индексов всех LOD'ов, от 0 до 5, но значения индексов нужно сдвинуть: например, индексы \verb|bunny1.obj| начинаются с нуля, но тогда они будут ссылаться на вершины \verb|bunny0.obj|, ведь они идут первыми в VBO, так что к индексам нужно прибавить количество вершин в \verb|bunny0.obj|; аналогично, к индексам \verb|bunny2.obj| нужно прибавить количество вершин из \verb|bunny0.obj| и \verb|bunny1.obj|, и т.д.
\end{itemize}
\end{itemize}
\end{frame}

\begin{frame}[fragile]
\fontsize{10pt}{10pt}
\frametitle{Задание 5 (продолжение)}
Добавляем LODs
\begin{itemize}
\item Нам нужно знать, где начинаются индексы для каждого LOD'а и сколько их: удобно завести массив размера \verb|N+1| (где \verb|N| \textendash{} количество LOD'ов, у нас \verb|N = 6|), где \verb|a[i]| \textendash{} количество вершин в сумме во всех LOD'ах с номерами, меньшими \verb|i| (в т.ч. \verb|a[0] = 0| и \verb|a[N]| \textendash{} суммарное количество вершин во всех LOD'ах), тогда индексы \verb|i|-ого LOD'а начинаются с \verb|a[i]|, и их \verb|a[i+1] - a[i]| штук
\begin{itemize}
\item N.B.: последний параметр \verb|glDrawElements| \textendash{} не номер индекса, а сдвиг в байтах, приведённый к типу \verb|void*|!
\end{itemize}
\pause
\item При рендеринге вычисляем номер LOD, который мы будем использовать (например, как расстояние до камеры, делённое на фиксированное значение; координаты камеры можно вычислить из матрицы \verb|view|)
\pause
\item Меняем вызов \verb|glDrawElements|, чтобы рисовался только выбранный LOD
\end{itemize}
\end{frame}

\end{document}