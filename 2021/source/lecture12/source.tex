% (c) Nikita Lisitsa, lisyarus@gmail.com, 2021

\documentclass{beamer}

\usepackage[T2A]{fontenc}
\usepackage[utf8]{inputenc}
\usepackage[russian]{babel}

\usepackage{graphicx}
\graphicspath{ {./images/} }

\usepackage{adjustbox}

\usepackage{color}
\usepackage{soul}

\usepackage{hyperref}

\usepackage{amsmath}

\usepackage{tikz}
\usetikzlibrary{decorations}
\usetikzlibrary{decorations.pathreplacing}
\usepackage{xifthen}

\definecolor{red}{rgb}{1,0,0}
\definecolor{green}{rgb}{0,0.5,0}
\definecolor{blue}{rgb}{0,0,1}
\definecolor{magenta}{rgb}{0.75,0,0.75}

\makeatletter
\newcommand{\slideimage}[1]{
  \begin{figure}
    \begin{adjustbox}{width=\textwidth, totalheight=\textheight-2\baselineskip-2\baselineskip,keepaspectratio}
      \includegraphics{#1}
    \end{adjustbox}
  \end{figure}
}
\makeatother

\title{Компьютерная графика}
\subtitle{Лекция 12: Анимации, easing functions, bitmap-анимации, кватернионы, иерархии объектов, скелетная анимация}
\date{2021}

\setbeamertemplate{footline}[frame number]

\begin{document}

\frame{\titlepage}

\begin{frame}[fragile]
\frametitle{Анимации}
\begin{itemize}
\item Анимация (в общем смысле) \textendash{} что угодно, меняющееся со временем
\pause
\begin{itemize}
\item Двигающийся объект
\item Анимированный элемент интерфейса
\item Анимированная модель
\item Движущаяся камера
\item etc.
\end{itemize}
\pause
\item Сводится к вопросу о том, что и как мы меняем в зависимости от времени
\end{itemize}
\end{frame}

\begin{frame}[fragile]
\frametitle{Анимации: общие идеи}
\begin{itemize}
\item Вычисление анимаций лучше привязывать к реальному времени, а не к номеру кадра (\verb|state += speed * dt| а не \verb|state += speed|)
\pause
\begin{itemize}
\item Будет лучше работать при выключенном VSync
\item Будет лучше работать, когда CPU/GPU не справляются с нагрузкой
\end{itemize}
\pause
\item Дискретные движения (напр. поворот камеры) можно сглаживать: вместо \verb|x = x_target| делать \verb|x = lerp(x, x_target, exp(-speed * dt))| (\verb|lerp(x0,x1,t)| - функция линейной интерполяции, в GLSL она называется \verb|mix|)
\pause
\begin{itemize}
\item Это соответствует численному решению уравнения \begin{math}\dot x = speed \cdot (x_{target} - x)\end{math}
\item Не зависит от начального значения \verb|x|
\end{itemize}
\end{itemize}
\end{frame}

\begin{frame}[fragile]
\frametitle{Анимации: easing functions}
\begin{itemize}
\item При временнóй интерполяции между двумя значениями вместо линейной интерполяции \verb|x = lerp(x_start, x_end, t)| можно использовать т.н. easing functions
\pause
\item Применяются к параметру интерполяции \begin{math}t \in [0, 1]\end{math} и сглаживают анимацию: \verb|x = lerp(x_start, x_end, easing(t))|
\pause
\item Примеры easing functions:
\begin{itemize}
\item \begin{math}f(t) = t\end{math}
\item \begin{math}f(t) = 3t^2-2t^3\end{math}
\item \begin{math}f(t) = t^2\end{math}
\item \begin{math}f(t) = 1 - (1-t)^2\end{math}
\item \begin{math}f(t) = \sqrt{t}\end{math}
\item Больше примеров: \href{https://easings.net}{easings.net}
\end{itemize}
\end{itemize}
\end{frame}

\begin{frame}[fragile]
\frametitle{Анимации: сплайны}
\begin{itemize}
\item Часто значения интерполируют, используя сплайны: кривые, позволяющие удобно настраивать зависимость некой величины от параметра \verb|t|
\item Обычно сплайн строится по набору точек и, возможно, значений производных в этих точках
\pause
\item Виды сплайнов:
\begin{itemize}
\item Сплайны Безье
\item Кубические сплайны
\item B-сплайны
\item NURBS
\item etc.
\end{itemize}
\pause
\item N.B.: спектр применения сплайнов не ограничивается анимациями!
\begin{itemize}
\item Curve fitting
\item Представление сложных геометрических форм (напр. зданий, шрифтов)
\item etc.
\end{itemize}
\end{itemize}
\end{frame}

\begin{frame}[fragile]
\frametitle{Анимации: сплайны}
\slideimage{spline-editing.png}
\end{frame}

\begin{frame}[fragile]
\frametitle{Bitmap-анимации}
\begin{itemize}
\item Меняющееся со временем изоражение
\pause
\item 3D текстура
\begin{itemize}
\item Номер кадра - 3я текстурная координата (нормированная)
\item Интерполирует между кадрами
\end{itemize}
\pause
\item 2D array текстура
\begin{itemize}
\item Номер кадра - 3я текстурная координата (\textbf{не} нормированная)
\item \textbf{Не} интерполирует между кадрами
\end{itemize}
\pause
\item 2D текстурный атлас - текстура, хранящая несколько изображений бок о бок
\begin{itemize}
\item По номеру кадра вычисляются настоящие текстурные координаты
\item \textbf{Не} интерполирует между кадрами
\end{itemize}
\end{itemize}
\end{frame}

\begin{frame}[fragile]
\frametitle{Текстура-атлас с анимацией}
\slideimage{atlas1.jpg}
\end{frame}

\begin{frame}[fragile]
\frametitle{Текстура-атлас с анимацией}
\slideimage{atlas2.jpg}
\end{frame}

\begin{frame}[fragile]
\frametitle{Представление вращений}
\begin{itemize}
\item Обычно мы представляли вращения матрицей
\item Матрица \begin{math}3x3\end{math} - 9 значений, это много
\item Применить матрицу к вектору - минимум операций сложения и умножения
\item Композиция вращений - произведение матриц, довольно быстрая операция
\pause
\item Вращения образуют 3х-мерную группу, т.е. описываются 3мя параметрами, например углами Эйлера
\item Применить вращение, выраженное через углы Эйлера - много тригонометрических функций, медленно
\item Композиция таких вращений - сложная операция, много тригонометрических функций
\pause
\item Хочется компромисс между сложностью и объёмом хранения
\end{itemize}
\end{frame}

\begin{frame}[fragile]
\frametitle{Кватернионы}
\begin{itemize}
\item Кватернионы \begin{math}\mathbb H\end{math} \textendash{} четырёхмерная \textit{некоммутативная} алгебра над вещественными числами с тремя мнимыми единицами
\item Каждый элемент \begin{math}q\in \mathbb H\end{math} представляется в виде \begin{math}q = a + bi + cj + dk\end{math}, где \begin{math}a, b, c, d \in \mathbb R\end{math} - коэффициенты кватерниона
\pause
\item Правила умножения:
\begin{itemize}
\item \begin{math}i^2=j^2=k^2=-1\end{math}
\item \begin{math}ij=-ji=k\end{math}
\item \begin{math}jk=-kj=i\end{math}
\item \begin{math}ki=-ik=j\end{math}
\end{itemize}
\end{itemize}
\end{frame}

\begin{frame}[fragile]
\frametitle{Кватернионы}
\begin{itemize}
\item Сопряжённый кватернион определяется как \begin{math}\overline q = a - bi - cj - dk\end{math}
\item Норма кватерниона: \begin{math}q \cdot \overline q = a^2 + b^2 + c^2 + d^2 = |q|^2\end{math}
\item Обратный кватернион: \begin{math}q^{-1} = \frac{1}{|q|^2} \overline q\end{math}
\end{itemize}
\end{frame}

\begin{frame}[fragile]
\frametitle{Кватернионы: альтернативное представление}
\begin{itemize}
\item Для кватерниона \begin{math}q = a + bi + cj + dk\end{math} назовём его скалярной частью число \begin{math}a\end{math}, а векторной частью \textendash{} вектор \begin{math}v = (b, c, d)\end{math}
\item Кватернион \textendash{} пара скаляр + вектор: \begin{math}q = (a, v)\end{math}
\pause
\item Произведение кватернионов: \begin{math}(a_1, v_1) \cdot (a_2, v_2) = (a_1 \cdot a_2 - v_1 \cdot v_2, a_1 v_2 + a_2 v_1 + v_1 \times v_2\end{math}
\pause
\item В таком виде кватернионы удобно реализовывать в шейдерах
\end{itemize}
\end{frame}

\begin{frame}[fragile]
\frametitle{Кватернионы: вращения}
\begin{itemize}
\item Представим произвольный трёхмерный вектор \begin{math}v\end{math} как кватернион с нулевой скалярной частью \begin{math}(0, v)\end{math}
\item Вращение вокруг оси \begin{math}w\end{math} (единичный вектор) на угол \begin{math}\theta\end{math} можно реализовать как \begin{math}q \cdot (0, v) \cdot q^{-1}\end{math}, где \begin{math}q = (\cos \frac{\theta}{2}, w \cdot \sin \frac{\theta}{2})\end{math}
\pause
\item N.B.: \begin{math}q^{-1} = (\cos \frac{\theta}{2}, - w \cdot \sin \frac{\theta}{2})\end{math}
\pause
\item Только алгебраические операции \begin{math}\Rightarrow\end{math} быстро!
\pause
\item Композиция вращений \textendash{} произведение кватернионов: \begin{math}{\color{blue}q_2} \cdot ({\color{red}q_1} \cdot (0, v) \cdot {\color{red}q_1^{-1}}) \cdot {\color{blue}q_2^{-1}} = ({\color{blue}q_2} \cdot {\color{red}q_1}) \cdot (0, v) \cdot ({\color{red}q_1^{-1}} \cdot {\color{blue}q_2^{-1}}) = ({\color{blue}q_2}{\color{red}q_1}) \cdot (0, v) \cdot ({\color{blue}q_2}{\color{red}q_1})^{-1}\end{math}
\pause
\item Стандартный способ для представления вращений объектов в 3D движках
\end{itemize}
\end{frame}

\begin{frame}[fragile]
\frametitle{Кватернионы: ссылки}
\begin{itemize}
\item \href{https://en.wikipedia.org/wiki/Quaternion}{en.wikipedia.org/wiki/Quaternion}
\item \href{https://en.wikipedia.org/wiki/Quaternions_and_spatial_rotation}{en.wikipedia.org/wiki/Quaternions\_and\_spatial\_rotation}
\item \href{https://en.wikipedia.org/wiki/Rotation_formalisms_in_three_dimensions}{en.wikipedia.org/wiki/Rotation\_formalisms\_in\_three\_dimensions}
\end{itemize}
\end{frame}

\begin{frame}[fragile]
\frametitle{Преобразования объектов}
\begin{itemize}
\item Часто для задания преобразования, применяемого к объекту, нам не нужна целиком матрица аффинного преобразования
\item Обычно это поворот + масштабирование + сдвиг
\pause
\item Поворот \textendash{} кватернион \begin{math}q\end{math}
\item Масштабирование \textendash{} число \begin{math}s\end{math} (изотропное масштабирование) или три числа (разный масштаб по разным осям)
\item Сдвиг \textendash{} вектор сдвига \begin{math}t\end{math}
\pause
\begin{equation}
v \mapsto s \cdot qvq^{-1} + t
\end{equation}
\end{itemize}
\end{frame}

\begin{frame}[fragile]
\frametitle{Иерархия объектов}
\begin{itemize}
\item Часто объекты сцены/мира образуют иерархическую структуру (шапка на человеке, человек в машине, машина на корабле)
\item Удобно описывать преобразование (позиция + поворот + масштабирование) не относительно центра сцены/мира, а относительно родительского объекта
\begin{itemize}
\item N.B.: обычно в такой ситуации есть один корневой объект - сцена
\end{itemize}
\pause
\item Нужно уметь вычислять итоговое преобразование объекта, т.е. композицию всех преобразований от корня до нашего объекта
\end{itemize}
\end{frame}

\begin{frame}[fragile]
\frametitle{Композиция преобразований объектов}
\begin{multline}
({\color{blue}t_2},{\color{blue}s_2},{\color{blue}q_2}) \cdot ({\color{red}t_1},{\color{red}s_1},{\color{red}q_1}) \cdot v = {\color{blue}s_2}{\color{blue}q_2}({\color{red}s_1}{\color{red}q_1}v{\color{red}q_1^{-1}} + {\color{red}t_1}){\color{blue}q_2^{-1}} + {\color{blue}t_2} = \\
= {\color{blue}s_2}{\color{red}s_1} ({\color{blue}q_2}{\color{red}q_1}) v ({\color{blue}q_2}{\color{red}q_1})^{-1} + {\color{blue}s_2}{\color{blue}q_2}{\color{red}t_1}{\color{blue}q_2^{-1}} + {\color{blue}t_2} = \\
({\color{blue}s_2}{\color{blue}q_2}{\color{red}t_1}{\color{blue}q_2^{-1}} + {\color{blue}t_2}, {\color{blue}s_2}{\color{red}s_1}, {\color{blue}q_2}{\color{red}q_1}) \cdot v
\end{multline}
\end{frame}

\begin{frame}[fragile]
\frametitle{Анимация трёхмерных моделей}
\begin{itemize}
\item Анимация положения объекта в пространстве \textendash{} неплохо, но скучно
\item Хочется анимировать сам объект, т.е. двигать его вершины
\pause
\item 2 способа:
\begin{itemize}
\item Покадровая анимация (keyframe animation, morph-target animation)
\item Скелетная анимация
\end{itemize}
\end{itemize}
\end{frame}

\begin{frame}[fragile]
\frametitle{Покадровая анимация моделей}
\begin{itemize}
\item Анимация хранится в виде "кадров": фиксированных состояний модели (наборов координат вершин)
\pause
\item Вершинный шейдер принимает два набора атрибутов позиций вершин и интерполирует между ними
\begin{itemize}
\item N.B.: интерполяция может использовать easing
\end{itemize}
\end{itemize}
\end{frame}

\begin{frame}[fragile]
\frametitle{Покадровая анимация моделей}
\slideimage{keyframes.jpg}
\end{frame}

\begin{frame}[fragile]
\frametitle{Покадровая анимация моделей}
\begin{itemize}
\item Много вариантов реализации:
\begin{itemize}
\item При смене кадра анимации загружать в VBO новые данные
\item При смене кадра менять VBO/VAO
\item Хранить кадры отдельно (например, в buffer textures), передавать в шейдер только номер кадра
\end{itemize}
\pause
\item Много проблем:
\begin{itemize}
\item Сложно модифицировать: нужно двигать \textit{все} вершины модели
\item Требует много памяти
\item Для хорошего качества нужно много кадров, иначе будут артефакты интерполяции (напр. модель начнёт пересекать саму себя)
\end{itemize}
\pause
\item Обычно \textbf{не} используется для 3D моделей
\end{itemize}
\end{frame}

\begin{frame}[fragile]
\frametitle{Скелетная анимация моделей}
\begin{itemize}
\item Модель привязывается к виртуальному "скелету"
\pause
\item Скелет \textendash{} иерархия виртуальных "костей"
\pause
\item Каждая вершина привязана к одной или (чаще) нескольким костям
\pause
\item Каждой паре вершина-кость соответствует некоторый вес: насколько эта кость влияет на эту вершину (сумма весов для одной вершины должна равняться 1)
\pause
\item Кадры анимации задаются только для скелета
\pause
\item Интерполируются только преобразования костей (костей гораздо меньше, чем вершин \begin{math}\Rightarrow\end{math} это не страшно делать даже на CPU)
\pause
\item В вершинном шейдере вычисляется итоговое преобразование для вершины как среднее между преобразованиями связанных с ней костей
\end{itemize}
\end{frame}

\begin{frame}[fragile]
\frametitle{Скелетная анимация моделей}
\slideimage{skeletal.jpg}
\end{frame}

\begin{frame}[fragile]
\frametitle{Скелетная анимация моделей}
\begin{verbatim}
uniform mat4x4 bones[16];

layout (location = 0) in vec4 position;
layout (location = 1) in ivec2 bone_ids;
layout (location = 2) in vec2 bone_weights;

void main()
{
  gl_Position =
      bone_weights.x * bones[bone_ids.x] * position
    + bone_weights.y * bones[bone_ids.y] * position;
}
\end{verbatim}
\end{frame}

\begin{frame}[fragile]
\frametitle{Скелетная анимация моделей}
\begin{itemize}
\item К нормалям тоже нужно применять преобразования (но не сдвиги!)
\pause
\item Кости обычно тоже выстроены в иерархию \begin{math}\Rightarrow\end{math} перед применением нужно вычислить суммарное преобразование (композицию)
\begin{itemize}
\item Нужно быть внимательным к особенностям задания преобразований в разных редакторах и форматах
\end{itemize}
\end{itemize}
\end{frame}

\begin{frame}[fragile]
\frametitle{Скелетная анимация моделей}
\begin{itemize}
\item Удобно и интуитивно модифицировать (все 3D-редакторы имеют поддержку скелетных анимаций)
\pause
\item Небольшой расход памяти (модель не дублируется)
\pause
\item Самый распространённый способ анимировать модели
\end{itemize}
\end{frame}

\begin{frame}[fragile]
\frametitle{Скелетная анимация}
\begin{itemize}
\item Преобразования для скелета часто комбинируют с процедурными элементами, которые невозможно заранее сделать в 3D-редакторе
\pause
\begin{itemize}
\item Голова человека поворачивается в сторону собеседника
\item Нога человека встаёт на камень
\item Рука человека берёт предмет
\end{itemize}
\pause
\item Для этого нужно решать обратную задачу: хотим, чтобы кость была в известной точке, нужно найти преобразования, которыми этого можно добиться
\begin{itemize}
\item Inverse kinematics
\end{itemize}
\end{itemize}
\end{frame}

\begin{frame}[fragile]
\frametitle{Скелетная анимация}
\begin{itemize}
\item Есть способы для некоторых ситуаций полностью процедурно генерировать преобразования скелета
\pause
\item Анимация передвижения пауков: \href{https://www.youtube.com/watch?v=e6Gjhr1IP6w}{youtube.com/watch?v=e6Gjhr1IP6w}
\pause
\item Оффлайн генерация анимации движения для двуногих: \href{https://www.youtube.com/watch?v=pgaEE27nsQw}{youtube.com/watch?v=pgaEE27nsQw}
\end{itemize}
\end{frame}

\begin{frame}[fragile]
\frametitle{Скелетная анимация: ссылки}
\begin{itemize}
\item \href{https://en.wikipedia.org/wiki/Skeletal_animation}{en.wikipedia.org/wiki/Skeletal\_animation}
\item \href{https://learnopengl.com/Guest-Articles/2020/Skeletal-Animation}{learnopengl.com/Guest-Articles/2020/Skeletal-Animation}
\item \href{https://ogldev.org/www/tutorial38/tutorial38.html}{ogldev.org/www/tutorial38/tutorial38.html}
\item \href{https://www.youtube.com/watch?v=f3Cr8Yx3GGA}{youtube.com/watch?v=f3Cr8Yx3GGA}
\end{itemize}
\end{frame}

\end{document}