% (c) Nikita Lisitsa, lisyarus@gmail.com, 2022

\documentclass{beamer}

\usepackage[T2A]{fontenc}
\usepackage[utf8]{inputenc}
\usepackage[russian]{babel}

\usepackage{graphicx}
\graphicspath{ {./images/} }

\usepackage{adjustbox}

\usepackage{color}
\usepackage{soul}

\usepackage{hyperref}

\definecolor{blue}{rgb}{0,0,1}
\definecolor{red}{rgb}{1,0,0}

\makeatletter
\newcommand{\slideimage}[1]{
  \begin{figure}
    \begin{adjustbox}{width=\textwidth, totalheight=\textheight-2\baselineskip-2\baselineskip,keepaspectratio}
      \includegraphics{#1}
    \end{adjustbox}
  \end{figure}
}
\makeatother

\title{Компьютерная графика}
\subtitle{Лекция 1: Введение в курс}
\date{2021}

\setbeamertemplate{footline}[frame number]

\begin{document}

\frame{\titlepage}

\begin{frame}
\frametitle{}
\begin{itemize}
\item Лисица Никита Игоревич
\item \nolinkurl{lisyarus@gmail.com}
\item \nolinkurl{+7 952 276 70 50}
\end{itemize}
\end{frame}

\begin{frame}
\frametitle{Как устроен курс}
\href{https://github.com/lisyarus/graphics-course-slides/tree/master/2022/pdf}{\nolinkurl{github.com/lisyarus/graphics-course-slides/2022/pdf}}
\href{https://github.com/lisyarus/graphics-course-practice/tree/master/2022/pdf}{\nolinkurl{github.com/lisyarus/graphics-course-practice/2022}}
\pause
\begin{itemize}
\item Лекции (слайды в \href{https://github.com/lisyarus/graphics-course-slides/tree/master/2022/pdf}{github-репозитории})
\pause
\item Практики:
\pause
\begin{itemize}
\item Код-заготовка на C++ (в \href{https://github.com/lisyarus/graphics-course-practice/tree/master/2022}{github-репозитории})
\pause
\item Слайды с заданием (в \href{https://github.com/lisyarus/graphics-course-slides/tree/master/2022/pdf}{github-репозитории})
\pause
\item Сдача на занятии или отправкой кода
\end{itemize}
\pause
\item Домашние задания:
\pause
\begin{itemize}
\item Слайды с заданием (в \href{https://github.com/lisyarus/graphics-course-slides/tree/master/2022/pdf}{github-репозитории})
\pause
\item Сдача на практическом занятии 
\end{itemize}
\pause
\item Финальный проект
\begin{itemize}
\item В относительно свободной форме
\pause
\item Сдача в конце курса
\end{itemize}
\end{itemize}
\end{frame}

\begin{frame}
\frametitle{Баллы}
\begin{itemize}
\pause
\item ~25 баллов -- работа на практиках
\pause
\begin{itemize}
\item 1 балл -- получилось хоть что-нибудь
\item 2 балл -- получилось всё
\item 3 балла -- получилось всё + доп. задание
\pause
\item Можно прислать в течение 3х дней после практики (до 24:00 среды)
\pause
\item Можно прислать позже со штрафом в 1 балл
\end{itemize}
\pause
\item ~15 баллов -- каждое из 3х домашних заданий
\pause
\begin{itemize}
\item Можно получить неполный балл
\pause
\item Можно сдать после дня сдачи со штрафом в 50\% баллов
\end{itemize}
\pause
\item 50+ баллов -- финальный проект
\end{itemize}
\end{frame}

\begin{frame}
\frametitle{Оценка за курс}
\pause
\begin{itemize}
\item Зачет: 50 и более баллов
\pause
\item Экзамен:
\begin{itemize}
\item 50-59 баллов: E
\item 60-69 баллов: D
\item 70-79 баллов: C
\item 80-89 баллов: B
\item 90-100 баллов: A
\end{itemize}
\end{itemize}
\end{frame}

\begin{frame}
\frametitle{Пререквезиты}
\pause
\begin{itemize}
\item Программирование
\pause
\begin{itemize}
\item Основы C++
\pause
\item Компилировать и запускать программы в удобной вам среде
\end{itemize}
\pause
\item Математика
\pause
\begin{itemize}
\item Линейная алгебра (векторы, матрицы, умножение матриц, линейные системы, ортогональность)
\pause
\item Аналитическая геометрия (координаты, уравнения кривых и поверхностей)
\end{itemize}
\end{itemize}
\end{frame}

\begin{frame}<1-2>[label=what_is]
\frametitle{Что такое компьютерная графика?}
\begin{itemize}
\pause % 1
\item Кинематограф, мультипликация
\pause % 2
\item Компьютерные игры
\pause % 3
\item Рисунки, concept art
\pause % 4
\item Графический интерфейс
\pause % 5
\item Визуализация данных
\pause % 6
\item Научная визуализация
\pause % 7
\item Карты
\pause % 8
\item И т.д.
\end{itemize}
\end{frame}

\begin{frame}
\frametitle{The Matrix Revolutions (2003)}
\begin{figure}
\slideimage{matrix.jpg}
\end{figure}
\end{frame}

\begin{frame}
\frametitle{Avatar (2009)}
\begin{figure}
\slideimage{avatar.jpg}
\end{figure}
\end{frame}

\begin{frame}
\frametitle{The Avengers (2012)}
\begin{figure}
\slideimage{avengers.jpg}
\end{figure}
\end{frame}

\begin{frame}
\frametitle{Klaus (2019)}
\begin{figure}
\slideimage{klaus.jpg}
\end{figure}
\end{frame}

\againframe<3>{what_is}

\begin{frame}
\frametitle{Space Invaders (1978)}
\begin{figure}
\slideimage{space-invaders.jpg}
\end{figure}
\end{frame}

\begin{frame}
\frametitle{Doom (1993)}
\begin{figure}
\slideimage{doom.png}
\end{figure}
\end{frame}

\begin{frame}
\frametitle{Grand Theft Auto: Vice City (2002)}
\begin{figure}
\slideimage{vice-city.jpg}
\end{figure}
\end{frame}

\begin{frame}
\frametitle{Civilization V (2010)}
\begin{figure}
\slideimage{civ-5.jpg}
\end{figure}
\end{frame}

\begin{frame}
\frametitle{The Witcher 3: Wild Hunt (2015)}
\begin{figure}
\slideimage{witcher.jpg}
\end{figure}
\end{frame}

\begin{frame}
\frametitle{Cyberpunk 2077 (2020)}
\begin{figure}
\slideimage{cyberpunk.jpg}
\end{figure}
\end{frame}

\againframe<4>{what_is}

\begin{frame}
\begin{figure}
\slideimage{night.png}
\end{figure}
\end{frame}

\begin{frame}
\begin{figure}
\slideimage{tunnel.png}
\end{figure}
\end{frame}

\begin{frame}
\begin{figure}
\slideimage{forest.jpg}
\end{figure}
\end{frame}

\againframe<5>{what_is}

\begin{frame}
\frametitle{Mac OS Catalina}
\begin{figure}
\slideimage{macos.png}
\end{figure}
\end{frame}

\begin{frame}
\frametitle{Windows 10}
\begin{figure}
\slideimage{windows.png}
\end{figure}
\end{frame}

\begin{frame}
\frametitle{Europa Universalis 4}
\begin{figure}
\slideimage{eu4.png}
\end{figure}
\end{frame}

\begin{frame}
\frametitle{Dear ImGui}
\begin{figure}
\slideimage{imgui.png}
\end{figure}
\end{frame}

\againframe<6>{what_is}

\begin{frame}
\frametitle{Популярность браузеров в 2002-2009}
\begin{figure}
\slideimage{browsers.png}
\end{figure}
\end{frame}

\begin{frame}
\frametitle{Карта землетрясений}
\begin{figure}
\slideimage{earthquakes.png}
\end{figure}
\end{frame}

\begin{frame}
\frametitle{Количество случаев заражения COVID-19}
\begin{figure}
\slideimage{covid.png}
\end{figure}
\end{frame}

\againframe<7>{what_is}

\begin{frame}
\frametitle{Неустойчивость Рэлея — Тейлора}
\begin{figure}
\slideimage{fluids.jpg}
\end{figure}
\end{frame}

\begin{frame}
\frametitle{Молекулярные орбитали бензола}
\begin{figure}
\slideimage{benzene.png}
\end{figure}
\end{frame}

\begin{frame}
\frametitle{Симуляция напряжений в стенте методом конечных элементов}
\begin{figure}
\slideimage{stent.jpg}
\end{figure}
\end{frame}

\againframe<8>{what_is}

\begin{frame}
\frametitle{Схематическая карта}
\begin{figure}
\slideimage{map.png}
\end{figure}
\end{frame}

\begin{frame}
\frametitle{Спутниковая карта}
\begin{figure}
\slideimage{satellite.png}
\end{figure}
\end{frame}

\begin{frame}
\frametitle{Карта погоды}
\begin{figure}
\slideimage{weather.png}
\end{figure}
\end{frame}

\againframe<9>{what_is}

\begin{frame}<1-2>[label=classification]
\frametitle{Грубая и неточная классификация}
\pause
\begin{itemize}
\item \only<2>{2D / 3D} \pause \only<3->{2D / 2.5D / 3D}
\pause
\item Векторная / растровая
\pause
\item \only<5>{Realtime / offline} \pause \only<6->{Realtime / near real-time / offline}
\pause
\item Фотореалистичная / стилизованная
\pause
\item CPU / GPU
\end{itemize}
\end{frame}

\begin{frame}
\frametitle{Super Mario Bros. (1983) - 2D}
\begin{figure}
\slideimage{mario.jpg}
\end{figure}
\end{frame}

\begin{frame}
\frametitle{Red Dead Redemption 2 (2018) - 3D}
\begin{figure}
\slideimage{rdr2.jpg}
\end{figure}
\end{frame}

\againframe<2-3>{classification}

\begin{frame}
\frametitle{Civilization III (2001) - 2.5D}
\begin{figure}
\slideimage{civ3.png}
\end{figure}
\end{frame}

\againframe<3-4>{classification}

\begin{frame}
\frametitle{Векторная графика}
\begin{figure}
\slideimage{vector.jpg}
\end{figure}
\end{frame}

\begin{frame}
\frametitle{Растровая графика}
\begin{figure}
\slideimage{raster.png}
\end{figure}
\end{frame}

\againframe<4->{classification}

\begin{frame}
\frametitle{Чем мы будем заниматься?}
\begin{itemize}
\item \only<-1>{2D / 2.5D / 3D}\only<2->{{\color{blue}\underline{2D / 2.5D / 3D}}}
\item \only<-2>{Векторная / растровая}\only<3->{{\color{blue}\underline{Векторная / растровая}}}
\item \only<-3>{Realtime}\only<4->{{\color{blue}\underline{Realtime}}} / near real-time / offline
\item \only<-4>{Фотореалистичная / стилизованная}\only<5->{{\color{blue}\underline{Фотореалистичная / стилизованная}}}
\item CPU / \only<-5>{GPU}\only<6->{{\color{blue}\underline{GPU}}}
\end{itemize}
\end{frame}

\begin{frame}
\frametitle{Чем мы будем заниматься?}
\begin{itemize}
\pause
\item Как реализовывать графические движки
\pause
\item Как реализовывать графические эффекты
\pause
\item Как их оптимизировать
\pause
\item Graphics engineer
\end{itemize}
\end{frame}

\begin{frame}
\frametitle{Где это пригодится?}
\begin{itemize}
\pause
\item Разработка игр и движков
\pause
\item Разработка инструментов для художников/дизайнеров/архитекторов/etc
\pause
\item Разработка инструментов для научной визуализации/визуализации данных/etc
\pause
\item Разработка картографических приложений
\pause
\item Разработка графического интерфейса
\pause
\item И т.д.
\end{itemize}
\end{frame}

\begin{frame}
\frametitle{Чем мы \textbf{не} будем заниматься?}
\begin{itemize}
\pause
\item Учиться рисовать / моделировать и анимировать объекты / etc.
\pause
\begin{itemize}
\item Красивая картинка -- движок + данные (текстуры, модели, частицы, etc, -- assets)
\item Курс про \textit{движок}
\end{itemize}
\pause
\item Делать игры
\begin{itemize}
\item Игра -- гейм-дизайн + контент + графика + физика + механики + UI + аудио + сетевые компоненты + ...
\item Курс про \textit{графику}
\end{itemize}
\end{itemize}
\end{frame}

\begin{frame}
\frametitle{Примерный план курса}
\pause
\begin{itemize}
\item Основы OpenGL
\begin{itemize}
\item Как хранить данные на GPU
\item Как рисовать эти данные
\item Вершинные буферы, шейдеры, текстуры, фреймбуферы
\item Работа с камерой, перспективная проекция
\end{itemize}
\pause
\item Освещение
\begin{itemize}
\item Теория
\item Модели освещения и материалов
\item Тени, отражения, ambient occlusion
\item Обработка большого количества источников света
\end{itemize}
\pause
\item Эффекты и оптимизации
\begin{itemize}
\item Системы частиц (e.g. дым)
\item Скелетная анимация
\item Уровни детализации, frustum culling
\item Объёмный (volumetric) рендеринг
\item Рендеринг текста
\end{itemize}
\end{itemize}
\end{frame}

\begin{frame}<1>[label=history-1]
\frametitle{Краткая история real-time компьютерной графики}
\centerline{1960-е: Осциллографы}
\pause
\begin{itemize}
\item Tennis For Two (1958)
\pause
\item Spacewar! (1962, PDP-1)
\pause
\item Sketchpad (1963, TX-2, световое перо)
\pause
\item Освещение и тени (Аппель, 1968)
\pause
\item Векторная графика довольно плохого качества: линии одного цвета
\item Компьютеры довольно слабые (TX-2 занимал большую комнату)
\end{itemize}
\end{frame}

\begin{frame}
\frametitle{Осциллограф}
\slideimage{oscilloscope.jpg}
\end{frame}

\againframe<1-2>{history-1}

\begin{frame}
\frametitle{Tennis For Two}
\slideimage{tennis-for-two.jpg}
\end{frame}

\againframe<2-3>{history-1}

\begin{frame}
\frametitle{PDP-1}
\slideimage{pdp1.jpg}
\end{frame}

\begin{frame}
\frametitle{Spacewar!}
\slideimage{spacewar.jpg}
\end{frame}

\againframe<3-4>{history-1}

\begin{frame}
\frametitle{Sketchpad}
\slideimage{sketchpad.jpg}
\href{https://archive.org/details/AlanKeyD1987}{\nolinkurl{archive.org/details/AlanKeyD1987}}
\end{frame}

\againframe<4-5>{history-1}

\begin{frame}
\frametitle{Освещение (Аппель, 1968)}
\slideimage{appel-shading.png}
\href{https://graphics.stanford.edu/courses/Appel.pdf}{\nolinkurl{graphics.stanford.edu/courses/Appel.pdf}}
\end{frame}

\againframe<5->{history-1}

\begin{frame}<1-2>[label=history-2]
\frametitle{Краткая история real-time компьютерной графики}
\centerline{1970-е: Аркадные игры}
\pause
\begin{itemize}
\item Magnavox Odyssey (Magnavox, 1972) -- первая игровая консоль, подключалась к телевизору (CRT)
\pause
\item Pong (Atari, 1972) -- одна из первых аркадных игр
\pause
\item Speed Race (Taito, 1974)
\pause
\item Gun Fight (Taito, 1975)
\pause
\item Space Invaders (Taito, 1978)
\pause
\item Pac-Man (Namco, 1980)
\pause
\item Переход к растровой графике
\pause
\item Разрешение экрана ограничено объёмами памяти
\end{itemize}
\end{frame}

\begin{frame}
\frametitle{Magnavox Odyssey}
\slideimage{magnavox.jpg}
\end{frame}

\againframe<2-3>{history-2}

\begin{frame}
\frametitle{Pong}
\slideimage{pong-cabinet.jpg}
\end{frame}

\begin{frame}
\frametitle{Pong}
\slideimage{pong.png}
\end{frame}

\againframe<3-4>{history-2}

\begin{frame}
\frametitle{Speed Race}
\slideimage{speed-race.png}
\end{frame}

\againframe<4-5>{history-2}

\begin{frame}
\frametitle{Gun Fight}
\slideimage{gun-fight.png}
\end{frame}

\againframe<5-6>{history-2}

\begin{frame}
\frametitle{Space Invaders}
\slideimage{space-invaders.jpg}
\end{frame}

\againframe<6-7>{history-2}

\begin{frame}
\frametitle{Pac-Man}
\slideimage{pac-man.png}
\end{frame}

\begin{frame}
\frametitle{Space Invaders}
\pause
\begin{itemize}
\item Intel 8080, 8-bit, 2Mhz
\pause
\item Экран 256x224, монохромный (1-bit)
\pause
\item 8Kb ROM
\pause
\item 8Kb RAM, из которых 7Kb занимал экран (framebuffer)
\end{itemize}
\end{frame}

\againframe<7->{history-2}

\begin{frame}<1-2>[label=history-3]
\frametitle{Краткая история real-time компьютерной графики}
\centerline{1980-е: Векторные аркады}
\pause
\begin{itemize}
\item Asteroids (Atari, 1979)
\pause
\item Tempest (Atari, 1981)
\pause
\item Star Wars (Atari, 1983)
\pause
\item Требует специального оборудования (Atari's QuadraScan)
\pause
\item Может рисовать только линии
\end{itemize}
\end{frame}

\begin{frame}
\frametitle{Asteroids}
\slideimage{asteroids.png}
\end{frame}

\againframe<2-3>{history-3}

\begin{frame}
\frametitle{Tempest}
\slideimage{tempest.png}
\href{https://www.youtube.com/watch?v=eJVpYL44jUQ}{Arcade Machines look WEIRD in Slow Mo - The Slow Mo Guys}
\end{frame}

\againframe<3-4>{history-3}

\begin{frame}
\frametitle{Star Wars}
\slideimage{star-wars.png}
\end{frame}

\againframe<4->{history-3}

\begin{frame}<1-2>[label=history-4]
\frametitle{Краткая история real-time компьютерной графики}
\centerline{1980-е: 8-битные спрайтовые консоли}
\pause
\begin{itemize}
\item Atari 2600 (Atari, 1977)
\pause
\item NES (Nintendo, 1983)
\pause
\item Рисуют готовые изображения (спрайты) в указанных частях экрана
\pause
\item Не так требовательны к объёмам памяти
\end{itemize}
\end{frame}

\begin{frame}
\frametitle{Atari 2600}
\slideimage{atari-2600.jpg}
\end{frame}

\begin{frame}
\frametitle{Donkey Kong (Nintendo, 1981)}
\slideimage{donkey-kong.png}
\end{frame}

\begin{frame}
\frametitle{Pitfall! (Activision, 1982)}
\slideimage{pitfall.png}
\end{frame}

\againframe<2-3>{history-4}

\begin{frame}
\frametitle{Super Mario Bros. (Nintendo, 1985)}
\slideimage{mario.png}
\end{frame}

\begin{frame}
\frametitle{The Legend of Zelda. (Nintendo, 1986)}
\slideimage{zelda.png}
\end{frame}

\againframe<3->{history-4}

\begin{frame}<1-2>[label=history-5]
\frametitle{Краткая история real-time компьютерной графики}
\centerline{Конец 1980-х: 16-битные консоли и персональные компьютеры}
\pause
\begin{itemize}
\item Sega Mega Drive (Sega, 1988)
\pause
\item Super NES (Nintendo, 1990)
\pause
\item Больше памяти, быстрее процессоры
\pause
\item Поддерживают больше спрайтов и цветов
\end{itemize}
\end{frame}

\begin{frame}
\frametitle{Sonic the Hedgehog (Sega, 1991)}
\slideimage{sonic.png}
\end{frame}

\againframe<2-3>{history-5}

\begin{frame}
\frametitle{Super Mario World (Nintendo, 1990)}
\slideimage{super-mario.jpg}
\end{frame}

\againframe<3->{history-5}

\begin{frame}<1-3>[label=history-6]
\frametitle{Краткая история real-time компьютерной графики}
\centerline{1990-е: Raycasting}
\pause
\begin{itemize}
\item Алгоритм рисования двумерных уровней в 3D
\pause
\item Wolfenstein 3D (id Software, 1992)
\pause
\item Doom (id Software, 1993)
\pause
\item Quake (id Software, 1996)
\end{itemize}
\end{frame}

\begin{frame}
\frametitle{Wolfenstein 3D}
\slideimage{wolfenstein.png}
\end{frame}

\againframe<3-4>{history-6}

\begin{frame}
\frametitle{Doom}
\slideimage{doom-1993.png}
\end{frame}

\againframe<4-5>{history-6}

\begin{frame}
\frametitle{Quake}
\slideimage{quake.png}
\end{frame}

\begin{frame}<1-5>[label=history-7]
\frametitle{Краткая история real-time компьютерной графики}
\centerline{1990-е: 32-битные консоли и компьютеры}
\pause
\begin{itemize}
\item Графические процессоры с 3D графикой (Sony Playstation, Sega Saturn, Nintendo 64)
\pause
\item Умеют рисовать полигоны с текстурами и примитивным освещением (порядка нескольких тысяч за кадр)
\pause
\item Разрешения экрана до 640x480, 24-bit цвет
\pause
\item Virtua Racing (Sega, 1992)
\pause
\item Tomb Raider (Core Design, 1996)
\pause
\item Crash Bandicoot (Naughty Dog, 1996)
\pause
\item Первые 3D API (OpenGL, DirectX)
\end{itemize}
\end{frame}

\begin{frame}
\frametitle{Virtua Racing}
\slideimage{racing.png}
\end{frame}

\againframe<5-6>{history-7}

\begin{frame}
\frametitle{Tomb Raider}
\slideimage{tomb-raider.png}
\end{frame}

\againframe<6-7>{history-7}

\begin{frame}
\frametitle{Crash Bandicoot}
\slideimage{crash.jpg}
\end{frame}

\begin{frame}
\frametitle{Краткая история real-time компьютерной графики}
\centerline{2000-е и 2010-е}
\pause
\begin{itemize}
\item Растут мощности как CPU, так и GPU
\pause
\item Растут доступные объёмы памяти
\pause
\item Новые возможности GPU: шейдеры, рендеринг в текстуру, тесселяция
\pause
\item Фотореалистичная графика
\end{itemize}
\end{frame}

\begin{frame}
\frametitle{Ghost of Tsushima (Sucker Punch Productions, 2020)}
\slideimage{ghost.jpg}
\end{frame}

\begin{frame}
\frametitle{О треугольниках}
Почему основным примитивом рисования стал треугольник?
\pause

Более сложные геометрические фигуры (круг, многоугольник, и т.д.):
\pause
\begin{itemize}
\item Сложно нарисовать на экране
\pause
\item Сложно интерполировать атрибуты (накладывать цвет и текстуру, вычислять освещение)
\end{itemize}
\end{frame}

\begin{frame}
\frametitle{О треугольниках}
Плюсы треугольника:
\pause
\begin{itemize}
\item Образ под действием перспективной проекции -- тоже треугольник
\pause
\item Есть единственный разумный способ интерполяции (линейная, с барицентрическими координатами)
\pause
\item Позволяет рисовать спрайты (прямоугольник -- два треугольника)
\pause
\item Позволяет рисовать многоугольники (посредством триангуляции)
\pause
\item Позволяет рисовать линии (превращая их в тонкие многоугольники)
\pause
\item Позволяет рисовать более сложные фигуры (аппроксимируя)
\end{itemize}
\end{frame}

\begin{frame}
\frametitle{Как использовать GPU? \only<2->{Графические API}}
GPU -- Graphics Processing Unit
\pause
\pause
\begin{itemize}
\item Вендор-специфичные API (1980е -- 1990е)
\pause
\item OpenGL (Silicon Graphics, 1992)
\pause
\begin{itemize}
\item \only<-11>{OpenGL 3.3 (Khronos Group, 2010)}\only<12->{\color{blue}\underline{OpenGL 3.3 (Khronos Group, 2010)}}
\end{itemize}
\pause
\item DirectX (Microsoft, 1995)
\pause
\begin{itemize}
\item {\only<11->{\color{red}}DirectX 12 (Microsoft, 2015)}
\end{itemize}
\pause
\item {\only<11->{\color{red}}Metal (Apple, 2014)}
\pause
\item {\only<11->{\color{red}}Vulkan (Khronos Group, 2018)}
\pause
\item {\only<11->{\color{red}}WebGPU (W3C, в разработке)}
\end{itemize}
\pause
\bigskip
{\color{red}Современные API}
\end{frame}

\begin{frame}
\frametitle{Как использовать GPU? API общего назначения (GPGPU):}
GPGPU -- General-Purpose Graphics Processing Unit
\pause
\begin{itemize}
\item CUDA (Nvidia, 2007)
\pause
\item DirectX 11 DirectCompute (Microsoft, 2008)
\pause
\item OpenCL (Khronos Group, 2009)
\pause
\item OpenGL 4.3 Compute Shaders (Khronos Group, 2012)
\pause
\item Metal Compute Shaders (Apple, 2014)
\pause
\item Vulkan Compute Shaders (Khronos Group, 2018)
\end{itemize}
\end{frame}

\begin{frame}
\frametitle{Почему OpenGL 3.3?}
\pause
\begin{itemize}
\item Широкая поддержка: интегрированные GPU, встраиваемые устройства, телефоны, web, некоторые игровые приставки
\pause
\item Поддерживает всё, что нам нужно
\pause
\item +/- Кроссплатформенность \pause (спасибо, Apple)
\pause
\item Низкий порог вхождения (в сравнении с более современными API)
\pause
\item Достаточно старый
\begin{itemize}
\item Много вспомогательных библиотек
\item Известны best practices
\item Известны все грабли \pause (их много)
\end{itemize}
\end{itemize}
\end{frame}

\begin{frame}
\frametitle{Почему OpenGL 3.3?}
\pause
\begin{itemize}
\item Крупные движки переписывают на Vulkan / DirectX 12
\pause
\item Не у всех есть на это ресурсы
\pause
\item Не всем нужна самая крутая графика и производительность (работает в 60 fps -- и ладно)
\pause
\item OpenGL всё ещё широко используется
\pause
\item Все основные концепции OpenGL (шейдеры, атрибуты вершин, буферы с данными, текстуры, ...) есть в любом графическом API $\Rightarrow$ изучение OpenGL поможет в изучении более современных API
\end{itemize}
\end{frame}

\begin{frame}[fragile]
\frametitle{История графических API: OpenGL 1.0 (1992)}
\begin{verbatim}
for object in scene.objects:
    glBegin(GL_TRIANGLES)
    for triangle in object.triangles:
        for vertex in triangle.vertices:
            glColor3f(vertex.color)
            glNormal3f(vertex.normal)
            glVertex3f(vertex.position)
    glEnd(GL_TRIANGLES)
\end{verbatim}
\pause
\begin{itemize}
\item Данные хранятся в памяти CPU
\pause
\item Несколько OpenGL-вызовов на каждую вершину
\pause
\begin{itemize}
\item GPU становятся быстрее $\Longrightarrow$ основное время тратится не на рисование, а на накладные расходы самих OpenGL-вызовов
\end{itemize}
\end{itemize}
\end{frame}

\begin{frame}[fragile]
\frametitle{История графических API: OpenGL 1.0 (1992)}
\begin{itemize}
\item Как менять положение объектов и камеры?
\pause
\item $\Longrightarrow$ Матрицы преобразований (\verb|glMatrixMode(), glLoadMatrix()|)
\pause
\item Fixed-function pipeline: настраиваемая, но не расширяемая последовательность операций (применить матрицы к входным данным, нарисовать треугольник на экране, выполнить тест глубины, ...)
\pause
\item Асинхронный API: команды выполнятся на GPU когда-нибудь
\end{itemize}
\end{frame}

\begin{frame}[fragile]
\frametitle{История графических API: OpenGL 1.1 (1997)}
\begin{itemize}
\item Vertex array -- спецификация формата и расположения вершин
\begin{itemize}
\item Сказать, где находятся вершины одной командой \verb|glVertexPointer|
\item Нарисовать все вершины одной командой \verb|glDrawArrays|
\item Вершины всё ещё хранятся на CPU
\end{itemize}
\pause
\item Текстуры -- изображения в памяти GPU, натягиваемые на полигоны
\end{itemize}
\end{frame}

\begin{frame}[fragile]
\frametitle{История графических API: OpenGL 1.1 (1997)}
\begin{verbatim}
// на старте
for object in scene.objects:
    object.createVertexArray(object.vertices)
    object.createTexture()

// при рендеринге
for object in scene.objects:
    glBindTexture(GL_TEXTURE_2D, object.texture)
    glBindVertexArray(object.vertexArray)
    glDrawArrays(object.vertexCount)
\end{verbatim}
\end{frame}

\begin{frame}
\frametitle{История графических API: OpenGL 1.2 - 1.4 (1998 - 2002)}
\begin{itemize}
\item В текстурах можно записать очень много интересного, помимо цвета: normal map, material map, bump map
\item Хочется выполнять сложные вычисления на каждый пиксель
\pause
\item $\Longrightarrow$ Texture environments -- зачатки программируемости GPU (шейдеров)
\end{itemize}
\end{frame}

\begin{frame}[fragile]
\frametitle{История графических API: OpenGL 1.5 (2003)}
\begin{itemize}
\item Vertex buffer -- возможность хранить вершины в памяти GPU
\end{itemize}
\pause
\begin{verbatim}
// на старте
for object in scene.objects:
    object.createVertexBuffer(object.vertices)
    object.createVertexArray(object.vertexBuffer)
    object.createTexture()

// при рендеринге
for object in scene.objects:
    glBindTexture(GL_TEXTURE_2D, object.texture)
    glBindVertexArray(object.vertexArray)
    glDrawArrays(object.vertexCount)
\end{verbatim}
\end{frame}

\begin{frame}
\frametitle{История графических API: OpenGL 2.0 (2004)}
\begin{itemize}
\item Шейдеры -- программы на C-подобном языке GLSL, компилируемые под конкретную GPU
\item Заменяют fixed-function pipeline
\begin{itemize}
\item Необходимые части fixed-function pipeline остаются (растеризация, тест глубины, etc)
\end{itemize}
\end{itemize}
\end{frame}

\begin{frame}[fragile]
\frametitle{История графических API: OpenGL 2.0 (2004)}
\begin{verbatim}
// на старте
for object in scene.objects:
    object.createVertexBuffer(object.vertices)
    object.createVertexArray(object.vertexBuffer)
    object.createTexture()
    object.createShaderProgram()

// при рендеринге
for object in scene.objects:
    glBindTexture(GL_TEXTURE_2D, object.texture)
    glBindVertexArray(object.vertexArray)
    glUseProgram(object.shaderProgram)
    glDrawArrays(object.vertexCount)
\end{verbatim}
\pause
\begin{itemize}
\item Примерно так будет выглядеть наш код
\end{itemize}
\end{frame}

\begin{frame}[fragile]
\frametitle{История графических API: OpenGL 3.0 (2008)}
\begin{itemize}
\item Огромная часть API объявлена deprecated
\pause
\begin{itemize}
\item Immediate-mode рисование -- \verb|glBegin/glEnd|
\item Хранение данных на CPU -- \verb|glVertexPointer, ...|
\item Матрицы преобразований -- \verb|glLoadMatrix, ...|
\end{itemize}
\pause
\item Transform feedback -- возможность записать результат работы шейдеров обратно в вершинный буфер
\begin{itemize}
\item Зачатки GPGPU
\end{itemize}
\end{itemize}
\end{frame}

\begin{frame}
\frametitle{История графических API: OpenGL 3.1 (2009)}
\begin{itemize}
\item Объявленные deprecated возможности удалены
\pause
\item Instanced rendering -- нарисовать много копий одного объекта в разных местах одной командой
\end{itemize}
\end{frame}

\begin{frame}
\frametitle{История графических API: OpenGL 3.2 (2009)}
\begin{itemize}
\item Механизм профилей
\begin{itemize}
\item Core profile
\begin{itemize}
\item Обязан поддерживаться
\item Только функционал конкретной версии OpenGL
\end{itemize}
\item Compatibility profile
\begin{itemize}
\item Не обязан поддерживаться
\item Функционал этой и всех предыдущих версий OpenGL
\end{itemize}
\pause
\item Мы будем использовать core profile
\end{itemize}
\pause
\item Геометрические шейдеры -- возможность менять тип геометрии и количество вершин на лету (используются для систем частиц, травы, etc)
\end{itemize}
\end{frame}

\begin{frame}
\frametitle{История графических API: OpenGL 4.0 (2010)}
\begin{itemize}
\item Шейдеры тесселяции -- увеличивают детализацию геометрии на лету
\begin{itemize}
\item Гораздо меньше возможностей, чем у геометрических шейдеров, зато быстрее
\end{itemize}
\pause
\item Indirect drawing -- можно вычислять количество вершин и их расположение в памяти на лету на GPU, и использовать вычисленные значения для команд рисования
\end{itemize}
\end{frame}

\begin{frame}
\frametitle{История графических API: OpenGL 4.1 - 4.7 (2010 - 2017)}
\begin{itemize}
\item Compute shaders -- настоящее GPGPU внутри OpenGL
\item Проработка и детализация API
\item Атомарные операции в шейдерах
\item Вливание расширений в стандарт OpenGL
\item \href{https://khronos.org/opengl/wiki/History_of_OpenGL}{\nolinkurl{khronos.org/opengl/wiki/History\_of\_OpenGL}}
\end{itemize}
\end{frame}

\begin{frame}
\frametitle{История графических API: Vulkan 1.1 (2018)}
\begin{itemize}
\item 700 строк кода, чтобы нарисовать один треугольник
\pause
\item Крайне низкоуровневый API
\item Последовательности команд для выполнения на GPU (command queues) в явном виде
\begin{itemize}
\item Можно распараллелить генерацию command queues на несколько CPU
\end{itemize}
\pause
\item Похож на DirectX 12, Metal
\item \href{https://vulkan-tutorial.com}{\nolinkurl{vulkan-tutorial.com}}
\end{itemize}
\end{frame}

\begin{frame}
\frametitle{Разновидности OpenGL}
\begin{itemize}
\item OpenGL
\pause
\item OpenGL ES (Embedded Systems)
\begin{itemize}
\item OpenGL ES 1.0 $\approx$ OpenGL 1.3
\item OpenGL ES 2.0 $\approx$ OpenGL 2.0
\item OpenGL ES 3.0 $\approx$ OpenGL 3.0
\end{itemize}
\pause
\item WebGL
\begin{itemize}
\item WebGL 1.0 $\approx$ OpenGL ES 2.0
\item WebGL 2.0 $\approx$ OpenGL ES 3.0
\end{itemize}
\pause
\item OpenGL SC (Safety Critical)
\begin{itemize}
\item Убраны любые способы отстрелить себе ногу, в ущерб производительности
\end{itemize}
\end{itemize}
\end{frame}

\begin{frame}
\frametitle{Что такое OpenGL?}
\begin{itemize}
\item {\color{red}Не} библиотека
\item Спецификация API (документ) на языке C
\begin{itemize}
\item Описание констант-перечислений (тэгов)
\item Описание сигнатур функций и их семантики
\end{itemize}
\end{itemize}
\end{frame}

\begin{frame}[fragile]
\frametitle{Что такое реализация OpenGL?}
\pause
\begin{itemize}
\item Заголовочный файл, поставляемый системой или драйвером
\begin{itemize}
\item Определение типов, e.g. \verb|typedef unsigned int GLenum;|
\item Определение констант, e.g. \verb|#define GL_TEXTURE_2D 0x0DE1|
\item Объявление функций, e.g. \verb|void glBindTexture(GLenum target, GLuint texture);|
\end{itemize}
\pause
\item Бинарная реализация объявленных функций (обычно - динамическая библиотека), поставляемая системой и/или драйвером
\begin{itemize}
\item Может содержать непосредственную реализацию OpenGL как часть драйвера и общаться с GPU
\item Может быть промежуточным звеном, маршрутизирующим вызов до драйвера
\item Может быть заглушкой
\end{itemize}
\end{itemize}
\end{frame}

\begin{frame}[fragile]
\frametitle{Что такое реализация OpenGL?}
\begin{itemize}
\item Заголовочный файл
\begin{itemize}
\item Linux: \verb|GL/gl.h| - до OpenGL 1.3
\pause
\item Windows: \verb|GL/gl.h| - до OpenGL 1.1
\pause
\item MacOS: \verb|OpenGL/gl.h| - до OpenGL 2.1
\begin{itemize}
\item {\color{red}Не} \verb|OpenGL/OpenGL.h|
\end{itemize}
\item Все платформы: \verb|GL/glext.h| вместе с \verb|#define GL_GLEXT_PROTOTYPES| - до OpenGL 4.6
\end{itemize}
\end{itemize}
\end{frame}

\begin{frame}[fragile]
\frametitle{Что такое реализация OpenGL?}
\begin{itemize}
\item Динамическая библиотека
\begin{itemize}
\item Linux: \verb|libGL.so|
\item Windows: \verb|opengl32.dll|
\item MacOS: \verb|OpenGL framework|
\end{itemize}
\pause
\item Может не содержать функции всех версий OpenGL
\begin{itemize}
\item Под Linux обычно содержит
\end{itemize}
\item Остальные функции OpenGL нужно динамически загружать специфичными для платформы средствами
\item $\Longrightarrow$ Библиотеки-загрузчики OpenGL
\end{itemize}
\end{frame}

\begin{frame}[fragile]
\frametitle{Загрузчики OpenGL}
\begin{itemize}
\item С и C++ specific, для других языков обычно встроено в обёртку над OpenGL
\item \href{https://khronos.org/opengl/wiki/OpenGL_Loading_Library}{\nolinkurl{khronos.org/opengl/wiki/OpenGL\_Loading\_Library}}
\item Обычно содержат код, автоматически сгенерированный по XML-спецификации OpenGL
\item Есть мой, основанный на \verb|glLoadGen| (который перестали поддерживать): \href{https://github.com/lisyarus/opengl-loader-generator}{github.com/lisyarus/opengl-loader-generator}
\item Мы будем использовать \href{http://glew.sourceforge.net/}{\color{blue}\underline{GLEW}}
\end{itemize}
\end{frame}

\begin{frame}
\frametitle{Контекст OpenGL}
\begin{itemize}
\item Необходим для вызова любой функции OpenGL
\item Привязан к конкретной реализации OpenGL
\item Привязан к конкретным версии и профилю OpenGL
\item Привязан к экрану / окну оконной системы / изображению в памяти
\item Хранит текущее глобальное состояние OpenGL
\item \href{https://khronos.org/opengl/wiki/OpenGL_Context}{\nolinkurl{khronos.org/opengl/wiki/OpenGL\_Context}}
\item Создаётся специфичными для платформы средствами
\item $\Longrightarrow$ Библиотеки, создающие контекст OpenGL
\end{itemize}
\end{frame}

\begin{frame}
\frametitle{Библиотеки, создающие контекст OpenGL}
\begin{itemize}
\item Обычно привязывают контекст к окну и умеют обрабатывать события оконной системы
\item GLUT - устаревшая, плохой интерфейс
\item GLFW
\item {\color{blue}\underline{SDL2}} - умеет загружать изображения, выводить звук, и другое
\item \href{https://open.gl/context}{\nolinkurl{open.gl/context}}
\end{itemize}
\end{frame}

\begin{frame}[fragile]
\frametitle{Как начать работать с OpenGL?}
\begin{verbatim}
window = createWindow(title)
context = createGLContext(window, version, profile)
context.makeCurrent()
loadGLFunctions()
// тут можно работать с OpenGL!
\end{verbatim}
\end{frame}

\begin{frame}
\frametitle{Литература, ссылки}
\begin{itemize}
\item Realtime графика
\begin{itemize}
\item Computer Graphics: Principles and Practice - книжка начального уровня
\item Real-Time Rendering (4th edition) - обзор передовых алгоритмов индустрии
\item \href{https://developer.nvidia.com/gpugems}{GPU Gems 1, 2, 3} - журнал про техники и алгоритмы
\end{itemize}
\item OpenGL
\begin{itemize}
\item \href{https://khronos.org/opengl/wiki}{\nolinkurl{khronos.org/opengl/wiki}} - подробное изложение всех аспектов OpenGL
\item \href{http://docs.gl}{\nolinkurl{docs.gl}} - удобная документация по отдельным функциям
\item \href{https://learnopengl.com}{\nolinkurl{learnopengl.com}} - уроки по отдельным темам
\end{itemize}
\end{itemize}
\end{frame}

\end{document}