% (c) Nikita Lisitsa, lisyarus@gmail.com, 2023

\documentclass[10pt]{beamer}

\usepackage[T2A]{fontenc}
\usepackage[russian]{babel}
\usepackage{minted}

\usepackage{graphicx}
\graphicspath{ {./images/} }

\usepackage{adjustbox}

\usepackage{tikz}

\usepackage{color}
\usepackage{soul}

\usepackage{hyperref}

\usetheme{metropolis}
\setminted{fontsize=\footnotesize}

\definecolor{blue}{rgb}{0,0,1}
\definecolor{red}{rgb}{1,0,0}

\makeatletter
\newcommand{\slideimage}[1]{
  \begin{figure}
    \begin{adjustbox}{width=\textwidth, totalheight=\textheight-2\baselineskip-2\baselineskip,keepaspectratio}
      \includegraphics{#1}
    \end{adjustbox}
  \end{figure}
}
\makeatother

\title{Компьютерная графика}
\subtitle{Домашнее задание 1: График функции на плоскости с изолиниями}
\date{2023}

\setbeamertemplate{footline}[frame number]

\begin{document}

\frame{\titlepage}

\begin{frame}[fragile]
\frametitle{Задание}
\begin{itemize}
\item Функция на плоскости, зависящая от времени \begin{math}f(x, y, t)\end{math}
\pause
\item Фиксированный прямоугольник \begin{math}[x_0, x_1] \times [y_0, y_1]\end{math}, разбитый на сетку из \begin{math}W\times H\end{math} прямоугольных ячеек
\pause
\item Нужно нарисовать:
\begin{itemize}
\item `График' функции цветом, используя вершины сетки как основу
\pause
\item Изолинии -- линии \begin{math}f(x,y,t) = const\end{math} поверх графика
\end{itemize}
\pause
\item График и изолинии вычисляются заново на каждый кадр
\end{itemize}
\end{frame}

\begin{frame}[fragile]
\frametitle{Пример}
\slideimage{example-plot.png}
\end{frame}

\begin{frame}[fragile]
\frametitle{Функция}
Любая интересная меняющаяся во времени функция на плоскости
\begin{itemize}
\item Metaballs: \begin{math}f(x,y,t) = \sum c_i\exp\left(-\frac{(x-x_i)^2+(y-y_i)^2}{r_i^2}\right)\end{math}, где \begin{math}(x_i, y_i)\end{math} -- координаты движущейся по какому-то закону точки
\pause
\item Шум Перлина: строится на основе сетки двумерных единичных векторов, которые можно крутить в зависимости от времени (эта сетка никак не связана с сеткой использующейся для рендеринга)
\pause
\item Комбинация синусов/косинусов с разными амплитудами и фазами
\pause
\item etc.
\end{itemize}
\end{frame}

\begin{frame}[fragile]
\frametitle{График}
\begin{itemize}
\item Вершина сетки -- \begin{math}(x_i, y_j)\end{math} + цвет
\item Раскрасить в зависимости от значения функции
\pause
\item Прямоугольники сетки придётся разбить на пары треугольников
\end{itemize}
\end{frame}

\begin{frame}[fragile]
\frametitle{Изолинии}
\begin{itemize}
\item Линии \begin{math}z = f(x,y,t) = C_i\end{math} для некоторых значений \begin{math}C_i\end{math}
\item Набор значений \begin{math}C_1, C_2, C_3, \dots, C_k\end{math} выбрать на основе вашей функции
\pause
\item Строить алгоритмом \textit{marching squares} с линейной интерполяцией (или аналогичным алгоритмом на треугольниках) на основе той же сетки, что и график
\end{itemize}
\end{frame}

\begin{frame}[fragile]
\frametitle{Marching squares}
\begin{center}
\begin{tikzpicture}
\draw[black,thick,dashed] (0.0, 0.0) rectangle (4.0, 4.0);
\draw[black,thick,dashed] (6.0, 0.0) rectangle (10.0, 4.0);

\draw[black,thick] (0.0, 2.0) -- (2.0, 0.0);
\draw[black,thick] (6.0, 3.5) -- (7.0, 0.0);

\node[circle,draw=blue,fill=blue] at (0.0, 0.0) {};
\node[circle,draw=red,fill=red] at (4.0, 0.0) {};
\node[circle,draw=red,fill=red] at (0.0, 4.0) {};
\node[circle,draw=red,fill=red] at (4.0, 4.0) {};

\node[circle,draw=blue,fill=blue] at (6.0, 0.0) {};
\node[circle,draw=red,fill=red] at (10.0, 0.0) {};
\node[circle,draw=red,fill=red] at (6.0, 4.0) {};
\node[circle,draw=red,fill=red] at (10.0, 4.0) {};

\node at (0.0, -0.5) {\color{blue}\begin{math}f<C\end{math}};
\node at (4.0, -0.5) {\color{red}\begin{math}f>C\end{math}};
\node at (0.0, 4.5) {\color{red}\begin{math}f>C\end{math}};
\node at (4.0, 4.5) {\color{red}\begin{math}f>C\end{math}};

\node at (6.0, -0.5) {\color{blue}\begin{math}f<C\end{math}};
\node at (10.0, -0.5) {\color{red}\begin{math}f>C\end{math}};
\node at (6.0, 4.5) {\color{red}\begin{math}f>C\end{math}};
\node at (10.0, 4.5) {\color{red}\begin{math}f>C\end{math}};

\node[circle,draw=black,fill=black] at (0.0, 2.0) {};
\node[circle,draw=black,fill=black] at (2.0, 0.0) {};

\node[circle,draw=black,fill=black] at (6.0, 3.5) {};
\node[circle,draw=black,fill=black] at (7.0, 0.0) {};
\end{tikzpicture}
\end{center}
\end{frame}

\begin{frame}[fragile]
\frametitle{Marching squares}
\begin{itemize}
\item Есть вариант алгоритма, соединящий центры рёбер, -- его \textbf{\alert{не нужно использовать}}
\item Есть вариант алгоритма, линейно интерполирующий значение функции вдоль ребра чтобы найти точку \begin{math}f = C\end{math} -- \textbf{\alert{нужно использовать именно его}}
\pause
\item Есть вариант алгоритма (иногда называется marching triangles), использующий треугольники и избавляющийся от неоднозначностей алгоритма на квадратах, -- \textbf{\alert{можно}} использовать его
\pause
\item Ещё marching triangles называют трёхмерный алгоритм построения модели по облаку точек -- это \textbf{\alert{не то, что нам нужно}}
\end{itemize}
\end{frame}

\begin{frame}[fragile]
\frametitle{Немного деталей}
\usemintedstyle{solarized-light}
\begin{itemize}
\item Часть данных вершин будут обновляться каждый кадр -- цвета точек графика, координаты изолиний
\item Часть данных постоянна -- XY-координаты вершин сетки -- их имеет смысл хранить в отдельном VBO
\item Как для графика, так и для изолиний имеет смысл использовать индексированный рендеринг
\item Можно использовать простые \mintinline{cpp}|GL_LINES| и \mintinline{cpp}|GL_TRIANGLES|, а можно \mintinline{cpp}|GL_LINE_STRIP/GL_LINE_LOOP| и \mintinline{cpp}|GL_TRIANGLE_STRIP| вместе с \mintinline{cpp}|primitive restart|
\end{itemize}
\end{frame}

\begin{frame}[fragile]
\frametitle{Немного деталей}
\usemintedstyle{solarized-light}
\begin{itemize}
\item Чтобы избавиться от дублирования вершин в изолиниях, придётся для каждого ребра исходной сетки запомнить индекс вершины изолинии, лежащей на этом ребре: можно использовать, например, \mintinline{cpp}|std::unordered_map| или просто \mintinline{cpp}|std::vector|, придумав какую-то нумерацию для рёбер
\end{itemize}
\end{frame}

\begin{frame}[fragile]
\frametitle{Немного деталей}
\usemintedstyle{solarized-light}
\begin{itemize}
\item Обратите особое внимание на работу с VAO и буферами:
\begin{itemize}
\item Создание VAO и буферов должно быть \textbf{только} в начале программы
\item Настройка атрибутов вершин (т.е. настройка VAO) должна быть \textbf{один раз} в начале программы
\item Данные в буферы должны загружаться \textbf{только} при их реальном изменении
\end{itemize}
\pause
\item В идеале у вас должно быть 2 \textit{draw call} (вызова функции \mintinline{cpp}|glDraw*|) на кадр: один на график, один на изолинии
\end{itemize}
\end{frame}

\begin{frame}[fragile]
\frametitle{Баллы}
\footnotesize
\begin{itemize}
\item \textbf{3 балла}: рисуется динамический график функции (т.е. цвета вершин меняются во времени) за один draw call
\item \textbf{3 балла}: рисуются динамические изолинии за один draw call
\item \textbf{2 балла}: все данные в VBO обновляются только при их изменении, статические данные хранятся в отдельных VBO
\item \textbf{2 балла}: используется индексированный рендеринг для графика и его вершины не дублируются
\item \textbf{2 балла}: используется индексированный рендеринг для изолиний и их вершины не дублируются
\item \textbf{1 балл}:  можно динамически менять количество изолиний
\item \textbf{1 балл}:  можно динамически менять детализацию сетки
\item \textbf{1 балл}:  корректно обрабатывается \textit{aspect ratio}, т.е. соотношение ширины к высоте видимого графика не меняется при изменении размеров окна (свободное пространство окна можно оставить пустым), и график всегда полностью влезает в окно
\end{itemize}
Всего: \textbf{15 баллов}

Защита заданий на практике \textbf{через 2 недели}.
\end{frame}

\begin{frame}[fragile]
\frametitle{Ссылки}
\fontsize{8pt}{8pt}\selectfont
\begin{itemize}
\item \href{http://jamie-wong.com/2014/08/19/metaballs-and-marching-squares}{\nolinkurl{jamie-wong.com/2014/08/19/metaballs-and-marching-squares}}\item \href{http://jacobzelko.com/marching-squares/}{\nolinkurl{jacobzelko.com/marching-squares}}
\item \href{https://ckcollab.com/2020/11/08/Marching-Squares-Algorithm.html}{\nolinkurl{ckcollab.com/2020/11/08/Marching-Squares-Algorithm.html}}
\end{itemize}
\end{frame}

\end{document}
