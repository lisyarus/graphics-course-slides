% (c) Nikita Lisitsa, lisyarus@gmail.com, 2021

\documentclass{beamer}

\usepackage[T2A]{fontenc}
\usepackage[utf8]{inputenc}
\usepackage[russian]{babel}

\usepackage{graphicx}
\graphicspath{ {./images/} }

\usepackage{adjustbox}

\usepackage{tikz}

\usepackage{color}
\usepackage{soul}

\usepackage{hyperref}

\definecolor{blue}{rgb}{0,0,1}
\definecolor{red}{rgb}{1,0,0}

\makeatletter
\newcommand{\slideimage}[1]{
  \begin{figure}
    \begin{adjustbox}{width=\textwidth, totalheight=\textheight-2\baselineskip-2\baselineskip,keepaspectratio}
      \includegraphics{#1}
    \end{adjustbox}
  \end{figure}
}
\makeatother

\title{Компьютерная графика}
\subtitle{Домашнее задание 1: график функции на плоскости}
\date{2021}

\setbeamertemplate{footline}[frame number]

\begin{document}

\frame{\titlepage}

\begin{frame}[fragile]
\frametitle{Задание}
\begin{itemize}
\item Функция на плоскости, зависящая от времени \begin{math}f(x, y, t)\end{math}
\pause
\item Фиксированный прямоугольник \begin{math}[x_0, x_1] \times [y_0, y_1]\end{math}, разбитый на сетку из \begin{math}W\times H\end{math} прямоугольных ячеек
\pause
\item Нужно нарисовать:
\begin{itemize}
\item Трёхмерный график функции, т.е. поверхность \begin{math}z = f(x, y, t)\end{math}, используя вершины сетки как основу
\pause
\item Изолинии - линии \begin{math}f(x,y,t) = const\end{math} поверх трёхмерного графика
\end{itemize}
\pause
\item Поверхность и изолинии вычисляются заново на каждый кадр
\end{itemize}
\end{frame}

\begin{frame}[fragile]
\frametitle{Пример}
\slideimage{matlab-plot.png}
\end{frame}

\begin{frame}[fragile]
\frametitle{Функция}
Любая интересная меняющаяся во времени функция на плоскости
\begin{itemize}
\item Metaballs: \begin{math}f(x,y,t) = \sum c_i\exp\left(-\frac{(x-x_i)^2+(y-y_i)^2}{r_i^2}\right)\end{math}, где \begin{math}(x_i, y_i)\end{math} - координаты движущейся по какому-то закону точки
\pause
\item Шум Перлина: строится на основе сетки двумерных единичных векторов, которые можно крутить в зависимости от времени (эта сетка никак не связана с сеткой использующейся для рендеринга)
\pause
\item Комбинация синусов/косинусов с разными амплитудами и фазами
\pause
\item etc.
\end{itemize}
\end{frame}

\begin{frame}[fragile]
\frametitle{График}
\begin{itemize}
\item Вершина исходной сетки - \begin{math}(x_i, y_j)\end{math}
\item Вершина графика - \begin{math}(x_i, y_j, f(x_i, y_j, t))\end{math}
\item Раскрасить в зависимости от значения функции
\pause
\item Прямоугольники сетки придётся разбить на пары треугольников
\pause
\item Вместо \begin{math}z = f(x,y,t)\end{math} у вас может быть \begin{math}y = f(x,z,t)\end{math}, - это не принципиально, главное, чтобы график был над некой координатной плоскостью и менялся по вертикальной оси
\end{itemize}
\end{frame}

\begin{frame}[fragile]
\frametitle{Изолинии}
\begin{itemize}
\item Линии \begin{math}z = f(x,y,t) = C_i\end{math} для некоторых \begin{math}C_i\end{math}, разного цвета
\item Набор значений \begin{math}C_1, C_2, C_3, \dots, C_k\end{math} - выбрать на основе вашей функции
\pause
\item Строить алгоритмом marching squares (с линейной интерполяцией) на основе той же сетки, что и график
\item Изолиния \begin{math}f(x,y,t) = C\end{math} должна быть нарисована на высоте \begin{math}C\end{math}, т.е. лежать на графике
\pause
\item Чтобы не было z-fighting'а между графиком и изолиниями, можно прибавить к высоте изолинии небольшой сдвиг
\end{itemize}
\end{frame}

\begin{frame}[fragile]
\frametitle{Marching squares}
\begin{center}
\begin{tikzpicture}
\draw[black,thick,dashed] (0.0, 0.0) rectangle (4.0, 4.0);
\draw[black,thick,dashed] (6.0, 0.0) rectangle (10.0, 4.0);

\draw[black,thick] (0.0, 2.0) -- (2.0, 0.0);
\draw[black,thick] (6.0, 3.5) -- (7.0, 0.0);

\node at (0.0, 0.0) {\color{blue}\textbullet};
\node at (4.0, 0.0) {\color{red}\textbullet};
\node at (0.0, 4.0) {\color{red}\textbullet};
\node at (4.0, 4.0) {\color{red}\textbullet};

\node at (6.0, 0.0) {\color{blue}\textbullet};
\node at (10.0, 0.0) {\color{red}\textbullet};
\node at (6.0, 4.0) {\color{red}\textbullet};
\node at (10.0, 4.0) {\color{red}\textbullet};

\node at (0.0, -0.5) {\color{blue}\begin{math}f<C\end{math}};
\node at (4.0, -0.5) {\color{red}\begin{math}f>C\end{math}};
\node at (0.0, 4.5) {\color{red}\begin{math}f>C\end{math}};
\node at (4.0, 4.5) {\color{red}\begin{math}f>C\end{math}};

\node at (6.0, -0.5) {\color{blue}\begin{math}f<C\end{math}};
\node at (10.0, -0.5) {\color{red}\begin{math}f>C\end{math}};
\node at (6.0, 4.5) {\color{red}\begin{math}f>C\end{math}};
\node at (10.0, 4.5) {\color{red}\begin{math}f>C\end{math}};

\node at (0.0, 2.0) {\color{black}\textbullet};
\node at (2.0, 0.0) {\color{black}\textbullet};

\node at (6.0, 3.5) {\color{black}\textbullet};
\node at (7.0, 0.0) {\color{black}\textbullet};
\end{tikzpicture}
\end{center}
\begin{itemize}
\item Есть вариант алгоритма, соединящий центры рёбер
\item Есть вариант алгоритма, линейно интерполирующий значение функции вдоль ребра чтобы найти точку \begin{math}f = C\end{math} - {\color{red}нужно использовать его}
\end{itemize}
\end{frame}

\begin{frame}[fragile]
\frametitle{Немного деталей}
\begin{itemize}
\item Часть данных вершин будут обновляться каждый кадр - высоты точек графика, изолинии
\item Часть данных постоянна - XY-координаты вершин сетки - их имеет смысл хранить в отдельном VBO
\item Как для графика, так и для изолиний имеет смысл использовать индексированный рендеринг
\end{itemize}
\end{frame}

\begin{frame}[fragile]
\frametitle{Баллы}
\begin{itemize}
\item 3 балла: рисуется динамический график функции
\item 3 балла: рисуются динамические изолинии
\item 3 балла: можно вращать камеру вокруг графика (хотя бы вокруг вертикальной оси)
\item 2 балла: все данные в VBO обновляются только при их изменении
\item 2 балла: используется индексированный рендеринг и вершины (как графика, так и изолиний) не дублируются
\item 1 балл:  можно динамически менять количество изолиний
\item 1 балл:  можно динамически менять детализацию сетки
\end{itemize}
Всего: 15 баллов

Защита заданий на практике 12 октября
\end{frame}

\begin{frame}[fragile]
\frametitle{Ссылки}
\begin{itemize}
\item \href{http://jamie-wong.com/2014/08/19/metaballs-and-marching-squares}{jamie-wong.com/2014/08/19/metaballs-and-marching-squares}
\item \href{http://jacobzelko.com/marching-squares/}{jacobzelko.com/marching-squares}
\item \href{https://ckcollab.com/2020/11/08/Marching-Squares-Algorithm.html}{ckcollab.com/2020/11/08/Marching-Squares-Algorithm.html}
\end{itemize}
\end{frame}

\end{document}