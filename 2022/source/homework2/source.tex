% (c) Nikita Lisitsa, lisyarus@gmail.com, 2022

\documentclass{beamer}

\usepackage[T2A]{fontenc}
\usepackage[utf8]{inputenc}
\usepackage[russian]{babel}

\usepackage{graphicx}
\graphicspath{ {./images/} }

\usepackage{adjustbox}

\usepackage{tikz}

\usepackage{color}
\usepackage{soul}

\usepackage{hyperref}

\definecolor{blue}{rgb}{0,0,1}
\definecolor{red}{rgb}{1,0,0}

\makeatletter
\newcommand{\slideimage}[1]{
  \begin{figure}
    \begin{adjustbox}{width=\textwidth, totalheight=\textheight-2\baselineskip-2\baselineskip,keepaspectratio}
      \includegraphics{#1}
    \end{adjustbox}
  \end{figure}
}
\makeatother

\title{Компьютерная графика}
\subtitle{Домашнее задание 2: Визуализатор сцен с освещением и тенями}
\date{2021}

\setbeamertemplate{footline}[frame number]

\begin{document}

\frame{\titlepage}

\begin{frame}[fragile]
\frametitle{Задание}
\begin{itemize}
\item Сделать визуализатор сцен в формате OBJ с текстурами и освещением по Фонгу
\pause
\item Путь до сцены задаётся, например, параметром командной строки
\pause
\item Два источника света: направленный (`солнце') и точечный, оба как-то двигаются со временем
\pause
\item От обоих источников света есть тени, построенные алгоритмом shadow mapping (+ PCF или VSM, с размытием)
\pause
\item Камера должна управляться пользователем (любым способом, главное -- чтобы можно было всё разглядеть)
\end{itemize}
\end{frame}

\begin{frame}[fragile]
\frametitle{Сцена}
Хорошая сцена:
\begin{itemize}
\item Порядка 100k-1kk треугольников в сумме
\item Порядка 100-1000 различных объектов
\item У большинства объектов есть текстура альбедо
\end{itemize}
Рекомендую тестировать на \textit{Crytek Sponza} и \textit{Rungholt}, обе сцены можно найти здесь: \href{https://casual-effects.com/data/index.html}{\nolinkurl{casual-effects.com/data/index.html}}
\end{frame}

\begin{frame}[fragile]
\frametitle{Crytek Sponza}
\slideimage{crytek-sponza.jpg}
\end{frame}

\begin{frame}[fragile]
\frametitle{Crytek Sponza}
\slideimage{sponza_real.jpeg}
\end{frame}

\begin{frame}[fragile]
\frametitle{Crytek Sponza}
\begin{itemize}
\item Одна из самых популярных тестовых сцен в 3D графике
\item Модель атриума дворца в Дубровнике (Хорватия)
\end{itemize}
\end{frame}

\begin{frame}[fragile]
\frametitle{Rungholt}
\slideimage{rungholt.png}
\end{frame}

\begin{frame}[fragile]
\frametitle{Rungholt}
\begin{itemize}
\item Город, построенный в Minecraft, переведённый в формат OBJ
\item Достаточно большая сцена (~6 миллионов треугольников)
\item Также содержит небольшую модель дома (house.obj), на которой хорошо тестировать
\end{itemize}
\end{frame}

\begin{frame}[fragile]
\frametitle{Сцена: формат OBJ}
Wavefront OBJ -- один из общепринятых форматов сцен
\begin{itemize}
\item Текстовый, достаточно простой для чтения
\item Содержит координаты вершин, нормали и текстурные координаты
\item Содержит индексы вершин, образующих треугольники
\item Может содержать много объектов
\item Может ссылаться на MTL-файл (Material Template Library), содержащий описания материалов
\item MTL может в свою очередь ссылаться на текстуры (альбедо, нормали, и т.п.)
\end{itemize}
\end{frame}

\begin{frame}[fragile]
\frametitle{Сцена: формат OBJ}
Что нужно из описания материалов:
\begin{itemize}
\item Текстура альбедо \verb|map_Ka|
\item Тестура прозрачности \verb|map_d| (делаем discard, если в текстуре значение меньше 0.5; нужно использовать и в шейдере для shadow map)
\item Коэффициент отражения (glossiness) \verb|Ks|
\item Показатель отражения (power) \verb|Ns|
\end{itemize}
\end{frame}

\begin{frame}[fragile]
\frametitle{Сцена: формат OBJ}
\begin{itemize}
\item Можно написать загрузчик руками (это несложно)
\item Можно использовать какую-нибудь библиотеку, например Assimp или TinyOBJ
\item Для загрузки текстур можно использовать какую-нибудь библиотеку, например SDL\_image, stb\_image или Boost.GIL
\item \textbf{N.B.} Есть два соглашения о том, как идут текстурные координаты: снизу вверх, или сверху вниз. Если текстуры будут выглядеть странно, попробуйте отразить текстурные координаты по оси Y: \verb|y = 1 - y|
\end{itemize}
\end{frame}

\begin{frame}[fragile]
\frametitle{Советы}
\begin{itemize}
\item Можно иметь один общий набор VAO + VBO + EBO, и для каждого объекта хранить только диапазон индексов
\item Можно иметь по VAO + VBO + EBO на каждый объект
\item Скорее всего, у вас будет 1 draw call (\verb|glDrawElements| или т.п.) на один объект
\item Напоминание: для рендеринга VAO и EBO не нужны, только \verb|glBindVertexArray| и \verb|glDraw*|!
\end{itemize}
\end{frame}

\begin{frame}[fragile]
\frametitle{Советы}
\begin{itemize}
\item Для теней от точечного источника света можно использовать одну cubemap текстуру, а можно 6 обычных 2D текстур
\item В любом случае, в сумме у вас будет 7 FBO: один для теней от солнца и 6 для теней от точечного источника
\item Размер shadow map лучше взять побольше, если не будет тормозить (4096x4096)
\item Размывать тени можно прямо в результирующем шейдере, при чтении из shadow map
\item Радиус размытия -- как можно больше, пока не начинает тормозить :)
\end{itemize}
\end{frame}

\begin{frame}[fragile]
\frametitle{Советы}
\begin{itemize}
\item При загрузке полезно посчитать bounding box сцены, чтобы потом по нему вычислять матрицы для теней
\item Сцены могут быть разного размера \begin{math}\Rightarrow\end{math} полезно делать скорость движения камеры и движение источника света пропорциональными размерам сцены
\end{itemize}
\end{frame}

\begin{frame}[fragile]
\frametitle{Баллы}
\begin{itemize}
\item 2 балл: геометрия сцены загружается и рисуется
\item 2 балла: есть текстуры альбедо и прозрачности
\item 1 балл: камера произвольно двигается
\item 1 балл: есть ambient освещение + два источника света
\item 2 балла: есть тени от направленного источника света
\begin{itemize}
\item +1 балл: PCF + blur
\item либо +2 балла: VSM + blur
\end{itemize}
\item 3 балла: есть тени от точечного источника света
\begin{itemize}
\item +1 балл: PCF
\item либо +2 балла: VSM
\end{itemize}
\end{itemize}
Всего: 15 баллов

Защита заданий на практике 7 ноября
\end{frame}

\end{document}