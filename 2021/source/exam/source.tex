% (c) Nikita Lisitsa, lisyarus@gmail.com, 2021

\documentclass{article}

\usepackage[T2A]{fontenc}
\usepackage[utf8]{inputenc}
\usepackage[russian]{babel}

\usepackage{adjustbox}

\usepackage{color}
\usepackage{soul}

\usepackage{hyperref}

\setcounter{secnumdepth}{0}

\begin{document}

\section{Как будет проходить экзамен}

Специальная программа генерирует 5 случайных номеров вопросов; по желанию любой из выпавших вопросов вы можете заменить на другой (тоже случайный). Каждый вопрос оценивается в 10 баллов (частичный ответ может быть оценён в меньшее количество баллов). При подготовке (после получения вопросов) можно использовать любые материалы; время подготовки не ограничено (в разумных пределах).

\section{Список вопросов}

\begin{enumerate}
\subsection{Графические API}
\item \textbf{Графические API.} Какие бывают; чем отличаются; GPGPU API.
\item \textbf{История OpenGL API.} В чём основные отличия OpenGL 3.0+ от OpenGL 1.0; какие есть API, основанные на OpenGL.
\item \textbf{Технические детали OpenGL API.} Контекст OpenGL; что содержит реализация OpenGL; загрузка OpenGL-функций; расширения OpenGL; ошибки OpenGL.

\subsection{Математика}
\item \textbf{Аффинное пространство.} Что такое; зачем нужно.
\item \textbf{Аффинные комбинации, выпуклые комбинации, линейная интерполяция.} Что такое; какие свойства.
\item \textbf{Аффинные преобразования.} Что такое; что ими можно описать; формула композиции; формула обратного.
\item \textbf{Однородные координаты.} Как представляются точки и векторы; как представляются аффинные преобразования; матрицы сдвига, поворота, масштабирования.
\item \textbf{Ортографическая проекция.} Когда используется; какие особенности; как описывается; как строится матрица.
\item \textbf{Перспективная проекция.} Когда используется; какие особенности; как описывается; как строится матрица.
\item \textbf{Перевод из экранных координат в мировые.}
\item \textbf{*Вычисление видимой области.}
\item \textbf{Easing functions, сплайны.} Зачем нужны; примеры; виды сплайнов.
\item \textbf{Кватернионы.} Что такое; какие свойства; таблица умножения; формула обратного; связь с вращениями.

\subsection{Объекты OpenGL}
\item \textbf{Объекты OpenGL.} Что такое; общие операции работы с объектами; объект с нулевым ID.
\item \textbf{Буферы.} Что такое; зачем нужны; какие есть значения target; как загрузить данные; mapped buffer.
\item \textbf{Vertex array.} Что такое; зачем нужны; какие данные хранят.
\item \textbf{Текстуры.} Что такое; зачем нужны; какие есть значения target; виды текстур.
\item \textbf{Текстуры 2.} Как загрузить данные; форматы пикселей; фильтрация; wrapping; mipmaps.
\item \textbf{Фреймбуферы.} Что такое; зачем нужны; какие есть значения target; как настраиваются; framebuffer completeness; attachments.
\item \textbf{Renderbuffer'ы.} Что такое; зачем нужны; какие есть значения target.
\item \textbf{Шейдеры.} Что такое; зачем нужны; как скомпилировать.
\item \textbf{Шейдерные программы.} Что такое; зачем нужны; как слинковать.
\item \textbf{Texture units.} Что такое; как привязать текстуру к шейдеру.

\subsection{Язык шейдеров}
\item \textbf{Язык GLSL.} Синтаксис (в общих чертах); управляющие конструкции; типы данных (назвать несколько); встроенные функции (назвать несколько); принципиальные отличия от C, C++ и т.п.
\item \textbf{Uniform-переменные.} Что такое; зачем нужны; как задавать значение.
\item \textbf{Samplers.} Что такое; виды; какие есть функции для чтения текстуры из шейдера; что означают текстурные координаты для разных видов sampler'ов.

\subsection{Рендеринг в OpenGL}
\item \textbf{Графический конвейер.} Все этапы обработки данных в конвейере.
\item \textbf{Атрибуты вершин.} Какие есть форматы хранения; как указывается расположение данных атрибута; как описываются в вершинном шейдере.
\item \textbf{Индексация вершин.} Зачем нужно; где хранятся индексы; как вызвать индексированный рендеринг; primitive restart.
\item \textbf{Primitive assembly.} Что такое; какие есть примитивы; как они превращаются в линии/треугольники.
\item \textbf{Вершинный шейдер.} Зачем нужен; что принимает на вход; что выдаёт на выходе.
\item \textbf{Геометрический шейдер.} Зачем нужен; что принимает на вход; что выдаёт на выходе.
\item \textbf{Back-face culling.} Как работает; зачем нужно; как настраивается.
\item \textbf{Clipping.} Как работает; по каким уравнениям происходит отсечение.
\item \textbf{Perspective divide.} Зачем нужен; как работает.
\item \textbf{Растеризация.} Что такое; правила растеризации OpenGL; viewport.
\item \textbf{Фрагментный шейдер.} Зачем нужен; что принимает на вход; что выдаёт на выходе.
\item \textbf{Тест глубины.} Зачем нужен; как настраивается; Z-буфер и его формат; Z-fighting; Z-clamping.
\item \textbf{Stencil-тест.} Зачем нужен; как настраивается; stencil буфер и его формат.
\item \textbf{Blending.} Зачем нужен; как настраивается.

\subsection{Освещение}
\item \textbf{Освещение.} Зачем нужно; что происходит со светом; виды материалов; BRDF, BTDF, BSDF; уравнение рендеринга.
\item \textbf{Источники света.} Точечные источники; направленные источники; как описывается освещённость в точке.
\item \textbf{Модель Фонга.} Ambient освещение; диффузное освещение; specular освещение.
\item \textbf{Ambient occlusion.} Что такое; зачем нужно; baked ambient occlusion.
\item \textbf{Normal mapping.} Что такое; зачем нужно; как реализуется; система координат нормалей; формат хранения нормалей.
\item \textbf{Material mapping.} Что такое; зачем нужно; как реализуется.
\item \textbf{Отражения.} Плоское зеркало; enrironment mapping; reflection mapping.
\item \textbf{Тени.} Что такое; мягкие и жёсткие тени; umbra, penumbra, их размер.
\item \textbf{Shadow volumes.} Принцип работы; плюсы и минусы; детали реализации.
\item \textbf{Shadow mapping.} Принцип работы; плюсы и минусы; детали реализации; shadow acne, shadow bias; peter panning.
\item \textbf{Shadow mapping: проекции.} Как строятся; отличия между точечным и направленным источниками.
\item \textbf{Shadow mapping: вариации.} PCF; ESM; VSM; PSM; CSM.

\subsection{Эффекты}
\item \textbf{Размытие.} Что такое; как реализуется; separable kernels; bloom; depth blur.
\item \textbf{Toon shading.} Что такое; как реализуется; edge detection; color grading.
\item \textbf{Сглаживание.} Что такое; зачем нужно; supersampling; multisampling; FXAA.
\item \textbf{Tone mapping.} Что такое; зачем нужен; пример tone mapping оператора; как реализуется.
\item \textbf{Gamma-correction.} Что такое; зачем нужно; как реализуется; sRGB.
\item \textbf{Color banding, dithering.} Что такое; когда возникает; как реализуется; где применяется; варианты dither mask.

\subsection{Прочее}
\item \textbf{Billdoards.} Что такое; зачем нужны; как реализуются.
\item \textbf{Покадровая анимация (моделей и изображений).} Что такое; как реализуются.
\item \textbf{Скелетная анимация.} Чем лучше покадровой; как реализуется.
\item \textbf{Рендеринг текста.} Кодировки; шейпинг; bitmap-шрифты; векторные шрифты; SDF-шрифты.
\item \textbf{Volume rendering.} Что такое; зачем нужен; основные понятия и уравнения; виды рассеяния.
\item \textbf{Volume rendering: методы реализации.} Slicing; raymarching.

\subsection{Оптимизация}
\item \textbf{*Оптимизация.} Общие идеи; метод поиска bottleneck'а.
\item \textbf{*Instanced rendering.} Что такое; зачем нужно; как работает.
\item \textbf{*Frustum culling.} Что такое; зачем нужно; SAT.
\item \textbf{*Occlusion culling.} Что такое; зачем нужно; детали реализации.
\item \textbf{*LOD.} Что такое; зачем нужно; детали реализации.
\end{enumerate}

\end{document}