% (c) Nikita Lisitsa, lisyarus@gmail.com, 2021

\documentclass{beamer}

\usepackage[T2A]{fontenc}
\usepackage[utf8]{inputenc}
\usepackage[russian]{babel}

\usepackage{graphicx}
\graphicspath{ {./images/} }

\usepackage{adjustbox}

\usepackage{color}
\usepackage{soul}

\usepackage{hyperref}

\usepackage{amsmath}

\usepackage{tikz}
\usetikzlibrary{decorations}
\usetikzlibrary{decorations.pathreplacing}
\usepackage{xifthen}

\definecolor{red}{rgb}{1,0,0}
\definecolor{green}{rgb}{0,0.5,0}
\definecolor{blue}{rgb}{0,0,1}
\definecolor{magenta}{rgb}{0.75,0,0.75}

\makeatletter
\newcommand{\slideimage}[1]{
  \begin{figure}
    \begin{adjustbox}{width=\textwidth, totalheight=\textheight-2\baselineskip-2\baselineskip,keepaspectratio}
      \includegraphics{#1}
    \end{adjustbox}
  \end{figure}
}
\makeatother

\title{Компьютерная графика}
\subtitle{Лекция 8: Stencil bufer, framebuffer, renderbuffer, пост-обработка}
\date{2021}

\setbeamertemplate{footline}[frame number]

\begin{document}

\frame{\titlepage}

\begin{frame}[fragile]
\frametitle{Stencil buffer (буфер трафарета)}
\begin{itemize}
\item Особый буфер, хранящий произвольные данные, с особыми побитовыми операциями над ними
\pause
\item Чем-то аналогичен буферу глубины: тоже хранит какие-то данные, такого же размера (как и цветовой буфер), тоже позволяет рисовать или не рисовать пиксель по какому-то условию (depth test)
\pause
\item У дефолтного фреймбуфера часть есть 8-битный stencil буфер (зависит от настроек контекста OpenGL)
\pause
\item Можно (и нужно, если вы его используете) очищать как \verb|glClear(GL_STENCIL_BUFFER_BIT|)
\pause
\item Настроить значение, которым очищается буфер: \verb|glClearStencil|
\end{itemize}
\end{frame}

\begin{frame}[fragile]
\frametitle{Stencil тест}
\begin{itemize}
\item Включить/выключить stencil тест: \verb|glEnable/glDisable(GL_STENCIL_TEST)|
\pause
\item Настроить stencil тест: \verb|glStencilFunc|
\begin{itemize}
\item \verb|func| - одна из констант \verb|GL_ALWAYS|, \verb|GL_LESS|, \verb|GL_GREATER|, \verb|GL_EQUAL|, ...
\item \verb|ref| - референсное значение для теста
\item \verb|mask| - побитовая маска для теста
\end{itemize}
\item Stencil тест: \verb|func(ref & mask, stencil & mask)|
\pause
\item Так же, как с depth тестом: если stencil тест не прошёл, пиксель не будет нарисован
\end{itemize}
\end{frame}

\begin{frame}[fragile]
\frametitle{Stencil тест}
\begin{itemize}
\item Как записать значение в stencil буфер? \pause \verb|glStencilOp|
\begin{itemize}
\item \verb|sfail| - что делать, если пиксель не прошёл stencil тест
\item \verb|dpfail| - что делать, если пиксель не прошёл depth тест
\item \verb|dppass| - что делать, если пиксель прошёл оба теста
\end{itemize}
\pause
\item Возможные значение \verb|sfail|, \verb|dpfail| и \verb|dppass|:
\begin{itemize}
\item \verb|GL_KEEP| - не менять записанное значение
\item \verb|GL_ZERO| - записать 0
\item \verb|GL_INVERT| - побитово обратить
\item \verb|GL_REPLACE| - записать \verb|ref| из функции \verb|glStencilFunc|
\item \verb|GL_INCR| - увеличить на 1, если значение меньше максимального
\item \verb|GL_DECR| - уменьшить на 1, если значение больше минимального (0)
\item \verb|GL_INCR_WRAP| - увеличить на 1 с целочисленным переполнением
\item \verb|GL_DECR_WRAP| - уменьшить на 1 с целочисленным переполнением
\end{itemize}
\end{itemize}
\end{frame}

\begin{frame}[fragile]
\frametitle{Stencil тест}
\begin{itemize}
\item Дополнительно можно включать/выключать запись отдельных битов stencil буфера: \verb|glStencilMask|
\pause
\item Все параметры stencil теста можно настраивать отдельно для front и back граней функциями \verb|glStencilFuncSeparate|, \verb|glStencilOpSeparate|, \verb|glStencilMaskSeparate|
\end{itemize}
\end{frame}

\begin{frame}[fragile]
\frametitle{Stencil тест}
\slideimage{stencil-example-1.png}
\end{frame}

\begin{frame}[fragile]
\frametitle{Stencil тест}
\slideimage{stencil-example-2.png}
\end{frame}

\begin{frame}[fragile]
\frametitle{Stencil тест}
\slideimage{stencil-example-3.png}
\end{frame}

\begin{frame}<1>[fragile,label=stencil_examples]
\frametitle{Stencil буфер: применение}
\begin{itemize}
\item Некоторые алгоритмы рисования теней (shadow volumes)
\pause
\item Любая ситуация, в которой нужно ограничить рисование определённых пикселей:
\pause
\begin{itemize}
\item Симулятор самолёта: сначала рисуется внутренность самолёта, затем - окружающий мир, только там, где не был нарисован самолёт \begin{math}\Rightarrow\end{math} можно избежать проблем с точностью буфера глубины
\pause
\item UI, который нужно нарисовать в какой-то ограниченной области экрана (например, scroll)
\pause
\item Рисование невыпуклых полигонов (odd-even rule)
\end{itemize}
\end{itemize}
\end{frame}

\begin{frame}[fragile]
\frametitle{Shadow volumes (stencil shadows)}
\slideimage{shadow-volumes.jpeg}
\end{frame}

\againframe<2-3>{stencil_examples}

\begin{frame}[fragile]
\frametitle{Microsoft flight simulator}
\slideimage{flight-simulator.jpg}
\end{frame}

\againframe<4>{stencil_examples}

\begin{frame}[fragile]
\frametitle{Unity Scroll View}
\slideimage{scroll-view.png}
\end{frame}

\againframe<5>{stencil_examples}

\begin{frame}[fragile]
\frametitle{Невыпуклый полигон}
\slideimage{non-convex-polygon.png}
\end{frame}

\begin{frame}[fragile]
\frametitle{Stencil буфер: ссылки}
\begin{itemize}
\item \href{https://www.khronos.org/opengl/wiki/Stencil_Test}{khronos.org/opengl/wiki/Stencil\_Test}
\item \href{https://learnopengl.com/Advanced-OpenGL/Stencil-testing}{learnopengl.com/Advanced-OpenGL/Stencil-testing}
\item \href{https://open.gl/depthstencils}{open.gl/depthstencils}
\item \href{https://en.wikibooks.org/wiki/OpenGL_Programming/Stencil_buffer}{en.wikibooks.org/wiki/OpenGL\_Programming/Stencil\_buffer}
\end{itemize}
\end{frame}

\end{document}