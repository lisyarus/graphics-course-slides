% (c) Nikita Lisitsa, lisyarus@gmail.com, 2021

\documentclass{beamer}

\usepackage[T2A]{fontenc}
\usepackage[utf8]{inputenc}
\usepackage[russian]{babel}

\usepackage{graphicx}
\graphicspath{ {./images/} }

\usepackage{adjustbox}

\usepackage{color}
\usepackage{soul}

\usepackage{hyperref}

\definecolor{blue}{rgb}{0,0,1}
\definecolor{red}{rgb}{1,0,0}

\makeatletter
\newcommand{\slideimage}[1]{
  \begin{figure}
    \begin{adjustbox}{width=\textwidth, totalheight=\textheight-2\baselineskip-2\baselineskip,keepaspectratio}
      \includegraphics{#1}
    \end{adjustbox}
  \end{figure}
}
\makeatother

\title{Компьютерная графика}
\subtitle{Практика 9: Shadow mapping 2}
\date{2021}

\setbeamertemplate{footline}[frame number]

\begin{document}

\frame{\titlepage}

\begin{frame}[fragile]
\slideimage{0.png}
\end{frame}

\begin{frame}[fragile]
\begin{itemize}
\item В этой практике мы рисуем front-facing грани в shadow map, а не back-facing грани!
\end{itemize}
\end{frame}

\begin{frame}[fragile]
\frametitle{Задание 1}
Подбираем shadow bias
\begin{itemize}
\item Нужно прибавить в шейдере константу (bias) к значению, прочитанному из shadow map
\item Слишком большое значение приведёт к заметному peter panning (тень будет съезжать в сторону от модели)
\item Слишком маленького значения будет недостаточно, чтобы убрать shadow acne
\end{itemize}
\end{frame}

\begin{frame}[fragile]
\slideimage{1.png}
\end{frame}

\begin{frame}[fragile]
\frametitle{Задание 2}
Вычисляем хорошую матрицу проекции для shadow map
\begin{itemize}
\item Нужно найти центр видимой области и оси \begin{math}X, Y, Z\end{math}
\item Направления осей \begin{math}\hat X, \hat Y, \hat Z\end{math} уже посчитаны, нужно только вычислить их длину
\item После чтения сцены из файла нужно посчитать её bounding box (min/max по всем кординатам вершин), её центр \begin{math}C\end{math} -- центр видимой области
\item Пройдясь по всем 8 вершинам \begin{math}V\end{math} bounding box'а сцены, можно посчитать модуль скалярного произведения \begin{math}|(V - C) \cdot \hat X|\end{math}, максимум таких значений -- длина вектора \begin{math}X\end{math}
\item Аналогично для \begin{math}Y\end{math} и \begin{math}Z\end{math}
\item Используя \begin{math}X,Y,Z,C\end{math} можно построить матрицу \verb|transform| ортографической проекции (см. слайды 4ой лекции)
\item N.B.: старый код вычисления матрицы (цикл \verb|for| + параметр \verb|shadow_scale|) нужно убрать
\end{itemize}
\end{frame}

\begin{frame}[fragile]
\slideimage{2.png}
\end{frame}

\begin{frame}[fragile]
\frametitle{Задание 3}
\fontsize{10pt}{10pt}
Variance shadow maps
\begin{itemize}
\item Включаем для shadow map линейную фильтрацию
\item Устанавливаем текстуре shadow map internal format в \verb|GL_RG32F| (format и type не принипиальны, можно \verb|GL_RGBA| и \verb|GL_FLOAT|)
\item Добавляем shadow map к фреймбуферу как \verb|GL_COLOR_ATTACHMENT0| вместо \verb|GL_DEPTH_ATTACHMENT|
\item Создаём renderbuffer (и выделяем ему память через \verb|glRenderbufferStorage|, тип пикселя -- \verb|GL_DEPTH_COMPONENT24|), который будет использоваться как буфер глубины при рисовании shadow map, и добавляем его как \verb|GL_DEPTH_ATTACHMENT| фреймбуфера
\item Во фрагментном шейдере, рисующем shadow map, добавляем out-переменную и пишем в неё \verb|vec4(z, z * z, 0.0, 0.0)| (\verb|z| можно достать из \verb|gl_FragCoord|)
\item Во фрагментном шейдере дебажного прямоугольника читаем все компоненты текстуры, например \verb|out_color = texture(shadow_map, ...)|
\item Перед очисткой (\verb|glClear|) фреймбуфера для генерации shadow map нужно поставить правильный цвет очистки: \verb|glClearColor(1.f, 1.f, 0.f, 0.f)|
\end{itemize}
\end{frame}

\begin{frame}[fragile]
\slideimage{3.png}
\end{frame}

\begin{frame}[fragile]
\frametitle{Задание 4}
\fontsize{10pt}{10pt}
Variance shadow maps
\begin{itemize}
\item В основном фрагментном шейдере читаем данные из shadow map и используем неравенство Чебышёва для вычисления освещённости:
\begin{verbatim}
vec2 data = texture(shadow_map, shadow_pos.xy).rg;
float mu = data.r;
float sigma = data.g - mu * mu;
float z = shadow_pos.z;
float factor = (z < mu) ? 1.0
    : sigma / (sigma + (z - mu) * (z - mu));
\end{verbatim}
\end{itemize}
\end{frame}

\begin{frame}[fragile]
\slideimage{4.png}
\end{frame}

\begin{frame}[fragile]
\frametitle{Задание 5}
Исправляем артефакты
\begin{itemize}
\item Добавляем наклон поверхности к среднему значению квадрата глубины при генерации shadow map:
\begin{itemize}
\item Через \verb|dFdx(z)| и \verb|dFdy(z)| можно получить градиент глубины по X и Y
\item К квадрату глубины добавляем \begin{math}\frac{1}{4}\left[\left(\frac{\partial Z}{\partial X}\right)^2 + \left(\frac{\partial Z}{\partial Y}\right)^2\right]\end{math}
\end{itemize}
\item Добавляем shadow bias (вычитаем константу из значения \verb|z|, использующегося для вычисления освещённости)
\item Значение, получающееся из формулы неравенства Чебышёва, преобразуем: диапазон \begin{math}[0, \delta]\end{math} переходит в \begin{math}0\end{math}, а диапазон \begin{math}[\delta, 1]\end{math} переходит в \begin{math}[0, 1]\end{math} (\begin{math}\delta\end{math} -- некое фиксированное значение, например, \verb|0.125|)
\end{itemize}
\end{frame}

\begin{frame}[fragile]
\slideimage{5.png}
\end{frame}

\begin{frame}[fragile]
\frametitle{Задание 6*}
Размываем shadow map
\begin{itemize}
\item Вместо чтения одного пикселя из shadow map, читаем набор значений из соседних пикселей и усредняем по Гауссу
\item Полученный двумерный вектор используем для вычисления освещённости через неравенство Чебышёва
\item Размываем не результат вычисления освещённости, а сами данные из shadow map!
\item {\color{red}N.B.}: по-хорошему это размытие нужно делать отдельными проходами с отдельными шейдерами и отдельным размытием по X и Y (см. \textit{separable Gaussian blur})
\end{itemize}
\end{frame}

\begin{frame}[fragile]
\slideimage{6.png}
\end{frame}

\end{document}