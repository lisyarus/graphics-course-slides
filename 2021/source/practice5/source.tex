% (c) Nikita Lisitsa, lisyarus@gmail.com, 2021

\documentclass{beamer}

\usepackage[T2A]{fontenc}
\usepackage[utf8]{inputenc}
\usepackage[russian]{babel}

\usepackage{graphicx}
\graphicspath{ {./images/} }

\usepackage{adjustbox}

\usepackage{color}
\usepackage{soul}

\usepackage{hyperref}

\definecolor{blue}{rgb}{0,0,1}
\definecolor{red}{rgb}{1,0,0}

\makeatletter
\newcommand{\slideimage}[1]{
  \begin{figure}
    \begin{adjustbox}{width=\textwidth, totalheight=\textheight-2\baselineskip-2\baselineskip,keepaspectratio}
      \includegraphics{#1}
    \end{adjustbox}
  \end{figure}
}
\makeatother

\title{Компьютерная графика}
\subtitle{Практика 5: Текстуры}
\date{2021}

\setbeamertemplate{footline}[frame number]

\begin{document}

\frame{\titlepage}

\begin{frame}[fragile]
\frametitle{Задание 1}
Добавляем текстурные координаты
\begin{itemize}
\item Добавляем в вершины новое поле (например, \verb|std::uint16_t texcoords[2];| - тип не принципиален)
\pause
\item Добавляем значение этого поля в вершины массива \verb|plane_vertices|
\pause
\item Описываем новый атрибут для VAO
\pause
\item Добавляем входной атрибут (\verb|vec2|) в вершинном шейдере, и передаём его во фрагментный
\pause
\item Во фрагментном шейдере выводим текстурные координаты в качестве цвета (например, \verb|out_color = vec4(texcoord, 0.0, 1.0)|)
\end{itemize}
\end{frame}

\begin{frame}[fragile]
\frametitle{Задание 2}
Создаём и рисуем текстуру в один пиксель
\begin{itemize}
\item Создаём объект текстуры типа \verb|GL_TEXTURE_2D|
\pause
\item Выставляем для него min и mag фильтры в \verb|GL_NEAREST|
\pause
\item В качестве данных загружаем один пиксель (например, красного цвета) в формате \verb|GL_RGBA8| (пиксель может быть \verb|std::uint8_t[4]| или \verb|std::uint32_t|)
\pause
\item Во фрагментном шейдере добавляем uniform-переменную типа \verb|sampler2D| и берём из неё цвет
\pause
\item В значение uniform-переменной записываем 0 (\verb|glUniform1i|)
\pause
\item Перед рисованием делаем нашу текстуру текущей для texture unit'а номер 0 (\verb|glActiveTexture| + \verb|glBindTexture|)
\end{itemize}
\end{frame}

\begin{frame}[fragile]
\frametitle{Задание 3}
Меняем картинку на шахматную доску
\begin{itemize}
\item Выбираем размеры картинки (например, \verb|width = height = 256|)
\pause
\item Создаём массив пикселей (лучше всего - \verb|std::vector<std::uint32_t>| или \verb|std::vector<std::uint8_t>|)
\pause
\item В цикле заполняем пиксели чёрным или белым цветом в зависимости от \verb|(x + y) % 2|
\pause
\item Записываем эти пиксели в текстуру (\verb|glTexImage2D|)
\end{itemize}
\end{frame}

\begin{frame}[fragile]
\frametitle{Задание 4}
Добавляем mipmaps
\begin{itemize}
\item Меняем mag filter на \verb|GL_LINEAR|
\pause
\item Меняем min filter на \verb|GL_LINEAR_MIPMAP_NEAREST|
\pause
\item Мы не загружали mipmaps, поэтому картинка сейчас будет чёрной
\pause
\item Делаем \verb|glGenerateMipmap| после загрузки основной картинки
\end{itemize}
\end{frame}

\begin{frame}[fragile]
\frametitle{Задание 5}
Делаем mipmaps явно видимыми
\begin{itemize}
\item Можно увеличить размер картинки до 1024x1024, чтобы влезло больше mipmap-уровеней
\pause
\item После \verb|glGenerateMipmap| загружаем для 1, 2 и 3 уровней монотонные красную, синюю и зелёную картинки соответственно
\pause
\item N.B.: какой уровень загрузить - второй параметр \verb|glTexImage2D|
\pause
\item N.B.: если картинка 1024x1024, первый mipmap должен быть 512x512, второй - 256x256, и т.д.
\pause
\item N.B.: массив красных пикселей в формате \verb|uint32_t| можно создать как
\begin{verbatim}
std::vector<std::uint32_t> pixels(512 * 512,
    0xff0000ffu);
\end{verbatim}
\pause
\item Сейчас mipmap-уровни видны по отдельности - меняем min filter на \verb|GL_LINEAR_MIPMAP_LINEAR|
\end{itemize}
\end{frame}

\begin{frame}[fragile]
\frametitle{Задание 6}
Добавляем вторую текстуру
\begin{itemize}
\item Повторяем всё из задания 2: создаём и настраиваем текстуру, создаём uniform-переменную
\pause
\item В созданную uniform-переменную (\verb|sampler2D|) записываем 1 (\verb|glUniform1i|) - это значит, что соответствующая текстура будет взята из texture unit'а с номером 1
\pause
\item Перед рисованием делаем старую текстуру текущей для texture unit'а номер 0, а новую - для texture unit'а номер 1 (\verb|glActiveTexture| + \verb|glBindTexture|, два раза)
\pause
\item Загружаем в текстуру данные из \verb|test_image|, шириной \verb|test_image_width|, высотой \verb|test_image_height|, \verb|format = GL_RGB|, \verb|type = GL_UNSIGNED_BYTE|
\pause
\item Во фрагментном шейдере читаем из обеих текстур и в качестве цвета выводим среднее значение
\end{itemize}
\end{frame}

\end{document}