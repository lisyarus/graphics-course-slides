% (c) Nikita Lisitsa, lisyarus@gmail.com, 2023

\documentclass{beamer}

\usepackage[T2A]{fontenc}
\usepackage[russian]{babel}
\usepackage{minted}

\usepackage{graphicx}
\graphicspath{ {./images/} }

\usepackage{adjustbox}

\usepackage{tikz}

\usepackage{color}
\usepackage{soul}

\usepackage{hyperref}

\definecolor{blue}{rgb}{0,0,1}
\definecolor{red}{rgb}{1,0,0}

\usetheme{metropolis}
\setminted{fontsize=\footnotesize}

\makeatletter
\newcommand{\slideimage}[1]{
  \begin{figure}
    \begin{adjustbox}{width=\textwidth, totalheight=\textheight-2\baselineskip-2\baselineskip,keepaspectratio}
      \includegraphics{#1}
    \end{adjustbox}
  \end{figure}
}
\makeatother

\title{Компьютерная графика}
\subtitle{Финальный проект}
\date{2023}

\setbeamertemplate{footline}[frame number]

\begin{document}

\frame{\titlepage}

\begin{frame}[fragile]
\frametitle{Задание}
\begin{itemize}
\item Свободная форма: игра / визуализация / динамическая сцена / что хотите
\pause
\item Нужно реализовать один или несколько сложных алгоритмов, которые мы изучали не очень подробно (или вообще не изучали) в течение курса
\pause
\item Алгоритмы оцениваются в разное количество баллов в зависимости от сложности
\end{itemize}
\end{frame}

\begin{frame}[fragile]
\frametitle{Что можно использовать}
\begin{itemize}
\item Можно пользоваться вспомогательными библиотеками (например, для загрузки текстур, моделей и сцен)
\pause
\item Весь OpenGL-код должен быть написан вами, т.е. библиотека не должна загружать данные на GPU, создавать текстуры, и т.п.
\pause
\item Можно брать код из практик и домашних заданий, в т.ч. загрузчики моделей, анимаций и шрифтов (при необходимости их можно доработать)
\pause
\item Можно брать сцены и текстуры из практик и домашних заданий
\end{itemize}
\end{frame}

\begin{frame}[fragile]
\frametitle{Алгоритмы}
\begin{itemize}
\item \textbf{5 баллов}: toon shading
\item \textbf{10 баллов}: bloom
\item \textbf{15 баллов}: планета с volume-rendered атмосферой
\item \textbf{15 баллов}: slicing volume rendering \textit{(не подойдёт для планеты)}
\item \textbf{20 баллов}: cascaded shadow maps
\item \textbf{25 баллов}: SSAO + dithering
\end{itemize}
\end{frame}

\begin{frame}[fragile]
\frametitle{Алгоритмы}
\begin{itemize}
\item \textbf{30 баллов}: SSDO / HDAO / HBAO \textit{(алгоритмы ambient occlusion)}
\item \textbf{30 баллов}: screen-space reflections
\item \textbf{30 баллов}: настоящие мягкие тени (радиус размытия зависит от расстояния до объекта, бросившего тень -- ищите \textit{summed-area soft shadows})
\item \textbf{30 баллов}: deferred shading
\item \textbf{30 баллов}: tiled/clustered shading
\end{itemize}
\end{frame}

\begin{frame}[fragile]
\frametitle{Алгоритмы}
\begin{itemize}
\item \textbf{30 баллов}: real-time hatching
\item \textbf{30 баллов}: честный векторный текст (\underline{не} через предварительное рисование в текстуру и \underline{не} триангуляцией)
\item \textbf{30 баллов}: водные каустики
\item \textbf{30 баллов}: очень много травы (billboards + geometry shaders)
\item \textbf{30 баллов}: очень большой ландшафт (geometry clipmaps или LOD)
\end{itemize}
\end{frame}

\begin{frame}[fragile]
\frametitle{Алгоритмы}
\begin{itemize}
\item Если вы нашли статью / видео и не уверены, подходит ли оно под описанные алгоритмы -- напишите мне, обсудим
\pause
\item Если у вас есть другой алгоритм на примете -- напишите мне, обсудим
\pause
\item Сдача в день зачёта 25.12 или экзамена 15.01
\end{itemize}
\end{frame}

\end{document}