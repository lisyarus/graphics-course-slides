% (c) Nikita Lisitsa, lisyarus@gmail.com, 2023

\documentclass[10pt]{beamer}

\usepackage[T2A]{fontenc}
\usepackage[russian]{babel}
\usepackage{minted}

\usepackage{graphicx}
\graphicspath{ {./images/} }

\usepackage{adjustbox}

\usepackage{color}
\usepackage{soul}

\usepackage{hyperref}

\usepackage{amsmath}

\usepackage{tikz}
\usetikzlibrary{decorations}
\usetikzlibrary{decorations.pathreplacing}
\usepackage{xifthen}

\usetheme{metropolis}
\setminted{fontsize=\footnotesize}

\definecolor{red}{rgb}{1,0,0}
\definecolor{green}{rgb}{0,0.5,0}
\definecolor{blue}{rgb}{0,0,1}
\definecolor{magenta}{rgb}{0.75,0,0.75}
\definecolor{codebg}{RGB}{29,35,49}

\makeatletter
\newcommand{\slideimage}[1]{
  \begin{figure}
    \begin{adjustbox}{width=\textwidth, totalheight=\textheight-2\baselineskip-2\baselineskip,keepaspectratio}
      \includegraphics{#1}
    \end{adjustbox}
  \end{figure}
}
\makeatother

\title{Компьютерная графика}
\subtitle{Лекция 8: Геометрические шейдеры, shadow volumes, shadow mapping и его разновидности}
\date{2023}

\setbeamertemplate{footline}[frame number]

\begin{document}

\usemintedstyle{solarized-light}

\frame{\titlepage}

\begin{frame}[fragile]
\frametitle{Геометрические шейдеры}
\begin{itemize}
\item Тип шейдера, наравне в вершинным и фрагментным
\pause
\item Создаётся как \mintinline{cpp}|glCreateShader(GL_GEOMETRY_SHADER)|
\pause
\item Встраивается после вершинного шейдера, до perspective divide
\pause
\item Оперирует целыми примитивами (точками/линиями/треугольниками), т.е. наборами вершин
\pause
\item Может на лету менять тип примитива и количество вершин
\pause
\item \begin{math}\Longrightarrow\end{math} Может работать довольно медленно
\end{itemize}
\end{frame}

\begin{frame}<1>[fragile,label=geometry_shader_examples]
\frametitle{Геометрические шейдеры: примеры использования}
\fontsize{10pt}{10pt}
\begin{itemize}
\item Расчёт нормалей для flat shading'а
\pause
\begin{itemize}
\item Вход: треугольник (тройка вершин)
\item Выход: те же вершины с посчитанной нормалью
\end{itemize}
\pause
\item Визуализация нормалей
\pause
\begin{itemize}
\item Вход: вершина с нормалью
\item Выход: линия из двух вершин -- исходная вершина, исходная вершина + нормаль
\end{itemize}
\pause
\item Shadow volumes -- алгоритм рисования теней
\pause
\item Billboards -- плоские фигуры, всегда смотрящие в сторону камеры
\pause
\begin{itemize}
\item Вход: одна вершина (точка)
\item Выход: набор треугольников
\pause
\item Системы частиц
\pause
\item Облака
\pause
\item Далёкие деревья
\pause
\item Трава
\end{itemize}
\end{itemize}
\end{frame}

\begin{frame}[fragile]
\frametitle{Flat shading}
\slideimage{flat-shading.png}
\end{frame}

\againframe<1-3>{geometry_shader_examples}

\begin{frame}[fragile]
\frametitle{Визуализация нормалей}
\slideimage{normals-viz.png}
\end{frame}

\againframe<3-6>{geometry_shader_examples}

\begin{frame}[fragile]
\frametitle{Billboards}
\slideimage{billboards.jpg}
\end{frame}

\againframe<6-8>{geometry_shader_examples}

\begin{frame}[fragile]
\frametitle{Billboards: система частиц (дым)}
\slideimage{smoke.jpg}
\end{frame}

\againframe<8-9>{geometry_shader_examples}

\begin{frame}[fragile]
\frametitle{Billboards: облака}
\slideimage{clouds.jpg}
\end{frame}

\againframe<9-10>{geometry_shader_examples}

\begin{frame}[fragile]
\frametitle{Billboards: деревья}
\slideimage{trees.jpg}
\end{frame}

\againframe<10-11>{geometry_shader_examples}

\begin{frame}[fragile]
\frametitle{Billboards: трава}
\slideimage{grass.jpg}
\end{frame}

\begin{frame}[fragile]
\frametitle{Геометрические шейдеры: пример}
\fontsize{10pt}{10pt}
\usemintedstyle{lightbulb}
\begin{minted}[bgcolor=codebg,fontsize=\tiny]{glsl}
uniform mat4 transform;
// Входные примитивы - точки
layout (points) in;
// Выходные примитивы - линии, в сумме не больше 2х вершин
layout (line_strip, max_vertices = 2) out;
// Данные из вершинного шейдера
in vec3 normal[];
  
void main() {    
    gl_Position = transform * gl_in[0].gl_Position;
    EmitVertex();

    gl_Position = transform * (gl_in[0].gl_Position + vec4(normal[0], 0));
    EmitVertex();

    EndPrimitive();
} 
\end{minted}
\end{frame}

\begin{frame}[fragile]
\frametitle{Геометрические шейдеры: пример}
\usemintedstyle{lightbulb}
\begin{minted}[bgcolor=codebg,fontsize=\tiny]{glsl}
// Входные примитивы - линии
layout (lines) in;
// Выходные примитивы - треугольники, в сумме не больше 4х вершин
layout (triangle_strip, max_vertices = 4) out;
// Данные для фрагментного шейдера
out vec4 color;
  
void main() {
    gl_Position = gl_in[0].gl_Position + vec4(-1.0, -1.0, 0.0, 0.0);
    color = vec4(1.0, 0.0, 0.0, 1.0);
    EmitVertex();

    gl_Position = gl_in[0].gl_Position + vec4( 1.0, -1.0, 0.0, 0.0);
    color = vec4(0.0, 1.0, 0.0, 1.0);
    EmitVertex();

    gl_Position = gl_in[1].gl_Position + vec4(-1.0,  1.0, 0.0, 0.0);
    color = vec4(0.0, 0.0, 1.0, 1.0);
    EmitVertex();

    gl_Position = gl_in[1].gl_Position + vec4( 1.0,  1.0, 0.0, 0.0);
    color = vec4(1.0, 1.0, 1.0, 1.0);
    EmitVertex();

    EndPrimitive();
} 
\end{minted}
\end{frame}

\begin{frame}[fragile]
\frametitle{Геометрические шейдеры: ссылки}
\begin{itemize}
\item \href{https://www.khronos.org/opengl/wiki/Geometry_Shader}{\nolinkurl{khronos.org/opengl/wiki/Geometry\_Shader}}
\item \href{https://learnopengl.com/Advanced-OpenGL/Geometry-Shader}{\nolinkurl{learnopengl.com/Advanced-OpenGL/Geometry-Shader}}
\item \href{https://open.gl/geometry}{\nolinkurl{open.gl/geometry}}
\item \href{https://www.lighthouse3d.com/tutorials/glsl-tutorial/geometry-shader}{\nolinkurl{lighthouse3d.com/tutorials/glsl-tutorial/geometry-shader}}
\item \href{https://developer.download.nvidia.com/books/HTML/gpugems/gpugems_ch07.html}{\texttt{GPU Gems, Chapter 7. Rendering Countless Blades of Waving Grass}}
\end{itemize}
\end{frame}

\begin{frame}[fragile]
\frametitle{Тени: теория}
\begin{itemize}
\item Точка сцены, в которую не попадает (заблокирован чем-то) прямой свет из конкретного источника света
\item Свойство точки по отношению к конкретному источнику света
\end{itemize}
\end{frame}

\begin{frame}[fragile]
\frametitle{Тени}
\slideimage{shadows1.png}
\end{frame}

\begin{frame}[fragile]
\frametitle{Тени: теория}
\begin{itemize}
\item Если источник света точечный (или бесконечно удалённый), тень -- бинарное свойство: луч из точки сцены в источник света или пересекает что-то (точка в тени), или нет (точка не в тени) \begin{math}\Longrightarrow\end{math} жёсткие тени (\textit{hard shadows})
\pause
\item Если источник света объёмный, точка сцены может находиться в тени относительно части источника света, и не находиться в тени относительно другой его части \begin{math}\Longrightarrow\end{math} мягкие тени (\textit{soft shadows})
\pause
\begin{itemize}
\item Точки, полностью находящиеся в тени -- \textit{umbra}
\item Точки, частично находящиеся в тени -- \textit{penumbra} (\textit{полутень})
\end{itemize}
\end{itemize}
\end{frame}

\begin{frame}[fragile]
\frametitle{Тени: мягкие vs жёсткие}
\slideimage{shadow-scheme1.png}
\end{frame}

\begin{frame}[fragile]
\frametitle{Мягкие тени}
\slideimage{shadows2.png}
\end{frame}

\begin{frame}[fragile]
\frametitle{Солнечное затмение}
\slideimage{eclipse.png}
\end{frame}

\begin{frame}[fragile]
\frametitle{Тени: теория}
\begin{itemize}
\item Реальные источники света -- объёмные
\pause
\item Размер и форма penumbra зависит от размеров и формы источника света, а также от расстояния до него (чем дальше, тем меньше penumbra)
\pause
\begin{itemize}
\item В пределе расстояния \begin{math}\rightarrow\infty\end{math} мягкая тень вырождается в жёсткую
\end{itemize}
\pause
\item В real-time графике получить правильные тени (как жёсткие, так и мягкие) довольно сложно
\end{itemize}
\end{frame}

\begin{frame}[fragile]
\frametitle{Тени: raytracing}
\begin{itemize}
\item При простом raytracing'е (\textit{Whitted-style}, без полного решения rendering equation) для получения жёсткой тени можно послать дополнительный луч в направлении света (shadow ray): если он что-то пересёк, точка находится в тени
\pause
\item При raytracing'е с монте-карло интегрированием для global illumination тени (как жёсткие, так и мягкие) получаются автоматически
\end{itemize}
\end{frame}

\begin{frame}[fragile]
\frametitle{Тени: real-time алгоритмы}
\begin{itemize}
\item Shadow volumes
\pause
\begin{itemize}
\item {\color{green}+} Идеальные жёсткие тени
\item {\color{red}---} Алиасинг
\item {\color{red}---} Нужно на лету генерировать геометрию
\item {\color{red}---} Сложно сделать мягкие тени
\item {\color{red}---} Очень большой fill rate (количество обрабатываемых пикселей), плохо предсказумая производительность
\end{itemize}
\pause
\item Shadow mapping
\pause
\begin{itemize}
\item {\color{green}+} Производительность растёт +/- линейно с ростом сложности сцены
\item {\color{red}---} Крупный алиасинг ("пиксельные" тени)
\item {\color{green}+} Много вариаций, улучшающих качество и позволяющих делать как жёсткие, так и мягкие тени
\end{itemize}
\end{itemize}
\end{frame}

\begin{frame}[fragile]
\frametitle{Shadow volumes (они же stencil shadows)}
\begin{itemize}
\item Тень от конкретного объекта -- некая (полубесконечная) трёхмерная область пространства (shadow volume)
\end{itemize}
\slideimage{shadow-volume.png}
\end{frame}

\begin{frame}[fragile]
\frametitle{Shadow volumes (они же stencil shadows)}
\begin{itemize}
\item Тень от конкретного объекта -- некая (полубесконечная) трёхмерная область пространства (shadow volume)
\pause
\item Нарисуем её как трёхмерный объект (построим и нарисуем её грани)
\pause
\item Если в конкретный пиксель попало одинаковое количество front-facing и back-facing граней, то он не в тени, иначе -- в тени
\pause
\item Количество пересечений луча из камеры с гранями shadow volume вычисляем с помощью stencil буфера
\end{itemize}
\end{frame}

\begin{frame}[fragile]
\frametitle{Shadow volumes (они же stencil shadows)}
\slideimage{shadow-volumes-scheme.png}
\end{frame}

\begin{frame}[fragile]
\frametitle{Shadow volumes: алгоритм}
\begin{itemize}
\item Выключаем рисование в цветовой буфер и буфер глубины (\mintinline{cpp}|glColorMask|, \mintinline{cpp}|glDepthMask|) (но не выключаем тест глубины!)
\pause
\item Настраиваем stencil буфер: \mintinline{cpp}|+1| для front-facing треугольников, \mintinline{cpp}|-1| для back-facing
\pause
\item Вычисляем и рисуем shadow volume для каждого объекта
\pause
\item Обратно включаем рисование цвета и глубины
\pause
\item Настраиваем stencil-тест так, чтобы рисовать только пиксели, в которых \mintinline{cpp}|stencil == 0|, и рисуем сцену шейдером, вычисляющим освещение
\pause
\item Настраиваем stencil-тест так, чтобы рисовать только пиксели, в которых \mintinline{cpp}|stencil != 0|, и рисуем сцену шейдером, игнорирующим освещение
\pause
\item Повторяем для каждого источника света
\end{itemize}
\end{frame}

\begin{frame}[fragile]
\frametitle{Shadow volume}
\slideimage{shadow-volume2.jpg}
\end{frame}

\begin{frame}[fragile]
\frametitle{Shadow volume}
\slideimage{shadow-volume3.png}
\end{frame}

\begin{frame}[fragile]
\frametitle{Shadow volumes (Doom 3)}
\slideimage{doom3-shadows.png}
\end{frame}

\begin{frame}[fragile]
\frametitle{Shadow volumes: как вычислять shadow volume?}
\begin{itemize}
\item Шаг 1: определить shadow edges -- рёбра модели, у которых один соседний треугольник обращён к источнику света \begin{math}\left(\vec n \cdot \vec r > 0\right)\end{math}, а второй -- против источника света \begin{math}\left(\vec n \cdot \vec r \leq 0\right)\end{math}
\pause
\item Щаг 2: для каждого shadow edge построить четырёхугольную грань shadow volume -- к точкам ребра \begin{math}p_1, p_2\end{math} добавлются точки \begin{math}p_1 + D \cdot \frac{p_1 - o}{|p_1 - o|}\end{math} и аналогично для \begin{math}p_2\end{math}
\begin{itemize}
\item \begin{math}o\end{math} -- координаты источника света
\item \begin{math}D\end{math} -- расстояние, до которого отбрасывается тень (в идеале мы хотим \begin{math}D\rightarrow\infty\end{math}, для этого можно воспользоваться однородными координатами)
\end{itemize}
\end{itemize}
\end{frame}

\begin{frame}[fragile]
\frametitle{Shadow volumes (Doom 3)}
\slideimage{shadow-edge.jpg}
\end{frame}

\begin{frame}[fragile]
\frametitle{Shadow volumes: как вычислять shadow volume?}
\begin{itemize}
\item Два варианта: на CPU или на GPU
\pause
\item На CPU: очень дорого (положение модели и источника света могут меняться каждый кадр)
\pause
\item На GPU: геометрические шейдеры!
\begin{itemize}
\item Нужен входной тип примитива \textit{lines adjacency}, позволяющий передать 4 вершины как один примитив (две вершины ребра + две вершины соседних треугольников)
\pause
\item \begin{math}\Longrightarrow\end{math} Модель должна быть специально адаптирована для генерации shadow volume
\pause
\item Шейдер проверяет, является ли ребро shadow edge, и генерирует грань shadow volume
\end{itemize}
\end{itemize}
\end{frame}

\begin{frame}[fragile]
\frametitle{Shadow volumes}
\begin{itemize}
\item {\color{green}+} Pixel-perfect жёсткие тени
\item {\color{red}---} Алиасинг
\item {\color{red}---} Сложно получить мягкие тени
\item {\color{red}---} Нужно на лету генерировать геометрию
\item {\color{red}---} Требует правильной входной геометрии (без дырок и дублирования)
\item {\color{red}---} Не работает, если камера находится внутри тени (исправляется с помощью т.н. Carmack's Reverse)
\item {\color{red}---} Ресурсозатратен: даже для маленького объекта его shadow volume может занимать значительную часть сцены \begin{math}\Longrightarrow\end{math} слишком много растеризации, чтения/записи памяти (stencil buffer)
\end{itemize}
\end{frame}

\begin{frame}[fragile]
\frametitle{Shadow volumes с плохой геометрией}
\slideimage{shadow-volume-bad-mesh.jpg}
\end{frame}

\begin{frame}[fragile]
\frametitle{Shadow volumes}
\begin{itemize}
\item Thief 3
\item Doom 3
\item Quake 4
\item Prey
\item Far Cry
\item F.E.A.R 1, 2, 3
\item S.T.A.L.K.E.R.
\item Timeshift
\item и др.
\end{itemize}
\end{frame}

\begin{frame}[fragile]
\frametitle{Shadow volumes: ссылки}
\begin{itemize}
\item \href{https://en.wikipedia.org/wiki/Shadow_volume}{\nolinkurl{en.wikipedia.org/wiki/Shadow\_volume}}
\item \href{https://ogldev.org/www/tutorial40/tutorial40.html}{\nolinkurl{ogldev.org/www/tutorial40/tutorial40.html}}
\item \href{https://developer.download.nvidia.com/books/HTML/gpugems/gpugems_ch09.html}{\texttt{GPU Gems, Chapter 9. Efficient Shadow Volume Rendering}}
\end{itemize}
\end{frame}

\begin{frame}[fragile]
\frametitle{Shadow mapping: идея}
\begin{itemize}
\item Самое популярное семейство алгоритмов рисования теней
\pause
\item Идея: представим, что источник света -- это камера (наблюдатель)
\item Тень -- это то, что \textit{не видит} источник света
\pause
\item Как понять, что видит источник света? \pause Нарисовать!
\end{itemize}
\end{frame}

\begin{frame}[fragile]
\frametitle{Shadow mapping: алгоритм}
\begin{itemize}
\item Рисуем сцену в текстуру (\textit{shadow map}, карта теней) с точки зрения источника света (цветовой буфер не нужен, достаточно буфера глубины)
\pause
\item Рисуем сцену на экран, фрагментный шейдер читает \textit{буфер глубины} (shadow map), нарисованный с точки зрения источника света -- там содержатся расстояния от источника до ближайшей (к источнику) поверхности
\item Если расстояние от источника до рисуемого пикселя больше, чем значение из буфера глубины, -- текущий пиксель находится в тени
\pause
\item \textbf{\alert{N.B.}} Линейная фильтрация для буфера глубины \underline{не имеет смысла} -- будут артефакты на границах объектов
\end{itemize}
\end{frame}

\begin{frame}[fragile]
\frametitle{Shadow mapping: алгоритм}
\slideimage{shadow-mapping1.png}
\end{frame}

\begin{frame}[fragile]
\frametitle{Shadow mapping: фрагментный шейдер}
\usemintedstyle{lightbulb}
\begin{minted}[bgcolor=codebg,fontsize=\tiny]{glsl}
// Матрица проекции, с которой рисовался shadow map
uniform mat4 shadow_projection;
// Буфер глубины (shadow map)
uniform sampler2D shadow_depth;

in vec3 position;

void main(){
  vec4 shadow_coord = shadow_projection * vec4(position, 1.0);
  shadow_coord /= shadow_coord.w; // perspective divide

  if (abs(shadow_coord.x) < 1.0 && abs(shadow_coord.y) < 1.0){
    shadow_coord = shadow_coord * 0.5 + vec2(0.5);

    bool in_shadow = texture(shadow_depth, shadow_coord.xy).r
        < shadow_coord.z;

    if (!in_shadow) {
        // рассчёт освещения
        ...
    }
  }
}
\end{minted}
\end{frame}

\begin{frame}[fragile]
\frametitle{Shadow mapping: буфер глубины}
\slideimage{shadow-mapping-depth.png}
\end{frame}

\begin{frame}[fragile]
\frametitle{Shadow mapping: результат}
\slideimage{shadow-mapping2.png}
\end{frame}

\begin{frame}[fragile]
\frametitle{Shadow mapping: проекция}
\begin{itemize}
\item Как выбрать проекцию камеры (\verb|shadow_transform|) для shadow map?
\pause
\item Удалённый источник освещает всю сцену с одного направления \begin{math}\Longrightarrow\end{math} удобно использовать ортографическую проекцию
\pause
\item Проекцию надо подобрать так, чтобы в видимую область попала вся сцена
\pause
\item Можно спроецировать все объекты на плоскость, перпендикулярную направлению на свет (которое можно считать Z-осью этой камеры), взять любые два ортогональных вектора X и Y в этой плоскости, и найти bounding box сцены в координатах XYZ -- это даст размеры и ориентацию области shadow map
\end{itemize}
\end{frame}

\begin{frame}[fragile]
\frametitle{Shadow mapping: проекция}
\begin{itemize}
\item Точечный источник светит во все стороны \begin{math}\Longrightarrow\end{math} удобно использовать перспективную проекцию и cubemap-текстуру
\pause
\item 6 shadow map (по одной на каждую грань) с соответствующей перспективной проекцией
\pause
\item Положение камеры совпадает с положением источника света, виртуальный `экран' совпадает с отрисовываемой гранью shadow map
\end{itemize}
\end{frame}

\begin{frame}[fragile]
\frametitle{Shadow acne}
\slideimage{shadow-acne.png}
\end{frame}

\begin{frame}<1>[fragile,label=acne]
\frametitle{Shadow acne}
\begin{itemize}
\item Артефакт, возникающий из-за дискретности shadow map: если луч света не перпендикулярен плоскости объекта, рядом с центром пикселя shadow map будут точки с меньшей или большей глубиной
\pause
\item Первый способ исправить: добавить к значению, прочитанному из из shadow map, некую константу (shadow bias)
\begin{itemize}
\item Часто эту константу умножают на тангенс угла падения света
\end{itemize}
\pause
\item Второй способ исправить: рисовать только back-facing грани (т.е. обращённые против света) в shadow map
\begin{itemize}
\item Теперь shadow acne возникнет на back-facing гранях, но мы и без shadow map знаем, что они в тени!
\item Работает только с правильной геометрией (двусторонней, без дырок, и т.п.)
\end{itemize}
\end{itemize}
\end{frame}

\begin{frame}[fragile]
\frametitle{Shadow acne}
\slideimage{shadow-acne-scheme.png}
\end{frame}

\againframe<1->{acne}

\begin{frame}[fragile]
\frametitle{Back-facing shadows}
\slideimage{shadow-mapping-backface.png}
\end{frame}

\begin{frame}[fragile]
\frametitle{Peter panning}
\slideimage{peter-panning.png}
\end{frame}

\begin{frame}[fragile]
\frametitle{Peter panning}
\begin{itemize}
\item Объекты кажутся приподнятыми над поверхностью из-за того, что их тень начинается не сразу под объектом
\pause
\item Усиливается при использовании shadow bias
\pause
\item Лучший способ исправить -- не использовать слишком тонкую геометрию
\end{itemize}
\end{frame}

\begin{frame}[fragile]
\frametitle{Shadow mapping}
\slideimage{shadow-mapping3.png}
\end{frame}

\begin{frame}[fragile]
\frametitle{Разрешение shadow map}
\slideimage{shadow-map-resolution.png}
\end{frame}

\begin{frame}[fragile]
\frametitle{Shadow mapping}
\fontsize{10pt}{10pt}
\begin{itemize}
\item При низком разрешении shadow map хорошо видны `пиксели' теней
\pause
\item Чем выше разрешение, тем лучше жёсткая тень
\pause
\begin{itemize}
\item Алгоритм silhouette mapping позволяет получить жёсткую тень без слишком большого разрешения
\end{itemize}
\pause
\item Можно слегка размыть тень, избавившись от `пикселей' и получив аппроксимацию мягких теней
\pause
\begin{itemize}
\item Размывать нужно результат сравнения глубины со значением в shadow map, а \textbf{не} саму shadow map!
\pause
\item Percentage-closer filtering (PCF)
\pause
\item В OpenGL есть встроенная реализация: shadow samplers
\end{itemize}
\pause
\item Более современные вариации shadow maps пишут в текстуру не глубину, а какие-нибудь значения, которые имеет смысл складывать и усреднять
\pause
\begin{itemize}
\item Позволяет использовать фильтрацию текстур, mipmaps, размытие shadow map
\pause
\item Exponential shadow maps, variance shadow maps
\end{itemize}
\end{itemize}
\end{frame}

\begin{frame}[fragile]
\frametitle{Shadow sampler}
\begin{itemize}
\item В OpenGL есть ограниченная поддержка PCF в виде \textit{shadow samplers}
\pause
\item Для текстуры нужно настроить параметры:
\begin{itemize}
\item \mintinline{cpp}|GL_TEXTURE_COMPARE_MODE| в значение \mintinline{cpp}|GL_COMPARE_REF_TO_TEXTURE|
\item \mintinline{cpp}|GL_TEXTURE_COMPARE_FUNC| в значение \mintinline{cpp}|GL_LEQUAL| (теоретически могут быть \mintinline{cpp}|GL_ALWAYS| и т.п.)
\end{itemize}
\pause
\item Текстуру с \mintinline{cpp}|GL_COMPARE_REF_TO_TEXTURE| \textbf{нельзя} использовать как \mintinline{glsl}|sampler2D|, для неё есть отдельный тип sampler'а: \mintinline{glsl}|sampler2DShadow|
\end{itemize}
\end{frame}

\begin{frame}[fragile]
\frametitle{Shadow sampler}
\begin{itemize}
\item Функция \mintinline{glsl}|texture(shadow_map, texcoord)| принимает трёхмерный вектор текстурных координат:
\begin{itemize}
\item Первые две -- обычные X,Y текстурные координаты
\item Третья -- используется для сравнения со значением в shadow map
\end{itemize}
\pause
\item Если сравнение успешно, возвращается 1, иначе 0 (функция сравнения настраивается \mintinline{cpp}|GL_TEXTURE_COMPARE_FUNC|)
\pause
\item Для такой текстуры можно включить линейную фильтрацию -- интерполироваться будут не значения глубины, а \textit{результаты сравнения}
\end{itemize}
\end{frame}

\begin{frame}[fragile]
\frametitle{Shadow sampler с линейной фильтрацией}
\slideimage{pcf.png}
\end{frame}

\begin{frame}[fragile]
\frametitle{Shadow sampler с линейной фильтрацией и 7x7 размытием по Гауссу}
\slideimage{pcf_gauss.png}
\end{frame}

\begin{frame}[fragile]
\frametitle{Shadow mapping: ссылки}
\begin{itemize}
\item \href{https://learnopengl.com/Advanced-Lighting/Shadows/Shadow-Mapping}{\nolinkurl{learnopengl.com/Advanced-Lighting/Shadows/Shadow-Mapping}}
\item \href{http://www.opengl-tutorial.org/intermediate-tutorials/tutorial-16-shadow-mapping}{\nolinkurl{opengl-tutorial.org/intermediate-tutorials/tutorial-16-shadow-mapping}}
\item \href{https://ogldev.org/www/tutorial23/tutorial23.html}{\nolinkurl{ogldev.org/www/tutorial23/tutorial23.html}}
\item \href{https://developer.nvidia.com/gpugems/gpugems2/part-ii-shading-lighting-and-shadows/chapter-17-efficient-soft-edged-shadows-using}{\texttt{GPU Gems 2, Chapter 17. Efficient Soft-Edged Shadows Using Pixel Shader Branching}}
\end{itemize}
\end{frame}

\end{document}