% (c) Nikita Lisitsa, lisyarus@gmail.com, 2023

\documentclass[10pt]{beamer}

\usepackage[T2A]{fontenc}
\usepackage[russian]{babel}
\usepackage{minted}

\usepackage{graphicx}
\graphicspath{ {./images/} }

\usepackage{adjustbox}

\usetheme{metropolis}

\usepackage{color}
\usepackage{soul}

\usepackage{hyperref}

\definecolor{blue}{rgb}{0,0,1}
\definecolor{red}{rgb}{1,0,0}
\definecolor{codebg}{RGB}{29,35,49}

\makeatletter
\newcommand{\slideimage}[1]{
  \begin{figure}
    \begin{adjustbox}{width=\textwidth, totalheight=\textheight-2\baselineskip-2\baselineskip,keepaspectratio}
      \includegraphics{#1}
    \end{adjustbox}
  \end{figure}
}
\makeatother

\title{Компьютерная графика}
\subtitle{Практика 11: Система частиц}
\date{2023}

\usemintedstyle{solarized-light}

\setbeamertemplate{footline}[frame number]

\begin{document}

\frame{\titlepage}

\begin{frame}[fragile]
\slideimage{0.png}
\end{frame}

\begin{frame}[fragile]
N.B: Камеру можно крутить и двигать стрелочками
\end{frame}

\begin{frame}[fragile]
\frametitle{Задание 1}
Рисуем частицы как квадраты
\begin{itemize}
\item В структуру \mintinline{glsl}|particle| добавляем параметр \mintinline{glsl}|float size|, инициализируем его в случайное значение (например, от \mintinline{glsl}|0.2| до \mintinline{glsl}|0.4|)
\item Добавляем соответствующий атрибут для VAO и в вершинном шейдере
\item Вершинный шейдер просто передаёт значение (\mintinline{glsl}|out float size|) в геометрический шейдер (где оно принимается как \mintinline{glsl}|in float size[]| -- геометрический шейдер обрабатывает сразу целый примитив, т.е. несколько вершин, поэтому нужен массив)
\item В геометрическом шейдере меняем тип выходной геометрии: \mintinline{glsl}|triangle_strip|, \mintinline{glsl}|max_vertices = 4|
\item В геометрическом шейдере вместо генерации одной вершины генерируем 4 вершины с координатами \begin{math}center + (\pm size, \pm size, 0)\end{math}
\end{itemize}
\end{frame}

\begin{frame}[fragile]
\slideimage{1.png}
\end{frame}

\begin{frame}[fragile]
\frametitle{Задание 2}
Добавляем текстурные координаты
\begin{itemize}
\item В геометрическом шейдере генерируем для вершин текстурные координаты (от 0 до 1 по каждой оси)
\item В геометрическом шейдере \mintinline{glsl}|out vec2 texcoord|
\item Во фрагментном шейдере, как обычно, \mintinline{glsl}|in vec2 texcoord|
\item Во фрагментном шейдере используем текстурные координаты в качестве цвета: \mintinline{glsl}|vec4(texcoord, 0.0, 1.0)|
\end{itemize}
\end{frame}

\begin{frame}[fragile]
\slideimage{2.png}
\end{frame}

\begin{frame}[fragile]
\frametitle{Задание 3}
Поворачиваем частицы в сторону камеры
\begin{itemize}
\item В геометрическом шейдере вычисляем X, Y, Z оси для частицы:
\begin{itemize}
\item Z -- направление из центра частицы на камеру
\item X, Y -- любые, перпендикулярные Z и друг другу
\end{itemize}
\item Вычисляем координаты вершин частицы так, чтобы она была параллельна плоскости XY
\end{itemize}
\end{frame}

\begin{frame}[fragile]
\slideimage{3.png}
\end{frame}

\begin{frame}[fragile]
\frametitle{Задание 4}
Cимулируем физику частиц
\begin{itemize}
\item Добавляем частице поле `скорость' типа \mintinline{glsl}|vec3| (как атрибут в шейдере оно не нужно), инициализируем его чем-то случайным
\item На каждый кадр, перед обновлением VBO, для каждой частицы, при условии \mintinline{glsl}|if (!paused)|:
\begin{itemize}
\item Увеличиваем Y-составляющую скорости на некую величину \mintinline{glsl}|velocity.y += dt * A|
\item Интегрируем скорость \mintinline{glsl}|position += velocity * dt|
\item Можно добавить трение \mintinline{glsl}|velocity *= exp(- C * dt)|
\item Можно уменьшать размер частицы \mintinline{glsl}|size *= exp(- D * dt)|
\end{itemize}
\end{itemize}
\end{frame}

\begin{frame}[fragile]
\slideimage{4.png}
\end{frame}

\begin{frame}[fragile]
\frametitle{Задание 5}
Вращаем частицы
\begin{itemize}
\item Добавляем частице атрибут `угол поворота' типа \mintinline{glsl}|float| (поле в структуру, входной параметр вершинного шейдера, настройка атрибута для VAO)
\item В геометрическом шейдере поворачиваем оси X, Y на этот угол
\item Добавляем частице поле `угловая скорость' типа \mintinline{glsl}|float| (как атрибут в шейдере оно не нужно), инициализируем его чем-то случайным
\item На каждый кадр, перед обновлением VBO, для каждой частицы, при условии \mintinline{glsl}|if (!paused)|:
\begin{itemize}
\item Интегрируем угловую скорость \mintinline{glsl}|rotation += angular_velocity * dt|
\end{itemize}
\end{itemize}
\end{frame}

\begin{frame}[fragile]
\slideimage{5.png}
\end{frame}

\begin{frame}[fragile]
\frametitle{Задание 6}
Создаём emitter частиц
\begin{itemize}
\item На каждый кадр, перед обновлением VBO, при условии \mintinline{glsl}|if (!paused)|:
\begin{itemize}
\item Создаём частицы не разом, а по одной в кадр, пока их не станет 256
\item Пересоздаём частицы с новыми случайными параметрами, если выполняется какое-то условие (например, Y-координата больше некого порогового значения, или размер меньше некого порогового значения)
\end{itemize}
\end{itemize}
\end{frame}

\begin{frame}[fragile]
\slideimage{6.png}
\end{frame}

\begin{frame}[fragile]
\frametitle{Задание 7}
Текстурируем частицы
\begin{itemize}
\item Загружаем изображение \mintinline{glsl}|particle.png| из файла в директории с проектом (путь до него уже есть в коде)
\item Создаём текстуру и загружаем в неё это изображение: internal format = \mintinline{glsl}|GL_RGBA8|, format = \mintinline{glsl}|GL_RGBA|, type = \mintinline{glsl}|GL_UNSIGNED_BYTE|, настраиваем линейную фильтрацию с mipmaps, генерируем mipmaps
\item Используем эту текстуру во фрагментном шейдере: текстура в оттенках серого, берём только первую координату цвета (\mintinline{glsl}|texture(...).r|) и используем как альфа-канал результирующего цвета (сам цвет можно сделать, например, белым)
\item Включаем аддитивный блендинг (blend func = \mintinline{glsl}|GL_SRC_ALPHA, GL_ONE|) и выключаем тест глубины
\end{itemize}
\end{frame}

\begin{frame}[fragile]
\slideimage{7.png}
\end{frame}

\begin{frame}[fragile]
\frametitle{Задание 8*}
\begin{footnotesize}
Раскрашиваем частицы
\begin{itemize}
\item Создаём одномерную текстуру с цветовой палитрой: \mintinline{glsl}|GL_TEXTURE_1D|, линейная фильтрация (без mipmaps), несколько вручную описанных пикселей (например, чёрный, оранжевый, жёлтый, белый), данные загружаются через \mintinline{glsl}|glTexImage1D|
\item Передаём эту текстуру во врагментный шейдер используя texture unit 1 (\mintinline{glsl}|GL_TEXTURE1|), в шейдере тип sampler'а -- \mintinline{glsl}|sampler1D|
\item Используем значение из первой текстуры (оно же -- альфа-канал результирующего цвета) для индексации в текстуру с палитрой, результирующий цвет = цвет из палитры + альфа-канал из первой текстуры
\item Можно дополнительно умножить текстурную координату для палитры (оно же -- значение альфа) на некую функцию от размера частицы (чтобы маленькие частицы были темнее; размер придётся передать во фрагментный шейдер)
\end{itemize}
\end{footnotesize}
\end{frame}

\begin{frame}[fragile]
\slideimage{8.png}
\end{frame}

\end{document}