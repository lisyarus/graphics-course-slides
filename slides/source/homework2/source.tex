% (c) Nikita Lisitsa, lisyarus@gmail.com, 2021

\documentclass{beamer}

\usepackage[T2A]{fontenc}
\usepackage[utf8]{inputenc}
\usepackage[russian]{babel}

\usepackage{graphicx}
\graphicspath{ {./images/} }

\usepackage{adjustbox}

\usepackage{tikz}

\usepackage{color}
\usepackage{soul}

\usepackage{hyperref}

\definecolor{blue}{rgb}{0,0,1}
\definecolor{red}{rgb}{1,0,0}

\makeatletter
\newcommand{\slideimage}[1]{
  \begin{figure}
    \begin{adjustbox}{width=\textwidth, totalheight=\textheight-2\baselineskip-2\baselineskip,keepaspectratio}
      \includegraphics{#1}
    \end{adjustbox}
  \end{figure}
}
\makeatother

\title{Компьютерная графика}
\subtitle{Домашнее задание 2: сцена с освещением, тенями и отражениями}
\date{2021}

\setbeamertemplate{footline}[frame number]

\begin{document}

\frame{\titlepage}

\begin{frame}[fragile]
\frametitle{Задание}
\begin{itemize}
\item Выбрать некоторую сцену, нарисовать её с текстурами, normal mapping'ом, несколькими источниками света и тенями
\pause
\item Выбрать некоторую модель, нарисовать её с reflection mapping'ом (отражениями), в отражении должна быть видна основная сцена, модель должна двигаться по сцене (сама или управляемая пользователем)
\pause
\item Камера должна управляться пользователем (любым способом, главное - чтобы можно было разглядеть все реализованные эффекты)
\end{itemize}
\end{frame}

\begin{frame}[fragile]
\frametitle{Сцена}
Ограничения на сцену:
\begin{itemize}
\item Порядка 100k-1kk треугольников в сумме
\item Порядка 100-1000 различных объектов
\item У большинства объектов есть текстуры альбедо и нормалей
\end{itemize}
Рекомендую \textit{Crytek Sponza}
\end{frame}

\begin{frame}[fragile]
\frametitle{Crytek Sponza}
\slideimage{crytek-sponza.jpg}
\end{frame}

\begin{frame}[fragile]
\frametitle{Сцена: формат}
Один из общепринятых форматов сцен - Wavefront OBJ
\begin{itemize}
\item Текстовый, достаточно простой для чтения
\item Содержит координаты вершин, нормали и текстурные координаты
\item Содержит индексы вершин, образующих треугольники
\item Может содержать много объектов
\item Может ссылаться на MTL-файл (Material Template Library), содержащий описания материалов
\item MTL может в свою очередь ссылаться на текстуры (альбедо, нормали, и т.п.)
\end{itemize}
При желании можно использовать и другие форматы, это не принципиально
\end{frame}

\begin{frame}[fragile]
\frametitle{Модель}
В качестве модели можно взять что угодно, например:
\begin{itemize}
\item Stanford dragon (7ая практика)
\item Stanford bunny (8ая практика)
\item Utah teapot
\item Happy buddha
\item Suzanne
\end{itemize}
От неё нужны только координаты вершин и нормали (нормали можно и посчитать самим)
\end{frame}

\begin{frame}[fragile]
\frametitle{Текстуры}
Текстуры нужно как-то прочитать, варианты:
\begin{itemize}
\item Предварительно сконвертировать в максимально простой формат (raw rgb / netpbm) - это можно сделать утилитой convert пакета imagemagick или любым серьёзным редактором изображений (Photoshop/GIMP/etc)
\item Использовать библиотеку загрузки изображений: SDL\_image / stb\_image / Boost.GIL (рекомендую stb\_image)
\end{itemize}
\end{frame}

\begin{frame}[fragile]
\frametitle{Советы}
\begin{itemize}
\item Скорее всего, у вас будет по VAO + VBO + EBO + набор текстур на объект, удобно это занести в некую структуру/класс
\item Скорее всего, у вас будет 1 draw call (\verb|glDrawElements| или т.п.) на один объект
\item Скорее всего, у вас будет 7 FBO: 1 для рисования теней, ещё 6 для рисования отражений (по 1 на каждую грань cubemap-текстуры)
\item Объём данных немаленький, возможно проще будет сначала тестировать на упрощённой сцене
\item Возможно будет удобнее предварительно перевести сцену в свой бинарный формат
\item При использовании формата OBJ \textbf{внимательно} прочитайте документацию по нему, треугольники там описываются немного неочевидным образом
\item Для теста отражений можно сначала использовать куб вместо сложной модели
\end{itemize}
\end{frame}

\begin{frame}[fragile]
\frametitle{Баллы}
\begin{itemize}
\item 4 балла: рисуется сцена с ambient освещением (AO необязательно) и текстурами, двигается камера
\item 1 балл: есть направленный источник света ("солнце"{}, откуда-нибудь сверху) и несколько точечных
\item 2 балла: освещение учитывает normal mapping
\item 3 балла: есть тени от "солнца"
\item 1 балл: рисуется двигающаяся модель (расчёт освещения для неё не принципиален)
\item 4 балла: на модели видны отражения окружающей сцены (должны рассчитываться на каждый кадр)
\end{itemize}
Всего: 15 баллов

Защита заданий на практике 16 ноября
\end{frame}

\begin{frame}[fragile]
\frametitle{Ссылки}
\begin{itemize}
\item \href{https://casual-effects.com/g3d/data10/index.html#mesh8}{Crytek Sponza} (на этом же сайте есть много других моделей)
\item \href{https://en.wikipedia.org/wiki/Wavefront_.obj_file}{Описание формата Wavefront OBJ}
\item \href{https://github.com/nothings/stb/blob/master/stb_image.h}{Библиотека stb\_image}
\item \href{https://learnopengl.com/Advanced-OpenGL/Cubemaps}{Туториал про environment mapping}
\item \href{https://www.youtube.com/watch?v=xutvBtrG23A}{Видео-туториал про reflection mapping}
\end{itemize}
\end{frame}

\end{document}