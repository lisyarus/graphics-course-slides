% (c) Nikita Lisitsa, lisyarus@gmail.com, 2021

\documentclass{beamer}

\usepackage[T2A]{fontenc}
\usepackage[utf8]{inputenc}
\usepackage[russian]{babel}

\usepackage{graphicx}
\graphicspath{ {./images/} }

\usepackage{adjustbox}

\usepackage{tikz}

\usepackage{color}
\usepackage{soul}

\usepackage{hyperref}

\definecolor{blue}{rgb}{0,0,1}
\definecolor{red}{rgb}{1,0,0}

\makeatletter
\newcommand{\slideimage}[1]{
  \begin{figure}
    \begin{adjustbox}{width=\textwidth, totalheight=\textheight-2\baselineskip-2\baselineskip,keepaspectratio}
      \includegraphics{#1}
    \end{adjustbox}
  \end{figure}
}
\makeatother

\title{Компьютерная графика}
\subtitle{Финальный проект}
\date{2021}

\setbeamertemplate{footline}[frame number]

\begin{document}

\frame{\titlepage}

\begin{frame}[fragile]
\frametitle{Задание}
\begin{itemize}
\item Свободная форма: игра / визуализация / динамическая сцена / etc
\pause
\item Можно пользоваться вспомогательными библиотеками (например, для загрузки текстур, моделей и сцен)
\begin{itemize}
\item Весь OpenGL-код должен быть написан вами, т.е. библиотека не должна загружать данные на GPU  т.п.
\end{itemize}
\pause
\item Нельзя брать сцены/модели из практик и домашних заданий
\pause
\item Оценивается наличие конкретных эффектов
\end{itemize}
\end{frame}

\begin{frame}[fragile]
\frametitle{Баллы}
\begin{itemize}
\item 1 балл: что-нибудь рисуется с помощью OpenGL
\item 1 балл: движущаяся камера (управляемая пользователем или кодом)
\item 1 балл: управляемые пользователем объекты
\item 1 балл: движущиеся сами по себе объекты
\item 2 балла: иерархии объектов
\item 3 балла: скелетная анимация
\end{itemize}
\end{frame}

\begin{frame}[fragile]
\frametitle{Баллы}
\begin{itemize}
\item 2 балла: освещение (ambient+diffuse+specular)
\item 3 балла: тени от направленного источника
\item 3 балла: тени от точечного источника
\begin{itemize}
\item +2 балла: ESM/VSM с размытием
\end{itemize}
\item 3 балла: полупрозрачные объекты (с сортировкой)
\end{itemize}
\end{frame}

\begin{frame}[fragile]
\frametitle{Баллы}
\begin{itemize}
\item 1 балл: альбедо-текстуры
\item 2 балла: normal mapping
\item 2 балла: material mapping
\item 3 балла: bitmap-анимации
\end{itemize}
\end{frame}

\begin{frame}[fragile]
\frametitle{Баллы}
\begin{itemize}
\item 1 балл: skybox
\item 2 балла: environment mapping
\item 3 балла: отражения
\end{itemize}
\end{frame}

\begin{frame}[fragile]
\frametitle{Баллы}
\begin{itemize}
\item 1 балл: tone mapping + gamma correction
\item 2 балла: blur
\begin{itemize}
\item +2 балла: separable kernel
\item +3 балла: depth blur
\end{itemize}
\item 2 балла: toon shading (edge detection + cel shading)
\item 2 балла: dithering
\item 3 балла: bloom
\end{itemize}
\end{frame}

\begin{frame}[fragile]
\frametitle{Баллы}
\begin{itemize}
\item 2 балла: bitmap-текст
\item 3 балла: SDF-текст
\item 3 балла: raymarched volume rendering
\item 3 балла: slicing volume rendering
\end{itemize}
\end{frame}

\begin{frame}[fragile]
\frametitle{Баллы}
\begin{itemize}
\item 3 балла: LOD для моделей
\item 3 балла: frustum culling
\end{itemize}
\end{frame}

\begin{frame}[fragile]
\frametitle{Баллы}
\begin{itemize}
\item 5 баллов: cascaded shadow maps
\item 7 баллов: screen-space ambient occlusion
\item 7 баллов: screen-space reflections
\item 10 баллов: deferred shading
\end{itemize}
\end{frame}

\end{document}