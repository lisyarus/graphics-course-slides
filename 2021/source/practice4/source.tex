% (c) Nikita Lisitsa, lisyarus@gmail.com, 2021

\documentclass{beamer}

\usepackage[T2A]{fontenc}
\usepackage[utf8]{inputenc}
\usepackage[russian]{babel}

\usepackage{graphicx}
\graphicspath{ {./images/} }

\usepackage{adjustbox}

\usepackage{color}
\usepackage{soul}

\usepackage{hyperref}

\definecolor{blue}{rgb}{0,0,1}
\definecolor{red}{rgb}{1,0,0}

\makeatletter
\newcommand{\slideimage}[1]{
  \begin{figure}
    \begin{adjustbox}{width=\textwidth, totalheight=\textheight-2\baselineskip-2\baselineskip,keepaspectratio}
      \includegraphics{#1}
    \end{adjustbox}
  \end{figure}
}
\makeatother

\title{Компьютерная графика}
\subtitle{Практика 4: Индексы, перспективная проекция, буфер глубины}
\date{2021}

\setbeamertemplate{footline}[frame number]

\begin{document}

\frame{\titlepage}

\begin{frame}[fragile]
\frametitle{Напоминание про VBO и VAO}
\begin{itemize}
\item VBO - хранит данные, ничего не знает о формате
\item VAO - описывает формат и расположение атрибутов вершин
\item Концептуально, расположение = id буфера + сдвиг
\item VAO также хранит id текущего EBO (\verb|GL_ELEMENT_ARRAY_BUFFER|) - оттуда берутся индексы
\item Для рендеринга нужен только VAO
\item Чтобы обновить данные, нужен только VBO
\item Чтобы обновить индексы, нужен только EBO (можно использовать \verb|target = GL_ARRAY_BUFFER|)
\end{itemize}
\end{frame}

\begin{frame}[fragile]
\frametitle{Задание 1}
Рисуем куб
\begin{itemize}
\item Создаём VAO, VBO, EBO
\pause
\item Загружаем данные в VBO и EBO
\pause
\item Настраиваем атрибуты для VAO
\pause
\item Рисуем с помощью \verb|glDrawElements|, \verb|mode = GL_TRIANGLES|, обращаем внимание на тип индексов (\verb|GL_UNSIGNED_INT|)
\end{itemize}
\end{frame}

\begin{frame}[fragile]
\frametitle{Задание 2}
Вращаем куб
\begin{itemize}
\item Меняем матрицу \verb|transform| чтобы куб крутился в плоскости XZ
\pause
\item В качестве угла можно взять что-нибудь зависящее от времени, например \verb|float angle = time;|
\pause
\item Куб будет обрезаться по \begin{math}z \in [-1, 1]\end{math}
\pause
\item Отмасштабируем его по всем осям \verb|float scale = 0.5f|, тоже с помощью матрицы \verb|transform|
\end{itemize}
\end{frame}

\begin{frame}[fragile]
\frametitle{Задание 3}
Добавляем перспективу
\begin{itemize}
\item Выбираем значения \verb|near|, \verb|far|, \verb|top|, \verb|right|
\begin{itemize}
\item \verb|near| - маленький, но не слишком, в духе \begin{math}0.001 \dots 0.1\end{math}
\item \verb|far| - большой, но не слишком, в духе \begin{math}10.0 \dots 1000.0\end{math}
\item Отношение \verb|right/near| - тангенс угла обзора, например \verb|right = near| это \begin{math}45^\circ\end{math}
\item Отношение \verb|right/top| - aspect ratio экрана (\verb|width/height|)
\end{itemize}
\pause
\item В матрицу \verb|view| записываем матрицу проекции с использованием выбранных значений
\pause
\item Куб будет виден изнутри
\end{itemize}
\end{frame}

\begin{frame}[fragile]
\frametitle{Задание 4}
Включаем тест глубины
\begin{itemize}
\item Сдвигаем куб по оси Z на какое-то расстояние (например, на 5 единиц)
\begin{itemize}
\item N.B: Камера смотрит в сторону -Z, т.е. сдвиг должен быть отрицательным
\end{itemize}
\pause
\item Куб будет рисоваться неправильно: задние грани перекрывают передние
\pause
\item Включим тест глубины: \verb|glEnable(GL_DEPTH_TEST)|
\pause
\item Не забываем очищать буфер глубины в начале каждого кадра: \verb|glClear(GL_DEPTH_BUFFER_BIT)|
\end{itemize}
\end{frame}

\begin{frame}[fragile]
\frametitle{Задание 5}
Двигаем куб
\begin{itemize}
\item Заведём переменные \verb|cube_x| и \verb|cube_y| с координатами центра куба по X и Y
\pause
\item Изменим матрицу \verb|transform|, добавив соответствующие сдвиги по X и Y
\pause
\item Перед рисованием каждого кадра обновим положение куба:
\begin{itemize}
\item Если нажата \verb|SDLK_LEFT|, сдвинем куб влево: \verb|cube_x -= speed * dt|
\item Аналогично \verb|SDLK_RIGHT|, \verb|SDLK_DOWN|, \verb|SDLK_UP|
\end{itemize}
\end{itemize}
\end{frame}

\begin{frame}[fragile]
\frametitle{Задание 6}
Играем с face culling
\begin{itemize}
\item Включим back-face culling: \verb|glEnable(GL_CULL_FACE)|
\pause
\item Ничего не изменится: куб сделан так, чтобы все треугольники были CCW
\pause
\item Изменим режим: \verb|glCullFace(GL_FRONT)|
\pause
\item Должны быть видны задние грани куба и не видны передние
\end{itemize}
\end{frame}

\begin{frame}[fragile]
\frametitle{Задание 7}
Три вращающихся куба
\begin{itemize}
\item Рисуем три куба в разных местах, вращающихся в плоскостях XY, XZ, YZ соответственно
\pause
\item Три раза \verb|glUniformMatrix| + \verb|glDrawElements|
\pause
\item Всё ещё можно двигать стрелочками, т.е. должны учитываться \verb|cube_x| и \verb|cube_y|
\end{itemize}
\end{frame}

\end{document}