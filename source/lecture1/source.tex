% (c) Nikita Lisitsa, lisyarus@gmail.com, 2025

\documentclass[10pt]{beamer}

\usepackage[T2A]{fontenc}
\usepackage[russian]{babel}
\usepackage{minted}

\usepackage{graphicx}
\graphicspath{ {./images/} }

\usepackage{adjustbox}

\usepackage{color}
\usepackage{soul}

\usepackage{hyperref}

\usetheme{metropolis}

\definecolor{red}{rgb}{1,0,0}
\definecolor{codebg}{RGB}{29,35,49}
\setminted{fontsize=\footnotesize}

\makeatletter
\newcommand{\slideimage}[1]{
  \begin{figure}
    \begin{adjustbox}{width=\textwidth, totalheight=\textheight-2\baselineskip-2\baselineskip,keepaspectratio}
      \includegraphics{#1}
    \end{adjustbox}
  \end{figure}
}
\makeatother

\title{Компьютерная графика}
\subtitle{Лекция 1: Введение в курс. История realtime графики. Графические API.}
\date{2025}

\setbeamertemplate{footline}[frame number]

\begin{document}

\frame{\titlepage}

\begin{frame}
\frametitle{}
\begin{itemize}
\item Лисица Никита Игоревич (Яндекс)
\item \nolinkurl{lisyarus@gmail.com}
\item \nolinkurl{+7(952)276-70-50}
\end{itemize}
\end{frame}

\begin{frame}
\frametitle{Как устроен курс}
\begin{itemize}
\item Репозиторий со слайдами: \href{https://github.com/lisyarus/graphics-course-slides/tree/master/pdf}{\nolinkurl{github.com/lisyarus/graphics-course-slides}}
\item Репозиторий с заготовками для практик: \href{https://github.com/lisyarus/graphics-course-practice}{\nolinkurl{github.com/lisyarus/graphics-course-practice}}
\end{itemize}
\end{frame}

\begin{frame}
\frametitle{Как устроен курс}
\begin{itemize}
\item \textbf{Лекции}
\begin{itemize}
\item Слайды (в репозитории \href{https://github.com/lisyarus/graphics-course-slides/tree/master/pdf}{\nolinkurl{slides}})
\end{itemize}
\pause
\item \textbf{Практики}
\pause
\begin{itemize}
\item Слайды с заданием (в репозитории \href{https://github.com/lisyarus/graphics-course-slides/tree/master/pdf}{\nolinkurl{slides}})
\pause
\item Код-заготовка на C++ (в репозитории \href{https://github.com/lisyarus/graphics-course-practice}{\nolinkurl{practice}})
\pause
\item Сдача на занятии или отправкой кода
\pause
\item В репозитории \href{https://github.com/lisyarus/graphics-course-practice}{\nolinkurl{practice}} есть \textit{подробная инструкция} по развёртке тестового проекта, -- лучше это сделать \textit{заранее}, чтобы не тратить время на первой практике
\end{itemize}
\pause
\item \textbf{Домашние задания}
\pause
\begin{itemize}
\item Слайды с заданием (в репозитории \href{https://github.com/lisyarus/graphics-course-slides/tree/master/pdf}{\nolinkurl{slides}})
\pause
\item Сдача на практическом занятии
\end{itemize}
\pause
\item \textbf{Финальный проект}
\begin{itemize}
\item Большое, сложное, интересное задание по выбору
\pause
\item Сдача в конце курса
\end{itemize}
\end{itemize}
\end{frame}

\begin{frame}
\frametitle{Баллы}
\begin{itemize}
\pause
\item \textbf{30-45 баллов} -- работа на практиках
\pause
\begin{itemize}
\item \textbf{1 балл} -- получилось хоть что-нибудь
\item \textbf{2 балл} -- получилось всё
\item \textbf{3 балла} -- получилось всё + доп. задание
\pause
\item Можно прислать код (или сдать лично) в течение недели (до полуночи перед следующей практикой)
\pause
\item Можно прислать позже со штрафом в \alert{\textbf{1 балл}}
\end{itemize}
\pause
\item \textbf{45 баллов} -- 3 домашних задания по \textbf{15 баллов}
\pause
\begin{itemize}
\item Можно получить неполный балл
\pause
\item Можно сдать после дня сдачи со штрафом в \alert{\textbf{50\% баллов}}
\end{itemize}
\pause
\item \textbf{30+ баллов} -- финальный проект
\end{itemize}
\end{frame}

\begin{frame}
\frametitle{Оценка за курс}
\pause
\begin{itemize}
\item Зачет: \textbf{50 и более} баллов
\pause
\item Экзамен:
\begin{itemize}
\item \textbf{50-59} баллов: \textbf{E}
\item \textbf{60-69} баллов: \textbf{D}
\item \textbf{70-79} баллов: \textbf{C}
\item \textbf{80-89} баллов: \textbf{B}
\item \textbf{90-100} баллов: \textbf{A}
\end{itemize}
\end{itemize}
\end{frame}

\begin{frame}
\frametitle{Пререквезиты}
\pause
\begin{itemize}
\item \textbf{Программирование}
\pause
\begin{itemize}
\item Основы C++
\pause
\item Компилировать и запускать программы в удобной вам среде
\end{itemize}
\pause
\item \textbf{Математика}
\pause
\begin{itemize}
\item Линейная алгебра (векторы, матрицы, умножение матриц, линейные системы, ортогональность)
\pause
\item Аналитическая геометрия (координаты, уравнения кривых и поверхностей)
\pause
\item Анализ (производные, интегралы, трансцендентные функции)
\end{itemize}
\end{itemize}
\end{frame}

\begin{frame}<1-2>[label=what_is]
\frametitle{Что такое компьютерная графика?}
\begin{itemize}
\pause % 1
\item Кинематограф, мультипликация
\pause % 2
\item Компьютерные игры
\pause % 3
\item Рисунки, concept art
\pause % 4
\item Графический интерфейс
\pause % 5
\item Визуализация данных
\pause % 6
\item Научная визуализация
\pause % 7
\item Карты
\pause % 8
\item И т.д.
\end{itemize}
\end{frame}

\begin{frame}
\frametitle{The Matrix Revolutions (2003)}
\begin{figure}
\slideimage{matrix.jpg}
\end{figure}
\end{frame}

\begin{frame}
\frametitle{Avatar (2009)}
\begin{figure}
\slideimage{avatar.jpg}
\end{figure}
\end{frame}

\begin{frame}
\frametitle{The Avengers (2012)}
\begin{figure}
\slideimage{avengers.jpg}
\end{figure}
\end{frame}

\begin{frame}
\frametitle{Klaus (2019)}
\begin{figure}
\slideimage{klaus.jpg}
\end{figure}
\end{frame}

\againframe<3>{what_is}

\begin{frame}
\frametitle{Space Invaders (1978)}
\begin{figure}
\slideimage{space-invaders.jpg}
\end{figure}
\end{frame}

\begin{frame}
\frametitle{Doom (1993)}
\begin{figure}
\slideimage{doom.png}
\end{figure}
\end{frame}

\begin{frame}
\frametitle{Grand Theft Auto: Vice City (2002)}
\begin{figure}
\slideimage{vice-city.jpg}
\end{figure}
\end{frame}

\begin{frame}
\frametitle{Civilization V (2010)}
\begin{figure}
\slideimage{civ-5.jpg}
\end{figure}
\end{frame}

\begin{frame}
\frametitle{The Witcher 3: Wild Hunt (2015)}
\begin{figure}
\slideimage{witcher.jpg}
\end{figure}
\end{frame}

\begin{frame}
\frametitle{Cyberpunk 2077 (2020)}
\begin{figure}
\slideimage{cyberpunk.jpg}
\end{figure}
\end{frame}

\againframe<4>{what_is}

\begin{frame}
\begin{figure}
\slideimage{night.png}
\end{figure}
\end{frame}

\begin{frame}
\begin{figure}
\slideimage{tunnel.png}
\end{figure}
\end{frame}

\begin{frame}
\begin{figure}
\slideimage{forest.jpg}
\end{figure}
\end{frame}

\againframe<5>{what_is}

\begin{frame}
\frametitle{Mac OS Catalina}
\begin{figure}
\slideimage{macos.png}
\end{figure}
\end{frame}

\begin{frame}
\frametitle{Windows 10}
\begin{figure}
\slideimage{windows.png}
\end{figure}
\end{frame}

\begin{frame}
\frametitle{Europa Universalis 4}
\begin{figure}
\slideimage{eu4.png}
\end{figure}
\end{frame}

\begin{frame}
\frametitle{Dear ImGui}
\begin{figure}
\slideimage{imgui.png}
\end{figure}
\end{frame}

\againframe<6>{what_is}

\begin{frame}
\frametitle{Популярность браузеров в 2002-2009}
\begin{figure}
\slideimage{browsers.png}
\end{figure}
\end{frame}

\begin{frame}
\frametitle{Карта землетрясений}
\begin{figure}
\slideimage{earthquakes.png}
\end{figure}
\end{frame}

\begin{frame}
\frametitle{Количество случаев заражения COVID-19}
\begin{figure}
\slideimage{covid.png}
\end{figure}
\end{frame}

\againframe<7>{what_is}

\begin{frame}
\frametitle{Неустойчивость Рэлея — Тейлора}
\begin{figure}
\slideimage{fluids.jpg}
\end{figure}
\end{frame}

\begin{frame}
\frametitle{Молекулярные орбитали бензола}
\begin{figure}
\slideimage{benzene.png}
\end{figure}
\end{frame}

\begin{frame}
\frametitle{Симуляция напряжений в стенте методом конечных элементов}
\begin{figure}
\slideimage{stent.jpg}
\end{figure}
\end{frame}

\againframe<8>{what_is}

\begin{frame}
\frametitle{Схематическая карта}
\begin{figure}
\slideimage{map.png}
\end{figure}
\end{frame}

\begin{frame}
\frametitle{Спутниковая карта}
\begin{figure}
\slideimage{satellite.png}
\end{figure}
\end{frame}

\begin{frame}
\frametitle{Карта погоды}
\begin{figure}
\slideimage{weather.png}
\end{figure}
\end{frame}

\againframe<9>{what_is}

\begin{frame}<1-2>[label=classification]
\frametitle{Грубая и неточная классификация}
\pause
\begin{itemize}
\item \only<2>{2D / 3D} \pause \only<3->{2D / 2.5D / 3D}
\pause
\item Векторная / растровая
\pause
\item \only<5>{Realtime / offline} \pause \only<6->{Realtime / near real-time / offline}
\pause
\item Фотореалистичная / стилизованная
\pause
\item CPU / GPU
\end{itemize}
\end{frame}

\begin{frame}
\frametitle{Super Mario Bros. (1983) - 2D}
\begin{figure}
\slideimage{mario.jpg}
\end{figure}
\end{frame}

\begin{frame}
\frametitle{Red Dead Redemption 2 (2018) - 3D}
\begin{figure}
\slideimage{rdr2.jpg}
\end{figure}
\end{frame}

\againframe<2-3>{classification}

\begin{frame}
\frametitle{Civilization III (2001) - 2.5D}
\begin{figure}
\slideimage{civ3.png}
\end{figure}
\end{frame}

\againframe<3-4>{classification}

\begin{frame}
\frametitle{Векторная графика}
\begin{figure}
\slideimage{vector.jpg}
\end{figure}
\end{frame}

\begin{frame}
\frametitle{Растровая графика}
\begin{figure}
\slideimage{raster.png}
\end{figure}
\end{frame}

\againframe<4->{classification}

\begin{frame}
\frametitle{Чем мы будем заниматься?}
\begin{itemize}
\item \only<-1>{2D / 2.5D / 3D}\only<2->{{\alert{\textbf{\underline{2D / 2.5D / 3D}}}}}
\item \only<-2>{Векторная / растровая}\only<3->{{\alert{\textbf{\underline{Векторная / растровая}}}}}
\item \only<-3>{Realtime}\only<4->{{\alert{\textbf{\underline{Realtime}}}}} / near real-time / offline
\item \only<-4>{Фотореалистичная / стилизованная}\only<5->{{\alert{\textbf{\underline{Фотореалистичная / стилизованная}}}}}
\item CPU / \only<-5>{GPU}\only<6->{{\alert{\textbf{\underline{GPU}}}}}
\end{itemize}
\end{frame}

\begin{frame}
\frametitle{Чем мы будем заниматься?}
\begin{itemize}
\pause
\item Как пользоваться \textit{графическими API}
\pause
\item Как реализовывать \textit{графические движки}
\pause
\item Как реализовывать \textit{графические эффекты}
\pause
\item Как их \textit{оптимизировать}
\pause
\item Профессия \alert{\textbf{\textit{\underline{graphics engineer}}}}
\end{itemize}
\end{frame}

\begin{frame}
\frametitle{Где это пригодится?}
\begin{itemize}
\pause
\item Разработка \textit{игр и игровых движков}
\pause
\item Разработка \textit{инструментов для художников/дизайнеров/архитекторов}
\pause
\item Разработка \textit{инструментов для научной визуализации/визуализации данных}
\pause
\item Разработка \textit{картографических приложений}
\pause
\item Разработка \textit{движков графического интерфейса}
\pause
\item И т.д.
\end{itemize}
\end{frame}

\begin{frame}
\frametitle{Чем мы \underline{не} будем заниматься?}
\begin{itemize}
\pause
\item \textbf{Учиться рисовать / моделировать и анимировать объекты / etc.}
\pause
\begin{itemize}
\item Красивая картинка -- движок + данные (текстуры, модели, частицы, etc, -- \textit{assets})
\item Курс про \alert{\textbf{\underline{\textit{движок}}}}
\end{itemize}
\pause
\item \textbf{Делать игры}
\begin{itemize}
\item Игра -- гейм-дизайн + контент + графика + физика + механики + UI + аудио + сетевые компоненты + ...
\item Курс про \alert{\textbf{\underline{\textit{графику}}}}
\end{itemize}
\end{itemize}
\end{frame}

\begin{frame}
\frametitle{Примерный план курса}
\pause
\begin{itemize}
\item \textbf{Основы OpenGL}
\begin{itemize}
\item Как хранить данные на GPU
\item Как рисовать эти данные
\item Вершинные буферы, шейдеры, текстуры, фреймбуферы
\item Работа с камерой, перспективная проекция
\end{itemize}
\pause
\item \textbf{Освещение}
\begin{itemize}
\item Теория
\item Модели освещения и материалов
\item Тени, отражения, ambient occlusion
\item Обработка большого количества источников света
\end{itemize}
\pause
\item \textbf{Эффекты и оптимизации}
\begin{itemize}
\item Системы частиц (e.g. дым)
\item Скелетная анимация
\item Уровни детализации, frustum culling
\item Объёмный (volumetric) рендеринг
\item Рендеринг текста
\end{itemize}
\end{itemize}
\end{frame}

\begin{frame}<1>[label=history-1]
\frametitle{Краткая история real-time компьютерной графики}
\textbf{1960-е: Осциллографы}
\pause
\begin{itemize}
\item Tennis For Two (1958)
\pause
\item Spacewar! (1962, PDP-1)
\pause
\item Sketchpad (1963, TX-2, световое перо)
\pause
\item Освещение и тени (Аппель, 1968)
\pause
\item Векторная графика довольно плохого качества: линии одного цвета
\item Компьютеры довольно слабые (TX-2 занимал большую комнату)
\end{itemize}
\end{frame}

\begin{frame}
\frametitle{Осциллограф}
\slideimage{oscilloscope.jpg}
\end{frame}

\againframe<1-2>{history-1}

\begin{frame}
\frametitle{Tennis For Two}
\slideimage{tennis-for-two.jpg}
\end{frame}

\againframe<2-3>{history-1}

\begin{frame}
\frametitle{PDP-1}
\slideimage{pdp1.jpg}
\end{frame}

\begin{frame}
\frametitle{Spacewar!}
\slideimage{spacewar.jpg}
\end{frame}

\againframe<3-4>{history-1}

\begin{frame}
\frametitle{Sketchpad}
\slideimage{sketchpad.jpg}
\href{https://archive.org/details/AlanKeyD1987}{\nolinkurl{archive.org/details/AlanKeyD1987}}
\end{frame}

\againframe<4-5>{history-1}

\begin{frame}
\frametitle{Освещение (Аппель, 1968)}
\slideimage{appel-shading.png}
\href{https://graphics.stanford.edu/courses/Appel.pdf}{\nolinkurl{graphics.stanford.edu/courses/Appel.pdf}}
\end{frame}

\againframe<5->{history-1}

\begin{frame}<1-2>[label=history-2]
\frametitle{Краткая история real-time компьютерной графики}
\textbf{1970-е: Аркадные игры}
\pause
\begin{itemize}
\item Magnavox Odyssey (Magnavox, 1972) -- первая игровая консоль, подключалась к телевизору (CRT)
\pause
\item Pong (Atari, 1972) -- одна из первых аркадных игр
\pause
\item Speed Race (Taito, 1974)
\pause
\item Gun Fight (Taito, 1975)
\pause
\item Space Invaders (Taito, 1978)
\pause
\item Pac-Man (Namco, 1980)
\pause
\item Переход к растровой графике
\pause
\item Разрешение экрана ограничено объёмами памяти
\end{itemize}
\end{frame}

\begin{frame}
\frametitle{Magnavox Odyssey}
\slideimage{magnavox.jpg}
\end{frame}

\againframe<2-3>{history-2}

\begin{frame}
\frametitle{Pong}
\slideimage{pong-cabinet.jpg}
\end{frame}

\begin{frame}
\frametitle{Pong}
\slideimage{pong.png}
\end{frame}

\againframe<3-4>{history-2}

\begin{frame}
\frametitle{Speed Race}
\slideimage{speed-race.png}
\end{frame}

\againframe<4-5>{history-2}

\begin{frame}
\frametitle{Gun Fight}
\slideimage{gun-fight.png}
\end{frame}

\againframe<5-6>{history-2}

\begin{frame}
\frametitle{Space Invaders}
\slideimage{space-invaders.jpg}
\end{frame}

\againframe<6-7>{history-2}

\begin{frame}
\frametitle{Pac-Man}
\slideimage{pac-man.png}
\end{frame}

\begin{frame}
\frametitle{Space Invaders}
\textbf{Технические характеристики:}
\pause
\begin{itemize}
\item Intel 8080, 8-bit, 2Mhz
\pause
\item Экран 256x224, монохромный (1-bit)
\pause
\item 8Kb ROM
\pause
\item 8Kb RAM, из которых 7Kb занимал экран (framebuffer)
\end{itemize}
\end{frame}

\againframe<7->{history-2}

\begin{frame}<1-2>[label=history-3]
\frametitle{Краткая история real-time компьютерной графики}
\textbf{1980-е: Векторные аркады}
\pause
\begin{itemize}
\item Asteroids (Atari, 1979)
\pause
\item Tempest (Atari, 1981)
\pause
\item Star Wars (Atari, 1983)
\pause
\item Требует специального оборудования (Atari's QuadraScan)
\pause
\item Может рисовать только линии
\end{itemize}
\end{frame}

\begin{frame}
\frametitle{Asteroids}
\slideimage{asteroids.png}
\end{frame}

\againframe<2-3>{history-3}

\begin{frame}
\frametitle{Tempest}
\slideimage{tempest.png}
\href{https://www.youtube.com/watch?v=eJVpYL44jUQ}{Arcade Machines look WEIRD in Slow Mo -- The Slow Mo Guys}
\end{frame}

\againframe<3-4>{history-3}

\begin{frame}
\frametitle{Star Wars}
\slideimage{star-wars.png}
\end{frame}

\againframe<4->{history-3}

\begin{frame}<1-2>[label=history-4]
\frametitle{Краткая история real-time компьютерной графики}
\textbf{1980-е: 8-битные спрайтовые консоли}
\pause
\begin{itemize}
\item Atari 2600 (Atari, 1977)
\pause
\item NES (Nintendo, 1983)
\pause
\item Рисуют готовые изображения (спрайты) в указанных частях экрана
\pause
\item Не так требовательны к объёмам памяти
\end{itemize}
\end{frame}

\begin{frame}
\frametitle{Atari 2600}
\slideimage{atari-2600.jpg}
\end{frame}

\begin{frame}
\frametitle{Donkey Kong (Nintendo, 1981)}
\slideimage{donkey-kong.png}
\end{frame}

\begin{frame}
\frametitle{Pitfall! (Activision, 1982)}
\slideimage{pitfall.png}
\end{frame}

\againframe<2-3>{history-4}

\begin{frame}
\frametitle{Super Mario Bros. (Nintendo, 1985)}
\slideimage{mario.png}
\end{frame}

\begin{frame}
\frametitle{The Legend of Zelda. (Nintendo, 1986)}
\slideimage{zelda.png}
\end{frame}

\againframe<3->{history-4}

\begin{frame}<1-2>[label=history-5]
\frametitle{Краткая история real-time компьютерной графики}
\textbf{Конец 1980-х: 16-битные консоли и персональные компьютеры}
\pause
\begin{itemize}
\item Sega Mega Drive (Sega, 1988)
\pause
\item Super NES (Nintendo, 1990)
\pause
\item Больше памяти, быстрее процессоры
\pause
\item Поддерживают больше спрайтов и цветов
\end{itemize}
\end{frame}

\begin{frame}
\frametitle{Sonic the Hedgehog (Sega, 1991)}
\slideimage{sonic.png}
\end{frame}

\againframe<2-3>{history-5}

\begin{frame}
\frametitle{Super Mario World (Nintendo, 1990)}
\slideimage{super-mario.jpg}
\end{frame}

\againframe<3->{history-5}

\begin{frame}<1-3>[label=history-6]
\frametitle{Краткая история real-time компьютерной графики}
\textbf{1990-е: Raycasting}
\pause
\begin{itemize}
\item Алгоритм рисования двумерных уровней в 3D
\pause
\item Wolfenstein 3D (id Software, 1992)
\pause
\item Doom (id Software, 1993)
\pause
\item Quake (id Software, 1996)
\end{itemize}
\end{frame}

\begin{frame}
\frametitle{Wolfenstein 3D}
\slideimage{wolfenstein.png}
\end{frame}

\againframe<3-4>{history-6}

\begin{frame}
\frametitle{Doom}
\slideimage{doom-1993.png}
\end{frame}

\againframe<4-5>{history-6}

\begin{frame}
\frametitle{Quake}
\slideimage{quake.png}
\end{frame}

\begin{frame}<1-5>[label=history-7]
\frametitle{Краткая история real-time компьютерной графики}
\textbf{1990-е: 32-битные консоли и компьютеры}
\pause
\begin{itemize}
\item Графические процессоры с 3D графикой (Sony Playstation, Sega Saturn, Nintendo 64)
\pause
\item Умеют рисовать полигоны с текстурами и примитивным освещением (порядка нескольких тысяч за кадр)
\pause
\item Разрешения экрана до 640x480, 24-bit цвет
\pause
\item Virtua Racing (Sega, 1992)
\pause
\item Tomb Raider (Core Design, 1996)
\pause
\item Crash Bandicoot (Naughty Dog, 1996)
\pause
\item Первые 3D API (OpenGL, DirectX)
\end{itemize}
\end{frame}

\begin{frame}
\frametitle{Virtua Racing}
\slideimage{racing.png}
\end{frame}

\againframe<5-6>{history-7}

\begin{frame}
\frametitle{Tomb Raider}
\slideimage{tomb-raider.png}
\end{frame}

\againframe<6-7>{history-7}

\begin{frame}
\frametitle{Crash Bandicoot}
\slideimage{crash.jpg}
\end{frame}

\begin{frame}
\frametitle{Краткая история real-time компьютерной графики}
\textbf{2000-е и дальше}
\pause
\begin{itemize}
\item Растут мощности как CPU, так и GPU
\pause
\item Растут доступные объёмы памяти
\pause
\item Новые возможности GPU: шейдеры, рендеринг в текстуру, тесселяция
\pause
\item Фотореалистичная графика
\end{itemize}
\end{frame}

\begin{frame}
\frametitle{Ghost of Tsushima (Sucker Punch Productions, 2020)}
\slideimage{ghost.jpg}
\end{frame}

% \begin{frame}
% \frametitle{О треугольниках}
% Почему основным примитивом рисования стал треугольник?
% \pause

% Более сложные геометрические фигуры (круг, многоугольник, и т.д.):
% \pause
% \begin{itemize}
% \item Сложно нарисовать на экране
% \pause
% \item Сложно интерполировать атрибуты (накладывать цвет и текстуру, вычислять освещение)
% \end{itemize}
% \end{frame}

% \begin{frame}
% \frametitle{О треугольниках}
% Плюсы треугольника:
% \pause
% \begin{itemize}
% \item Образ под действием перспективной проекции -- тоже треугольник
% \pause
% \item Есть единственный разумный способ интерполяции (линейная, с барицентрическими координатами)
% \pause
% \item Позволяет рисовать спрайты (прямоугольник -- два треугольника)
% \pause
% \item Позволяет рисовать многоугольники (посредством триангуляции)
% \pause
% \item Позволяет рисовать линии (превращая их в тонкие многоугольники)
% \pause
% \item Позволяет рисовать более сложные фигуры (аппроксимируя)
% \end{itemize}
% \end{frame}

\begin{frame}
\frametitle{Как использовать GPU? \only<2->{Графические API}}
\textbf{GPU -- Graphics Processing Unit}
\pause
\pause
\begin{itemize}
\item \textit{Вендор-специфичные} API (1980е -- 1990е)
\pause
\item \textbf{OpenGL} (Silicon Graphics, 1992)
\pause
\begin{itemize}
\item \only<-11>{\textbf{OpenGL 3.3} (Khronos Group, 2010)}\only<12->{\alert{\textbf{\underline{OpenGL 3.3}}} (Khronos Group, 2010)}
\end{itemize}
\pause
\item \textbf{DirectX} (Microsoft, 1995)
\pause
\begin{itemize}
\item {\only<11->{\color{mLightGreen}}\textbf{DirectX 12}} (Microsoft, 2015)
\end{itemize}
\pause
\item {\only<11->{\color{mLightGreen}}\textbf{Metal}} (Apple, 2014)
\pause
\item {\only<11->{\color{mLightGreen}}\textbf{Vulkan}} (Khronos Group, 2018)
\pause
\item {\only<11->{\color{mLightGreen}}\textbf{WebGPU}} (W3C, working draft)
\end{itemize}
\pause
\bigskip
{\color{mLightGreen}\textbf{\textit{Современные} API}}
\end{frame}

\begin{frame}
\frametitle{Как использовать GPU? API общего назначения (GPGPU):}
\textbf{GPGPU -- General-Purpose Graphics Processing Unit}
\pause
\begin{itemize}
\item \textbf{CUDA} (Nvidia, 2007)
\pause
\item \textbf{DirectX 11 DirectCompute} (Microsoft, 2008)
\pause
\item \textbf{OpenCL} (Khronos Group, 2009)
\pause
\item \textbf{OpenGL 4.3 Compute Shaders} (Khronos Group, 2012)
\pause
\item \textbf{Metal Compute Shaders} (Apple, 2014)
\pause
\item \textbf{Vulkan Compute Shaders} (Khronos Group, 2018)
\pause
\item \textbf{WebGPU Compute Shaders} (W3C, 2021)
\end{itemize}
\end{frame}

\begin{frame}
\frametitle{Почему OpenGL 3.3?}
\pause
\begin{itemize}
\item \textit{Широкая поддержка}: интегрированные GPU, встраиваемые устройства, телефоны, web, некоторые игровые приставки
\pause
\item Поддерживает \textit{все нужные нам возможности GPU}
\pause
\item +/- \textit{Кроссплатформенность} \pause (спасибо, Apple)
\pause
\item \textit{Низкий} порог вхождения (в сравнении с более современными API)
\pause
\item Достаточно \textit{старый}
\begin{itemize}
\item \textit{Много} вспомогательных библиотек
\item Известны \textit{best practices}
\item Известны все \textit{грабли} \pause (их много)
\end{itemize}
\end{itemize}
\end{frame}

\begin{frame}
\frametitle{Почему OpenGL 3.3?}
\pause
\begin{itemize}
\item Крупные движки переписывают на \textbf{Vulkan / DirectX 12}
\pause
\item Не у всех есть на это ресурсы
\pause
\item Не всем нужна самая крутая графика и производительность (работает в 60 fps -- и ладно)
\pause
\item \textbf{OpenGL} всё ещё широко используется
\pause
\item Все основные концепции \textbf{OpenGL} \textit{(шейдеры, атрибуты вершин, буферы с данными, текстуры, ...)} есть в любом графическом API $\Rightarrow$ изучение OpenGL \textit{поможет в изучении} более современных API
\end{itemize}
\end{frame}

\begin{frame}[fragile]
\frametitle{История графических API: OpenGL 1.0 (1992)}
\slideimage{opengl.png}
\usemintedstyle{lightbulb}
\begin{minted}[bgcolor=codebg]{python}
for object in scene.objects:
    glBegin(GL_TRIANGLES)
    for triangle in object.triangles:
        for vertex in triangle.vertices:
            glColor3f(vertex.color)
            glNormal3f(vertex.normal)
            glVertex3f(vertex.position)
    glEnd(GL_TRIANGLES)
\end{minted}
\end{frame}

\begin{frame}[fragile]
\frametitle{История графических API: OpenGL 1.0 (1992)}
\begin{itemize}
\item Данные хранятся \textit{в памяти CPU}
\pause
\item Несколько OpenGL-вызовов \textit{на каждую вершину}
\pause
\begin{itemize}
\item GPU становятся быстрее $\Longrightarrow$ основное время тратится не на рисование, а на \textit{накладные расходы} самих OpenGL-вызовов
\end{itemize}
\end{itemize}
\end{frame}

\begin{frame}[fragile]
\frametitle{История графических API: OpenGL 1.0 (1992)}
\usemintedstyle{solarized-light}
\begin{itemize}
\item Как менять положение объектов и камеры?
\pause
\item $\Longrightarrow$ \textbf{Матрицы преобразований} (\mintinline{cpp}|glMatrixMode, glLoadMatrix|)
\pause
\item \textbf{Fixed-function pipeline}: настраиваемая, но \textit{не расширяемая} последовательность операций (применить матрицы к входным данным, нарисовать треугольник на экране, выполнить тест глубины, ...)
\pause
\item \textbf{Асинхронный API}: команды выполнятся на GPU \textit{когда-нибудь}
\end{itemize}
\end{frame}

\begin{frame}[fragile]
\frametitle{История графических API: OpenGL 1.1 (1997)}
\usemintedstyle{solarized-light}
\begin{itemize}
\item \textbf{Vertex array} -- спецификация формата и расположения вершин
\begin{itemize}
\item Сказать, где находятся вершины одной командой \mintinline{cpp}|glVertexPointer|
\item Нарисовать все вершины одной командой \mintinline{cpp}|glDrawArrays|
\item Вершины всё ещё хранятся на CPU
\end{itemize}
\pause
\item \textbf{Текстуры} -- изображения в памяти GPU, натягиваемые на полигоны
\end{itemize}
\end{frame}

\begin{frame}[fragile]
\frametitle{История графических API: OpenGL 1.1 (1997)}
\usemintedstyle{lightbulb}
\begin{minted}[bgcolor=codebg]{python}
# на старте
for object in scene.objects:
    object.createVertexArray(object.vertices)
    object.createTexture()

# при рендеринге
for object in scene.objects:
    glBindTexture(GL_TEXTURE_2D, object.texture)
    glBindVertexArray(object.vertexArray)
    glDrawArrays(object.vertexCount)
\end{minted}
\end{frame}

\begin{frame}
\frametitle{История графических API: OpenGL 1.2 - 1.4 (1998 - 2002)}
\begin{itemize}
\item В текстурах можно записать \textit{очень много интересного}, помимо цвета: normal map, material map, bump map
\item Хочется выполнять \textit{сложные вычисления} на каждый пиксель
\pause
\item $\Longrightarrow$ \textbf{Texture environments} -- зачатки программируемости GPU (шейдеров)
\end{itemize}
\end{frame}

\begin{frame}[fragile]
\frametitle{История графических API: OpenGL 1.5 (2003)}
\begin{itemize}
\item \textbf{Vertex buffer} -- возможность хранить вершины \textit{в памяти GPU}
\end{itemize}
\pause
\usemintedstyle{lightbulb}
\begin{minted}[bgcolor=codebg]{python}
# на старте
for object in scene.objects:
    object.createVertexBuffer(object.vertices)
    object.createVertexArray(object.vertexBuffer)
    object.createTexture()

# при рендеринге
for object in scene.objects:
    glBindTexture(GL_TEXTURE_2D, object.texture)
    glBindVertexArray(object.vertexArray)
    glDrawArrays(object.vertexCount)
\end{minted}
\end{frame}

\begin{frame}
\frametitle{История графических API: OpenGL 2.0 (2004)}
\begin{itemize}
\item \textbf{Шейдеры} -- программы на C-подобном языке GLSL, компилируемые под конкретную GPU
\begin{itemize}
\item \textit{Заменяют} fixed-function pipeline
\item Необходимые части fixed-function pipeline \textit{остаются} (растеризация, тест глубины, etc)
\end{itemize}
\end{itemize}
\end{frame}

\begin{frame}[fragile]
\frametitle{История графических API: OpenGL 2.0 (2004)}
\usemintedstyle{lightbulb}
\begin{minted}[bgcolor=codebg]{python}
# на старте
for object in scene.objects:
    object.createVertexBuffer(object.vertices)
    object.createVertexArray(object.vertexBuffer)
    object.createTexture()
    object.createShaderProgram()

# при рендеринге
for object in scene.objects:
    glBindTexture(GL_TEXTURE_2D, object.texture)
    glBindVertexArray(object.vertexArray)
    glUseProgram(object.shaderProgram)
    glDrawArrays(object.vertexCount)
\end{minted}
\pause
\begin{itemize}
\item \underline{Примерно так будет выглядеть наш код}
\end{itemize}
\end{frame}

\begin{frame}[fragile]
\frametitle{История графических API: OpenGL 3.0 (2008)}
\usemintedstyle{solarized-light}
\begin{itemize}
\item Огромная часть API объявлена \textit{deprecated}
\pause
\begin{itemize}
\item Immediate-mode рисование -- \mintinline{cpp}|glBegin/glEnd|
\item Хранение данных на CPU -- \mintinline{cpp}|glVertexPointer, ...|
\item Матрицы преобразований -- \mintinline{cpp}|glLoadMatrix, ...|
\end{itemize}
\pause
\item \textbf{Transform feedback} -- возможность записать результат работы шейдеров обратно в вершинный буфер
\begin{itemize}
\item Зачатки \textit{GPGPU}
\end{itemize}
\end{itemize}
\end{frame}

\begin{frame}
\frametitle{История графических API: OpenGL 3.1 (2009)}
\begin{itemize}
\item Объявленные deprecated возможности \textit{удалены}
\pause
\item \textbf{Instanced rendering} -- нарисовать \textit{много копий} одного объекта в разных местах \textit{одной командой}
\end{itemize}
\end{frame}

\begin{frame}
\frametitle{История графических API: OpenGL 3.2 (2009)}
\begin{itemize}
\item \textbf{Механизм профилей}
\begin{itemize}
\item \textbf{Core profile}
\begin{itemize}
\item Обязан поддерживаться
\item Только функционал конкретной версии OpenGL
\end{itemize}
\item \textbf{Compatibility profile}
\begin{itemize}
\item Не обязан поддерживаться
\item Функционал этой и всех предыдущих версий OpenGL
\end{itemize}
\pause
\item Мы будем использовать \alert{\textbf{\textit{\underline{core profile}}}}
\end{itemize}
\pause
\item \textbf{Геометрические шейдеры} -- возможность менять тип геометрии и количество вершин \textit{на лету} (используются для систем частиц, травы, etc)
\end{itemize}
\end{frame}

\begin{frame}
\frametitle{История графических API: OpenGL 4.0 (2010)}
\begin{itemize}
\item \textbf{Шейдеры тесселяции} -- увеличивают детализацию геометрии на лету
\begin{itemize}
\item Гораздо меньше возможностей, чем у геометрических шейдеров, \textit{зато быстрее}
\end{itemize}
\pause
\item \textbf{Indirect drawing} -- можно вычислять количество вершин и их расположение в памяти на лету \textit{на GPU}, и использовать вычисленные значения для команд рисования
\end{itemize}
\end{frame}

\begin{frame}
\frametitle{История графических API: OpenGL 4.1 - 4.7 (2010 - 2017)}
\begin{itemize}
\item \textbf{Compute шейдеры} -- \textit{настоящее GPGPU} внутри OpenGL
\item Проработка и детализация API
\item \textbf{Атомарные операции} в шейдерах
\item Вливание расширений в стандарт OpenGL
\item \href{https://khronos.org/opengl/wiki/History_of_OpenGL}{\nolinkurl{khronos.org/opengl/wiki/History\_of\_OpenGL}}
\end{itemize}
\end{frame}

\begin{frame}
\frametitle{История графических API: Vulkan 1.1 (2018)}
\slideimage{vulkan.png}
\begin{itemize}
\item \textit{700 строк кода}, чтобы нарисовать один треугольник
\pause
\item Не поддерживается на macOS и iOS, но есть прослойка MoltenVK, реализующая Vulkan поверх Metal API
\pause
\item \textit{Крайне низкоуровневый} и гибкий API
\begin{itemize}
\item Последовательности команд для выполнения на GPU (command queues) в явном виде
\item Явное управление памятью и синхронизацией
\end{itemize}
\pause
\item Похож на DirectX 12, Metal
\item \href{https://vulkan-tutorial.com}{\nolinkurl{vulkan-tutorial.com}}
\end{itemize}
\end{frame}

\begin{frame}
\frametitle{История графических API: WebGPU (2021 - ...)}
\slideimage{webgpu.png}
\begin{itemize}
\item \textit{200 строк кода}, чтобы нарисовать один треугольник
\pause
\item Всё ещё в состоянии \textit{working draft}, но некоторые браузеры (Chrome с апреля 2023) уже его поддерживают
\pause
\item Ориентирован на web, но есть реализации \textit{под десктоп} (\href{https://dawn.googlesource.com/dawn/+/refs/heads/chromium-gpu-experimental/README.md}{\texttt{Google Dawn}}, \href{https://github.com/gfx-rs/wgpu-native}{\nolinkurl{wgpu-native}}) поверх платформенных API (DirectX 12/Vulkan/Metal)
\pause
\item \textit{Разумно низкоуровневый} API: сложнее (но приятнее), чем OpenGL, и легче, чем Vulkan
\item \textit{Хороший выбор} для изучения после OpenGL
\item \href{https://sotrh.github.io/learn-wgpu/}{\nolinkurl{sotrh.github.io/learn-wgpu}}
\end{itemize}
\end{frame}

\begin{frame}
\frametitle{Разновидности OpenGL}
\begin{itemize}
\item \textbf{OpenGL}
\pause
\item \textbf{OpenGL ES} (Embedded Systems)
\begin{itemize}
\item \textbf{OpenGL ES 1.0} $\approx$ \textbf{OpenGL 1.3}
\item \textbf{OpenGL ES 2.0} $\approx$ \textbf{OpenGL 2.0}
\item \textbf{OpenGL ES 3.0} $\approx$ \textbf{OpenGL 3.0}
\end{itemize}
\pause
\item \textbf{WebGL}
\begin{itemize}
\item \textbf{WebGL 1.0} $\approx$ \textbf{OpenGL ES 2.0}
\item \textbf{WebGL 2.0} $\approx$ \textbf{OpenGL ES 3.0}
\end{itemize}
\pause
\item \textbf{OpenGL SC} (Safety Critical)
\begin{itemize}
\item Убраны любые способы отстрелить себе ногу, \textit{в ущерб производительности}
\end{itemize}
\end{itemize}
\end{frame}

\begin{frame}
\frametitle{Что такое OpenGL?}
\begin{itemize}
\item \textbf{{\color{red}Не} библиотека!}
\item \textbf{Спецификация API (документ) на языке C}
\begin{itemize}
\item Описание констант-перечислений (тэгов)
\item Описание сигнатур функций и их семантики
\end{itemize}
\pause
\item \href{https://registry.khronos.org/OpenGL/specs/gl/glspec46.core.pdf}{\texttt{OpenGL 4.6 Specification}}
\end{itemize}
\end{frame}

\begin{frame}[fragile]
\frametitle{Что такое реализация OpenGL?}
\pause
\textbf{Заголовочный файл}, поставляемый системой или драйвером
\usemintedstyle{solarized-light}
\begin{itemize}
\item Определение типов, e.g. \mintinline{cpp}|typedef unsigned int GLenum;|
\item Определение констант, e.g. \mintinline{cpp}|#define GL_TEXTURE_2D 0x0DE1|
\item Объявление функций, e.g. \mintinline{cpp}|void glBindTexture(GLenum target,GLuint texture);|
\end{itemize}
\end{frame}

\begin{frame}[fragile]
\frametitle{Что такое реализация OpenGL?}
\pause
\textbf{Бинарная реализация} объявленных функций (обычно -- динамическая библиотека), поставляемая системой и/или драйвером
\begin{itemize}
\item Может содержать \textit{непосредственную реализацию OpenGL} как часть драйвера и общаться с GPU
\item Может быть \textit{промежуточным звеном}, маршрутизирующим вызов до драйвера
\item Может быть \textit{заглушкой}
\end{itemize}
\end{frame}

\begin{frame}[fragile]
\frametitle{Что такое реализация OpenGL?}
\usemintedstyle{solarized-light}
\begin{itemize}
\item \textbf{Заголовочный файл}
\begin{itemize}
\item Linux: \mintinline{cpp}|GL/gl.h| -- до OpenGL 1.3
\pause
\item Windows: \mintinline{cpp}|GL/gl.h| -- до OpenGL 1.1
\pause
\item MacOS: \mintinline{cpp}|OpenGL/gl.h| -- до OpenGL 2.1
\begin{itemize}
\item {\color{red}Не} \mintinline{cpp}|OpenGL/OpenGL.h|
\end{itemize}
\item Все платформы: \mintinline{cpp}|GL/glext.h| вместе с \mintinline{cpp}|#define GL_GLEXT_PROTOTYPES| -- до OpenGL 4.6
\end{itemize}
\end{itemize}
\end{frame}

\begin{frame}[fragile]
\frametitle{Что такое реализация OpenGL?}
\usemintedstyle{solarized-light}
\begin{itemize}
\item \textbf{Динамическая библиотека}
\begin{itemize}
\item Linux: \mintinline{cpp}|libGL.so|
\item Windows: \mintinline{cpp}|opengl32.dll|
\item MacOS: \mintinline{cpp}|OpenGL framework|
\end{itemize}
\pause
\item Может содержать функции \textit{не всех версий} OpenGL
\begin{itemize}
\item Под Linux обычно содержит
\end{itemize}
\item Остальные функции OpenGL нужно динамически загружать \textit{специфичными для платформы средствами}
\item $\Longrightarrow$ Библиотеки-загрузчики OpenGL
\end{itemize}
\end{frame}

\begin{frame}[fragile]
\frametitle{Загрузчики OpenGL}
\begin{itemize}
\item С и C++ specific, для других языков обычно встроено в обёртку над OpenGL
\item \href{https://khronos.org/opengl/wiki/OpenGL_Loading_Library}{\nolinkurl{khronos.org/opengl/wiki/OpenGL\_Loading\_Library}}
\item Обычно содержат код, автоматически сгенерированный по XML-спецификации OpenGL
\item Есть мой, основанный на \verb|glLoadGen| (который перестали поддерживать): \href{https://github.com/lisyarus/opengl-loader-generator}{\nolinkurl{github.com/lisyarus/opengl-loader-generator}}
\item Мы будем использовать \href{http://glew.sourceforge.net/}{\alert{\textbf{\underline{GLEW}}}}
\end{itemize}
\end{frame}

\begin{frame}
\frametitle{Контекст OpenGL}
\begin{itemize}
\item Необходим для вызова любой функции OpenGL
\item Привязан к \textit{конкретной реализации} OpenGL
\item Привязан к \textit{конкретным версии и профилю} OpenGL
\item Привязан к \textit{экрану / окну оконной системы / изображению в памяти}
\item Хранит текущее \textit{глобальное состояние} OpenGL
\item \href{https://khronos.org/opengl/wiki/OpenGL_Context}{\nolinkurl{khronos.org/opengl/wiki/OpenGL\_Context}}
\item Создаётся \textit{специфичными для платформы} средствами
\item $\Longrightarrow$ Библиотеки, создающие контекст OpenGL
\end{itemize}
\end{frame}

\begin{frame}
\frametitle{Библиотеки, создающие контекст OpenGL}
\begin{itemize}
\item Обычно привязывают контекст к окну и умеют обрабатывать события оконной системы
\item GLUT -- устаревшая, плохой интерфейс
\item GLFW -- простая, широко распространённая
\item \href{https://www.libsdl.org/}{\alert{\textbf{\underline{SDL2}}}} -- умеет загружать изображения, выводить звук, и другое
\item \href{https://open.gl/context}{\nolinkurl{open.gl/context}}
\end{itemize}
\end{frame}

\begin{frame}[fragile]
\frametitle{Как начать работать с OpenGL?}
\usemintedstyle{lightbulb}
\begin{minted}[bgcolor=codebg]{python}
window = createWindow(title)
context = createGLContext(window, version, profile)
context.makeCurrent()
loadGLFunctions()
# тут можно работать с OpenGL!
\end{minted}
\end{frame}

\begin{frame}
\frametitle{Литература, ссылки}
\begin{itemize}
\item \textbf{Realtime графика}
\begin{itemize}
\item \href{https://cgpp.net/about.xml}{\texttt{Computer Graphics: Principles and Practice}} -- книжка начального уровня
\item \href{https://www.amazon.com/Real-Time-Rendering-Fourth-Tomas-Akenine-M\%C3\%B6ller/dp/1138627003}{\texttt{Real-Time Rendering (4th edition)}} -- обзор передовых алгоритмов индустрии
\item \href{https://developer.nvidia.com/gpugems}{\texttt{GPU Gems 1, 2, 3}} -- журнал про техники и алгоритмы
\end{itemize}
\item \textbf{OpenGL}
\begin{itemize}
\item \href{https://khronos.org/opengl/wiki}{\nolinkurl{khronos.org/opengl/wiki}} -- подробное изложение всех аспектов OpenGL
\item \href{http://docs.gl}{\nolinkurl{docs.gl}} -- удобная документация по отдельным функциям
\item \href{https://learnopengl.com}{\nolinkurl{learnopengl.com}} -- уроки по отдельным темам
\end{itemize}
\end{itemize}
\end{frame}

\end{document}