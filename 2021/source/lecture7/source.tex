% (c) Nikita Lisitsa, lisyarus@gmail.com, 2021

\documentclass{beamer}

\usepackage[T2A]{fontenc}
\usepackage[utf8]{inputenc}
\usepackage[russian]{babel}

\usepackage{graphicx}
\graphicspath{ {./images/} }

\usepackage{adjustbox}

\usepackage{color}
\usepackage{soul}

\usepackage{hyperref}

\usepackage{amsmath}

\usepackage{tikz}
\usetikzlibrary{decorations}
\usetikzlibrary{decorations.pathreplacing}
\usepackage{xifthen}

\definecolor{red}{rgb}{1,0,0}
\definecolor{green}{rgb}{0,0.5,0}
\definecolor{blue}{rgb}{0,0,1}
\definecolor{magenta}{rgb}{0.75,0,0.75}

\makeatletter
\newcommand{\slideimage}[1]{
  \begin{figure}
    \begin{adjustbox}{width=\textwidth, totalheight=\textheight-2\baselineskip-2\baselineskip,keepaspectratio}
      \includegraphics{#1}
    \end{adjustbox}
  \end{figure}
}
\makeatother

\title{Компьютерная графика}
\subtitle{Лекция 7: Несколько источников света, normal mapping, material mapping, ambient occlusion, baking, отражения, environment mapping, HDR}
\date{2021}

\setbeamertemplate{footline}[frame number]

\begin{document}

\frame{\titlepage}

\begin{frame}[fragile]
\frametitle{Модель фонга}
\begin{itemize}
\item Ambient + diffuse + specular
\end{itemize}
\slideimage{phong.png}
\end{frame}

\begin{frame}[fragile]
\frametitle{Specular highlight}
\begin{itemize}
\item Грубая аппроксимация отражений: прямой свет от источника света обычно значительно ярче, чем свет, отражённый другими объектами
\pause
\item \verb|light_dir| - направление на источник света
\item Вычисляем направление отражённого света
\begin{verbatim}
reflected = 2.0 * normal * dot(normal, light_dir)
    - light_dir
\end{verbatim}
\pause
\item Если отражённый луч светит \textit{примерно} в направлении камеры, добавим к цвету пикселя specular-компоненту
\pause
\item Обычно вычисляется как \verb|pow(max(0.0, dot(reflected, camera_dir)), 16.0)|
\begin{itemize}
\item \verb|camera_dir| - направление на камеру
\item 16 - настраиваемый параметр: чем меньше, тем ярче specular highlight
\end{itemize}
\pause
\item N.B. В таком виде получается ненормализованная BRDF
\end{itemize}
\end{frame}

\begin{frame}[fragile]
\frametitle{Несколько источников света}
\slideimage{many-lights.png}
\end{frame}

\begin{frame}[fragile]
\frametitle{Несколько источников света}
\begin{itemize}
\item Простейший способ - использовать по отдельному набору uniform-переменных для каждого источника света, и обрабатывать их все сразу в шейдере
\pause
\item Можно использовать массивы uniform-переменных
\pause
\begin{itemize}
\item Задаются как \verb|vec3 light_position[N]| (\verb|N| - некая константа)
\item Каждый элемент массива - отдельная uniform-переменная, для неё нужно отдельно вызвать \verb|glGetUniformLocation|
\item Имя этой переменной - имя массива + индекс, например \verb|light_position[0]|
\end{itemize}
\end{itemize}
\end{frame}

\begin{frame}[fragile]
\frametitle{Несколько источников света}
\begin{itemize}
\item Сколько источников света можно так обработать?
\pause
\item OpenGL гарантирует, что под uniform-переменные есть как минимум 1024 выделенных компоненты (\verb|vec3| занимает 3 компоненты, и т.п.)
\pause
\item В варианте 3 \verb|vec3| на источник (position, color, attenuation) - около 113 источников света
\pause
\item Чем больше источников, тем больше вычислений в шейдере, тем больше нагрузка на GPU
\end{itemize}
\end{frame}

\begin{frame}[fragile]
\frametitle{Несколько источников света}
\begin{itemize}
\item Такой подход выполняет много лишних операций:
\begin{itemize}
\item Вычисляется полное освещение для объекта, который может быть скрыт другим объектом
\pause
\item Вычисляется вклад в освещение для объекта, находящегося слишком далеко от источника света
\end{itemize}
\pause
\item Можно частично решить, вычисляя для каждого объекта список источников, которыми он освещён \begin{math}\Rightarrow\end{math} больше изменений uniform-переменных при рендеринге, меньше возможность для batching'а (объединения объектов в группы или в общие буферы)
\pause
\item Более современные решения: deferred shading, tiled/clustered shading
\end{itemize}
\end{frame}

\begin{frame}[fragile]
\frametitle{Несколько источников света}
\begin{itemize}
\item \href{https://learnopengl.com/Lighting/Multiple-lights}{learnopengl.com/Lighting/Multiple-lights}
\item \href{https://www.tomdalling.com/blog/modern-opengl/08-even-more-lighting-directional-lights-spotlights-multiple-lights}{tomdalling.com/blog/modern-opengl/08-even-more-lighting-directional-lights-spotlights-multiple-lights}
\item \href{https://en.wikibooks.org/wiki/GLSL_Programming/GLUT/Multiple_Lights}{en.wikibooks.org/wiki/GLSL\_Programming/GLUT/Multiple\_Lights}
\end{itemize}
\end{frame}

\begin{frame}[fragile]
\frametitle{Normal mapping}
\begin{itemize}
\item Хотим высокую детализацию поверхности объекта
\end{itemize}
\slideimage{orange.png}
\end{frame}

\begin{frame}[fragile]
\frametitle{Normal mapping}
\begin{itemize}
\item Можно взять очень много вершин
\pause
\item Очень дорого, особенно когда объектов много
\end{itemize}
\slideimage{orange-mesh.png}
\end{frame}

\begin{frame}[fragile]
\frametitle{Normal mapping}
\begin{itemize}
\item Наблюдение: высокая детализация даёт 2 вещи
\begin{itemize}
\item Детализацию координат вершин - слабо заметна
\item Детализацию нормалей вершин - сильно заметна (влияет на освещение)
\end{itemize}
\pause
\item Давайте откажемся от детализации координат!
\end{itemize}
\end{frame}

\begin{frame}[fragile]
\frametitle{Normal mapping}
\slideimage{orange-normal-map.png}
\end{frame}

\begin{frame}[fragile]
\frametitle{Normal mapping}
\begin{itemize}
\item Normal mapping: используем текстуру, в которой закодированы нормали в точках объекта
\pause
\item Вместо нормалей, приходящих в вершинах, достаём нормали из текстуры во фрагментном шейдере
\pause
\item При этом можно использовать слабо детализированную геометрию
\end{itemize}
\end{frame}

\begin{frame}[fragile]
\frametitle{Normal mapping}
\slideimage{orange-normals.jpg}
\end{frame}

\begin{frame}[fragile]
\frametitle{Normal mapping}
\slideimage{bricks-normal-map.png}
\end{frame}

\begin{frame}[fragile]
\frametitle{Normal mapping}
\slideimage{bricks-normals.png}
\end{frame}

\begin{frame}[fragile]
\frametitle{Normal mapping}
\slideimage{pupil-normals.jpg}
\end{frame}

\begin{frame}[fragile]
\frametitle{Normal mapping: формат хранения}
\fontsize{10pt}{10pt}
\begin{itemize}
\item Как хранить нормали в текстуре?
\pause
\item Нормаль - 3х-компонентный нормированный вектор
\pause
\item Floating-point текстуры (\verb|GL_RGB32F| или \verb|GL_RGB16F|)
\begin{itemize}
\item Занимают много места (3x16-bit = 6 байт на пиксель)
\item Излишняя точность
\item Значительная часть возмозных значений (координаты больше 1 по модулю) вообще не используется
\end{itemize}
\pause
\item Signed normalized текстуры (\verb|GL_RGB8_SNORM|)
\begin{itemize}
\item Обычно достаточно точности
\item Компактные
\item Форматы изображений обычно не умеют работать с signed пикселями
\end{itemize}
\pause
\item Обычные unsigned normalized текстуры (\verb|GL_RGB8|, \verb|GL_RGB10_A2|, \verb|GL_RGB565|, \verb|GL_RGB5_A1|, \verb|GL_R3_G3_B2|)
\begin{itemize}
\item Координаты хранятся в виде, отмасштабированном в диапазон [0, 1], т.е. \verb|0.5 * (x + 1)|
\end{itemize}
\end{itemize}
\end{frame}

\begin{frame}[fragile]
\frametitle{Normal mapping: система координат}
\begin{itemize}
\item В какой системе координат хранить нормали в текстуре?
\pause
\item В системе координат объекта
\begin{itemize}
\item Нет лишних операций, просто читаем нормаль из текстуры
\end{itemize}
\pause
\item В локальной системе координат пикселя, учитывая геометрию объекта
\begin{itemize}
\item Нужно восстанавливать локальную систему координат в шейдере (используя нормаль вершины)
\item Можно сэкономить и хранить только две координаты нормали
\item Обычно такая текстура с нормалями легче воспринимается на глаз
\pause
\item Такие текстуры часто выглядят синими, потому что направление Z считается нормалью вершины
\end{itemize}
\end{itemize}
\end{frame}

\begin{frame}[fragile]
\frametitle{Normal mapping: туториалы}
\begin{itemize}
\item \href{https://learnopengl.com/Advanced-Lighting/Normal-Mapping}{learnopengl.com/Advanced-Lighting/Normal-Mapping}
\item \href{http://www.opengl-tutorial.org/intermediate-tutorials/tutorial-13-normal-mapping}{opengl-tutorial.org/intermediate-tutorials/tutorial-13-normal-mapping}
\item \href{https://sudonull.com/post/14500-Learn-OpenGL-Lesson-55-Normal-Mapping}{sudonull.com/post/14500-Learn-OpenGL-Lesson-55-Normal-Mapping}
\item \href{https://habr.com/ru/post/415579}{habr.com/ru/post/415579}
\end{itemize}
\end{frame}

\begin{frame}[fragile]
\frametitle{Material mapping}
\begin{itemize}
\item В обработке освещения часто встречаются настраиваемые параметры, помимо цвета (альбедо) объекта
\pause
\item Например, показатель степени для specular highlight
\pause
\item Можно его тоже варьировать, читая из текстуры
\end{itemize}
\end{frame}

\begin{frame}[fragile]
\frametitle{Material mapping}
\slideimage{donut-specular.png}
\end{frame}

\begin{frame}[fragile]
\frametitle{Ambient occlusion}
\begin{itemize}
\item Ambient освещение - светит "отовсюду"
\pause
\item Это свет, значит от него может быть тень
\pause
\item Какие-то части объекта получают больше ambient света, какие-то - меньше
\pause
\item Обычно затеняются углы, углубления, трещины, стыки объектов, и т.п.
\pause
\item Очень сильно увеличивает реализм изображения
\end{itemize}
\end{frame}

\begin{frame}[fragile]
\frametitle{Ambient occlusion (Crysis, 2007)}
\slideimage{crysis-ao.jpg}
\end{frame}

\begin{frame}[fragile]
\frametitle{Ambient occlusion}
\slideimage{unity-ao.jpg}
\end{frame}

\begin{frame}[fragile]
\frametitle{Ambient occlusion}
\slideimage{artstation-ao.jpg}
\end{frame}

\begin{frame}[fragile]
\frametitle{Ambient occlusion}
\slideimage{birch-combined.png}
\end{frame}

\begin{frame}[fragile]
\frametitle{Ambient occlusion}
\begin{itemize}
\item В современности обычно используются real-time алгоритмы, использующие только геометрию сцены: SSAO, HBAO, HBAO+, HDAO, VXAO
\pause
\item Можно предподсчитать ambient occlusion для модели, записать её в текстуру и использовать в шейдере
\pause
\item Работает быстрее, чем общие real-time алгоритмы
\pause
\item Не учитывает occlusion между разными объектами
\end{itemize}
\end{frame}

\begin{frame}[fragile]
\frametitle{Ambient occlusion}
\slideimage{baked-ao.png}
\end{frame}

\begin{frame}[fragile]
\frametitle{Baking}
\begin{itemize}
\item Записывание предподсчитанных свойств объектов в текстуры в общем случае называется baking (запеканием)
\pause
\item Все 3D-редакторы умеют запекать нормали, ambient occlusion, и многое другое
\end{itemize}
\end{frame}

\begin{frame}[fragile]
\frametitle{Отражения}
\begin{itemize}
\item Расчёт отражений в real-time - очень сложная задача
\pause
\item Для плоских зеркал можно нарисовать сцену целиком, применив отражающее преобразование и нарисовав зеркало предварительно в stencil буфер
\pause
\begin{itemize}
\item Рисовать заново всю сцену - дорого
\item Не работает для неплоских зеркал
\end{itemize}
\pause
\item Можно нарисовать сцену целиком, вычисляя проекцию объекта на зеркало в вершинном шейдере
\begin{itemize}
\item Тоже дорого
\item Могут быть артефакты с растягиванием вершин
\end{itemize}
\end{itemize}
\end{frame}

\begin{frame}[fragile]
\frametitle{Отражения}
\begin{itemize}
\item Аппроксимация: рисовать сцену в текстуру, эту текстуру накладывать на объект
\pause
\begin{itemize}
\item Мы пока не умеем рисовать в текстуры
\end{itemize}
\pause
\item Можно использовать предподсчитанную текстуру!
\end{itemize}
\end{frame}

\begin{frame}[fragile]
\frametitle{Environment mapping}
\begin{itemize}
\item Предподсчитанная cubemap-текстура окружения
\pause
\item Не учитывает динамические объекты
\pause
\item Хорошо работает для нечётких/слабых зеркал (капот автомобиля, дверная ручка, гильза)
\end{itemize}
\end{frame}

\begin{frame}[fragile]
\frametitle{Environment mapping}
\slideimage{env-map.jpg}
\end{frame}

\begin{frame}[fragile]
\frametitle{Environment mapping: реализация}
\begin{itemize}
\item Используя направление на камеру и нормаль поверхности, вычисляем направление отражённого луча
\pause
\item Используем направление отражённого луча в качестве параметра для чтения из cubemap-текстуры
\end{itemize}
\end{frame}

\begin{frame}[fragile]
\frametitle{Environment mapping: ссылки}
\begin{itemize}
\item \href{https://learnopengl.com/Advanced-OpenGL/Cubemaps}{learnopengl.com/Advanced-OpenGL/Cubemaps}
\end{itemize}
\end{frame}

\begin{frame}[fragile]
\frametitle{HDR}
\begin{itemize}
\item Проблема: после вычисления освещённости мы можем получить значения компонент цвета большие 1, и они будут сжаты до значений [0, 1]
\pause
\item В реальном мире интенсивность света может меняться на ~15 порядков
\pause
\item High Dynamic Range (HDR) - термин, означающий большой (не ограниченный [0, 1]) диапазон интенсивностей
\pause
\item HDR текстура (например, environment map) - содержит значения, выходящие за диапазон [0, 1] (обычно 32-bit floating-point)
\pause
\item HDR рендеринг - позволяет отобразить HDR-освещение
\end{itemize}
\end{frame}

\begin{frame}[fragile]
\frametitle{Tone mapping}
\begin{itemize}
\item Нужно превратить диапазон освещённостей \begin{math}[0, \infty)\end{math} в диапазон \begin{math}[0, 1]\end{math}
\pause
\item В принципе, подойдёт любая монотонно возрастающая функция \begin{math}[0, \infty)\rightarrow [0, 1]\end{math}
\begin{itemize}
\item \begin{math}x \mapsto \frac{x}{1+x}\end{math} - Reinhard operator
\item \begin{math}x \mapsto \arctan(x)\end{math}
\item \begin{math}x \mapsto \frac{2}{1+e^{-x}}-1\end{math}
\end{itemize}
\pause
\item Иногда используют более сложные функции, взятые из кинематографа (filmic tone mapping)
\end{itemize}
\end{frame}

\begin{frame}[fragile]
\frametitle{Reinhard operator}
\slideimage{reinhard.png}
\end{frame}

\begin{frame}[fragile]
\frametitle{Filmic tone mapping curve (ACES)}
\slideimage{aces.png}
\end{frame}

\begin{frame}[fragile]
\frametitle{HDR \& tone mapping: ссылки}
\begin{itemize}
\item \href{https://learnopengl.com/Advanced-Lighting/HDR}{learnopengl.com/Advanced-Lighting/HDR}
\item \href{https://skylum.com/blog/what-is-tone-mapping}{skylum.com/blog/what-is-tone-mapping}
\item \href{https://www.veneratech.com/what-is-hdr-tone-mapping}{veneratech.com/what-is-hdr-tone-mapping}
\end{itemize}
\end{frame}

\end{document}