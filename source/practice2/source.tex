% (c) Nikita Lisitsa, lisyarus@gmail.com, 2023

\documentclass[10pt]{beamer}

\usepackage[T2A]{fontenc}
\usepackage[russian]{babel}
\usepackage{minted}

\usepackage{graphicx}
\graphicspath{ {./images/} }

\usetheme{metropolis}

\usepackage{adjustbox}

\usepackage{color}
\usepackage{soul}

\usepackage{hyperref}

\definecolor{codebg}{RGB}{29,35,49}

\setminted{fontsize=\footnotesize}

\makeatletter
\newcommand{\slideimage}[1]{
  \begin{figure}
    \begin{adjustbox}{width=\textwidth, totalheight=\textheight-2\baselineskip-2\baselineskip,keepaspectratio}
      \includegraphics{#1}
    \end{adjustbox}
  \end{figure}
}
\makeatother

\title{Компьютерная графика}
\subtitle{Практика 2: Uniform'ы и матрицы преобразований}
\date{2023}

\setbeamertemplate{footline}[frame number]

\begin{document}

\frame{\titlepage}

\begin{frame}
\slideimage{0.png}
\end{frame}

\begin{frame}[fragile]
\frametitle{Задание 1}
Уменьшим треугольник в 2 раза, используя uniform переменную
\usemintedstyle{solarized-light}
\begin{itemize}
\item В коде вершинного шейдера:
\begin{itemize}
\item \mintinline{glsl}|uniform float scale;|
\item Нужно где-то умножить вектор координат на \mintinline{glsl}|scale|
\item \textbf{\alert{NB}}: последняя координата \mintinline{glsl}|gl_Position| \textit{должна остаться равной 1}
\end{itemize}
\item После создания программы, до основного цикла:
\begin{itemize}
\item \mintinline{cpp}|glUseProgram|
\item \mintinline{cpp}|glGetUniformLocation| -- возвращает уникальный идентификатор, ползволяющий работать с этой uniform-переменной
\item \mintinline{cpp}|glUniform1f| -- устанавливает значение конкретной uniform-переменной типа \mintinline{cpp}|float|
\end{itemize}
\end{itemize}
\end{frame}

\begin{frame}
\frametitle{Задание 1}
\slideimage{1.png}
\end{frame}

\begin{frame}[fragile]
\frametitle{Задание 2}
Добавим анимацию вращения
\usemintedstyle{solarized-light}
\begin{itemize}
\item В коде вершинного шейдера:
\begin{itemize}
\item \mintinline{glsl}|uniform float angle;|
\item Нужно где-то повернуть вектор координат на \mintinline{glsl}|angle|
\item Формула поворота есть в слайдах лекции (пока нужны только первые 2 координаты)
\end{itemize}
\item После создания программы, до основного цикла:
\begin{itemize}
\item \mintinline{cpp}|glUseProgram|
\item \mintinline{cpp}|glGetUniformLocation|
\item \mintinline{cpp}|float time = 0.f;|
\end{itemize}
\item В теле основного цикла рендеринга, после вычисления \mintinline{cpp}|dt|
\begin{itemize}
\item \mintinline{cpp}|time += dt;|
\end{itemize}
\item В теле основного цикла рендеринга, после \mintinline{cpp}|glUseProgram|, до \mintinline{cpp}|glDrawArrays|
\begin{itemize}
\item \mintinline{cpp}|glUniform1f|
\item В качестве значения можно использовать \mintinline{cpp}|time|
\end{itemize}
\end{itemize}
\end{frame}

\begin{frame}
\frametitle{Задание 2}
\slideimage{2.png}
\end{frame}

\begin{frame}[fragile]
\frametitle{Задание 3}
\usemintedstyle{solarized-light}
Заменим ручное применение преобразования на матрицу
\begin{itemize}
\item В коде шейдера:
\begin{itemize}
\item Заменяем две uniform-переменные на одну: \mintinline{glsl}|uniform mat4 transform|
\item Заменяем ручное вращение и масштабирование на умножение на матрицу: \mintinline{glsl}|gl_Position = transform * vec4(...);|
\end{itemize}
\item Обновляем вызов \mintinline{cpp}|glGetUniformLocation|
\end{itemize}
\end{frame}

\begin{frame}[fragile]
\frametitle{Задание 3}
\usemintedstyle{solarized-light}
Заменим ручное применение преобразования на матрицу
\begin{itemize}
\item В теле основного цикла рендеринга
\begin{itemize}
\item Создаём матрицу \begin{math}4\times 4\end{math} -- массив из 16 \mintinline{cpp}|float|'ов
\usemintedstyle{lightbulb}
\begin{minted}[bgcolor=codebg]{cpp}
float transform[16] =
{
    ?, ?, ?, ?, // 1 строка
    ?, ?, ?, ?, // 2 строка
    ?, ?, ?, ?, // 3 строка
    ?, ?, ?, ?, // 4 строка
};
\end{minted}
\usemintedstyle{solarized-light}
\item Заполняем матрицу значениями, чтобы это была матрица применения поворота и масштабирования
\item \mintinline{cpp}|glUniformMatrix4fv|, \mintinline{cpp}|count = 1|, \mintinline{cpp}|transpose = GL_TRUE|
\end{itemize}
\end{itemize}
\end{frame}

\begin{frame}
\frametitle{Задание 3}
\slideimage{2.png}
\end{frame}

\begin{frame}[fragile]
\frametitle{Задание 4}
\usemintedstyle{solarized-light}
Добавляем в матрицу сдвиг, зависящий от времени
\begin{itemize}
\item В теле основного цикла рендеринга
\begin{itemize}
\item Заводим переменные под сдвиг:
\mintinline{cpp}|float x = ?;|
\mintinline{cpp}|float y = ?;|
\item Обновляем матрицу преобразования:
\mintinline{cpp}|float transform[16] = ...;|
\end{itemize}
\end{itemize}
\end{frame}

\begin{frame}
\frametitle{Задание 4}
\slideimage{4.png}
\end{frame}

\begin{frame}[fragile]
\frametitle{Задание 5}
\usemintedstyle{solarized-light}
Добавляем учёт \textit{aspect ratio} экрана
\begin{itemize}
\item В коде шейдера:
\begin{itemize}
\item Добавляем uniform-переменную \mintinline{cpp}|view|, аналогичную переменной \mintinline{cpp}|transform|
\item Применяем обе матрицы: \mintinline{glsl}|gl_Position = view * transform * ...;|
\end{itemize}
\item После создания программы, до основного цикла:
\begin{itemize}
\item Добавляем \mintinline{cpp}|glGetUniformLocation|
\end{itemize}
\item В теле основного цикла рендеринга
\begin{itemize}
\item Вычисляем \mintinline{cpp}|aspect_ratio = width / height| (\textbf{\alert{NB}}: если написать буквально так, будет целочисленное деление; нам нужно деление во floating point)
\item Создаём новую матрицу, которая делит x-координату на \mintinline{cpp}|aspect_ratio|:
\mintinline{cpp}|float view[16] = ...;|
\item Устанавливаем значение новой uniform-переменной с помозью \mintinline{cpp}|glUniformMatrix4fv|
\end{itemize}
\end{itemize}
\end{frame}

\begin{frame}
\frametitle{Задание 5}
\slideimage{5.png}
\end{frame}

\begin{frame}[fragile]
\frametitle{Задание 6}
\usemintedstyle{solarized-light}
Выключаем VSync
\begin{itemize}
\item В теле основного цикла рендеринга
\begin{itemize}
\item Выводим в лог значение переменной \mintinline{cpp}|dt| (время, потраченное на один кадр в секундах) -- скорее всего, будет в районе \mintinline{cpp}|0.016|
\end{itemize}
\item После создания OpenGL-контекста:
\begin{itemize}
\item \mintinline{cpp}|SDL_GL_SetSwapInterval(0);|
\item Проверяем значение переменной \mintinline{cpp}|dt| -- должно стать значительно меньше (например, \mintinline{cpp}|0.001|)
\end{itemize}
\item В теле основного цикла рендеринга
\begin{itemize}
\item Заменяем вычисление \mintinline{cpp}|dt| на какую-нибудь константу, например \mintinline{cpp}|float dt = 0.016f;|
\item Должен получиться эффект, похожий на \href{https://en.wikipedia.org/wiki/Wagon-wheel_effect}{\texttt{wagon-wheel effect}}
\end{itemize}
\item \textbf{\alert{NB}}: может не сработать (зависит от системы, драйвера, и т.п.), ничего страшного
\end{itemize}
\end{frame}

\begin{frame}[fragile]
\frametitle{Задание 7*}
\usemintedstyle{solarized-light}
Заменяем треугольник на управляемый шестиугольник
\begin{itemize}
\item Нужно поменять количество вершин в вызове \mintinline{cpp}|glDrawArrays|
\item Нужно правильно вычислить координаты вершин в вершинном шейдере
\item Нужно, чтобы шестиугольник можно было двигать по обеим осям
\begin{itemize}
\item Вам пригодится словарь \mintinline{cpp}|key_down| и константы \mintinline{cpp}|SDL_SCANCODE_LEFT|, \mintinline{cpp}|SDL_SCANCODE_RIGHT|, \mintinline{cpp}|SDL_SCANCODE_UP|, \mintinline{cpp}|SDL_SCANCODE_DOWN|
\end{itemize}
\end{itemize}
\end{frame}

\begin{frame}
\frametitle{Задание 7}
\slideimage{7.png}
\end{frame}

\end{document}