% (c) Nikita Lisitsa, lisyarus@gmail.com, 2023

\documentclass{beamer}

\usepackage[T2A]{fontenc}
\usepackage[russian]{babel}
\usepackage{minted}

\usepackage{graphicx}
\graphicspath{ {./images/} }

\usepackage{adjustbox}

\usepackage{color}
\usepackage{soul}

\usepackage{hyperref}

\usepackage{amsmath}

\usepackage{tikz}
\usetikzlibrary{decorations}
\usetikzlibrary{decorations.pathreplacing}
\usepackage{xifthen}

\usetheme{metropolis}
\setminted{fontsize=\footnotesize}

\definecolor{red}{rgb}{1,0,0}
\definecolor{green}{rgb}{0,0.5,0}
\definecolor{blue}{rgb}{0,0,1}
\definecolor{magenta}{rgb}{0.75,0,0.75}
\definecolor{codebg}{RGB}{29,35,49}

\makeatletter
\newcommand{\slideimage}[1]{
  \begin{figure}
    \begin{adjustbox}{width=\textwidth, totalheight=\textheight-2\baselineskip-2\baselineskip,keepaspectratio}
      \includegraphics{#1}
    \end{adjustbox}
  \end{figure}
}
\makeatother

\title{Компьютерная графика}
\subtitle{Лекция 15: рендеринг текста, bitmap-шрифты, векторные шрифты, (M)SDF-шрифты, чем заняться дальше}
\date{2023}

\setbeamertemplate{footline}[frame number]

\usemintedstyle{solarized-light}

\begin{document}

\frame{\titlepage}

\begin{frame}[fragile]
\frametitle{Рендеринг текста}
\begin{itemize}
\item Текст -- очень сложная штука
\pause
\item Слева направо (латиница, кириллица), справа налево (арабский, иврит), сверху вниз (хангыль, традиционное японское и китайское письмо, старомонгольское письмо)
\pause
\item Иероглифы (кандзи), слоговое письмо (катакана, хирагана), консонантно-слоговое письмо (индийская, эфиопская письменность), консонантное письмо (арабский), консонантно-вокалическое письмо (латиница, кириллица)
\pause
\item Одна \textit{графема} может представлять один или несколько звуков/слогов/объектов
\pause
\item Могут быть сложные правила по соединению символов между собой (арабский, лигатуры в латинице), дополнения к символам (диакритика)
\end{itemize}
\end{frame}

\begin{frame}[fragile]
\frametitle{Алфавиты}
\slideimage{alphabet.jpg}
\end{frame}

\begin{frame}[fragile]
\frametitle{Корейская письменность}
\slideimage{korean.png}
\end{frame}

\begin{frame}[fragile]
\frametitle{Арабская письменность}
\slideimage{arabic.png}
\end{frame}

\begin{frame}[fragile]
\frametitle{Этапы рендеринга текста}
\begin{itemize}
\item Абстрактный текст
\pause
\item + \textit{кодировка} \begin{math}\Longrightarrow\end{math} машинное представление текста
\pause
\item + \textit{шрифт} + настройки \textit{шейпинга (shaping)} \begin{math}\Longrightarrow\end{math} набор глифов (изображений символов) и их координат
\pause
\item + \textit{алгоритм} рендеринга \begin{math}\Longrightarrow\end{math} нарисованный текст
\end{itemize}
\end{frame}

\begin{frame}[fragile]
\frametitle{Кодировки}
\begin{itemize}
\item Описывают машинное представление текста, т.е. соответствие \alert{\textbf{последовательностей символов}} и \alert{\textbf{последовательностей бит}}
\end{itemize}
\end{frame}

\begin{frame}[fragile]
\frametitle{Кодировки: ASCII}
\begin{itemize}
\item 7 бит (обычно дополняется нулевым старшим битом до 8 бит), первые 32 символа - управляющие (\verb|\r|, \verb|\n|, \verb|\t|, ...), остальные 96 -- буквы английского алфавита (большие и маленькие) и прочие символы (различные скобки, арифметические операции, пунктуация, пробел, ...)
\pause
\item Многие кодировки совпадают с ASCII в диапазоне 0-127 или 32-127
\end{itemize}
\end{frame}

\begin{frame}[fragile]
\frametitle{Кодировки: 8-битные}
\begin{itemize}
\item \textbf{ISO/IEC 8859} -- 15 разных вариантов (ISO/IEC 8859-5 для русского языка)
\pause
\item \textbf{Code page XXX} -- много разных кодировок для DOS (Code page 866 для русского языка)
\pause
\item \textbf{Windows code pages} (Windows-1251 для русского языка)
\pause
\item \textbf{KOI-8} и вариации -- для русского языка
\pause
\item И т.д.
\end{itemize}
\end{frame}

\begin{frame}[fragile]
\frametitle{Кодировки: unicode}
\begin{itemize}
\item \textbf{Unicode} -- стандарт, описывающий соответствие абстрактных символов целочисленным кодам (\textit{code points}) в диапазоне \verb|0x0..0x10FFFFh| исключая \verb|0xD800h..0xDFFFh| для суррогатных пар в UTF-16 (итого 1112064 code point'а), и рекомендации по их интерпретации и визуализации
\pause
\item На сегодняшний день описывает 154998 символов (в прошлом году было 149186)
\pause
\item Сам unicode -- \textbf{не кодировка}, а таблица соответствия, но есть основанные на нём кодировки
\end{itemize}
\end{frame}

\begin{frame}[fragile]
\frametitle{Кодировки: unicode}
\begin{itemize}
\item \textbf{UTF-8}: от 1 до 4 байт на символ, совпадает с ASCII в диапазоне \verb|0..7Fh|, самая распространённая сегодня кодировка (95\% интернета)
\pause
\item \textbf{UCS-2}: устаревшая, 2 байта на символ, не поддерживает весь unicode
\pause
\item \textbf{UTF-16}: 2 или 4 байта на символ
\pause
\item \textbf{UTF-32}: 4 байта на символ
\pause
\item \textbf{GB 18030}: специальная кодировка для китайских иероглифов (но тоже поддерживает весь unicode)
\end{itemize}
\end{frame}

\begin{frame}[fragile]
\frametitle{Кодировки: unicode}
\begin{itemize}
\item В кодировках UTF-8 и UTF-16 символ занимает \textit{переменное} количество байт -- в этой кодировке сложнее двигаться по тексту на целое число code point'ов
\pause
\item UTF-16 часто используется в WinAPI как альтернатива более старым функциям, поддерживавшим только 8-битные кодировки
\pause
\item \textbf{Моё мнение}: лучше хранить текст в UTF-8, редактировать его в UTF-32, и конвертировать в UTF-16 только для передачи в WinAPI :)
\end{itemize}
\end{frame}

\begin{frame}[fragile]
\frametitle{Code points, глифы и графемы}
\begin{itemize}
\item \textbf{Code point} -- один unicode элемент (абстрактный символ)
\pause
\item \textbf{Глиф} -- одно изображение в шрифте
\pause
\item \textbf{Графема} -- один визуальный символ (один или несколько глифов)
\pause
\item В общем случае один code point \alert{\textbf{не соответствует}} одному глифу или одной графеме
\pause
\item Примеры:
\begin{itemize}
\item Два символа \verb|ff| могут быть представлены двумя глифами или одним глифом (лигатурой) ff
\pause
\item Символ \verb|Ò| может быть одним или двумя code point'ами и одним или двумя (\verb|O + `|) глифами, но считается одной графемой
\end{itemize}
\end{itemize}
\end{frame}

\begin{frame}[fragile]
\frametitle{Шрифты}
\begin{itemize}
\item Содержит набор \textit{глифов} (изображений символов в каком-либо виде) и правил их использования
\pause
\item Виды шрифтов:
\pause
\begin{itemize}
\item \textbf{Bitmap-шрифты}: глиф -- готовое изображение (bitmap)
\pause
\item \textbf{Векторные шрифты}: глиф описывается как геометрическая фигура
\pause
\item \textbf{(M)SDF-шрифты}: глиф описывается с помощью \textit{signed distance field} (SDF)
\end{itemize}
\end{itemize}
\end{frame}

\begin{frame}[fragile]
\frametitle{Шрифты}
\begin{itemize}
\item Современные форматы шрифтов (\verb|.ttf| -- TrueType, \verb|.otf| -- OpenType) -- векторные, описывают границу глифа как набор отрезков и квадратичных кривых Безье
\pause
\item Bitmap и SDF шрифты часто строятся по векторным шрифтам
\pause
\item \verb|FreeType| -- самая распространённая библиотека для чтения векторных шрифтов; умеет растеризовать в bitmap и (с версии 2.11.0, июль 2021) в SDF
\end{itemize}
\end{frame}

\begin{frame}[fragile]
\frametitle{Шейпинг (shaping)}
\begin{itemize}
\item Процесс преобразования последовательности символов в набор отпозиционированных глифов
\pause
\item Может включать в себя:
\pause
\begin{itemize}
\item Настройки: направление (слева-направо, справа-налево, сверху-вниз, снизу-вверх), размер шрифта, межбуквенное расстояние, стиль (жирный, курсив, и т.п.)
\pause
\item Hinting: сдвиг глифов, чтобы они были лучше выровнены по пиксельной сетке
\pause
\item Kerning: изменение расстояния между соседними глифами для лучшего восприятия
\pause
\item Лигатуры: последовательности несвязанных символов, представленные одним глифом (ff, fi, <=>)
\end{itemize}
\end{itemize}
\end{frame}

\begin{frame}[fragile]
\frametitle{Шейпинг (shaping)}
\begin{itemize}
\item Шейпинг \alert{\textbf{не занимается}} расстановкой строк и абзацев, а также двунаправленным (bidirectional) текстом (идущим как слева-направо, так и справа-налево)
\pause
\item Для простых моноширинных шрифтов шейпинг может сводиться к расположению глифов на равных расстояниях друг от друга
\pause
\item \verb|harfbuzz| -- одна из самых распространённых библиотек для шейпинга текста, используется всеми на свете
\pause
\item \verb|FreeType| позволяет сделать шейпинг, но хуже, чем harfbuzz
\pause
\item \verb|FriBidi| -- бибилиотека для двунаправленного текста
\end{itemize}
\end{frame}

\begin{frame}[fragile]
\frametitle{Hinting}
\slideimage{hinting.png}
\end{frame}

\begin{frame}[fragile]
\frametitle{Kerning}
\slideimage{kerning.png}
\end{frame}

\begin{frame}[fragile]
\frametitle{Лигатуры}
\slideimage{ligatures.png}
\end{frame}

\begin{frame}[fragile]
\frametitle{Рендеринг bitmap-шрифтов}
\begin{itemize}
\item Обычно представлены в виде texture atlas: одна текстура, содержащая все глифы шрифта
\pause
\item Содержат информацию о расположении глифов в текстуре (текстурные координаты левого верхнего и правого нижнего пикселя)
\pause
\item Глифы рисуются как текстурированные прямоугольники
\pause
\item Плохо ведут себя при масштабировнии (как увеличении, так и уменьшении), mipmap'ы не особо помогают
\pause
\item Очень просты в реализации
\pause
\item Часто используются для дебажного текста, инди-игр, и т.п.
\end{itemize}
\end{frame}

\begin{frame}[fragile]
\frametitle{Bitmap-шрифт}
\slideimage{bitmap-font.png}
\end{frame}

\begin{frame}[fragile]
\frametitle{Bitmap-шрифт: описание в коде}
\usemintedstyle{lightbulb}
\begin{minted}[bgcolor=codebg]{cpp}
struct bitmap_font
{
  GLuint texture_id;

  struct glyph
  {
    vec2 top_left;
    vec2 bottom_right;
  };

  std::unordered_map<std::char32_t, glyph> glyphs;
};
\end{minted}
\end{frame}

\begin{frame}[fragile]
\frametitle{Bitmap-шрифт: фрагментный шейдер}
\usemintedstyle{lightbulb}
\begin{minted}[bgcolor=codebg]{glsl}
uniform sampler2D font_texture;
uniform vec3 text_color;

in vec2 texcoord;

layout (location = 0) out vec4 out_color;

void main()
{
  float alpha = texture(font_texture, texcoord).r;
  out_color = vec4(text_color, alpha);
}
\end{minted}
\end{frame}

\begin{frame}[fragile]
\frametitle{Рендеринг векторных шрифтов}
\begin{itemize}
\item Глиф описывается как набор геометрических фигур (фигура может описывать `дырку' в другой фигуре, как дырка в букве `О'), граница фигуры -- набор отрезков и квадратичных кривых Безье
\pause
\item Много разных способов рендеринга:
\pause
\begin{itemize}
\item Аппроксимация набором треугольников
\pause
\item Запаковка фигур в текстуру + шейдер, который честно вычисляет площадь пересечения фигуры и пикселя
\pause
\item Полигональная аппроксимация глифа (рисуется с использованием stencil буфера) + треугольник со специальным шейдером для каждой кривой Безье
\pause
\item Slug algorithm (запатентован)
\end{itemize}
\pause
\item Обычно легко переносит масштабирование
\pause
\item Сложен в реализации
\pause
\item Используется для текста максимально возможного качества
\end{itemize}
\end{frame}

\begin{frame}[fragile]
\frametitle{Векторный глиф}
\slideimage{vector-glyph.png}
\end{frame}

\begin{frame}[fragile]
\frametitle{Slug algorithm}
\slideimage{slug.jpg}
\end{frame}

\begin{frame}[fragile]
\frametitle{Signed distance field (SDF)}
\begin{itemize}
\item Описание двумерного или трёхмерного объекта/фигуры \textit{функцией расстояния} до границы объекта
\pause
\item Обычно положительна снаружи объекта и отрицательна внутри (поэтому \textit{signed}), \begin{math}f(p) = 0\end{math} -- граница объекта
\pause
\item SDF может быть представлена явной формулой (напр. \begin{math}f(p) = \|p - O\| - R\end{math} -- расстояние до сферы радиуса \begin{math}R\end{math} с центром в точке \begin{math}O\end{math}), комбинацией более простых SDF, или текстурой
\pause
\item SDF-сцены часто используются для экспериментального рендеринга и удобны для raymarching'а
\end{itemize}
\end{frame}

\begin{frame}[fragile]
\frametitle{Signed distance field (SDF)}
\begin{itemize}
\item SDF-сцены часто используются для экспериментального рендеринга и удобны для raymarching'а
\pause
\item Часто, использующиеся на практике функции не являются SDF, и удовлетворяют более слабым условиям (например, \begin{math}f(p)\end{math} не больше расстояния до объекта, или \begin{math}|\nabla f|\leq L\end{math})
\pause
\item Блог Inigo Quilez (\href{https://iquilezles.org/}{\texttt{iquilezles.org}}) -- один из лучших ресурсов про SDF и рендеринг с их помощью
\end{itemize}
\end{frame}

\begin{frame}[fragile]
\frametitle{Рендеринг SDF-шрифтов (Valve, 2007)}
\begin{itemize}
\item Описывается так же, как bitmap-шрифт, но текстура хранит значения SDF для глифов
\pause
\item Фрагментный шейдер читает значение SDF из текстуры шрифта: если оно меньше 0, то пиксель находится внутри глифа (e.g. чёрный пиксель), иначе -- нет (e.g. прозрачный пиксель)
\pause
\item Прост в реализации
\pause
\item Требует чуть больше места под глифы, но менее требователен к разрешению
\pause
\item Неплохо масштабируется (есть артефакты, но менее серьёзные, чем для bitmap-шрифтов)
\pause
\item Один из самых популярных способов рендеринга шрифтов
\end{itemize}
\end{frame}

\begin{frame}[fragile]
\frametitle{SDF-шрифт}
\slideimage{sdf-font.png}
\end{frame}

\begin{frame}[fragile]
\frametitle{SDF-шрифт: артефакты при magnification}
\slideimage{sdf-artifacts.jpg}
\end{frame}

\begin{frame}[fragile]
\frametitle{SDF-текстура}
\begin{itemize}
\item Тестуры позволяют хранить значения от 0 до 1
\pause
\item Мы хотим хранить произвольные, но не очень большие по модулю числа (расстояние в пикселях)
\pause
\item \begin{math}\Longrightarrow\end{math} В текстуре придётся хранить что-то в духе \begin{math}0.5 + sdf / scale\end{math}, где \begin{math}scale\end{math} -- максимальное представимое расстояние
\pause
\item \begin{math}\Longrightarrow\end{math} При чтении из текстуры нужно применять обратное преобразование \begin{math}(sdf - 0.5) \cdot scale\end{math}
\pause
\item Для трёхмерного текста лучше включить для этой текстуры анизотропную фильтрацию, чтобы текст хорошо выглядел `сбоку'
\end{itemize}
\end{frame}

\begin{frame}[fragile]
\frametitle{Рендеринг SDF-шрифтов}
\begin{itemize}
\item Можно легко реализовать много дополнительных эффектов:
\pause
\begin{itemize}
\item Сглаживание: вместо жёсткой границы плавно меняем прозрачность пикселя в зависимости от значения SDF
\pause
\item Сглаживание с учётом масштаба и перспективы: по \mintinline{glsl}|dFfx, dFfy| можно вычислить диапазон значений SDF, чтобы сглаживание было ровно в 1 пиксель в экранных координатах
\pause
\item Обводка текста другим цветом: рисуем цвет обводки, если \begin{math}0 \leq f(p) \leq \varepsilon\end{math}
\pause
\item Псевдотрёхмерный текст: по градиенту SDF можно восстановить нормаль к глифу
\end{itemize}
\end{itemize}
\end{frame}

\begin{frame}[fragile]
\frametitle{SDF-шрифт с эффектами}
\slideimage{sdf-effects.png}
\end{frame}

\begin{frame}[fragile]
\frametitle{SDF-шрифт: фрагментный шейдер}
\usemintedstyle{lightbulb}
\begin{minted}[bgcolor=codebg]{glsl}
uniform sampler2D sdfTexture;
uniform float sdfScale;
uniform vec3 textColor;

in vec2 texcoord;

layout (location = 0) out vec4 out_color;

void main()
{
  float textureValue = texture(sdfTexture, texcoord).r;
  float sdfValue = sdfScale * (textureValue - 0.5);
  // сглаживание
  float alpha = smoothstep(-0.5, 0.5, sdfValue);
  out_color = vec4(textColor, alpha);
}
\end{minted}
\end{frame}

\begin{frame}[fragile]
\frametitle{MSDF (Chlumský, 2015)}
\begin{itemize}
\item У SDF-текста есть типичные артефакты: острые углы сглаживаются, из-за чего приходится брать SDF-текстуру большого разрешения
\pause
\item Идея: билинейная интерполяция не портит прямые линии \begin{math}\Longrightarrow\end{math} представим глиф \textit{пересечением} нескольких объектов, чтобы острые углы, представленные пересечением нескольких прямых, не сглаживались
\pause
\item В текстуре есть 4 канала (RGBA) \begin{math}\Longrightarrow\end{math} можем сохранить сразу 4 разных SDF в одной текстуре!
\end{itemize}
\end{frame}

\begin{frame}[fragile]
\frametitle{MSDF}
\begin{itemize}
\item \textbf{MSDF}: Multi-channel signed distance field
\pause
\item На практике оказывается, что хватает 3 каналов, т.е. трёх различных SDF
\pause
\item Вместо пересечения (т.е. минимума из трёх значений SDF) лучше брать \textit{медиану} (т.е. значение посередине между двумя другими)
\pause
\item Эффекты на этих трёх каналах уже не реализовать, но для этого можно записать обычную SDF в четвёртый (альфа) канал
\pause
\item Есть инструменты (программа, библиотека, сайт) для генерации таких текстур по шрифту
\end{itemize}
\end{frame}

\begin{frame}[fragile]
\frametitle{MSDF-шрифт}
\slideimage{msdf-font.png}
\end{frame}

\begin{frame}[fragile]
\frametitle{MSDF: сравнение с SDF}
\slideimage{msdf.png}
\end{frame}

\begin{frame}[fragile]
\frametitle{MSDF: пример кода}
\usemintedstyle{lightbulb}
\begin{minted}[bgcolor=codebg,fontsize=\scriptsize]{glsl}
uniform sampler2D sdfTexture;
uniform float sdfScale;
uniform vec3 textColor;

in vec2 texcoord;

layout (location = 0) out vec4 out_color;

float median(vec3 v) {
    return max(min(v.r, v.g), min(max(v.r, v.g), v.b));
}

void main()
{
  float textureValue =
    median(texture(sdfTexture, texcoord).rgb);
  float sdfValue = sdfScale * (textureValue - 0.5);
  float alpha = smoothstep(-0.5, 0.5, sdfValue);
  out_color = vec4(textColor, alpha);
}
\end{minted}
\end{frame}

\begin{frame}[fragile]
\frametitle{Шрифты и шейпинг: ссылки}
\begin{itemize}
\item \href{https://freetype.org}{\texttt{FreeType}}
\item \href{https://harfbuzz.github.io}{\texttt{harfbuzz}}
\item \href{https://github.com/fribidi/fribidi}{\texttt{FriBidi}}
\end{itemize}
\end{frame}

\begin{frame}[fragile]
\frametitle{Векторные и bitmap-шрифты: ссылки}
\begin{itemize}
\item \href{https://learnopengl.com/In-Practice/Text-Rendering}{\texttt{Туториал по рендерингу bitmap-шрифтов}}
\item \href{https://wdobbie.com/post/gpu-text-rendering-with-vector-textures}{\texttt{Один способ рендеринга векторных шрифтов}}
\item \href{https://medium.com/@evanwallace/easy-scalable-text-rendering-on-the-gpu-c3f4d782c5ac}{\texttt{Другой способ рендеринга векторных шрифтов}}
\item \href{https://jcgt.org/published/0006/02/02}{\texttt{Slug algorithm}}
\item \href{https://sluglibrary.com}{\texttt{Slug library}}
\end{itemize}
\end{frame}

\begin{frame}[fragile]
\frametitle{SDF и MSDF-шрифты: ссылки}
\begin{itemize}
\item \href{https://steamcdn-a.akamaihd.net/apps/valve/2007/SIGGRAPH2007_AlphaTestedMagnification.pdf}{\texttt{Статья от Valve про SDF-текст}}
\item \href{https://blog.mapbox.com/drawing-text-with-signed-distance-fields-in-mapbox-gl-b0933af6f817}{\texttt{Туториал по рендерингу SDF-шрифтов}}
\item \href{https://evanw.github.io/font-texture-generator/}{\texttt{SDF font generator}}
\item \href{https://github.com/Chlumsky/msdfgen}{\texttt{Репозиторий с генератором MSDF-шрифтов от автора этой техники}}
\item \href{https://github.com/Chlumsky/msdfgen/files/3050967/thesis.pdf}{\texttt{Shape Decomposition for Multi-channel Distance Fields (Chlumský, 2015)}}
\item \href{https://msdf-bmfont.donmccurdy.com/}{\texttt{MSDF font generator}}
\end{itemize}
\end{frame}

\begin{frame}[fragile]
\frametitle{Чем заняться дальше?}
\begin{itemize}
\item Очень много источников света
\item Очень много объектов
\item Очень большая сцена
\item Сложные объекты
\item Другие API
\item Raytracing
\item Статьи и конференции
\end{itemize}
\end{frame}

\begin{frame}[fragile]
\frametitle{Очень много источников света}
\begin{itemize}
\item Deferred shading
\item Tiled/Clustered shading
\item Compute shaders (OpenGL 4.3 или расширение)
\item \href{https://hal.science/hal-02155101v1/file/LTC_linearlights_GPUZen.pdf}{\texttt{Линейные}} и \href{https://learnopengl.com/Guest-Articles/2022/Area-Lights}{\texttt{площадные источники света}} с помощью linearly transformed cosines (LTC)
\end{itemize}
\end{frame}

\begin{frame}[fragile]
\frametitle{Очень много объектов}
\begin{itemize}
\item Batching
\item Frustum culling
\item Occlusion culling
\item LOD
\end{itemize}
\end{frame}

\begin{frame}[fragile]
\frametitle{Очень большая сцена (e.g. планета)}
\begin{itemize}
\item Иерархический LOD для ландшафта (например, в виде квадродерева)
\item Проблемы с точностью (\mintinline{cpp}|float| не хватает; нужно рисовать относительно некой anchor-точки в \mintinline{cpp}|double|)
\item Проблемы с буфером глубины (\href{https://nlguillemot.wordpress.com/2016/12/07/reversed-z-in-opengl/}{\texttt{reversed z}})
\end{itemize}
\end{frame}

\begin{frame}[fragile]
\frametitle{Сложные объекты}
\begin{itemize}
\item Вода: отражение + преломление (по Френелю) + `туман' или честное рассеивание в плотности воды + \href{https://en.wikipedia.org/wiki/Trochoidal_wave}{\texttt{волны Герстнера}}
\item Растительность (vegetation): трава, кусты, деревья (см. \href{https://developer.nvidia.com/gpugems/gpugems/part-i-natural-effects/chapter-7-rendering-countless-blades-waving-grass}{\texttt{статья в GPU Gems}}, \href{https://www.youtube.com/watch?v=Ibe1JBF5i5Y}{\texttt{доклад про траву в Ghost of Tsushima}}, \href{https://www.youtube.com/watch?app=desktop&v=wavnKZNSYqU}{доклад про vegetation в Horizon Zero Dawn})
\item Облака + небо (volume rendering) (см. \href{https://www.guerrilla-games.com/read/the-real-time-volumetric-cloudscapes-of-horizon-zero-dawn}{\texttt{доклад про облака в Horizon Zero Dawn}})
\item Сложные BRDF (см. \href{https://boksajak.github.io/files/CrashCourseBRDF.pdf}{\texttt{Crash Course in BRDF Implementation}})
\end{itemize}
\end{frame}

\begin{frame}[fragile]
\frametitle{Другие API}
\begin{itemize}
\item OpenGL 4.0+: compute shaders, indirect rendering, direct state access
\item Vulkan: \href{https://vulkan-tutorial.com}{\texttt{vulkan-tutorial.com}}
\item WebGPU: \href{https://eliemichel.github.io/LearnWebGPU/index.html}{\texttt{WebGPU C++ Guide}}
\end{itemize}
\end{frame}

\begin{frame}[fragile]
\frametitle{Raytracing + real-time GI}
\begin{itemize}
\item \href{https://raytracing.github.io/books/RayTracingInOneWeekend.html}{\texttt{Ray Tracing in One Weekend}} (\textbf{не} real-time)
\item \href{https://developer.nvidia.com/rtx/raytracing/vkray}{\texttt{Vulkan raytracing tutorial}}
\item \href{https://research.nvidia.com/sites/default/files/publications/GIVoxels-pg2011-authors.pdf}{\texttt{Voxel cone tracing}}
\item \href{https://www.ea.com/seed/news/siggraph21-global-illumination-surfels}{\texttt{Surflet-based GI}}
\item \href{https://research.nvidia.com/publication/2021-06_restir-gi-path-resampling-real-time-path-tracing}{\texttt{ReSTIR GI}}
\item \href{https://radiance-cascades.com/}{\texttt{Radiance cascades}}
\end{itemize}
\end{frame}

\begin{frame}[fragile]
\frametitle{Статьи и конференции}
\begin{itemize}
\item \href{https://developer.nvidia.com/gpugems}{\texttt{GPU Gems 1, 2, 3}}
\item SIGGRAPH (e.g. \href{https://kesen.realtimerendering.com/sig2022.html}{\texttt{2022}})
\item \href{GDC}{\texttt{https://www.gdcvault.com/free/}}
\item \href{https://www.jendrikillner.com/tags/weekly/}{\texttt{Graphics Programming Weekly}}
\end{itemize}
\end{frame}

\begin{frame}[fragile]
\frametitle{В заключение}
\begin{itemize}
\item Область real-time рендеринга очень активно развивается
\item О рисовании любого объекта/эффекта можно найти десятки статей и даже PhD
\item Есть тысячи туториалов по всему на свете
\item Не бойтесь гуглить и читать непонятные статьи, со временем станет понятнее
\item Не бойтесь писать мне :)
\end{itemize}
\end{frame}

\end{document}