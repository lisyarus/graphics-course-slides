% (c) Nikita Lisitsa, lisyarus@gmail.com, 2022

\documentclass{beamer}

\usepackage[T2A]{fontenc}
\usepackage[utf8]{inputenc}
\usepackage[russian]{babel}

\usepackage{graphicx}
\graphicspath{ {./images/} }

\usepackage{adjustbox}

\usepackage{color}
\usepackage{soul}

\usepackage{hyperref}

\definecolor{blue}{rgb}{0,0,1}
\definecolor{red}{rgb}{1,0,0}

\makeatletter
\newcommand{\slideimage}[1]{
  \begin{figure}
    \begin{adjustbox}{width=\textwidth, totalheight=\textheight-2\baselineskip-2\baselineskip,keepaspectratio}
      \includegraphics{#1}
    \end{adjustbox}
  \end{figure}
}
\makeatother

\title{Компьютерная графика}
\subtitle{Практика 4: Индексы, перспективная проекция, буфер глубины}
\date{2021}

\setbeamertemplate{footline}[frame number]

\begin{document}

\frame{\titlepage}

\begin{frame}[fragile]
\frametitle{Напоминание про VBO и VAO}
\begin{itemize}
\item VBO -- хранит данные, ничего не знает о формате
\item VAO -- описывает формат и расположение атрибутов вершин
\item Концептуально, расположение = id буфера + сдвиг
\item VAO также хранит id текущего EBO (\verb|GL_ELEMENT_ARRAY_BUFFER|) -- оттуда берутся индексы
\item Для рендеринга нужен только VAO
\item Чтобы обновить данные, нужен только VBO
\item Чтобы обновить индексы, нужен только EBO (можно использовать \verb|target = GL_ARRAY_BUFFER|)
\end{itemize}
\end{frame}

\begin{frame}[fragile]
\frametitle{Практика 4}
\begin{itemize}
\item В этой практике нельзя менять код шейдеров!
\end{itemize}
\end{frame}

\begin{frame}[fragile]
\frametitle{Практика 4}
\slideimage{0.png}
\end{frame}

\begin{frame}[fragile]
\frametitle{Задание 1}
Рисуем зайца
\begin{itemize}
\item Создаём VAO, VBO, EBO
\item Загружаем данные (\verb|bunny.vertices| и \verb|bunny.indices|) в VBO и EBO
\item Настраиваем атрибуты для VAO (нужно понять правильные настройки по вершинному шейдеру и по описанию структуры \verb|obj_data::vertex| в \verb|obj_parser.hpp|)
\item Рисуем с помощью \verb|glDrawElements|, \verb|mode = GL_TRIANGLES|, обращаем внимание на тип индексов (\verb|GL_UNSIGNED_INT|)
\item N.B. заяц будет рисоваться странно из-за отключенного теста глубины, так и задумано
\end{itemize}
\end{frame}

\begin{frame}[fragile]
\frametitle{Задание 1}
\slideimage{1.png}
\end{frame}

\begin{frame}[fragile]
\frametitle{Задание 2}
Вращаем зайца
\begin{itemize}
\item Меняем матрицу \verb|transform| чтобы куб крутился в плоскости XZ
\item В качестве угла нужно взять что-нибудь зависящее от времени, например \verb|float angle = time;|
\item Заяц будет обрезаться по \begin{math}z \in [-1, 1]\end{math} (особенно хорошо видно, что иногда обрезается хвост)
\item Отмасштабируем его по всем осям \verb|float scale = 0.5f|, тоже с помощью матрицы \verb|transform|
\end{itemize}
\end{frame}

\begin{frame}[fragile]
\frametitle{Задание 2}
\slideimage{2.png}
\end{frame}

\begin{frame}[fragile]
\frametitle{Задание 3}
Добавляем перспективу
\begin{itemize}
\item Выбираем значения \verb|near|, \verb|far|, \verb|top|, \verb|right|
\begin{itemize}
\item \verb|near| -- маленький, но не слишком, в духе \begin{math}0.001 \dots 0.1\end{math}
\item \verb|far| -- большой, но не слишком, в духе \begin{math}10.0 \dots 1000.0\end{math}
\item Отношение \verb|right/near| -- тангенс половины угла обзора, например \verb|right = near| это \begin{math}90^\circ\end{math}
\item Отношение \verb|right/top| -- aspect ratio экрана (\verb|width/height|)
\end{itemize}
\item В матрицу \verb|view| записываем матрицу проекции с использованием выбранных значений (см. слайд с лекции)
\item Заяц будет виден изнутри
\end{itemize}
\end{frame}

\begin{frame}[fragile]
\frametitle{Задание 3}
\slideimage{3.png}
\end{frame}

\begin{frame}[fragile]
\frametitle{Задание 4}
Включаем тест глубины
\begin{itemize}
\item Сдвигаем зайца по оси Z на какое-то расстояние (например, на 3 единицы)
\begin{itemize}
\item N.B: Камера смотрит в сторону -Z, т.е. сдвиг должен быть отрицательным
\end{itemize}
\item Заяц будет рисоваться неправильно: задние грани перекрывают передние
\item Включим тест глубины: \verb|glEnable(GL_DEPTH_TEST)|
\item Не забываем очищать буфер глубины в начале каждого кадра: \verb|glClear(GL_DEPTH_BUFFER_BIT)|
\begin{itemize}
\item Можно это делать одновременно с очисткой цветового буфера:  \verb=glClear(GL_DEPTH_BUFFER_BIT | GL_COLOR_BUFFER_BIT)=
\end{itemize}
\end{itemize}
\end{frame}

\begin{frame}[fragile]
\frametitle{Задание 4}
\slideimage{4.png}
\end{frame}

\begin{frame}[fragile]
\frametitle{Задание 5}
Двигаем зайца
\begin{itemize}
\item Заведём переменные \verb|bunny_x| и \verb|bunny_y| с координатами центра зайца по X и Y
\item Изменим матрицу \verb|transform|, добавив соответствующие сдвиги по X и Y
\item Перед рисованием каждого кадра обновим положение зайца:
\begin{itemize}
\item Если нажата клавиша влево \verb|SDLK_LEFT|, сдвинем куб влево: \verb|cube_x -= speed * dt|
\item Аналогично \verb|SDLK_RIGHT|, \verb|SDLK_DOWN|, \verb|SDLK_UP|
\item Состояние нажатости клавиш уже доступно в словаре \verb|button_down|
\end{itemize}
\end{itemize}
\end{frame}

\begin{frame}[fragile]
\frametitle{Задание 5}
\slideimage{5.png}
\end{frame}

\begin{frame}[fragile]
\frametitle{Задание 6}
Играем с face culling
\begin{itemize}
\item Включим back-face culling: \verb|glEnable(GL_CULL_FACE)|
\item Ничего не изменится: заяц сделан так, чтобы все треугольники были CCW
\item Изменим режим: \verb|glCullFace(GL_FRONT)|
\item Должны быть видны задние грани куба и не видны передние
\item Выглядеть будет странно -- не пугайтесь, попробуйте увидеть и понять, что происходит :)
\end{itemize}
\end{frame}

\begin{frame}[fragile]
\frametitle{Задание 6}
\slideimage{6.png}
\end{frame}

\begin{frame}[fragile]
\frametitle{Задание 7*}
Три вращающихся зайца
\begin{itemize}
\item Рисуем три зайца в разных местах, вращающихся в плоскостях XY, XZ, YZ соответственно
\item Три раза \verb|glUniformMatrix| + \verb|glDrawElements|
\item Всё ещё можно двигать стрелочками, т.е. должны учитываться \verb|bunny_x| и \verb|bunny_y|
\end{itemize}
\end{frame}

\begin{frame}[fragile]
\frametitle{Задание 7}
\slideimage{7.png}
\end{frame}

\end{document}