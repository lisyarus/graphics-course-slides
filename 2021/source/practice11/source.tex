% (c) Nikita Lisitsa, lisyarus@gmail.com, 2021

\documentclass{beamer}

\usepackage[T2A]{fontenc}
\usepackage[utf8]{inputenc}
\usepackage[russian]{babel}

\usepackage{graphicx}
\graphicspath{ {./images/} }

\usepackage{adjustbox}

\usepackage{color}
\usepackage{soul}

\usepackage{hyperref}

\definecolor{blue}{rgb}{0,0,1}
\definecolor{red}{rgb}{1,0,0}

\makeatletter
\newcommand{\slideimage}[1]{
  \begin{figure}
    \begin{adjustbox}{width=\textwidth, totalheight=\textheight-2\baselineskip-2\baselineskip,keepaspectratio}
      \includegraphics{#1}
    \end{adjustbox}
  \end{figure}
}
\makeatother

\title{Компьютерная графика}
\subtitle{Практика 11: Система частиц}
\date{2021}

\setbeamertemplate{footline}[frame number]

\begin{document}

\frame{\titlepage}

\begin{frame}[fragile]
\frametitle{Задание 1}
Рисуем частицы как квадраты
\begin{itemize}
\item В структуру \verb|particle| добавляем параметр \verb|float size|, инициализируем его в случайное значение (например, от \verb|0.2| до \verb|0.4|)
\pause
\item Добавляем соответствующий атрибут для VAO и в вершинном шейдере
\pause
\item Вершинный шейдер просто передаёт значение \verb|size| в геометрический шейдер
\pause
\item В геометрическом шейдере меняем тип выходной геометрии: \verb|triangle_strip|, \verb|max_vertices = 4|
\pause
\item В геометрическом шейдере вместо генерации одной вершины генерируем 4 вершины с координатами \begin{math}center + (\pm size, \pm size, 0)\end{math}
\pause
\item Из геометрического шейдера во фрагментный передаём текстурные координаты и используем их в качестве цвета
\end{itemize}
\end{frame}

\begin{frame}[fragile]
\frametitle{Задание 2}
Поворачиваем частицы в сторону камеры
\begin{itemize}
\item В геометрическом шейдере заводим uniform-переменную для позиции камеры и устанавливаем её при рендеринге
\pause
\item Вычисляем X, Y, Z оси для частицы:
\begin{itemize}
\item Z - направление из центра частицы на камеру
\item X, Y - любые, перпендикулярные Z и друг другу
\end{itemize}
\pause
\item Частица должна быть параллельна плоскости XY
\end{itemize}
\end{frame}

\begin{frame}[fragile]
\frametitle{Задание 3}
\fontsize{8pt}{8pt}
Вращаем частицы и симулируем физику частиц
\begin{itemize}
\item Добавляем частице атрибут "угол поворота" (поле в структуру, входной параметр вершинного шейдере, настройка атрибута для VAO)
\pause
\item В геометрическом шейдере поворачиваем оси X, Y на этот угол
\pause
\item Добавляем частице поля "скорость" (\verb|vec3|) и "угловая скорость" (\verb|float|) (как атрибуты в шейдере они не нужны), инициализируем их чем-то случайным
\pause
\item Создаём частицы не разом, а по одной в кадр, пока их не станет 256
\pause
\item Заставляем частицы лететь вверх
\begin{itemize}
\item Увеличиваем Y-составляющую скорости на некую величину \verb|velocity.y += dt * A|
\item Интегрируем скорость и угловую скорость (\verb|position += velocity * dt|)
\item Можно добавить трение (\verb|velocity *= exp(- C * dt)|)
\item Можно уменьшать размер частицы (\verb|size *= exp(- D * dt)|)
\end{itemize}
\pause
\item Частицу, достигшую некой Y-координаты (скажем, \verb|y >= 3|), пересоздаём с новыми случайными параметрами
\pause
\item N.B.: VBO теперь нужно обновлять каждый кадр
\item N.B.: создание частиц и симуляцию физики лучше выполнять при условии \verb|if (!paused)|
\end{itemize}
\end{frame}

\begin{frame}[fragile]
\frametitle{Задание 4}
Текстурируем частицы
\begin{itemize}
\item Загружаем изображение \verb|particle.gray| из файла в директории с проектом: 1024x1024, 8 bbp (bits per pixel)
\pause
\item Создаём текстуру и загружаем в неё это изображение: internal format = \verb|GL_R8|, format = \verb|GL_RED|, type = \verb|GL_UNSIGNED_BYTE|, настраиваем линейную фильтрацию с mipmaps, генерируем mipmaps
\pause
\item Используем эту текстуру во фрагментном шейдере: текстура одноканальная, берём только первую координату цвета (\verb|texture(...).r|) и используем как альфа-канал результирующего цвета
\pause
\item Включаем аддитивный блендинг (blend func = \verb|GL_SRC_ALPHA, GL_ONE|)
\end{itemize}
\end{frame}

\begin{frame}[fragile]
\frametitle{Задание 5}
Раскрашиваем частицы
\begin{itemize}
\item Создаём одномерную текстуру с цветовой палитрой: \verb|GL_TEXTURE_1D|, линейная фильтрация (без mipmaps), несколько вручную описанных пикселей (например, чёрный, оранжевый, жёлтый, белый)
\pause
\item Передаём эту текстуру во врагментный шейдер используя texture unit 1 (\verb|GL_TEXTURE1|), в шейдере тип sampler'а - \verb|sampler1D|
\pause
\item Используем значение из первой текстуры (оно же - альфа-канал результирующего цвета) для индексации в текстуру с палитрой, результирующий цвет = цвет из палитры + альфа-канал из первой текстуры
\pause
\item Можно дополнительно умножить текстурную координату для палитры (оно же - значение альфа) на некую функцию от размера частицы (чтобы маленькие частицы были темнее; размер придётся передать во фрагментный шейдер)
\end{itemize}
\end{frame}

\end{document}