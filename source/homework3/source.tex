% (c) Nikita Lisitsa, lisyarus@gmail.com, 2023

\documentclass[10pt]{beamer}

\usepackage[T2A]{fontenc}
\usepackage[russian]{babel}
\usepackage{minted}

\usepackage{graphicx}
\graphicspath{ {./images/} }

\usepackage{adjustbox}

\usepackage{tikz}

\usepackage{color}
\usepackage{soul}

\usepackage{hyperref}

\usetheme{metropolis}
\setminted{fontsize=\footnotesize}


\definecolor{blue}{rgb}{0,0,1}
\definecolor{red}{rgb}{1,0,0}
\definecolor{codebg}{RGB}{29,35,49}

\makeatletter
\newcommand{\slideimage}[1]{
  \begin{figure}
    \begin{adjustbox}{width=\textwidth, totalheight=\textheight-2\baselineskip-2\baselineskip,keepaspectratio}
      \includegraphics{#1}
    \end{adjustbox}
  \end{figure}
}
\makeatother

\title{Компьютерная графика}
\subtitle{Домашнее задание 3: Визуализатор сцен v2}
\date{2021}

\setbeamertemplate{footline}[frame number]

\usemintedstyle{solarized-light}

\begin{document}

\frame{\titlepage}

\begin{frame}[fragile]
\frametitle{Задание}
\begin{itemize}
\item Сделать визуализатор сцен (в любом формате) с текстурами и освещением по Фонгу, normal mapping'ом, material mapping'ом, tone mapping'ом, гамма-коррекцией, отражениями, туманом и объёмными тенями
\pause
\item Удобнее иметь два файла: один -- с большой сценой (напр. Sponza), второй -- с маленьким отражающим объектом (напр. Stanford bunny)
\pause
\item Камера должна управляться пользователем (любым способом, главное -- чтобы можно было всё разглядеть)
\pause
\item Тени нужны только от направленного источника (солнца)
\pause
\item Источники света -- какие хотите, главное, чтобы были видны все эффекты; один из источников должен иметь яркость \textbf{больше единицы}
\pause
\item Можно переиспользовать код из 2ого домашнего задания
\end{itemize}
\end{frame}

\begin{frame}[fragile]
\frametitle{Задание}
\slideimage{example.png}
\end{frame}

\begin{frame}[fragile]
\frametitle{Normal mapping}
\begin{itemize}
\item Если в вашей модели есть normal maps, вам нужны tangent/bitangent векторы, которые не поддерживаются форматом Wavefront OBJ (но поддерживаются в glTF)
\pause
\item Если в вашей модели есть bump maps, то вам не нужны дополнительные данные в вершинах, но сложнее алгоритм вычисления итоговой нормали
\pause
\item Второй способ описан, например, в статье \href{https://mmikk.github.io/papers3d/mm_sfgrad_bump.pdf}{\texttt{Morten S. Mikkelsen - Bump Mapping Unparametrized Surfaces on the GPU}}, за него будет \textbf{больше баллов}
\end{itemize}
\end{frame}

\begin{frame}[fragile]
\frametitle{Bump map vs normal map}
\slideimage{bump_vs_normal.png}
\end{frame}

\begin{frame}[fragile]
\frametitle{Эффект normal mapping}
\slideimage{normal_map.png}
\end{frame}

\begin{frame}[fragile]
\frametitle{Material mapping}
\begin{itemize}
\item Скорее всего, у вас будут обычные gloss maps (\mintinline{cpp}|map_Ks| в OBJ)
\pause
\item Их нужно применять как множитель для specular составляющей освещения
\end{itemize}
\end{frame}

\begin{frame}[fragile]
\frametitle{Tone mapping}
\begin{itemize}
\item Возьмите любую функцию, которая вам понравится
\pause
\item Лучше ACES или Uncharted
\pause
\item Применять \textit{перед} гамма-коррекцией
\pause
\item Текстуры альбедо из входных данных нужно загружать как \textbf{sRGB-текстуры}, иначе они будут выглядеть слишком ярко из-за гамма-коррекции
\end{itemize}
\end{frame}

\begin{frame}[fragile]
\frametitle{Отражения}
\begin{itemize}
\item Возьмите один небольшой движущийся по сцене объект
\pause
\item Стройте cubemap-отражения из центра этого объекта
\pause
\item Применяйте отражения только к этому объекту
\pause
\item Не страшно, если в отражениях не будет виден туман, но тени \textbf{должны быть видны}
\end{itemize}
\end{frame}

\begin{frame}[fragile]
\frametitle{Отражения}
\slideimage{reflection.png}
\end{frame}

\begin{frame}[fragile]
\frametitle{Отражения}
\begin{itemize}
\item Текстуру отражений можно сделать довольно маленькой, например 256x256
\pause
\item В текстуру отражений нужно записывать цвет \textbf{без tone-mapping'а и гамма-коррекции}, т.е. сырое количество света
\pause
\item Это значение может быть больше единицы \begin{math}\Longrightarrow\end{math} для отражений нужно или использовать floating-point текстуру, или записывать в неё цвет делённый на некую максимальную яркость (и не забыть домножить обратно при чтении из этой текстуры)
\pause
\item Если использовать второй способ, может возникнуть banding из-за ограниченной точности текстуры -- тогда лучше сделать её sRGB, или вручную конвертировать при записи и чтении
\end{itemize}
\end{frame}

\begin{frame}[fragile]
\frametitle{Banding}
\slideimage{banding.png}
\end{frame}

\begin{frame}[fragile]
\frametitle{Туман и объёмные тени}
\begin{itemize}
\item Обычный туман можно сделать, вычислив оптическую глубину и подставив в формулу вида \mintinline{glsl}|mix(ambient_light, color, exp(-optical_depth/smth))|
\pause
\item Для объёмных теней нужно делать интегрирование вдоль луча от камеры до рисуемого пикселя (или наоборот)
\pause
\item Считаем, что в сцене повсюду однородная `пыль', которая частично поглощает свет, частично излучает ambient свет, и рассеивает солнечный свет
\pause
\item В рассеивании нужно учитывать, находится ли точка луча в тени (так же, как при обычном shadow mapping'е)
\pause
\item Функцию рассеивания можно взять независящей от углов, и просто подогнать коэффициенты чтобы выглядело красиво
\pause
\item Если число шагов интегрирования будет маленьким, будут артефакты, -- ничего страшного, лишь бы не 3 FPS :)
\end{itemize}
\end{frame}

\begin{frame}[fragile]
\frametitle{Туман}
\slideimage{fog.png}
\end{frame}

\begin{frame}[fragile]
\frametitle{Объёмные тени}
\slideimage{volumetric.png}
\end{frame}

\begin{frame}[fragile]
\frametitle{Баллы}
\begin{itemize}
\item \textbf{2 балла}: геометрия сцены загружается и рисуется, движется камера
\item \textbf{1 балл}: есть источники света, и один имеет яркость больше единицы
\item \textbf{1 балл}: текстуры альбедо (и только они!) загружаются как sRGB
\item \textbf{2 балла}: есть normal mapping с текстурами нормалей
\item либо \textbf{3 балла}: есть normal mapping с bump-текстурами
\item \textbf{4 балла}: есть cubemap-отражения хотя бы на одном движущемся объекте
\item \textbf{1 балла}: есть простой туман
\item \textbf{3 балла}: есть объёмные тени
\end{itemize}
Всего: \textbf{15 баллов}

Защита заданий на практике \textbf{9 декабря}
\end{frame}

\end{document}