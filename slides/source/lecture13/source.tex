% (c) Nikita Lisitsa, lisyarus@gmail.com, 2021

\documentclass{beamer}

\usepackage[T2A]{fontenc}
\usepackage[utf8]{inputenc}
\usepackage[russian]{babel}

\usepackage{graphicx}
\graphicspath{ {./images/} }

\usepackage{adjustbox}

\usepackage{color}
\usepackage{soul}

\usepackage{hyperref}

\usepackage{amsmath}

\usepackage{tikz}
\usetikzlibrary{decorations}
\usetikzlibrary{decorations.pathreplacing}
\usepackage{xifthen}

\definecolor{red}{rgb}{1,0,0}
\definecolor{green}{rgb}{0,0.5,0}
\definecolor{blue}{rgb}{0,0,1}
\definecolor{magenta}{rgb}{0.75,0,0.75}

\makeatletter
\newcommand{\slideimage}[1]{
  \begin{figure}
    \begin{adjustbox}{width=\textwidth, totalheight=\textheight-2\baselineskip-2\baselineskip,keepaspectratio}
      \includegraphics{#1}
    \end{adjustbox}
  \end{figure}
}
\makeatother

\title{Компьютерная графика}
\subtitle{Лекция 13: состояние OpenGL (напоминание), матрицы проекций (напоминание), рендеринг в cubemap, дистрибуция приложений на OpenGL}
\date{2021}

\setbeamertemplate{footline}[frame number]

\begin{document}

\frame{\titlepage}

\begin{frame}[fragile]
\frametitle{Настройка графического конвейера}
\begin{itemize}
\item Графический конвейер (pipeline) - набор всех операций, происходящих с данными от момента вызова \verb|glDraw*| до появления пикселей на экране (или текстуре/рендербуфере)
\pause
\item Графический конвейер = programmable pipeline + fixed-function pipeline
\pause
\item Настройка programmable pipeline: шейдеры (шейдерная программа)
\pause
\item Настройка fixed-function pipeline: включение/выключение (\verb|glEnable/glDisable|) конкретных операций и их специфическая настройка
\end{itemize}
\end{frame}

\begin{frame}[fragile]
\frametitle{Настройка fixed-function pipeline}
\begin{itemize}
\item Depth clamp
\begin{itemize}
\item По умолчанию, все примитивы обрезаются по уравнению \begin{math}z \leq |w|\end{math}
\item Можно заменить обрезание clamping'ом через \verb|glEnable(GL_DEPTH_CLAMP)|
\end{itemize}
\pause
\item Culling
\begin{itemize}
\item Можно не рисовать back-facing или front-facing полигоны
\item Включить: \verb|glEnable(GL_CULL_FACE)|
\item Настроить, что \textbf{не} рисуется: \verb|glCullFace|
\item Настроить, что считается back-facing, а что front-facing: \verb|glFrontFace|
\end{itemize}
\pause
\item Viewport
\begin{itemize}
\item Настроить перевод из NDC (normalized device coordinates, [-1..1]) в пиксельные координаты: \verb|glViewport|
\item Обычно нужно делать каждый раз при изменении размеров окна или при переключении фреймбуферов
\end{itemize}
\end{itemize}
\end{frame}

\begin{frame}[fragile]
\frametitle{Настройка fixed-function pipeline}
\begin{itemize}
\item Depth test
\begin{itemize}
\item Можно не рисовать пиксели, находящиеся сзади уже нарисованных пикселей
\item Включить: \verb|glEnable(GL_DEPTH_TEST)|
\item Настроить: \verb|glDepthFunc|
\item Настроить преобразование из NDC в [0, 1]: \verb|glDepthRangef|
\item Включить/выключить запись значений глубины: \verb|glDepthMask|
\end{itemize}
\pause
\item Stencil test
\begin{itemize}
\item Включить: \verb|glEnable(GL_STENCIL_TEST)|
\item Настроить: \verb|glStencilFunc|, \verb|glStencilOp|, \verb|glStencilMask|
\end{itemize}
\pause
\item Scissor test
\begin{itemize}
\item Можно не рисовать пиксели, находящиеся вне некоторого прямоугольника
\item Включить: \verb|glEnable(GL_SCISSOR_TEST)|
\item Настроить: \verb|glScissor|
\end{itemize}
\end{itemize}
\end{frame}

\begin{frame}[fragile]
\frametitle{Настройка fixed-function pipeline}
\begin{itemize}
\item Color mask
\begin{itemize}
\item Настроить запись в конкретные цветовые каналы: \verb|glColorMask|
\end{itemize}
\pause
\item Blending
\begin{itemize}
\item Можно записывать значение некоторой функции от входного цвета и уже записанного цвета
\item Включить: \verb|glEnable(GL_BLEND)|
\item Настроить: \verb|glBlendFunc|/\verb|glBlendFuncSeparate|, \verb|glBlendEquation|, \verb|glBlendColor|
\end{itemize}
\pause
\item Color logical operation
\begin{itemize}
\item Можно записывать результат некоторой побитовой операции от входного цвета и уже записанного цвета
\item Выключает blending
\item Включить: \verb|glEnable(GL_COLOR_LOGIC_OP)|
\item Настроить: \verb|glLogicOp|
\end{itemize}
\end{itemize}
\end{frame}

\begin{frame}[fragile]
\frametitle{Проекции}
\begin{itemize}
\item Ничто не мешает делать с вершинами абсолютно любые преобразования в вершинном шейдере
\pause
\item Обычно используют ортографическую или перспективную проекцию, так как они выражаются матрицами (с учётом perspective divide)
\pause
\item Ортографическая проекция: \textbf{не использует} perspective divide, последняя строка равна (0 0 0 1)
\pause
\item Перспективная проекция: \textbf{использует} perspective divide, последняя строка содержит зависимость от X, Y или Z
\end{itemize}
\end{frame}

\begin{frame}[fragile]
\frametitle{Ортографическая проекция}
\fontsize{10pt}{10pt}
\begin{itemize}
\item Проецирует параллельно некому вектору: точки пространства, отличающиеся на вектор, параллельный направлению взгляда, проецируются в одну точку экрана
\pause
\item Видимая область пространства - параллелепипед
\pause
\item Видимый размер объектов \textbf{не зависит} от расстояния до них
\pause
\item Удобно описать центром видимой области \begin{math}C\end{math} и осями \begin{math}X, Y, Z\end{math}
\pause
\begin{itemize}
\item Центр \begin{math}C\end{math} переходит в центр экрана ((0, 0, 0) в NDC)
\pause
\item Точки, отличающиеся на вектор параллельный \begin{math}X\end{math}, после проекции отличаются только X-координатой
\pause
\item Точки, отличающиеся на вектор параллельный \begin{math}Y\end{math}, после проекции отличаются только Y-координатой
\pause
\item Точки, отличающиеся на вектор параллельный \begin{math}Z\end{math}, после проекции попадают в один пиксель (и отличаются только Z-координатой)
\end{itemize}
\pause
\item N.B.: можно понимать её как композицию аффинного преобразования, двигающего камеру в точку C с осями XYZ, и стандартной ортографической проекции (с единичной матрицей)
\pause
\item Матрица проекции: см. лекцию 4
\end{itemize}
\end{frame}

\begin{frame}[fragile]
\frametitle{Перспективная проекция}
\begin{itemize}
\item Проецирует из некой точки: точки пространства, лежащие на одной прямой с центром проекции, проецируются в одну точку экрана
\pause
\item Видимая область пространства - усечённая пирамида
\pause
\item Видимый размер объектов \textbf{зависит} от расстояния до них: чем объект дальше, тем он мельче
\pause
\item Удобно описать параметрами проекции near, far, fovx, fovy и аффинными преобразованием, двигающим камеру (как с ортографической проекцией)
\pause
\item Обычно за направление взгляда берут ось -Z, чтобы получить левую систему координат
\pause
\begin{itemize}
\item near: всё ближе этого значения (по Z-координате) будет отсекаться
\pause
\item far: всё дальше этого значения (по Z-координате) будет отсекаться
\pause
\item fovx, fovy - угол обзора по X и Y
\end{itemize}
\pause
\item Матрица проекции: см. лекцию 4
\end{itemize}
\end{frame}

\begin{frame}[fragile]
\frametitle{Cubemaps}
\begin{itemize}
\item Cubemap-текстура: (концептуально) набор из шести 2D текстур, понимаемых как грани \textit{виртуального} куба
\pause
\item Удобно хранить изображения, натянутые на куб/сферу
\pause
\item Текстурная координата в шейдере - вектор направления; вычисляется пересечение луча из центра куба в этом направлении с поверхностью куба (не зависит от длины вектора направления и от размеров куба), по этому пересечению вычисляется пиксель в конкретной грани cubemap текстуры
\end{itemize}
\end{frame}

\begin{frame}[fragile]
\frametitle{Cubemaps: примеры использования}
\begin{itemize}
\item Хочется нарисовать некое бесконечно удалённое окружение сцены (skybox) - небо, далёкие горы, и т.п.: нужна текстура для всех направлений \begin{math}\Rightarrow\end{math} cubemaps!
\pause
\item Environment mapping (как skybox, только для отражений) - нужно знать, как выглядит сцена в некотором направлении из отражающей точки \begin{math}\Rightarrow\end{math} cubemaps!
\pause
\item Отражения (как environment mapping, только динамический) \begin{math}\Rightarrow\end{math} cubemaps!
\pause
\item Shadow maps от точечного источника - источник светит во все стороны из фиксированной точки: нужно знать расстояние до ближайшего объекта в каждом направлении \begin{math}\Rightarrow\end{math} cubemaps!
\end{itemize}
\end{frame}

\begin{frame}[fragile]
\frametitle{Рендеринг в cubemap}
\begin{itemize}
\item Cubemap - это текстура \begin{math}\Rightarrow\end{math} чтобы рисовать в неё, нужен фреймбуфер
\pause
\item Cubemap целиком нельзя сделать attachment'ом фреймбуфера, можно только отдельные её грани
\pause
\item \verb|glFramebufferTexture2D| - принимает (в отличие от \verb|glFramebufferTexture|) дополнительный параметр \verb|textarget|, который можно (в том числе) использовать как индикатор конкретной грани cubemap текстуры:
\begin{itemize}
\item \verb|GL_TEXTURE_CUBE_MAP_POSITIVE_X|
\item \verb|GL_TEXTURE_CUBE_MAP_NEGATIVE_X|
\item \verb|GL_TEXTURE_CUBE_MAP_POSITIVE_Y|
\item \verb|GL_TEXTURE_CUBE_MAP_NEGATIVE_Y|
\item \verb|GL_TEXTURE_CUBE_MAP_POSITIVE_Z|
\item \verb|GL_TEXTURE_CUBE_MAP_NEGATIVE_Z|
\end{itemize}
\end{itemize}
\end{frame}

\begin{frame}[fragile]
\frametitle{Рендеринг в cubemap}
\begin{itemize}
\item Нужно 6 фреймбуферов, к каждому нужно присоединить соответствующую грань cubemap текстуры
\pause
\item Рисуем сцену 6 раз, по одному разу для каждой грани
\pause
\item Для каждой грани нужна перспективная проекция
\begin{itemize}
\item Центр - точка, с точки зрения которой рисуется cubemap (для теней - позиция источника света, для отражений - координаты отражающей точки)
\pause
\item fovx = fovy = \begin{math}90^\circ\end{math} (из геометрии куба)
\end{itemize}
\end{itemize}
\end{frame}

\begin{frame}[fragile]
\frametitle{Рендеринг в cubemap: псевдокод}
\fontsize{10pt}{10pt}
\begin{verbatim}
// Инициализация:
GLuint fbos[6];
... 
for (i in 0..5) {
  glBindFramebuffer(GL_DRAW_FRAMEBUFFER, fbos[i]);
  glFramebufferTexture2D(GL_DRAW_FRAMEBUFFER,
    GL_COLOR_ATTACHMENT0, 
    GL_TEXTURE_CUBE_MAP_POSITIVE_X + i, cubemap, 0);
}

// Рендеринг:
...
glViewport(0, 0, cubemap_size, cubemap_size);
set_uniform("projection", cubemap_projection);
for (i in 0..5) {
  glBindFramebuffer(GL_DRAW_FRAMEBUFFER, fbos[i]);
  set_uniform("view", cubemap_view[i]);
  draw_scene();
}
\end{verbatim}
\end{frame}

\begin{frame}[fragile]
\frametitle{Дистрибуция программ на C++}
\begin{itemize}
\item Чтобы программа запустилась на конкретном компьютере, нужно, чтобы:
\pause
\begin{itemize}
\item Операционная система понимала формат исполняемого файла
\pause
\item Бинарный код подходил для используемого процессора и операционной системы
\pause
\item Присутствовали все необходимые динамические библиотеки
\end{itemize}
\end{itemize}
\end{frame}

\begin{frame}[fragile]
\frametitle{Форматы исполняемых файлов}
\begin{itemize}
\item Windows: PE (Portable Executable)
\item Linux: ELF (Executable and Linkable Format)
\item MacOS: Mach-O (Mach Object)
\end{itemize}
\end{frame}

\begin{frame}[fragile]
\frametitle{Форматы исполняемых файлов}
\begin{itemize}
\item Содержат:
\pause
\begin{itemize}
\item Magic number для идентификации формата
\pause
\item Метаинформация: битность (32/64), CPU (набор инструкций - ISA), ABI, endianess, etc
\pause
\item Блоки с данными и кодом
\pause
\item Таблицы релокации (для функций из динамических библиотек)
\pause
\item Список зависимостей - динамических библиотек
\end{itemize}
\end{itemize}
\end{frame}

\begin{frame}[fragile]
\frametitle{Форматы исполняемых файлов}
\begin{itemize}
\item Обычно компилятор по умолчанию генерирует исполняемый файл для системы, на которой происходит сборка (host)
\pause
\item Кросс-компиляция: компиляция программы, предназначенной для запуска на системе (target), отличной от той, на которой происходит сборка (host)
\pause
\item Кросс-компиляция с Linux (host) на Windows (target): MinGW
\pause
\item Кросс-компиляция с Windows (host) на Linux (target): Cygwin + crosstool-ng
\pause
\item Кросс-компиляция на MacOS (target): нет
\begin{itemize}
\item Apple требует, чтобы программы для MacOS собирались на MacOS, на железе от Apple, и были подписаны платными сертификатами разработчика
\end{itemize}
\end{itemize}
\end{frame}

\begin{frame}[fragile]
\frametitle{CPU}
\begin{itemize}
\item Скомпилированный код - байт-код для исполнения процессором
\pause
\item Если конкретный процессор не понимает этот байт-код, программу не запустить
\pause
\item Обычно линейки процессоров сохраняют обратную совместимость: программу для старого процессора можно запустить на более новом процессоре
\pause
\item Компиляторы позволяют настраивать то, под какой процессор генерируется код, e.g. для GCC: \verb|gcc -march=haswell ...| (полный список для GCC: \href{https://gcc.gnu.org/onlinedocs/gcc/x86-Options.html}{gcc.gnu.org/onlinedocs/gcc/x86-Options.html})
\end{itemize}
\end{frame}

\begin{frame}[fragile]
\frametitle{ABI}
\begin{itemize}
\item Скомпилированный код содержит вызовы внешних функций (системные вызовы и функции из динамически загруженных библиотек)
\pause
\item На уровне ассемблера нет понятия вызова функции \begin{math}\Rightarrow\end{math} нужны соглашения о вызове (calling conventions) - как передаются аргументы в функции, как возвращается результат, как передаётся параметр \verb|this|, и т.п.
\pause
\item Скомпилированный код содержит работу со структурами данных
\pause
\item На уровне ассемблера нет структур данных \begin{math}\Rightarrow\end{math} нужно соглашение о том, как выглядят структуры в памяти (memory layout)
\pause
\item (для C++) нужно соглашение о деталях обработки исключений, разрешения namespace'ов и перегрузки функций
\pause
\item За всё это отвечает ABI (Application Binary Interface)
\end{itemize}
\end{frame}

\begin{frame}[fragile]
\frametitle{ABI}
\begin{itemize}
\item GCC: начиная с 3ей версии использует стандартизованный Itanium ABI
\pause
\item MSVC: начиная с Visual Studio 2015 сохраняет ABI-совместимость кода, скомпилированного разными версиями MSVC
\pause
\item Clang: почти ABI-совместим с GCC; умеет генерировать код, ABI-совместимый с MSVC
\pause
\item Код может быть ABI-совместим с другим кодом с точки зрения C, но не с точки зрения C++
\end{itemize}
\end{frame}

\begin{frame}[fragile]
\frametitle{Динамические библиотеки}
\begin{itemize}
\item Стандартная библиотека C (\verb|libc.so| в Linux, \verb|msvcrt.dll| в Windows) и C++ (\verb|libstdc++.so| у GCC, \verb|libc++.so| у Clang, \verb|msvcp140.dll| и т.п. у Visual Studio, etc.)
\pause
\item Библиотека математических функций (\verb|libm.so|)
\pause
\item Библиотека с реализацией OpenGL (\verb|libGL.so| в Linux, \verb|opengl32.dll| в Windows)
\pause
\item Вспомогательные библиотеки (SDL, загрузчик OpenGL, etc)
\end{itemize}
\end{frame}

\begin{frame}[fragile]
\frametitle{Динамические библиотеки}
\begin{itemize}
\item Некоторые библиотеки можно слинковать статически - их код будет влит в код исполняемого файла, и при исполнении они уже не нужны
\pause
\begin{itemize}
\item Стандартная библиотека C++
\pause
\item Загрузчик OpenGL
\end{itemize}
\pause
\item Остальные нужно где-то найти!
\end{itemize}
\end{frame}

\begin{frame}[fragile]
\frametitle{Динамические библиотеки}
\begin{itemize}
\item Многие библиотеки (стандартная библиотека C, математические функции, реализация OpenGL) есть в любой системе и лежат по каким-то стандартным путям (\verb|/usr/lib| в Linux, \verb|C:\Program Files| в Windows)
\pause
\item Поиск дополнительных зависимостей специфичен для системы:
\pause
\begin{itemize}
\item Windows: если библиотеки не найдены в системных путях, они ищутся в директории с исполняемым файлом
\pause
\item Linux, MacOS: у исполняемых файлов есть свойство RPATH (runtime path) - список путей, по которым нужно искать зависимости; в качестве RPATH можно указать текущую директорию (\verb|.| для Linux, специальные макросы для MacOS)
\end{itemize}
\pause
\item С таким подходом зависимости можно поставлять в собранном виде рядом с исполняемым файлом
\end{itemize}
\end{frame}

\end{document}