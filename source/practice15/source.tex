% (c) Nikita Lisitsa, lisyarus@gmail.com, 2022

\documentclass{beamer}

\usepackage[T2A]{fontenc}
\usepackage[utf8]{inputenc}
\usepackage[russian]{babel}
\usepackage{minted}

\usepackage{graphicx}
\graphicspath{ {./images/} }

\usepackage{adjustbox}

\usetheme{metropolis}

\usepackage{color}
\usepackage{soul}

\usepackage{hyperref}

\definecolor{blue}{rgb}{0,0,1}
\definecolor{red}{rgb}{1,0,0}
\definecolor{codebg}{RGB}{29,35,49}

\makeatletter
\newcommand{\slideimage}[1]{
  \begin{figure}
    \begin{adjustbox}{width=\textwidth, totalheight=\textheight-2\baselineskip-2\baselineskip,keepaspectratio}
      \includegraphics{#1}
    \end{adjustbox}
  \end{figure}
}
\makeatother

\title{Компьютерная графика}
\subtitle{Практика 15: MSDF-текст}
\date{2022}

\usemintedstyle{solarized-light}

\setbeamertemplate{footline}[frame number]

\begin{document}

\frame{\titlepage}

\begin{frame}
\slideimage{0.png}
\end{frame}

\begin{frame}[fragile]
\begin{itemize}
\item В этой практике есть текст \mintinline{cpp}|"Hello, world!"| по умолчанию, но можно его стирать и печатать свой прямо в программе :)
\end{itemize}
\end{frame}

\begin{frame}[fragile]
\frametitle{Задание 1}
Рисуем треугольник
\begin{itemize}
\item Заводим структуру вершины с полями \mintinline{glsl}|vec2 position| и \mintinline{glsl}|vec2 texcoord|
\item Заводим VAO + VBO для них, настраиваем атрибуты
\item Используем атрибуты в шейдерной программе: \mintinline{glsl}|gl_Position = vec4(position, 0.0, 1.0)|, в качестве цвета выводим \mintinline{glsl}|vec4(texcoord, 0.0, 1.0)|
\item Инициализируем VBO данными при создании: треугольник с координатами \mintinline{glsl}|(0,0)|, \mintinline{glsl}|(100,0)| и \mintinline{glsl}|(0,100)| и текстурными координатами \mintinline{glsl}|(0,0)|, \mintinline{glsl}|(1,0)| и \mintinline{glsl}|(0,1)|
\item Рисуем этот треугольник (коодинаты сильно выходят за диапазон \mintinline{glsl}|[-1..1]|, так что мы увидим только угол треугольника)
\end{itemize}
\end{frame}

\begin{frame}
\frametitle{Задание 1}
\slideimage{1.png}
\end{frame}

\begin{frame}[fragile]
\frametitle{Задание 2}
Настраиваем матрицу
\begin{itemize}
\item Заводим матрицу \mintinline{glsl}|transform|, которая переводит из экранных координат в OpenGL-ные: \begin{math}[0, width] \times [height, 0] \mapsto [-1, 1]\times[-1,1]\end{math}
\item Передаём эту матрицу в качестве значения uniform-переменной \mintinline{glsl}|transform|
\item В шейдере применяем к вершине матрицу \mintinline{glsl}|transform| (сама матрица там уже есть)
\item N.B. экранная Y-координата идёт \textbf{сверху вниз}, а OpenGL-ная -- \textbf{снизу вверх}!
\item N.B. треугольник должен появиться в верхнем левом углу и быть размером ровно 100 пикселей
\end{itemize}
\end{frame}

\begin{frame}
\frametitle{Задание 2}
\slideimage{2.png}
\end{frame}

\begin{frame}[fragile]
\frametitle{Задание 3}
\fontsize{8pt}{8pt}
\selectfont
Генерируем глифы
\begin{itemize}
\item Убираем заполнение VBO на старте
\item Вместо этого в цикле рисования в случае, если флаг \mintinline{cpp}|text_changed| имеет значение \mintinline{cpp}|true|, генерируем новый массив вершин:
\begin{itemize}
\fontsize{8pt}{8pt}
\selectfont
\item Заводим координаты `пера' \mintinline{cpp}|vec2 pen(0.0)|, это точка отсчёта для текущего символа
\item Проходимся по всем буквам переменной \mintinline{cpp}|text|, находим соответствующий глиф в \mintinline{cpp}|font.glyphs|
\item Для каждой буквы генерируем прямоугольник из 4 вершин (разбивая вручную на 2 треугольника, т.е. в итоге 6 вершин)
\item Координаты вершины -- \mintinline{cpp}|[glyph.xoffset .. glyph.xoffset + glyph.width] + pen.x|, аналогично для Y
\item Текстурные координаты вершины -- \mintinline{cpp}|[glyph.x .. glyph.x + glyph.width] / texture_width|, аналогично для Y
\item После каждого символа нужно сдвинуть перо по X на \mintinline{cpp}|glyph.advance|
\end{itemize}
\item Загружаем эти вершины в VBO, запоминаем количество вершин, и очищаем флаг \mintinline{cpp}|text_changed|
\end{itemize}
\end{frame}

\begin{frame}
\frametitle{Задание 3}
\slideimage{3.png}
\end{frame}

\begin{frame}[fragile]
\frametitle{Задание 4}
Выводим MSDF-глифы
\begin{itemize}
\item Передаём значение \mintinline{cpp}|font.sdf_scale| в новую uniform-переменную \mintinline{glsl}|float sdf_scale|
\item Заводим uniform-переменную для текстуры шрифта \mintinline{glsl}|sampler2D sdf_texture| (она уже выставлена для нулевого texture unit'а, ничего дополнительно делать не нужно)
\item Выводим буквы чёрного цвета с прозрачностью, посчитанной через SDF (см. слайды лекции)
\end{itemize}
\end{frame}

\begin{frame}
\frametitle{Задание 4}
\slideimage{4.png}
\end{frame}

\begin{frame}[fragile]
\frametitle{Задание 5}
Центрируем текст
\begin{itemize}
\item При обновлении текста вычисляем bounding box всех вершин (т.е. максимальные и минимальные X и Y координаты)
\item Дополняем матрицу \mintinline{cpp}|transform| так, чтобы центр текста был в центре экрана
\end{itemize}
\end{frame}

\begin{frame}
\frametitle{Задание 5}
\slideimage{5.png}
\end{frame}

\begin{frame}[fragile]
\frametitle{Задание 6}
Увеличиваем текст
\begin{itemize}
\item Дополняем матрицу \mintinline{cpp}|transform| так, чтобы буквы стали больше (примерно в 5-6 раз, не принципиально)
\item Сглаживание на границе букв не учитывает растяжение и будет размытым
\item Чтобы сделать чёткое сглаживание, вместо значения \mintinline{glsl}|0.5| в функции \mintinline{glsl}|smoothstep| используем величину
\begin{minted}[bgcolor=codebg]{glsl}
length(vec2(dFdx(sdfValue), dFdy(sdfValue)))
    / sqrt(2.0)
\end{minted}

\end{itemize}
\end{frame}

\begin{frame}
\frametitle{Задание 6}
\slideimage{6.png}
\end{frame}

\begin{frame}[fragile]
\frametitle{Задание 7*}
Добавляем обводку текста
\begin{itemize}
\item Дорабатываем шейдер, чтобы у чёрного текста появилась белая обводка
\item Параметры подберите на свой вкус, главное -- чтобы при изменении размеров экрана всё ещё выглядело красиво :)
\end{itemize}
\end{frame}

\begin{frame}
\frametitle{Задание 7*}
\slideimage{7.png}
\end{frame}

\end{document}