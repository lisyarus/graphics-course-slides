% (c) Nikita Lisitsa, lisyarus@gmail.com, 2021

\documentclass{beamer}

\usepackage[T2A]{fontenc}
\usepackage[utf8]{inputenc}
\usepackage[russian]{babel}

\usepackage{graphicx}
\graphicspath{ {./images/} }

\usepackage{adjustbox}

\usepackage{color}
\usepackage{soul}

\usepackage{hyperref}

\definecolor{blue}{rgb}{0,0,1}
\definecolor{red}{rgb}{1,0,0}

\makeatletter
\newcommand{\slideimage}[1]{
  \begin{figure}
    \begin{adjustbox}{width=\textwidth, totalheight=\textheight-2\baselineskip-2\baselineskip,keepaspectratio}
      \includegraphics{#1}
    \end{adjustbox}
  \end{figure}
}
\makeatother

\title{Компьютерная графика}
\subtitle{Лекция 4: Камера, проекции, буфер глубины}
\date{2021}

\setbeamertemplate{footline}[frame number]

\begin{document}

\frame{\titlepage}

\begin{frame}[fragile]
\frametitle{Проекции}
\begin{itemize}
\item Мир - двухмерный/трёхмерный, произвольного размера, с произвольного ракурса
\pause
\item Экран - двухмерный, \begin{math}[-1, 1] \times [-1, 1]\end{math}
\pause
\item За преобразование отвечает проекция
\end{itemize}
\end{frame}

\begin{frame}[fragile]
\frametitle{Ортографическая проекция}
\begin{itemize}
\item Самый простой способ: \pause выкинуть третью координату
\begin{center}\begin{math}(x, y, z) \mapsto (x, y)\end{math}\end{center}
\begin{center}
\begin{math}
\begin{pmatrix}
1 & 0 & 0 & 0 \\
0 & 1 & 0 & 0 \\
0 & 0 & 1 & 0 \\
0 & 0 & 0 & 1
\end{pmatrix}
\end{math}
\end{center}
\pause
\begin{itemize}
\item X и Y всё ещё \begin{math}[-1, 1]\end{math}
\end{itemize}
\pause
\item Как сделать \begin{math}X \in [-W, W]\end{math} и \begin{math}Y \in [-H, H]\end{math}?
\pause
\begin{center}
\begin{math}
\begin{pmatrix}
\frac{1}{W} & 0 & 0 & 0 \\
0 & \frac{1}{H} & 0 & 0 \\
0 & 0 & 1 & 0 \\
0 & 0 & 0 & 1
\end{pmatrix}
\end{math}
\end{center}
\end{itemize}
\end{frame}

\begin{frame}[fragile]
\frametitle{Ортографическая проекция}
\begin{itemize}
\item Как сделать \begin{math}X \in [X_0 - W, X_0 + W]\end{math} и \begin{math}Y \in [Y_c - H, Y_c + H]\end{math}?
\pause
\begin{itemize}
\item Сдвинуть на \begin{math}(-X_0, -Y_0)\end{math}, а затем применить матрицу с предыдущего слайда:
\end{itemize}
\begin{center}
\begin{math}
\begin{pmatrix}
\frac{1}{W} & 0 & 0 & 0 \\
0 & \frac{1}{H} & 0 & 0 \\
0 & 0 & 1 & 0 \\
0 & 0 & 0 & 1
\end{pmatrix}
\cdot
\begin{pmatrix}
1 & 0 & 0 & -X_0 \\
0 & 1 & 0 & -Y_0 \\
0 & 0 & 1 & 0 \\
0 & 0 & 0 & 1
\end{pmatrix}
\end{math}
\end{center}
\pause
\item Если размер области \begin{math}W\end{math}, нужно разделить на \begin{math}W\end{math}
\item Если центр в \begin{math}X_0\end{math}, нужно сдвинуть на \begin{math}-X_0\end{math}
\pause
\item Общая идея: если камера получена каким-то преобразованием, нужно применить обратное преобразование
\end{itemize}
\end{frame}

\begin{frame}[fragile]
\frametitle{Ортографическая проекция: общий случай}
\begin{itemize}
\item В общем случае ортографическую камеру можно задать
\pause
\begin{itemize}
\item Положением камеры \begin{math}C = (C_x, C_y, C_z)\end{math}
\pause
\item Осями координат камеры \begin{math}X = (X_x, X_y, X_z), Y = (Y_x, Y_y, Y_z), Z = (Z_x, Z_y, Z_z)\end{math}
\end{itemize}
\pause
\item Делает проекцию параллельно вектору \begin{math}Z\end{math} на параллелограмм \begin{math}C \pm X \pm Y\end{math}
\item Параллелограмм отождествляется с экраном
\pause
\item Обычно \begin{math}X, Y, Z\end{math} взаимно ортогональны
\pause
\item Как выразить эту проекцию матрицей?
\pause
\item Преобразование из стандартной системы координат OpenGL \begin{math}[-1, 1]^3\end{math} в координаты области, видимой через эту камеру: смена системы координат + параллельный перенос
\begin{center}
\begin{math}
\begin{pmatrix}
X_x & Y_x & Z_x & C_x \\
X_y & Y_y & Z_y & C_y \\
X_z & Y_z & Z_z & C_z \\
0 & 0 & 0 & 1
\end{pmatrix}
\end{math}
\end{center}
\pause
\item Обратное преобразование (проекция) - обратная матрица
\end{itemize}
\end{frame}

\begin{frame}[fragile]
\frametitle{Ортографическая проекция: общий случай}
\begin{itemize}
\item Можно представить \begin{math}X, Y, Z\end{math} как произведение длины и нормированного вектора:
\begin{center}
\begin{math}X = W \cdot \hat X \quad Y = H \cdot \hat Y \quad Z = D \cdot \hat Z\end{math}
\end{center}
\pause
\item Матрицу можно разбить на перенос, масштабирование и поворот
\begin{center}
\begin{math}
\begin{pmatrix}
X_x & Y_x & Z_x & C_x \\
X_y & Y_y & Z_y & C_y \\
X_z & Y_z & Z_z & C_z \\
0 & 0 & 0 & 1
\end{pmatrix}
=
\begin{pmatrix}
1 & 0 & 0 & C_x \\
0 & 1 & 0 & C_y \\
0 & 0 & 1 & C_z \\
0 & 0 & 0 & 1
\end{pmatrix}
\cdot
\begin{pmatrix}
\hat X_x & \hat Y_x & \hat Z_x & 0 \\
\hat X_y & \hat Y_y & \hat Z_y & 0 \\
\hat X_z & \hat Y_z & \hat Z_z & 0 \\
0 & 0 & 0 & 1
\end{pmatrix}
\cdot
\begin{pmatrix}
W & 0 & 0 & 0 \\
0 & H & 0 & 0 \\
0 & 0 & D & 0 \\
0 & 0 & 0 & 1
\end{pmatrix}
\end{math}
\end{center}
\end{itemize}
\end{frame}

\begin{frame}[fragile]
\frametitle{Ортографическая проекция: общий случай}
\begin{itemize}
\item Тогда обратная матрица (матрица проекции):
\begin{center}
\begin{math}
\begin{pmatrix}
X_x & Y_x & Z_x & C_x \\
X_y & Y_y & Z_y & C_y \\
X_z & Y_z & Z_z & C_z \\
0 & 0 & 0 & 1
\end{pmatrix}^{-1}
=
\begin{pmatrix}
\frac{1}{W} & 0 & 0 & 0 \\
0 & \frac{1}{H} & 0 & 0 \\
0 & 0 & \frac{1}{D} & 0 \\
0 & 0 & 0 & 1
\end{pmatrix}
\cdot
\begin{pmatrix}
\hat X_x & \hat Y_x & \hat Z_x & 0 \\
\hat X_y & \hat Y_y & \hat Z_y & 0 \\
\hat X_z & \hat Y_z & \hat Z_z & 0 \\
0 & 0 & 0 & 1
\end{pmatrix}^{-1}
\cdot
\begin{pmatrix}
1 & 0 & 0 & -C_x \\
0 & 1 & 0 & -C_y \\
0 & 0 & 1 & -C_z \\
0 & 0 & 0 & 1
\end{pmatrix}
\end{math}
\end{center}
\end{itemize}
\end{frame}

\begin{frame}[fragile]
\frametitle{Ортографическая проекция: ортогональный случай}
\begin{itemize}
\item Если \begin{math}X, Y, Z\end{math} ортогональны, то 
\begin{center}
\begin{math}
\begin{pmatrix}
\hat X_x & \hat Y_x & \hat Z_x & 0 \\
\hat X_y & \hat Y_y & \hat Z_y & 0 \\
\hat X_z & \hat Y_z & \hat Z_z & 0 \\
0 & 0 & 0 & 1
\end{pmatrix}^{-1}
=
\begin{pmatrix}
\hat X_x & \hat Y_x & \hat Z_x & 0 \\
\hat X_y & \hat Y_y & \hat Z_y & 0 \\
\hat X_z & \hat Y_z & \hat Z_z & 0 \\
0 & 0 & 0 & 1
\end{pmatrix}^T
=
\begin{pmatrix}
\hat X_x & \hat X_y & \hat X_z & 0 \\
\hat Y_x & \hat Y_y & \hat Y_z & 0 \\
\hat Z_x & \hat Z_y & \hat Z_z & 0 \\
0 & 0 & 0 & 1
\end{pmatrix}
\end{math}
\end{center}
\end{itemize}
\end{frame}

\end{document}