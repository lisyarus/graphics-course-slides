% (c) Nikita Lisitsa, lisyarus@gmail.com, 2022

\documentclass{beamer}

\usepackage[T2A]{fontenc}
\usepackage[utf8]{inputenc}
\usepackage[russian]{babel}
\usepackage{minted}

\usepackage{graphicx}
\graphicspath{ {./images/} }

\usepackage{adjustbox}

\usetheme{metropolis}

\usepackage{color}
\usepackage{soul}

\usepackage{hyperref}

\definecolor{blue}{rgb}{0,0,1}
\definecolor{red}{rgb}{1,0,0}

\makeatletter
\newcommand{\slideimage}[1]{
  \begin{figure}
    \begin{adjustbox}{width=\textwidth, totalheight=\textheight-2\baselineskip-2\baselineskip,keepaspectratio}
      \includegraphics{#1}
    \end{adjustbox}
  \end{figure}
}
\makeatother

\title{Компьютерная графика}
\subtitle{Практика 14: Timer queries, instancing, frustum culling, LOD}
\date{2021}

\usemintedstyle{solarized-light}

\setbeamertemplate{footline}[frame number]

\begin{document}

\frame{\titlepage}

\begin{frame}
\slideimage{0.png}
\end{frame}

\begin{frame}[fragile]
\begin{itemize}
\item В этой практике имеет смысл делать Release сборку проекта
\end{itemize}
\end{frame}

\begin{frame}[fragile]
\fontsize{8pt}{8pt}
\frametitle{Задание 1}
Замеряем время рисования кадра с помощью timer queries
\begin{itemize}
\item Заводим \mintinline{cpp}|std::vector| для ID query объектов и ещё один для запоминания, свободен объект или нет
\item В начале каждого кадра находим индекс первого свободного query объекта, или, если такого нет, создаём новый и добавляем в массив (и помечаем как свободный)
\item Помечаем выбранный query объект как занятый, и вставляем \mintinline{cpp}|glBeginQuery(GL_TIME_ELAPSED, id)| в начало кадра (перед \mintinline{cpp}|glClear|) и соответствующий \mintinline{cpp}|glEndQuery| в конец кадра (перед \mintinline{cpp}|SwapBuffers|)
\item В конце кадра проверяем каждый query объект на то, готовы ли его данные: если готовы -- достаём их и логируем (лучше разделить на 10\textsuperscript{6} или 10\textsuperscript{9} чтобы получить миллисекунды или секунды, соответственно; делить нужно во floating-point), и помечаем как свободный
\item N.B.: на некоторых GPU у timer queries очень большая погрешность, и могут получиться даже отрицательные числа
\item В дальнейших заданиях имеет смысл смотреть на это значение и сравнивать с предыдущими заданиями (всё-таки мы занимаемся оптимизацией)
\item В конце программы можно залогировать размер массива \mintinline{cpp}|query| объектов: это примерное отставание (в кадрах) GPU от CPU
\end{itemize}
\end{frame}

\begin{frame}
\frametitle{Задание 1}
\slideimage{0.png}
\end{frame}

\begin{frame}[fragile]
\frametitle{Задание 2}
Рисуем 1024 копии объекта
\begin{itemize}
\item В цикле рисуем копии объекта, например, сеткой, с координатами в диапазоне \mintinline{cpp}|[-16 .. 16)| с шагом в 1 по X и Z
\item Для сдвига можно использовать model-матрицу (она уже есть в коде, нужно только обновлять значение через \mintinline{cpp}|glUniformMatrix4fv|)
\end{itemize}
\end{frame}

\begin{frame}
\frametitle{Задание 2}
\slideimage{2.png}
\end{frame}

\begin{frame}[fragile]
\fontsize{10pt}{10pt}
\selectfont
\frametitle{Задание 3}
Рисуем 1024 копии объекта, используя instancing
\begin{itemize}
\item В вершинном шейдере заводим новый атрибут вершины для позиции instance (\mintinline{glsl}|vec3|, \mintinline{glsl}|location = 3|) и прибавляем к \mintinline{glsl}|in_position| перед применением матриц
\item Заводим массив \mintinline{cpp}|std::vector| сдвигов \mintinline{cpp}|glm::vec3|, использовавшихся в предыдущем задании, и заполняем при старте программы
\item Заводим VBO для этих сдвигов и загружаем в него этот массив (так же, как в обычный VBO)
\item Настраиваем атрибут с \mintinline{cpp}|index = 3|, беря данные из нового VBO, и делаем \mintinline{cpp}|glVertexAttribDivisor(3, 1)| (чтобы этот атрибут менялся один раз на instance, а не на вершину)
\begin{itemize}
\fontsize{10pt}{10pt}
\selectfont
\item В коде есть несколько VAO, настроить нужно каждый!
\item Перед \mintinline{cpp}|glVertexAttribPointer| нужно будет сделать текущим ваш новый VBO
\end{itemize}
\item Вместо цикла по 1024 объектам делаем один вызов \mintinline{cpp}|glDrawElementsInstanced|
\item Model-матрицу меняем обратно на единичную
\end{itemize}
\end{frame}

\begin{frame}
\frametitle{Задание 3}
\slideimage{2.png}
\end{frame}

\begin{frame}[fragile]
\fontsize{10pt}{10pt}
\frametitle{Задание 4}
Добавляем frustum culling
\begin{itemize}
\item Каждый кадр создаём объект \mintinline{cpp}|frustum| (из \mintinline{cpp}|frustum.hpp|), передавая ему матрицу \mintinline{cpp}|projection * view|
\item Вместо задания массива instance на старте, мы пересоздаём его на каждрый кадр
\item В цикле прохода по всем объектам (который теперь снова в цикле рисования) создаём объект типа \mintinline{cpp}|aabb| (из \mintinline{cpp}|aabb.hpp|), используя bbox модели (\mintinline{cpp}|input_model.meshes[0].min/max|) и прибавляя offset текущего объекта
\item Если \mintinline{cpp}|aabb| объекта пересекает \mintinline{cpp}|frustum| (функция \mintinline{cpp}|intersect| из \mintinline{cpp}|intersect.hpp|), добавляем его в массив instance'ов
\item Обновляем VBO с позициями instance'ов
\item Логируем число нарисованных объектов (если вы не меняли начальные параметры камеры, на старте будет 500 объектов)
\end{itemize}
\end{frame}

\begin{frame}
\frametitle{Задание 4}
\slideimage{2.png}
\end{frame}

\begin{frame}[fragile]
\fontsize{10pt}{10pt}
\frametitle{Задание 5*}
Добавляем LOD
\begin{itemize}
\item Вместо одного массива instance заводим 6 штук -- по одному на каждый уровень детализации
\item Для каждого объекта вычисляем его номер LOD (например, как расстояние до камеры, делённое на фиксированное значение) и кладём в соответствующий массив
\item После frustum culling'а делаем цикл по всем LOD \mintinline{cpp}|[0..5]|, где
\begin{itemize}
\item Загружаем массив instance для данного LOD в ваш VBO для данных instance
\item Рисуем модель \mintinline{cpp}|input_model.meshes[lod]| используя VAO \mintinline{cpp}|vaos[lod]|
\end{itemize}
\end{itemize}
\end{frame}

\begin{frame}
\frametitle{Задание 5*}
\slideimage{5.png}
\end{frame}

\end{document}