% (c) Nikita Lisitsa, lisyarus@gmail.com, 2021

\documentclass{beamer}

\usepackage[T2A]{fontenc}
\usepackage[utf8]{inputenc}
\usepackage[russian]{babel}

\usepackage{graphicx}
\graphicspath{ {./images/} }

\usepackage{adjustbox}

\usepackage{color}
\usepackage{soul}

\usepackage{hyperref}

\usepackage{amsmath}

\usepackage{tikz}
\usetikzlibrary{decorations}
\usetikzlibrary{decorations.pathreplacing}
\usepackage{xifthen}

\definecolor{red}{rgb}{1,0,0}
\definecolor{green}{rgb}{0,0.5,0}
\definecolor{blue}{rgb}{0,0,1}
\definecolor{magenta}{rgb}{0.75,0,0.75}

\makeatletter
\newcommand{\slideimage}[1]{
  \begin{figure}
    \begin{adjustbox}{width=\textwidth, totalheight=\textheight-2\baselineskip-2\baselineskip,keepaspectratio}
      \includegraphics{#1}
    \end{adjustbox}
  \end{figure}
}
\makeatother

\title{Компьютерная графика}
\subtitle{Лекция 6: Ошибки OpenGL, расширения OpenGL, blending, освещение}
\date{2021}

\setbeamertemplate{footline}[frame number]

\begin{document}

\frame{\titlepage}

\begin{frame}[fragile]
\frametitle{Ошибки OpenGL}
\begin{itemize}
\item Часто драйвер может понять, что в определённом OpenGL-вызове содержится ошибка
\pause
\begin{itemize}
\item \verb|glTexImage2D| с параметром \verb|target|, не являющимся одним из типов текстур
\pause
\item \verb|glTexParameteri| устанавливающая mag filter в что-то отличное от \verb|GL_LINEAR| или \verb|GL_NEAREST|
\pause
\item \verb|glBindBuffer| с ID буффера, который не был получен через \verb|glGenBuffers|
\pause
\item etc.
\end{itemize}
\pause
\item В таких случаях генерируется ошибка как особое значение типа \verb|GLenum|
\pause
\item Как правило, вызвавшая ошибку операция не выполняется
\pause
\item \verb|glGetError()| - получить значение ошибки, если она была
\end{itemize}
\end{frame}

\begin{frame}[fragile]
\frametitle{Ошибки OpenGL}
Возможные значения ошибок:
\begin{itemize}
\item \verb|GL_NO_ERROR| - ошибки нет
\pause
\item \verb|GL_INVALID_ENUM| - недопустимое значения перечисления (например, \verb|GL_TEXTURE_2D| как первый параметр \verb|glBindBuffer|)
\pause
\item \verb|GL_INVALID_VALUE| - недопустимое значение числового аргумента (например, отрицательная ширина текстуры в \verb|glTexImage2D|)
\pause
\item \verb|GL_INVALID_OPERATION| - недопустимая операция (например, \verb|glUniform1i| при отсутствии текущей шейдерной программы)
\pause
\item \verb|GL_INVALID_FRAMEBUFFER_OPERATION| - операция рисования/чтения пикселей с невалидным framebuffer'ом (о них мы поговорим позже)
\pause
\item \verb|GL_OUT_OF_MEMORY| - закончилась память на GPU (может быть вызвана любой OpenGL-командой, обычно не обрабатывается)
\end{itemize}
\end{frame}

\begin{frame}[fragile]
\frametitle{Ошибки OpenGL}
\begin{itemize}
\item Документация к каждой функции описывает все возможные ошибки, которые эта функция может вызвать, и в каких случаях
\pause
\begin{itemize}
\item N.B. \verb|GL_OUT_OF_MEMORY| нигде не указан, потому что может появиться где угодно
\end{itemize}
\pause
\item Не все ошибки покрываются этим механизмом (например: \verb|glDrawArrays| с правильно настроенным VAO но с пустым VBO с вершинами)
\pause
\item Нет информации о том, какая именно функция вызвала ошибку (любая из тех, что были вызваны до \verb|glGetError|)
\pause
\begin{itemize}
\item \begin{math}\Rightarrow\end{math} Придётся расставлять проверку после каждого OpenGL-вызова
\end{itemize}
\end{itemize}
\end{frame}

\begin{frame}[fragile]
\frametitle{Ошибки OpenGL}
\begin{itemize}
\item Драйвер может хранить один флаг ошибки, и пропускать все последующие ошибки до вызова \verb|glGetError|
\pause
\item Драйвер может хранить несколько флагов, и \verb|glGetError| возвращает и очищает любой из них
\pause
\item \begin{math}\Rightarrow\end{math} Чтобы очистить ошибки, нужно вызывать \verb|glGetError| в цикле:
\end{itemize}
\begin{verbatim}
while (glGetError() != GL_NO_ERROR);
\end{verbatim}
\end{frame}

\begin{frame}[fragile]
\frametitle{Ошибки OpenGL}
\begin{itemize}
\item OpenGL 4.3+ (или расширение \verb|GL_ARB_debug_output|): более удобный механизм, \verb|glDebugMessageCallback|
\pause
\item \href{https://www.khronos.org/opengl/wiki/OpenGL_Error}{khronos.org/opengl/wiki/OpenGL\_Error}
\item \href{https://docs.gl/gl3/glGetError}{docs.gl/gl3/glGetError}
\end{itemize}
\end{frame}

\begin{frame}[fragile]
\frametitle{Расширения OpenGL}
\begin{itemize}
\item Предоставляют дополнительную функциональность за рамками возможностей конкретной версии OpenGL\only<2->{, например}
\pause
\begin{itemize}
\item Конкретный производитель выпустил новую функциональность
\pause
\item Фунцкиональность доступна на большинстве реализаций OpenGL, но ещё не успела войти в новую версию OpenGL
\pause
\item Функциональность доступна в новой версии OpenGL, но крайне распространена и хочется ей пользоваться из старой версии
\end{itemize}
\end{itemize}
\end{frame}

\begin{frame}[fragile]
\frametitle{Расширения OpenGL}
\begin{itemize}
\item Делятся на
\begin{itemize}
\item ARB (\textit{Architectural Review Board}) - функциональность, которая (скорее всего) войдёт в следующий стандарт
\item EXT - широко распространённая функциональность
\item NV - расширение от Nvidia (возможно, доступно и на других видеокартах)
\item AMD - расширение от AMD (возможно, доступно и на других видеокартах)
\item APPLE - расширение от APPLE (возможно, доступно и на других видеокартах)
\item И т.д.
\end{itemize}
\end{itemize}
\end{frame}

\begin{frame}[fragile]
\frametitle{Расширения OpenGL}
\begin{itemize}
\item Расширение - набор констант и функций (как и конкретная версия OpenGL)
\pause
\item Функции нужно загружать, так же, как и функции самого OpenGL
\pause
\item Константы и перечисления заканчиваются на суффикс в зависимости от типа расширения:
\begin{itemize}
\item \verb|ARB_debug_output|: \verb|glDebugMessageCallbackARB|, \verb|GL_DEBUG_SEVERITY_HIGH_ARB|
\item \verb|EXT_texture_filter_anisotropic|: \verb|GL_TEXTURE_MAX_ANISOTROPY_EXT|
\item \verb|NV_texture_barrier|: \verb|glTextureBarrierNV|
\end{itemize}
\pause
\item Исключение: core extensions - предоставляют функциональность новых версий OpenGL в старых версиях OpenGL
\begin{itemize}
\item Имеют тип \verb|ARB|
\item Не имеют суффиксов в названиях функций и констант
\end{itemize}  
\end{itemize}
\end{frame}

\begin{frame}[fragile]
\frametitle{Расширения OpenGL}
\begin{itemize}
\item Получить список поддерживаемых расширений:
\end{itemize}
\begin{verbatim}
GLint numExtensions;
glGetIntegerv(GL_NUM_EXTENSIONS, numExtensions);
for (GLint i = 0; i < numExtensions; ++i) {
    std::cout << glGetStringi(GL_EXTENSIONS, i) << "\n";
}
\end{verbatim}
\pause
\begin{itemize}
\item GLEW загружает их автоматически:
\end{itemize}
\begin{verbatim}
if (GLEW_EXT_texture_filter_anisotropic) {
    std::cout << "Anisotropic filtering is supported\n";
}
\end{verbatim}
\end{frame}

\begin{frame}[fragile]
\frametitle{Расширения OpenGL}
\begin{itemize}
\item \href{https://www.khronos.org/opengl/wiki/OpenGL_Extension}{khronos.org/opengl/wiki/OpenGL\_Extension}
\item \href{https://www.khronos.org/registry/OpenGL/index_gl.php}{khronos.org/registry/OpenGL/index\_gl.php} - список всех зарегистрированных расширений
\end{itemize}
\end{frame}

\begin{frame}<1-3>[fragile,label=blending]
\frametitle{Blending (смешивание)}
\begin{itemize}
\item Результат фрагментного шейдера (\verb|layout (location = 0) vec4 out_color;|) записывается в пиксель экрана, стирая то, что было в нём до этого
\pause
\item Иногда мы хотим "смешать" результат и пиксель экрана, и записать результат смешивания
\pause
\begin{itemize}
\item Полупрозрачные объекты
\pause
\item Растительность
\pause
\item Алгоритмы освещения и теней
\pause
\item Heatmaps
\pause
\item Ручное сглаживание
\pause
\item И т.д.
\end{itemize}
\end{itemize}
\end{frame}

\begin{frame}
\frametitle{Полупрозрачность}
\begin{figure}
\slideimage{window.png}
\end{figure}
\end{frame}

\begin{frame}
\frametitle{Полупрозрачность}
\begin{figure}
\slideimage{macos-ui.png}
\end{figure}
\end{frame}

\againframe<3-4>{blending}

\begin{frame}
\frametitle{Деревья}
\begin{figure}
\slideimage{birch.png}
\end{figure}
\end{frame}

\againframe<4-6>{blending}

\begin{frame}
\frametitle{Heatmap}
\begin{figure}
\slideimage{heatmap.jpg}
\end{figure}
\end{frame}

\againframe<6-7>{blending}

\begin{frame}
\frametitle{Yandex maps}
\begin{figure}
\slideimage{yandex-maps.jpg}
\end{figure}
\end{frame}

\againframe<7->{blending}

\begin{frame}[fragile]
\frametitle{Blending (смешивание)}
\begin{itemize}
\item Включается/выключается: \verb|glEnable(GL_BLEND)|
\pause
\item \begin{math}src\end{math} - результат фрагментного шейдера
\item \begin{math}dst\end{math} - пиксель на экране
\item Уравнение blending'а:
\begin{center}
\begin{math}
dst \leftarrow f(C_{src} \cdot src, C_{dst} \cdot dst)
\end{math}
\end{center}
\pause
\item \begin{math}f\end{math} - настраиваемая функция
\item \begin{math}C_{src}\end{math} и \begin{math}C_{dst}\end{math} - настраиваемые веса
\end{itemize}
\end{frame}

\begin{frame}[fragile]
\frametitle{Blending (смешивание)}
\begin{itemize}
\item Настройка функции \begin{math}f\end{math}: \verb|glBlendEquation(GLenum equation)|:
\pause
\begin{itemize}
\item \verb|GL_FUNC_ADD|: \begin{math}f(src, dst) = src + dst\end{math}
\item \verb|GL_FUNC_SUBTRACT|: \begin{math}f(src, dst) = src - dst\end{math}
\item \verb|GL_FUNC_REVERSE_SUBTRACT|: \begin{math}f(src, dst) = dst - src\end{math}
\item \verb|GL_MIN|: \begin{math}f(src, dst) = \min(src, dst)\end{math}
\item \verb|GL_MAX|: \begin{math}f(src, dst) = \max(src, dst)\end{math}
\end{itemize}
\end{itemize}
\end{frame}

\begin{frame}[fragile]
\frametitle{Blending (смешивание)}
\begin{itemize}
\item Настройка весов \begin{math}C_{src}\end{math} и \begin{math}C_{dst}\end{math}: \verb|glBlendFunc(GLenum src, GLenum dst)|:
\pause
\begin{itemize}
\item \verb|GL_ZERO|: \begin{math}C = 0\end{math}
\item \verb|GL_ONE|: \begin{math}C = 1\end{math}
\item \verb|GL_SRC_COLOR|: \begin{math}C = src\end{math}
\item \verb|GL_ONE_MINUS_SRC_COLOR|: \begin{math}C = 1 - src\end{math}
\item \verb|GL_SRC_ALPHA|: \begin{math}C = src_A\end{math}
\item \verb|GL_ONE_MINUS_SRC_ALPHA|: \begin{math}C = 1 - src_A\end{math}
\item И т.д.
\end{itemize}
\end{itemize}
\end{frame}

\begin{frame}[fragile]
\frametitle{Blending (смешивание)}
\begin{itemize}
\item Обычно \begin{math}src=(R,G,B,A)\end{math} означает, что цвет имеет полупрозрачность \begin{math}A\end{math}
\begin{itemize}
\item \begin{math}A=0\end{math} - полностью прозрачный цвет
\item \begin{math}A=1\end{math} - полностью непрозрачный цвет
\end{itemize}
\pause
\item Для этого нужна такая формула смешивания:
\begin{center}
\begin{math}
dst \leftarrow src_A \cdot src + (1 - src_A) \cdot dst
\end{math}
\end{center}
\pause
\item Этому соответстует настройка
\begin{verbatim}
glBlendEquation(GL_FUNC_ADD);
glBlendFunc(GL_SRC_ALPHA, GL_ONE_MINUS_SRC_ALPHA);
\end{verbatim}
\pause
\item Это самый типичный способ использовать blending
\end{itemize}
\end{frame}

\begin{frame}[fragile]
\frametitle{Blending (смешивание)}
\begin{itemize}
\item Такое смешивание некоммутативно, т.е. результат зависит от порядка рисования
\end{itemize}
\slideimage{two-squares.png}
\end{frame}

\begin{frame}[fragile]
\frametitle{Blending (смешивание)}
\begin{itemize}
\item Смешивание некорректно работает с тестом глубины
\end{itemize}
\slideimage{blending_incorrect_order.png}
\end{frame}

\begin{frame}[fragile]
\frametitle{Blending (смешивание)}
\begin{itemize}
\item В общем случае нужно сортировать полупрозрачные объекты и рисовать в порядке уменьшения расстояния до камеры
\item Order-independent transparency (OIT) - активная тема исследований
\end{itemize}
\end{frame}

\begin{frame}[fragile]
\frametitle{Blending (смешивание)}
\slideimage{blending_sorted.png}
\end{frame}

\begin{frame}[fragile]
\frametitle{Blending (смешивание)}
\begin{itemize}
\item \href{https://www.khronos.org/opengl/wiki/Blending}{khronos.org/opengl/wiki/Blending}
\item \href{http://docs.gl/gl3/glBlendFunc}{docs.gl/gl3/glBlendFunc}
\item \href{http://docs.gl/gl3/glBlendEquation}{docs.gl/gl3/glBlendEquation}
\item \href{http://www.opengl-tutorial.org/intermediate-tutorials/tutorial-10-transparency/}{opengl-tutorial.org/intermediate-tutorials/tutorial-10-transparency}
\item \href{https://learnopengl.com/Advanced-OpenGL/Blending}{learnopengl.com/Advanced-OpenGL/Blending}
\end{itemize}
\end{frame}

\begin{frame}[fragile]
\frametitle{Освещение}
Зачем нужно освещение?
\begin{itemize}
\item Реалистичный рендеринг: всё, что мы видим, это свет
\pause
\item Нереалистичный рендеринг:
\begin{itemize}
\item Помогает понять форму объектов
\item Помогает понять соотношение между объектами в пространстве
\item Создаёт атмосферу
\end{itemize}
\end{itemize}
\end{frame}

\begin{frame}[fragile]
\frametitle{Освещение}
\slideimage{suzanne.png}
\end{frame}

\begin{frame}[fragile]
\frametitle{Освещение}
\slideimage{suzanne-lit.png}
\end{frame}

\begin{frame}[fragile]
\frametitle{Освещение}
\slideimage{suzanne-box.png}
\end{frame}

\begin{frame}[fragile]
\frametitle{Освещение}
\slideimage{suzanne-box-lit.png}
\end{frame}

\begin{frame}[fragile]
\frametitle{Освещение}
\slideimage{zelda.png}
\end{frame}

\begin{frame}[fragile]
\frametitle{Свет}
\begin{itemize}
\item Свет - электромагнитная волна
\item Пара векторных полей (электрическое \begin{math}\mathbf{E}\end{math} и магнитное \begin{math}\mathbf{B}\end{math}), описываемых уравнениями Максвелла
\begin{center}
\begin{math}
\begin{matrix}
\nabla \cdot \mathbf{E} = 4 \pi \rho \\
\nabla \cdot \mathbf{B} = 0 \\
\nabla \times \mathbf{E} = -\frac{1}{c} \frac{\partial \mathbf B}{\partial t} \\
\nabla \times \mathbf{B} = \frac{1}{c} \left(4 \pi \mathbf J + \frac{\partial \mathbf E}{\partial t} \right)
\end{matrix}
\end{math}
\end{center}
\pause
\item Раскладывается в сумму волн фиксированной длины волны \begin{math}\lambda\end{math} (монохроматические волны)
\item Интерференция, дифракция, поляризация, взаимодействие с веществом
\end{itemize}
\end{frame}

\begin{frame}[fragile]
\frametitle{Свет}
\begin{itemize}
\item В реднеринге обычно достаточно геометрической оптики - предела при \begin{math}\lambda\rightarrow 0\end{math}
\pause
\item Свет распространяется прямыми лучами (исключение - граница раздела сред или неоднородные среды)
\item Свет распространяется бесконечно быстро (\begin{math}c \rightarrow \infty\end{math})
\item Луч света разбивается в сумму монохроматических лучей, каждая имеет свою амплитуду (количество света)
\pause
\item Луч света - распределение амплитуд по длинам волн
\end{itemize}
\end{frame}

\begin{frame}[fragile]
\frametitle{Свет}
\slideimage{blue-red-spectrum.png}
\end{frame}

\begin{frame}[fragile]
\frametitle{Видимый спектр}
\slideimage{visible-spectrum.jpg}
\end{frame}

\begin{frame}[fragile]
\frametitle{Косинус угла падения}
\begin{itemize}
\item На поверхность площадью \begin{math}A\end{math} падает световой поток с площадью поперечного сечения \begin{math}{\color{blue} \Phi}\end{math}
\item Тогда \begin{math}{\color{blue} \Phi} = A \cdot {\color{red} \cos \theta}\end{math}
\item \only<2->{\begin{math}\Rightarrow\end{math} На единицу площади приходится \begin{math}{\color{red} \cos \theta}\end{math} от общей плотности потока}
\end{itemize}
\begin{center}
\begin{tikzpicture}
\draw[black,thick] (0.0, 0.0) -- (10.0, 0.0);
\draw[black,thick] (4.5, 0.0) -- (8.5, 4.0);
\draw[black,thick] (5.5, 0.0) -- (9.0, 3.5);
\draw[black,thick] (4.5, -0.25) -- (4.5, 0.25);
\draw[black,thick] (5.5, -0.25) -- (5.5, 0.25);
\draw[blue,thick] (7.5, 3.0) -- (8.0, 2.5);
\draw[red,thick] (6.0, 0.0) arc (0:45:0.5);

\node[black] at (5.0, -0.5) {A};
\node[blue] at (8.0, 3.0) {\begin{math}\Phi\end{math}};
\node[red] at (6.5, 0.5) {\begin{math}\theta\end{math}};

\only<3->{
  \draw[black,thick,-stealth] (5.0, 3.0) -- (5.0, 3.5);
  \draw[blue,thick,-stealth] (5.0, 3.0) -- (5.35, 3.35);

  \node[black] at (5.0, 4.0) {\begin{math}\vec{n}\end{math}};
  \node[blue] at (5.7, 3.7) {\begin{math}\vec{\omega}\end{math}};

  \node[] at (2.0, 2.0) {\begin{math}{\color{red} \cos\theta} = \vec{n} \cdot {\color{blue} \vec{\omega}}\end{math}};
}
\end{tikzpicture}
\end{center}
\pause
\pause
\end{frame}

\begin{frame}[fragile]
\frametitle{Столкновение света с поверхностью}
\begin{itemize}
\item Что происходит, когда луч света сталкивается с некой поверхностью?
\pause
\item Зависит от её \textit{материала}:
\pause
\begin{itemize}
\item Абсолютно чёрное тело: поглощает свет
\pause
\item Зеркало: отражает свет в строго определённом направлении
\pause
\item Старое, плохо отполированное зеркало / лакированная поверхность: отражает свет примерно в определённом направлении
\pause
\item Диффузная (ламбертова) поверхность: отражает свет во все стороны равномерно
\pause
\item Стекло: преломляет свет
\pause
\item Вода: частично отражает, частично преломляет (закон Френеля)
\end{itemize}
\pause
\item \begin{math}\Rightarrow\end{math} Свет, выходящий в каком-то направлении, может зависеть от света, пришедшего из \textit{любого} направления
\end{itemize}
\end{frame}

\begin{frame}[fragile]
\frametitle{Абсолютно чёрное тело}
\begin{center}
\begin{tikzpicture}
\draw[black,thick] (0.0, 0.0) -- (10.0, 0.0);
\draw[blue,thick,-stealth] (8.0, 3.0) -- (5.5, 0.5);

\node[] at (5.0, 0.0) {\begin{math}\bullet\end{math}};
\node[] at (5.0, -0.5) {\begin{math}p\end{math}};
\node[blue] at (8.5, 3.5) {\begin{math}\vec\omega_{in}\end{math}};
\end{tikzpicture}
\end{center}
\end{frame}

\begin{frame}[fragile]
\frametitle{Зеркало}
\begin{center}
\begin{tikzpicture}
\draw[black,thick] (0.0, 0.0) -- (10.0, 0.0);
\draw[blue,thick,-stealth] (8.0, 3.0) -- (5.5, 0.5);
\draw[magenta,thick,-stealth] (4.5, 0.5) -- (2.0, 3.0);

\node[] at (5.0, 0.0) {\begin{math}\bullet\end{math}};
\node[] at (5.0, -0.5) {\begin{math}p\end{math}};
\node[blue] at (8.5, 3.5) {\begin{math}\vec\omega_{in}\end{math}};
\node[magenta] at (1.5, 3.5) {\begin{math}\vec\omega_{out}\end{math}};
\end{tikzpicture}
\end{center}
\end{frame}

\begin{frame}[fragile]
\frametitle{Старое зеркало / лакированная поверхность}
\begin{center}
\begin{tikzpicture}
\draw[black,thick] (0.0, 0.0) -- (10.0, 0.0);
\draw[blue,thick,-stealth] (8.0, 3.0) -- (5.5, 0.5);
\draw[magenta,thick,-stealth] (4.5, 0.5) -- (2.0, 3.0);
\draw[magenta,thick,-stealth] (4.64, 0.6) -- (3.5, 2.5);
\draw[magenta,thick,-stealth] (4.78, 0.67) -- (4.5, 1.5);
\draw[magenta,thick,-stealth] (4.4, 0.36) -- (2.5, 1.5);
\draw[magenta,thick,-stealth] (4.33, 0.22) -- (3.5, 0.5);

\node[] at (5.0, 0.0) {\begin{math}\bullet\end{math}};
\node[] at (5.0, -0.5) {\begin{math}p\end{math}};
\node[blue] at (8.5, 3.5) {\begin{math}\vec\omega_{in}\end{math}};
\node[magenta] at (1.5, 3.5) {\begin{math}\vec\omega_{out}\end{math}};
\end{tikzpicture}
\end{center}
\end{frame}

\begin{frame}[fragile]
\frametitle{Диффузная поверхность}
\begin{center}
\begin{tikzpicture}
\draw[black,thick] (0.0, 0.0) -- (10.0, 0.0);
\draw[blue,thick,-stealth] (8.0, 3.0) -- (5.5, 0.5);
\draw[magenta,thick,-stealth] (4.387627564304205, 0.35355339059327373) -- (1.3257653858252327, 2.121320343559642);
\draw[magenta,thick,-stealth] (4.646446609406726, 0.6123724356957946) -- (2.878679656440357, 3.674234614174767);
\draw[magenta,thick,-stealth] (5.0, 0.7071067811865476) -- (5.0, 4.242640687119285);
\draw[magenta,thick,-stealth] (5.353553390593274, 0.6123724356957946) -- (7.121320343559642, 3.6742346141747673);
\draw[magenta,thick,-stealth] (5.612372435695795, 0.35355339059327373) -- (8.674234614174768, 2.121320343559642);

\node[] at (5.0, 0.0) {\begin{math}\bullet\end{math}};
\node[] at (5.0, -0.5) {\begin{math}p\end{math}};
\node[blue] at (8.5, 3.5) {\begin{math}\vec\omega_{in}\end{math}};
\node[magenta] at (1.5, 3.5) {\begin{math}\vec\omega_{out}\end{math}};
\end{tikzpicture}
\end{center}
\end{frame}

\begin{frame}[fragile]
\frametitle{Стекло}
\begin{center}
\begin{tikzpicture}
\draw[black,thick] (0.0, 0.0) -- (10.0, 0.0);
\draw[blue,thick,-stealth] (8.0, 3.0) -- (5.5, 0.5);
\draw[magenta,thick,-stealth] (4.646446609406726, -0.6123724356957944) -- (2.878679656440356, -3.6742346141747655);

\node[] at (5.0, 0.0) {\begin{math}\bullet\end{math}};
\node[] at (5.0, -0.5) {\begin{math}p\end{math}};
\node[blue] at (8.5, 3.5) {\begin{math}\vec\omega_{in}\end{math}};
\node[magenta] at (2.5251262658470814, -4.28660704987056) {\begin{math}\vec\omega_{out}\end{math}};
\end{tikzpicture}
\end{center}
\end{frame}

\begin{frame}[fragile]
\frametitle{Уравнение рендеринга}
\begin{center}
\begin{equation}
{\color{magenta}I_{out}}(p, {\color{magenta}\vec\omega_{out}}, \lambda) = I_{e}(p, {\color{magenta}\vec\omega_{out}})+\int\limits_{\mathbb S^2} {\color{blue}I_{in}}(p, {\color{blue}\vec\omega_{in}}, \lambda) \cdot {\color{green}f}(p, {\color{blue}\vec\omega_{in}}, {\color{magenta}\vec\omega_{out}}, \lambda) \cdot ({\color{blue}\vec\omega_{in}} \cdot \vec{n}) d{\color{blue}\vec\omega_{in}}
\end{equation}
\end{center}
\pause
\begin{itemize}
\item Уравнение для конкретной точки \begin{math}p\end{math}
\pause
\item \begin{math}{\color{magenta}I_{out}}\end{math} - количество света, выходящего из точки \begin{math}p\end{math} в направлении \begin{math}{\color{magenta}\vec\omega_{out}}\end{math} с длиной волны \begin{math}\lambda\end{math}
\pause
\item \begin{math}I_{e}\end{math} - количество света, испускаемого поверхностью из точки \begin{math}p\end{math} в направлении \begin{math}{\color{magenta}\vec\omega_{out}}\end{math} с длиной волны \begin{math}\lambda\end{math}
\pause
\item \begin{math}{\color{blue}I_{in}}\end{math} - количество света, приходящего в точку \begin{math}p\end{math} из направления \begin{math}{\color{blue}\vec\omega_{in}}\end{math} с длиной волны \begin{math}\lambda\end{math}
\pause
\item \begin{math}{\color{green}f}\end{math} - функция, определяющая, сколько света с длиной волны \begin{math}\lambda\end{math}, пришедшего из направления \begin{math}{\color{blue}\vec\omega_{in}}\end{math}, отразится в направлении \begin{math}{\color{magenta}\vec\omega_{out}}\end{math}
\end{itemize}
\end{frame}

\begin{frame}[fragile]
\frametitle{Уравнение рендеринга}
\begin{itemize}
\item Абсолютно чёрное тело: \begin{math}{\color{green}f}(p, {\color{blue}\vec\omega_{in}}, {\color{magenta}\vec\omega_{out}}, \lambda) = 0\end{math}
\pause
\item Старое зеркало: \begin{math}{\color{green}f}(p, {\color{blue}\vec\omega_{in}}, {\color{magenta}\vec\omega_{out}}, \lambda) = \left(R_{\vec n}({\color{blue}\vec\omega_{in}})\cdot{\color{magenta}\vec\omega_{out}}\right)^7\end{math}
\begin{itemize}
\item \begin{math}R\end{math} - отраженный вектор: \begin{math}R_{\vec n}(\vec \omega) = 2\vec n \cdot (\vec n \cdot \vec \omega) - \vec \omega\end{math}
\end{itemize}
\pause
\item Диффузная поверхность: \begin{math}{\color{green}f}(p, {\color{blue}\vec\omega_{in}}, {\color{magenta}\vec\omega_{out}}, \lambda) = 1\end{math}
\end{itemize}
\end{frame}

\begin{frame}[fragile]
\frametitle{Цвет}
\begin{itemize}
\item Зависимость \begin{math}{\color{green}f}\end{math} от \begin{math}\lambda\end{math} обеспечивает цвет объектов
\pause
\item Мы видим свет, отражённый объектом
\item Объект синего цвета не отражает красный цвет (кажется чёрным при освещении красным светом)
\end{itemize}
\end{frame}

\begin{frame}[fragile]
\frametitle{Материал = BSDF}
\begin{itemize}
\item \begin{math}{\color{green}f}\end{math} определяет \textit{материал} объекта
\pause 
\item Если \begin{math}{\color{green}f}\end{math} только отражает свет, её называют BRDF: Bidirectional Reflectance Distribution Function
\pause 
\item Если \begin{math}{\color{green}f}\end{math} только преломляет свет, её называют BRTF: Bidirectional Transmittance Distribution Function
\pause 
\item В общем случае её называют BSTF: Bidirectional Scattering Distribution Function
\end{itemize}
\end{frame}

\begin{frame}[fragile]
\frametitle{BSDF}
\begin{itemize}
\item Обычно BSTF предполагается нормированной:
\end{itemize}
\begin{equation}
\int\limits_{\mathbb S^2} {\color{green}f}(p, {\color{blue}\vec\omega_{in}}, {\color{magenta}\vec\omega_{out}}, \lambda) d{\color{blue}\vec\omega_{in}} = 1
\end{equation}
\end{frame}

\begin{frame}[fragile]
\frametitle{Уравнение рендеринга}
\begin{itemize}
\item Откуда взять \begin{math}{\color{blue}I_{in}}\end{math}? \pause Ответ: это чей-то \begin{math}{\color{magenta}I_{out}}\end{math} 
\end{itemize}
\begin{center}
\begin{tikzpicture}
\draw[black,thick] (0.0, 0.0) -- (10.0, 0.0);
\draw[black,thick] (0.0, 0.0) -- (0.0, 6.0);

\node[] at (0.0, 3.0) {\begin{math}{\color{magenta}\bullet}\end{math}};
\node[] at (6.0, 0.0) {\begin{math}{\color{blue}\bullet}\end{math}};

\node[] at (-0.5, 3.0) {\begin{math}{\color{magenta}I_{out}}\end{math}};
\node[] at (0.5, 3.5) {\begin{math}{\color{magenta}\vec\omega_{out}}\end{math}};
\node[] at (6.0, -0.5) {\begin{math}{\color{blue}I_{in}}\end{math}};
\node[] at (6.5, 0.5) {\begin{math}{\color{blue}\vec\omega_{in}}\end{math}};

\draw[magenta,thick,-stealth] (0.5, 2.75) -- (2.0, 2.0);
\draw[blue,thick,-stealth] (4.0, 1.0) -- (5.5, 0.25);
\end{tikzpicture}
\end{center}
\end{frame}

\begin{frame}[fragile]
\frametitle{Уравнение рендеринга}
\begin{itemize}
\item Интегральное уравнение для каждой точки каждой поверхности сцены
\pause
\item Геометрия сцены связывает уравнения для разных точек
\pause
\item Обычно вместо всех возможных значений \begin{math}\lambda\end{math} берут дискретный набор ({\color{red}красный},{\color{green}зелёный},{\color{blue}синий})
\pause
\item Задача называется GI: Global Illumination
\pause
\item Решать \textbf{очень} сложно, есть бесчисленное количество методов
\pause
\item Реалистичный offline рендеринг: медленно, но очень хорошая картинка
\pause
\item Real-time рендеринг: нужны грубые аппроксимации
\end{itemize}
\end{frame}

\begin{frame}[fragile]
\frametitle{Только источник света}
\slideimage{two_balls_only_emissive.png}
\end{frame}

\begin{frame}[fragile]
\frametitle{Только прямое освещение}
\slideimage{two_balls_no_shadow.png}
\end{frame}

\begin{frame}[fragile]
\frametitle{Прямое освещение и тени}
\slideimage{two_balls_no_diffuse.png}
\end{frame}

\begin{frame}[fragile]
\frametitle{Прямое и непрямое освещение, тени}
\slideimage{two_balls_full.png}
\end{frame}

\begin{frame}[fragile]
\frametitle{Прозрачность и отражение}
\slideimage{two_balls_ref.png}
\end{frame}

\begin{frame}[fragile]
\frametitle{Ambient освещение}
\begin{itemize}
\item Свет, приходящий "отовсюду"
\item На улице: небо
\item В помещении: диффузное отражение от стен, пола, потолка, других объектов
\item Определяется своим цветом
\end{itemize}
\end{frame}

\begin{frame}[fragile]
\frametitle{Ambient освещение: фрагментный шейдер}
\begin{verbatim}
uniform vec3 ambient_color;

in vec3 color;

layout (location = 0) out vec4 out_color;

void main() {
  vec3 result_color = color * ambient_color;
  out_color = vec4(result_color, 1.0);
}
\end{verbatim}
\end{frame}

\begin{frame}[fragile]
\frametitle{Источники света}
\begin{itemize}
\item Два самых часто используемых типа: направленные и точечные
\pause
\item Направленные:
\begin{itemize}
\item Задаются направлением и яркостью
\item Моделируют удалённые источники - солнце, луна
\end{itemize}
\pause
\item Точечные:
\begin{itemize}
\item Задаются расположением и функцией зависимости яркости от расстояния
\item Моделируют (относительно) близкие источники - костёр, свеча, лампа
\end{itemize}
\end{itemize}
\end{frame}

\begin{frame}[fragile]
\frametitle{Прямое освещение}
\begin{itemize}
\item Нужно найти скалярное произведение между направлением на источник света и нормалью к поверхности
\pause
\item N.B. отрицательный результат означает, что поверхность обращена не по направлению к свету
\pause
\item Нужно найти яркость источника света в данной точке
\pause
\item Результат = произведение яркости, цвета объекта и коэффициента освещённости
\end{itemize}
\end{frame}

\begin{frame}[fragile]
\frametitle{Направленный источник света: фрагментный шейдер}
\begin{verbatim}
uniform vec3 light_direction;
uniform vec3 light_color;

in vec3 normal;
in vec3 color;

layout (location = 0) out vec4 out_color;

void main() {
  float cosine = dot(normal, light_direction);
  float light_factor = max(0.0, cosine);
  vec3 result_color = light_factor * color * light_color;
  out_color = vec4(result_color, 1.0);
}
\end{verbatim}
\end{frame}

\begin{frame}[fragile]
\frametitle{Точечный источник света}
\begin{itemize}
\item Обычно яркость убывает с расстоянием
\pause
\item Часто берут такую функцию затухания (attenuation):
\end{itemize}
\begin{equation}
f(r) = \frac{1}{C_0 + C_1 r + C_2 r^2}
\end{equation}
\end{frame}

\begin{frame}[fragile]
\frametitle{Точечный источник света: фрагментный шейдер}
\fontsize{10pt}{10pt}
\begin{verbatim}
uniform vec3 light_position;
uniform vec3 light_attenuation; // c0, c1, c2
uniform vec3 light_color;

in vec3 position;
in vec3 normal;
in vec3 color;

layout (location = 0) out vec4 out_color;

void main() {
  vec3 light_vector = light_position - position;
  vec3 light_direction = normalize(light_vector);
  float cosine = dot(normal, light_direction);
  float light_factor = max(0.0, cosine);

  float light_distance = length(light_vector);
  float light_intensity = dot(light_attenuation, 
    vec3(1.0, light_distance, light_distance * light_distance));

  vec3 result_color = light_factor * light_intensity * color
    * light_color;
  out_color = vec4(result_color, 1.0);
}
\end{verbatim}
\end{frame}

\begin{frame}[fragile]
\frametitle{Модель Фонга}
\begin{itemize}
\item Можно добавить отражённый свет (specular), имитируя гладкую поверхность
\pause
\item Ambient + direct (diffuse) + specular = модель Фонга
\end{itemize}
\slideimage{phong.png}
\end{frame}

\begin{frame}[fragile]
\frametitle{Модель Гуро}
\begin{itemize}
\item То же самое, но в вершинном шейдере, и интерполировать результат
\item Выглядит плохо, в современности не используется
\end{itemize}
\slideimage{gourand.png}
\end{frame}

\begin{frame}[fragile]
\frametitle{Ссылки}
\begin{itemize}
\item \href{https://learnopengl.com/Lighting/Basic-Lighting}{learnopengl.com/Lighting/Basic-Lighting}
\item \href{http://www.opengl-tutorial.org/beginners-tutorials/tutorial-8-basic-shading}{opengl-tutorial.org/beginners-tutorials/tutorial-8-basic-shading}
\item \href{https://www.lighthouse3d.com/tutorials/glsl-tutorial/point-lights}{lighthouse3d.com/tutorials/glsl-tutorial/point-lights}
\end{itemize}
\end{frame}

\end{document}