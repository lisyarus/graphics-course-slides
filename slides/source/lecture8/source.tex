% (c) Nikita Lisitsa, lisyarus@gmail.com, 2021

\documentclass{beamer}

\usepackage[T2A]{fontenc}
\usepackage[utf8]{inputenc}
\usepackage[russian]{babel}

\usepackage{graphicx}
\graphicspath{ {./images/} }

\usepackage{adjustbox}

\usepackage{color}
\usepackage{soul}

\usepackage{hyperref}

\usepackage{amsmath}

\usepackage{tikz}
\usetikzlibrary{decorations}
\usetikzlibrary{decorations.pathreplacing}
\usepackage{xifthen}

\definecolor{red}{rgb}{1,0,0}
\definecolor{green}{rgb}{0,0.5,0}
\definecolor{blue}{rgb}{0,0,1}
\definecolor{magenta}{rgb}{0.75,0,0.75}

\makeatletter
\newcommand{\slideimage}[1]{
  \begin{figure}
    \begin{adjustbox}{width=\textwidth, totalheight=\textheight-2\baselineskip-2\baselineskip,keepaspectratio}
      \includegraphics{#1}
    \end{adjustbox}
  \end{figure}
}
\makeatother

\title{Компьютерная графика}
\subtitle{Лекция 8: Stencil bufer, framebuffer, renderbuffer, пост-обработка}
\date{2021}

\setbeamertemplate{footline}[frame number]

\begin{document}

\frame{\titlepage}

\begin{frame}[fragile]
\frametitle{Stencil buffer (буфер трафарета)}
\begin{itemize}
\item Особый буфер, хранящий произвольные данные, с особыми побитовыми операциями над ними
\pause
\item Чем-то аналогичен буферу глубины: тоже хранит какие-то данные, такого же размера (как и цветовой буфер), тоже позволяет рисовать или не рисовать пиксель по какому-то условию (depth test)
\pause
\item У дефолтного фреймбуфера часть есть 8-битный stencil буфер (зависит от настроек контекста OpenGL)
\pause
\item Можно (и нужно, если вы его используете) очищать как \verb|glClear(GL_STENCIL_BUFFER_BIT|)
\pause
\item Настроить значение, которым очищается буфер: \verb|glClearStencil|
\end{itemize}
\end{frame}

\begin{frame}[fragile]
\frametitle{Stencil тест}
\begin{itemize}
\item Включить/выключить stencil тест: \verb|glEnable/glDisable(GL_STENCIL_TEST)|
\pause
\item Настроить stencil тест: \verb|glStencilFunc|
\begin{itemize}
\item \verb|func| - одна из констант \verb|GL_ALWAYS|, \verb|GL_LESS|, \verb|GL_GREATER|, \verb|GL_EQUAL|, ...
\item \verb|ref| - референсное значение для теста
\item \verb|mask| - побитовая маска для теста
\end{itemize}
\item Stencil тест: \verb|func(ref & mask, stencil & mask)|
\pause
\item Так же, как с depth тестом: если stencil тест не прошёл, пиксель не будет нарисован
\end{itemize}
\end{frame}

\begin{frame}[fragile]
\frametitle{Stencil тест}
\begin{itemize}
\item Как записать значение в stencil буфер? \pause \verb|glStencilOp|
\begin{itemize}
\item \verb|sfail| - что делать, если пиксель не прошёл stencil тест
\item \verb|dpfail| - что делать, если пиксель не прошёл depth тест
\item \verb|dppass| - что делать, если пиксель прошёл оба теста
\end{itemize}
\pause
\item Возможные значение \verb|sfail|, \verb|dpfail| и \verb|dppass|:
\begin{itemize}
\item \verb|GL_KEEP| - не менять записанное значение
\item \verb|GL_ZERO| - записать 0
\item \verb|GL_INVERT| - побитово обратить
\item \verb|GL_REPLACE| - записать \verb|ref| из функции \verb|glStencilFunc|
\item \verb|GL_INCR| - увеличить на 1, если значение меньше максимального
\item \verb|GL_DECR| - уменьшить на 1, если значение больше минимального (0)
\item \verb|GL_INCR_WRAP| - увеличить на 1 с целочисленным переполнением
\item \verb|GL_DECR_WRAP| - уменьшить на 1 с целочисленным переполнением
\end{itemize}
\end{itemize}
\end{frame}

\begin{frame}[fragile]
\frametitle{Stencil тест}
\begin{itemize}
\item Дополнительно можно включать/выключать запись отдельных битов stencil буфера: \verb|glStencilMask|
\pause
\item Все параметры stencil теста можно настраивать отдельно для front и back граней функциями \verb|glStencilFuncSeparate|, \verb|glStencilOpSeparate|, \verb|glStencilMaskSeparate|
\end{itemize}
\end{frame}

\begin{frame}[fragile]
\frametitle{Stencil тест}
\slideimage{stencil-example-1.png}
\end{frame}

\begin{frame}[fragile]
\frametitle{Stencil тест}
\slideimage{stencil-example-2.png}
\end{frame}

\begin{frame}[fragile]
\frametitle{Stencil тест}
\slideimage{stencil-example-3.png}
\end{frame}

\begin{frame}<1>[fragile,label=stencil_examples]
\frametitle{Stencil буфер: применение}
\begin{itemize}
\item Некоторые алгоритмы рисования теней (shadow volumes)
\pause
\item Любая ситуация, в которой нужно ограничить рисование определённых пикселей:
\pause
\begin{itemize}
\item Симулятор самолёта: сначала рисуется внутренность самолёта, затем - окружающий мир, только там, где не был нарисован самолёт \begin{math}\Rightarrow\end{math} можно избежать проблем с точностью буфера глубины
\pause
\item UI, который нужно нарисовать в какой-то ограниченной области экрана (например, scroll)
\pause
\item Рисование невыпуклых полигонов (odd-even rule)
\end{itemize}
\end{itemize}
\end{frame}

\begin{frame}[fragile]
\frametitle{Shadow volumes (stencil shadows)}
\slideimage{shadow-volumes.jpeg}
\end{frame}

\againframe<2-3>{stencil_examples}

\begin{frame}[fragile]
\frametitle{Microsoft flight simulator}
\slideimage{flight-simulator.jpg}
\end{frame}

\againframe<4>{stencil_examples}

\begin{frame}[fragile]
\frametitle{Unity Scroll View}
\slideimage{scroll-view.png}
\end{frame}

\againframe<5>{stencil_examples}

\begin{frame}[fragile]
\frametitle{Невыпуклый полигон}
\slideimage{non-convex-polygon.png}
\end{frame}

\begin{frame}[fragile]
\frametitle{Stencil буфер: ссылки}
\begin{itemize}
\item \href{https://www.khronos.org/opengl/wiki/Stencil_Test}{khronos.org/opengl/wiki/Stencil\_Test}
\item \href{https://learnopengl.com/Advanced-OpenGL/Stencil-testing}{learnopengl.com/Advanced-OpenGL/Stencil-testing}
\item \href{https://open.gl/depthstencils}{open.gl/depthstencils}
\item \href{https://en.wikibooks.org/wiki/OpenGL_Programming/Stencil_buffer}{en.wikibooks.org/wiki/OpenGL\_Programming/Stencil\_buffer}
\end{itemize}
\end{frame}

\begin{frame}[fragile]
\frametitle{Framebuffer (FBO, кадровый буфер)}
\begin{itemize}
\item OpenGL-объект, хранящий настройки того, куда осуществляется рисование
\pause
\item Не владеет памятью, только ссылается на неё
\pause
\item Может иметь depth buffer, stencil buffer, и несколько color buffer'ов
\pause
\item Общепринятая аббревиатура: FBO
\end{itemize}
\end{frame}

\begin{frame}[fragile]
\frametitle{Framebuffer}
\begin{itemize}
\item Создание/удаление: \verb|glGenFramebuffers|/\verb|glDeleteFramebuffers|
\pause
\item Сделать текущим: \verb|glBindFramebuffer|
\pause
\item Два значения target:
\begin{itemize}
\item \verb|GL_DRAW_FRAMEBUFFER| - в этот фреймбуфер рисуют все операции рисования (\verb|glDrawArrays|, etc), этот фреймбуфер очищает \verb|glClear|
\pause
\item \verb|GL_READ_FRAMEBUFFER| - из этого фреймбуфера читают операции чтения
\end{itemize}  
\pause
\item Можно вызвать \verb|glBindFramebuffer(GL_FRAMEBUFFER, fbo)| - это эквивалентно
\begin{verbatim}
glBindFramebuffer(GL_DRAW_FRAMEBUFFER, fbo);
glBindFramebuffer(GL_READ_FRAMEBUFFER, fbo);
\end{verbatim}
\end{itemize}
\end{frame}

\begin{frame}[fragile]
\frametitle{Default framebuffer}
\begin{itemize}
\item Особый объект, имеет ID = 0
\pause
\item Создаётся при создании OpenGL-контекста
\pause
\item Настроен, чтобы рисовать на экран, к которому привязан контекст
\pause
\item Для него форматы цвета, буфера глубины и stencil буфера настраиваются при создании контекста
\pause
\item Нельзя настроить по-другому или удалить
\end{itemize}
\end{frame}

\begin{frame}[fragile]
\frametitle{Чтение из фреймбуфера}
\begin{itemize}
\item \verb|glReadPixels| - достаёт пиксели из определённого участка текущего \verb|GL_READ_FRAMEBUFFER| (можно прочитать цвет, глубину или stencil)
\pause
\item \verb|glBlitFramebuffer| - копирует пиксели из определённого участка текущего \verb|GL_READ_FRAMEBUFFER| в определённый участок текущего \verb|GL_DRAW_FRAMEBUFFER| (можно скопировать цвет, глубину или stencil)
\end{itemize}
\end{frame}

\begin{frame}[fragile]
\frametitle{Framebuffer attachments}
\begin{itemize}
\item Текстура, на которую ссылается фреймбуфер, называется attachment
\pause
\item У каждого фреймбуфера есть набор attachment points
\pause
\item \verb|GL_COLOR_ATTACHMENT0|, ... \verb|GL_COLOR_ATTACHMENT7| - цветовые буферы
\pause
\begin{itemize}
\item Максимальное количество: \verb|glGet(GL_MAX_COLOR_ATTACHMENTS)|
\item Номер (0..7) - то, что мы указываем как location для выходной переменной фрагментного шейдера \verb|layout (location = 0) ...|
\end{itemize}
\pause
\item \verb|GL_DEPTH_ATTACHMENT| - буфер глубины
\item \verb|GL_STENCIL_ATTACHMENT| - stencil буфер
\end{itemize}
\end{frame}

\begin{frame}[fragile]
\frametitle{Framebuffer attachments}
\begin{itemize}
\item Связать текстуру с фреймбуфером: \verb|glFramebufferTexture|
\pause
\begin{itemize}
\item \verb|target| - текстура будет привязана к текущему фреймбуферу с таким таргетом (\verb|GL_READ_FRAMEBUFFER| или \verb|GL_DRAW_FRAMEBUFFER|)
\pause
\item \verb|attachment| - \verb|GL_COLOR_ATTACHMENT0|, ...
\pause
\item \verb|texture| - ID текстуры (делать её текущей не нужно)
\pause
\item \verb|level| - mipmap-уровень текстуры, в который будет осуществляться рисование
\end{itemize}
\end{itemize}
\end{frame}

\begin{frame}[fragile]
\frametitle{Framebuffer attachments}
\begin{itemize}
\item Можно привязать несколько текстур к одному фреймбуферу (в т.ч. иметь несколько цветовых буферов - соответственно, несколько \verb|out|-переменных во фрагментном шейдере)
\pause
\item Можно привязать одну текстуру к нескольким фреймбуферам
\pause
\item Можно привязать несколько разных mipmap-уровней одной текстуры к одному или нескольким фреймбуферам
\pause
\item Можно привязать одномерные и двумерные текстуры, грани cubemap-текстуры (\verb|glFramebufferTexture2D|), слои трёхмерных или 2D-array текстур (\verb|glFramebufferTexture3D| или \verb|glFramebufferTextureLayer|)
\end{itemize}
\end{frame}

\begin{frame}[fragile]
\frametitle{Framebuffer completeness}
\begin{itemize}
\item У фреймбуферов есть особое свойство - completeness: означает, можно ли рисовать в этот фреймбуфер
\pause
\item Фреймбуфер считается complete, если
\begin{itemize}
\item Размеры всех его attachment'ов совпадают
\pause
\item Все attachment'ы имеют правильный формат
\end{itemize}
\pause
\item Проверить completeness - \verb|glCheckFramebufferStatus|: вернёт \verb|GL_FRAMEBUFFER_COMPLETE| или некий код ошибки
\end{itemize}
\end{frame}

\begin{frame}[fragile]
\frametitle{Renderable formats}
\begin{itemize}
\item Что такое \textit{правильный формат}?
\pause
\item Для цветового буфера - color-renderable формат, т.е. любой цветовой формат: \verb|GL_RED|, \verb|GL_RGB8|, \verb|GL_RGBA8|, \verb|GL_RG32F|, ...
\pause
\item Для буфера глубины - depth-renderable формат: \verb|GL_DEPTH_COMPONENT16|, \verb|GL_DEPTH_COMPONENT24|, \verb|GL_DEPTH_COMPONENT32F|, \verb|GL_DEPTH24_STENCIL8|, \verb|GL_DEPTH32F_STENCIL8|
\pause
\item Для stencil буфера - stencil-renderable формат: \verb|GL_DEPTH24_STENCIL8|, \verb|GL_DEPTH32F_STENCIL8|
\end{itemize}
\end{frame}

\begin{frame}[fragile]
\frametitle{Viewport}
\begin{itemize}
\item \verb|glViewport| настраивает преобразование из координат \begin{math}[-1, 1]^2\end{math} в пиксельные координаты
\pause
\item Никак не связан с фреймбуфером
\pause
\item Но обычно мы хотим, чтобы viewport совпадал с размером текущего фреймбуфера
\pause
\item \begin{math}\Rightarrow\end{math} Перед рисованием с помощью фреймбуфера надо подумать о viewport'е
\pause
\item N.B. \verb|glClear| игнорирует viewport
\end{itemize}
\end{frame}

\begin{frame}[fragile]
\frametitle{Рисование в текстуру: код}
\fontsize{8pt}{8pt}
\begin{verbatim}
GLuint fbo;
glGenFramebuffers(1, &fbo);

// создаём текстуру
...

// привязываем текстуру
glBindFramebuffer(GL_DRAW_FRAMEBUFFER, fbo);
glFramebufferTexture(GL_DRAW_FRAMEBUFFER, GL_COLOR_ATTACHMENT0,
    fbo_color_texture, 0);

// проверяем completeness
if (glCheckFramebufferStatus(GL_DRAW_FRAMEBUFFER) != GL_FRAMEBUFFER_COMPLETE)
    throw std::runtime_error("Framebuffer incomplete");

...

// Рисование в FBO
glBindFramebuffer(GL_DRAW_FRAMEBUFFER, fbo);
glClear(GL_COLOR_BUFFER_BIT);
glViewport(0, 0, fbo_width, fbo_height);
drawSomething();

// Рисование в дефолтный фреймбуфер
// здесь можно использовать текстуру fbo_color_texture
glBindFramebuffer(GL_DRAW_FRAMEBUFFER, 0);
glClear(GL_COLOR_BUFFER_BIT);
glViewport(0, 0, screen_width, screen_height);
drawSomethingElse();
\end{verbatim}
\end{frame}

\begin{frame}[fragile]
\frametitle{Renderbuffers}
\begin{itemize}
\item Объекты OpenGL, хранящие пиксели
\pause
\item Нельзя загрузить данные, нет режимов фильтрации, нет mipmaps
\pause
\item Могут быть attachment'ами для фреймбуфера, вместо текстур
\end{itemize}
\end{frame}

\begin{frame}[fragile]
\frametitle{Renderbuffers}
\begin{itemize}
\item Создать/удалить: \verb|glGenRenderbuffers/glDeleteRenderbuffers|
\pause
\item Сделать текущим: \verb|glBindRenderbuffer|, target = \verb|GL_RENDERBUFFER|
\pause
\item Выделить память: \verb|glRenderbufferStorage|
\pause
\item Привязать renderbuffer к фреймбуферу: \verb|glFramebufferRenderbuffer|
\end{itemize}
\end{frame}

\begin{frame}[fragile]
\frametitle{Framebuffers \& renderbuffers: ссылки}
\begin{itemize}
\item \href{https://www.khronos.org/opengl/wiki/Framebuffer_Object}{www.khronos.org/opengl/wiki/Framebuffer\_Object}
\item \href{https://www.khronos.org/opengl/wiki/Renderbuffer_Object}{www.khronos.org/opengl/wiki/Renderbuffer\_Object}
\item \href{http://www.opengl-tutorial.org/intermediate-tutorials/tutorial-14-render-to-texture}{opengl-tutorial.org/intermediate-tutorials/tutorial-14-render-to-texture}
\item \href{https://learnopengl.com/Advanced-OpenGL/Framebuffers}{learnopengl.com/Advanced-OpenGL/Framebuffers}
\end{itemize}
\end{frame}

\begin{frame}<1-2>[fragile,label=fb_examples]
\frametitle{Примеры использования render to texture}
\begin{itemize}
\item Пост-обработка кадра
\pause
\begin{itemize}
\item Размытие
\pause
\item Свечение (bloom)
\pause
\item Toon shading (edge detection, color grading)
\pause
\item HDR, гамма-коррекция
\pause
\item Сглаживание (FXAA)
\pause
\item И т.д.
\end{itemize}
\pause
\item Тени (shadow maps)
\pause
\item Отражения (как environment maps, но рисующиеся в реальном времени)
\pause
\item Deferred shading
\pause
\item И т.д.
\end{itemize}
\end{frame}

\begin{frame}[fragile]
\frametitle{Размытие}
\slideimage{blur.png}
\end{frame}

\againframe<3>{fb_examples}

\begin{frame}[fragile]
\frametitle{Свечение}
\slideimage{bloom.png}
\end{frame}

\againframe<4>{fb_examples}

\begin{frame}[fragile]
\frametitle{Toon shading}
\slideimage{toon-shading.jpg}
\end{frame}

\againframe<5-6>{fb_examples}

\begin{frame}[fragile]
\frametitle{FXAA}
\slideimage{fxaa.jpg}
\end{frame}

\againframe<7-8>{fb_examples}

\begin{frame}[fragile]
\frametitle{Shadow mapping}
\slideimage{shadow-maps.png}
\end{frame}

\againframe<9->{fb_examples}

\begin{frame}[fragile]
\frametitle{Размытие}
\begin{itemize}
\item Усреднение значений соседних пикселей
\pause
\item Box blur: все веса одинаковые, получается слегка угловатым
\pause
\item Gaussian blur: веса пропорциональны гауссиане \begin{math}\exp\left(-\frac{x^2+y^2}{r^2}\right)\end{math}, получается равномерное сглаживание
\pause
\item Несколько итераций box blur похожи на один gaussian blur
\pause
\item Обычно gaussian blur делают в два прохода: один размывает по горизонтали, второй - по вертикали
\end{itemize}
\end{frame}

\begin{frame}[fragile]
\frametitle{Box blur}
\fontsize{10pt}{10pt}
\begin{verbatim}
uniform sampler2D source;

in vec2 texcoord;

out vec4 out_color;

void main()
{
    vec4 sum = vec4(0.0);
    const int N = 5;

    for (int x = -N; x <= N; ++x) {
        for (int y = -N; y <= N; ++y) {
            sum += texture(source, texcoord +
                vec2(x,y) / vec2(textureSize(source)));
        }
    }

    out_color = sum / float((2*N+1)*(2*N+1));
}
\end{verbatim}
\end{frame}

\begin{frame}[fragile]
\frametitle{Gaussian blur}
\fontsize{10pt}{10pt}
\begin{verbatim}
uniform sampler2D source;

in vec2 texcoord;

out vec4 out_color;

void main()
{
    vec4 sum = vec4(0.0);
    float sum_w = 0.0;
    const int N = 5;
    float radius = 3.0;

    for (int x = -N; x <= N; ++x) {
        for (int y = -N; y <= N; ++y) {
            float c = exp(-float(x*x + y*y) / (radius*radius));
            sum += c * texture(source, texcoord +
                vec2(x,y) / vec2(textureSize(source)));
            sum_w += c;
        }
    }

    out_color = sum / sum_w;
}
\end{verbatim}
\end{frame}

\end{document}