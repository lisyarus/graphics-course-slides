% (c) Nikita Lisitsa, lisyarus@gmail.com, 2023

\documentclass[10pt]{beamer}

\usepackage[T2A]{fontenc}
\usepackage[russian]{babel}
\usepackage{minted}

\usepackage{graphicx}
\graphicspath{ {./images/} }

\usepackage{adjustbox}

\usepackage{color}
\usepackage{soul}

\usepackage{hyperref}

\usepackage{amsmath}

\usepackage{tikz}
\usetikzlibrary{decorations}
\usetikzlibrary{decorations.pathreplacing}
\usepackage{xifthen}

\usetheme{metropolis}
\setminted{fontsize=\footnotesize}

\definecolor{red}{rgb}{1,0,0}
\definecolor{green}{rgb}{0,0.5,0}
\definecolor{blue}{rgb}{0,0,1}
\definecolor{magenta}{rgb}{0.75,0,0.75}

\definecolor{codebg}{RGB}{29,35,49}

\makeatletter
\newcommand{\slideimage}[1]{
  \begin{figure}
    \begin{adjustbox}{width=\textwidth, totalheight=\textheight-2\baselineskip-2\baselineskip,keepaspectratio}
      \includegraphics{#1}
    \end{adjustbox}
  \end{figure}
}
\makeatother

\title{Компьютерная графика}
\subtitle{Лекция 13: Анимации, easing functions, keyframes, bitmap-анимации, кватернионы, иерархии объектов, скелетная анимация, форматы 3D моделей}
\date{2023}

\setbeamertemplate{footline}[frame number]

\usemintedstyle{solarized-light}

\begin{document}

\frame{\titlepage}

\begin{frame}[fragile]
\frametitle{Анимация}
\begin{itemize}
\item Анимация (в общем смысле) -- что угодно, меняющееся со временем
\pause
\begin{itemize}
\item Двигающийся объект
\pause
\item Анимированный элемент интерфейса
\pause
\item Анимированная модель
\pause
\item Движущаяся камера
\pause
\item etc.
\end{itemize}
\pause
\item Сводится к вопросу о том, что и как мы меняем в зависимости от времени
\end{itemize}
\end{frame}

\begin{frame}[fragile]
\frametitle{Анимация}
\begin{itemize}
\item Зачем нужна анимация?
\pause
\begin{itemize}
\item Часто в ней заключается суть (напр. почти все игры)
\pause
\item Анимация выглядит приятнее и понятнее, чем дискретное изменение состояния (напр. в UI)
\end{itemize}  
\end{itemize}
\end{frame}

\begin{frame}[fragile]
\frametitle{Анимация: общие идеи}
\begin{itemize}
\item Вычисление анимаций лучше привязывать к реальному времени, а не к номеру кадра (напр. \mintinline{cpp}|state += speed * dt| а не \mintinline{cpp}|state += speed|)
\pause
\begin{itemize}
\item Будет лучше работать при выключенном VSync
\pause
\item Будет лучше работать, когда CPU/GPU не справляются с нагрузкой
\end{itemize}
\pause
\item В общем случае анимация -- это зависимость какой-то величины от времени \begin{math}x = f(t)\end{math}
\pause
\item Как задать функцию \begin{math}f(t)\end{math}?
\pause
\begin{itemize}
\item Явной формулой
\pause
\item Склеить из кусочков (сплайн)
\pause
\item Вычислять неявно на основе текущего состояния
\end{itemize}
\end{itemize}
\end{frame}

\begin{frame}[fragile]
\frametitle{Анимация одной величины}
\begin{itemize}
\item Пусть у нас есть некая величина \mintinline{cpp}|x|, и мы хотим плавно поменять её значение на \mintinline{cpp}|new_x|
\pause
\item \textbf{Вариант 1}: запомнить старое и новое значения, интерполировать между ними учитывая прошедшее время \mintinline{cpp}|x = lerp(old_x, new_x, time)|
\pause
\begin{itemize}
\item {\color{green}+} Легко настраивать форму интерполяции (\textit{easing functions})
\pause
\item {\color{red}---} Нужно обрабатывать ситуацию, когда новое значение изменилось в процессе анимации
\end{itemize}
\end{itemize}
\end{frame}

\begin{frame}[fragile]
\frametitle{Анимация одной величины}
\begin{itemize}
\item Пусть у нас есть некая величина \mintinline{cpp}|x|, и мы хотим плавно поменять её значение на \mintinline{cpp}|new_x|
\pause
\item \textbf{Вариант 2}: обновлять \mintinline{cpp}|x| \textit{только} на основе нового значения: \mintinline{cpp}|x = lerp(x, new_x, speed * dt)|
\begin{itemize}
\item Если \mintinline{cpp}|dt > 1/speed|, возникнет нестабильность
\pause
\item Более точная, стабильная формула: \mintinline{cpp}|x = lerp(x, new_x, 1 - exp(- speed * dt))|
\pause
\item Для точного вычисления \mintinline{cpp}|exp(x)-1| для маленьких значений \mintinline{cpp}|x| есть функция \mintinline{cpp}|std::expm1|
\pause
\item Решение дифференциального уравнения \begin{math}\frac{dx}{dt} = speed \cdot (x_{new} - x)\end{math}
\pause
\item Удобно для анимации камеры, элементов интерфейса
\pause
\item \href{https://lisyarus.github.io/blog/programming/2023/02/21/exponential-smoothing.html}{\texttt{Подробная статья про этот метод}}
\end{itemize}
\end{itemize}
\end{frame}

\begin{frame}[fragile]
\frametitle{Анимации: easing functions}
\begin{itemize}
\item При временнóй интерполяции между двумя значениями вместо линейной интерполяции \mintinline{cpp}|x = lerp(old_x, new_x, t)| можно использовать т.н. easing functions
\pause
\item Применяются к параметру интерполяции \begin{math}t \in [0, 1]\end{math} и сглаживают анимацию: \mintinline{cpp}|x = lerp(old_x, new_x, easing(t))|
\pause
\item Примеры easing functions:
\begin{itemize}
\item \begin{math}f(t) = t\end{math}
\item \begin{math}f(t) = 3t^2-2t^3\end{math}
\item \begin{math}f(t) = t^2\end{math}
\item \begin{math}f(t) = 1 - (1-t)^2\end{math}
\item \begin{math}f(t) = \sqrt{t}\end{math}
\item Больше примеров: \href{https://easings.net}{\texttt{easings.net}}
\end{itemize}
\end{itemize}
\end{frame}

\begin{frame}[fragile]
\frametitle{Анимация: easing functions}
\slideimage{easing.png}
\end{frame}

\begin{frame}[fragile]
\frametitle{Анимация: сплайны}
\begin{itemize}
\item Часто значения интерполируют, используя сплайны: кривые, позволяющие удобно настраивать зависимость некой величины от параметра \mintinline{cpp}|t|
\pause
\item Обычно сплайн строится по набору значений в точках (keyframes) и, возможно, значений производных в этих точках
\pause
\item Виды сплайнов:
\begin{itemize}
\item Сплайны Безье
\item Кубические сплайны
\item B-сплайны
\item NURBS
\item etc.
\end{itemize}
\pause
\item \alert{\textbf{NB}}: спектр применения сплайнов \underline{не ограничивается} анимациями!
\begin{itemize}
\item Curve fitting
\item Представление сложных геометрических форм (напр. зданий, шрифтов)
\item etc.
\end{itemize}
\end{itemize}
\end{frame}

\begin{frame}[fragile]
\frametitle{Анимация: сплайны}
\slideimage{spline-editing.png}
\end{frame}

\begin{frame}[fragile]
\frametitle{Bitmap-анимации}
\begin{itemize}
\item Меняющееся со временем изображение
\pause
\item 3D текстура
\begin{itemize}
\item Номер кадра -- 3я текстурная координата (нормированная)
\item Сама интерполирует между кадрами
\end{itemize}
\pause
\item 2D array текстура
\begin{itemize}
\item Номер кадра -- 3я текстурная координата (\textbf{не} нормированная)
\item \textbf{Не} интерполирует сама между кадрами
\end{itemize}
\pause
\item 2D текстурный атлас -- текстура, хранящая несколько изображений бок о бок
\begin{itemize}
\item По номеру кадра вычисляются настоящие текстурные координаты
\item \textbf{Не} интерполирует сам между кадрами
\end{itemize}
\end{itemize}
\end{frame}

\begin{frame}[fragile]
\frametitle{Текстура-атлас с анимацией}
\slideimage{atlas1.jpg}
\end{frame}

\begin{frame}[fragile]
\frametitle{Текстура-атлас с анимацией}
\slideimage{atlas2.jpg}
\end{frame}

\begin{frame}[fragile]
\frametitle{Анимирование вращений}
\begin{itemize}
\item Обычно мы представляли вращения матрицами:
\begin{itemize}
\item Матрица \begin{math}3\times 3\end{math} -- 9 значений, это много
\pause
\item Применить матрицу к вектору -- минимум операций сложения и умножения
\pause
\item Композиция вращений -- произведение матриц, довольно быстрая операция
\pause
\item Интерполяция между двумя матрицами вращения -- почти всегда \textbf{не} матрица вращения
\end{itemize}
\pause
\item Вращения образуют 3х-мерную группу \begin{math}\mathbb{SO}(3)\end{math}, т.е. описываются 3-мя параметрами, например углами Эйлера
\pause
\begin{itemize}
\item Применить вращение, выраженное через углы Эйлера -- много тригонометрических функций, медленно
\pause
\item Композиция таких вращений -- очень сложная операция, много тригонометрических функций
\end{itemize}
\pause
\item Хочется компромисс между сложностью, объёмом хранения и удобством использования
\end{itemize}
\end{frame}

\begin{frame}[fragile]
\frametitle{Кватернионы}
\begin{itemize}
\item Кватернионы \begin{math}\mathbb H\end{math} -- четырёхмерная \textit{некоммутативная} алгебра над вещественными числами с тремя мнимыми единицами \begin{math}i, j, k\end{math}
\item Каждый элемент \begin{math}q\in \mathbb H\end{math} представляется в виде \begin{math}q = a + bi + cj + dk\end{math}, где \begin{math}a, b, c, d \in \mathbb R\end{math} -- коэффициенты кватерниона
\pause
\item Правила умножения:
\begin{itemize}
\item \begin{math}i^2=j^2=k^2=-1\end{math}
\item \begin{math}ij=-ji=k\end{math}
\item \begin{math}jk=-kj=i\end{math}
\item \begin{math}ki=-ik=j\end{math}
\end{itemize}
\end{itemize}
\end{frame}

\begin{frame}[fragile]
\frametitle{Кватернионы}
\begin{itemize}
\item Сопряжённый кватернион определяется как \begin{math}\overline q = a - bi - cj - dk\end{math}
\item Норма кватерниона: число \begin{math}q \cdot \overline q = a^2 + b^2 + c^2 + d^2 = |q|^2\end{math}
\item Обратный кватернион: \begin{math}q^{-1} = \frac{1}{|q|^2} \overline q\end{math}
\item Единичный кватернион: \begin{math}|q| = 1\end{math}
\item Единичные кватернионы образуют трёхмерную сферу \begin{math}\mathbb{S}^3\end{math} (вложенную в четырёхмерное пространство \begin{math}\mathbb{R}^4 \cong \mathbb{H}\end{math})
\end{itemize}
\end{frame}

\begin{frame}[fragile]
\frametitle{Кватернионы: альтернативное представление}
\begin{itemize}
\item Для кватерниона \begin{math}q = a + bi + cj + dk\end{math} назовём его скалярной частью число \begin{math}a\end{math}, а векторной частью -- вектор \begin{math}v = (b, c, d)\end{math}
\item Кватернион -- пара скаляр + вектор: \begin{math}q = (a, v)\end{math}
\pause
\item Сопряжённый кватернион: \begin{math}(a, -v)\end{math}
\pause
\item Произведение кватернионов: \begin{math}(a_1, v_1) \cdot (a_2, v_2) = (a_1 \cdot a_2 - v_1 \cdot v_2, a_1 v_2 + a_2 v_1 + v_1 \times v_2)\end{math}
\pause
\item В таком виде кватернионы удобно реализовывать в шейдерах
\end{itemize}
\end{frame}

\begin{frame}[fragile]
\frametitle{Кватернионы: вращения}
\begin{itemize}
\item Представим произвольный трёхмерный вектор \begin{math}v\end{math} как кватернион с нулевой скалярной частью \begin{math}(0, v)\end{math}
\pause
\item Вращение вокруг оси \begin{math}w\end{math} (единичный вектор) на угол \begin{math}\theta\end{math} можно реализовать как \begin{math}q \cdot (0, v) \cdot q^{-1}\end{math}, где \begin{math}q = (\cos \frac{\theta}{2}, w \cdot \sin \frac{\theta}{2})\end{math}
\pause
\item \alert{\textbf{NB}}: \begin{math}|q| = 1\end{math}
\item \alert{\textbf{NB}}: \begin{math}q^{-1} = (\cos \frac{\theta}{2}, - w \cdot \sin \frac{\theta}{2})\end{math}
\pause
\item Любое вращение представляется единичным кватернионом, и любой единичный кватернион описывает вращение
\pause
\item \begin{math}q\end{math} и \begin{math}-q\end{math} описывают одно и то же вращение (и только они)
\end{itemize}
\end{frame}

\begin{frame}[fragile]
\frametitle{Кватернионы: вращения}
\begin{itemize}
\item Для вращения нужны только алгебраические операции \begin{math}\Longrightarrow\end{math} быстро!
\pause
\item Композиция вращений \textendash{} произведение кватернионов: \begin{math}{\color{blue}q_2} \cdot ({\color{red}q_1} \cdot (0, v) \cdot {\color{red}q_1^{-1}}) \cdot {\color{blue}q_2^{-1}} = ({\color{blue}q_2} \cdot {\color{red}q_1}) \cdot (0, v) \cdot ({\color{red}q_1^{-1}} \cdot {\color{blue}q_2^{-1}}) = ({\color{blue}q_2}{\color{red}q_1}) \cdot (0, v) \cdot ({\color{blue}q_2}{\color{red}q_1})^{-1}\end{math}
\pause
\item \underline{Стандартный способ} представления вращений объектов в 3D движках
\end{itemize}
\end{frame}

\begin{frame}[fragile]
\frametitle{Сферическая интерполяция}
\begin{itemize}
\item Линейная интерполяция двух единичных кватернионов -- почти всегда \textbf{не} единичный кватернион
\pause
\item Можно отнормировать результат, но это не будет соответствовать равномерной интерполяции
\pause
\item Правильный способ: использовать геодезическую кривую на поверхности сферы
\pause
\item Эта операция называется \mintinline{cpp}|slerp| (\textit{spherical linear interpolation}), имеет явную формулу и реализована в большинстве математических библиотек (в т.ч. \mintinline{cpp}|glm|)
\end{itemize}
\end{frame}

\begin{frame}[fragile]
\frametitle{Кватернионы: представление в коде}
\begin{itemize}
\item Есть два варианта представить кватернион \begin{math}q = w + xi + yj + zk\end{math}:
\pause
\begin{itemize}
\item Как четырёхмерный вектор \mintinline{cpp}|[w, x, y, z]| -- логичнее с математической точки зрения
\item Как четырёхмерный вектор \mintinline{cpp}|[x, y, z, w]| -- удобнее работать в GLSL (и других шейдерных языках)
\end{itemize}
\pause
\item Общепринятого варианта нет
\item В библиотеке \mintinline{cpp}|glm| -- \mintinline{cpp}|[w, x, y, z]| (есть \alert{\textbf{наполовину сломанная}} поддержка \mintinline{cpp}|[x, y, z, w]|)
\item Формат моделей \mintinline{cpp}|glTF| описывает вращения как \mintinline{cpp}|[x, y, z, w]|
\end{itemize}
\end{frame}

\begin{frame}[fragile]
\frametitle{Кватернионы: ссылки}
\fontsize{8pt}{8pt}
\selectfont
\begin{itemize}
\item \href{https://en.wikipedia.org/wiki/Quaternion}{\texttt{en.wikipedia.org/wiki/Quaternion}}
\item \href{https://en.wikipedia.org/wiki/Quaternions_and_spatial_rotation}{\texttt{en.wikipedia.org/wiki/Quaternions\_and\_spatial\_rotation}}
\item \href{https://en.wikipedia.org/wiki/Rotation_formalisms_in_three_dimensions}{\texttt{en.wikipedia.org/wiki/Rotation\_formalisms\_in\_three\_dimensions}}
\item \href{https://en.wikipedia.org/wiki/Slerp}{\texttt{en.wikipedia.org/wiki/Slerp}}
\end{itemize}
\end{frame}

\begin{frame}[fragile]
\frametitle{Преобразования объектов}
\begin{itemize}
\item Часто для задания преобразования, применяемого к объекту, нам не нужна целиком матрица аффинного преобразования
\pause
\item Обычно это масштабирование + поворот + сдвиг (в таком порядке!)
\pause
\item Масштабирование -- одно число \begin{math}s\end{math} (\textit{изотропное масштабирование}) или три числа (разный масштаб по разным осям)
\pause
\item Поворот -- кватернион \begin{math}q\end{math}
\pause
\item Сдвиг -- вектор сдвига \begin{math}t\end{math}
\pause
\item Преобразование вершин: \begin{math}v \mapsto q(s\cdot v)q^{-1} + t\end{math}
\pause
\item Преобразование нормалей: \begin{math}n \mapsto qnq^{-1}\end{math} либо \begin{math}n \mapsto \frac{q(s^{-1}\cdot n)q^{-1}}{\|q(s^{-1}\cdot n)q^{-1}\|}\end{math}
\end{itemize}
\end{frame}

\begin{frame}[fragile]
\frametitle{Преобразование нормалей}
\begin{itemize}
\item В общем случае, если вершины модели преобразуются какой-то матрицей \begin{math}M\end{math}, нормали могут \textbf{перестать} быть перпендикулярны к поверхности
\pause
\item Пусть \begin{math}n\end{math} -- нормаль к точке некой гладкой поверхности, и \begin{math}v\end{math} -- любой касательный вектор к поверхности в этой точке
\pause
\item Из ортогональности \begin{math}n\cdot v = 0\end{math} не следует, что \begin{math}(Mn)\cdot (Mv)=0\end{math}!
\pause
\item Но, по свойствам матриц и скалярного произведения \begin{math}0 = n\cdot v = n\cdot (M^{-1}M)v = n\cdot M^{-1}(Mv) = (M^{-1})^T n \cdot (Mv)\end{math}
\pause
\item Таким образом, вектор \begin{math}M^{-T} n\end{math} будет перпендикулярен преобразованной поверхности
\pause
\item \begin{math}\Longrightarrow\end{math} В общем случае нормали нужно преобразовывать матрицей \begin{math}M^{-T}\end{math}
\end{itemize}
\end{frame}

\begin{frame}[fragile]
\frametitle{Преобразование нормалей}
\begin{itemize}
\item Формула \begin{math}M^{-T}\end{math} требует обращения матрицы (дорого) и нестабильна для преобразований с маленьким определителем (например, масштаб по одной из осей близок к нулю)
\pause
\item Обратную матрицу \begin{math}A^{-1}\end{math} можно посчитать по формуле \begin{math}\frac{1}{\operatorname{det}A}\operatorname{adj}{A}\end{math}, где \begin{math}\operatorname{adj}{A}\end{math} -- \textit{присоединённая матрица}
\pause
\item \begin{math}\operatorname{adj}{A}=(\operatorname{cof}{A})^T\end{math}, где \begin{math}\operatorname{cof}{A}\end{math} -- матрица \textit{алгебраических дополнений}
\pause
\item Подставляем \begin{math}A=M^T\end{math}:
\begin{equation*}
M^{-T} = \frac{1}{\operatorname{det} M^T}\operatorname{adj}(M^T) = \frac{1}{\operatorname{det}M}\operatorname{cof}M
\end{equation*}
\pause
\item Определитель \begin{math}\operatorname{det}M\end{math} можно игнорировать: это просто число, оно исчезнет, когда мы отнормируем получившийся вектор нормали
\pause
\item Остаётся только \begin{math}\operatorname{cof}M\end{math}: она требует меньше вычислений, чем \begin{math}M^{-T}\end{math}, и не чувствительна к вырожденным преобразованиям
\end{itemize}
\end{frame}

\begin{frame}[fragile]
\frametitle{Преобразование нормалей: частные случаи}
\begin{itemize}
\item Если \begin{math}M\end{math} -- матрица вращения, то \begin{math}M^T = M^{-1}\end{math} и \begin{math}M^{-T}=M\end{math}, т.е. нормали вращаются так же, как сами вершины
\pause
\item Если \begin{math}M\end{math} -- диагональная матрица неизотропного масштабирования, то \begin{math}M^T = M\end{math} и \begin{math}M^{-T}=M^{-1}\end{math}, т.е. к нормалям применяется масштаб, обратный масштабу вершин 
\end{itemize}
\end{frame}

\begin{frame}[fragile]
\frametitle{Иерархия объектов}
\begin{itemize}
\item Часто объекты сцены/мира образуют иерархическую структуру (шапка на человеке, человек в машине, машина на корабле)
\pause
\item Удобно описывать полное преобразование объекта (позиция + поворот + масштабирование) не относительно центра сцены/мира, а относительно родительского объекта
\pause
\item Обычно в такой ситуации есть один корневой объект -- сцена
\pause
\item \begin{math}\Longrightarrow\end{math} Нужно уметь вычислять итоговое преобразование объекта, т.е. композицию всех преобразований от корня до нашего объекта
\end{itemize}
\end{frame}

\begin{frame}[fragile]
\frametitle{Композиция аффинных преобразований}
\begin{equation}
({\color{blue}t_2},{\color{blue}q_2},{\color{blue}s_2}) \cdot ({\color{red}t_1},{\color{red}q_1},{\color{red}s_1}) \cdot v = ({\color{green}t},{\color{green}q},{\color{green}s}) \cdot v 
\end{equation}
\begin{multline}
({\color{blue}t_2},{\color{blue}q_2},{\color{blue}s_2}) \cdot ({\color{red}t_1},{\color{red}q_1},{\color{red}s_1}) \cdot v = {\color{blue}q_2}{\color{blue}s_2}({\color{red}q_1}{\color{red}s_1}v{\color{red}q_1^{-1}} + {\color{red}t_1}){\color{blue}q_2^{-1}} + {\color{blue}t_2} = \\
= {\color{blue}s_2}{\color{red}s_1} ({\color{blue}q_2}{\color{red}q_1}) v ({\color{blue}q_2}{\color{red}q_1})^{-1} + {\color{blue}s_2}{\color{blue}q_2}{\color{red}t_1}{\color{blue}q_2^{-1}} + {\color{blue}t_2} = \\
({\color{blue}s_2}{\color{blue}q_2}{\color{red}t_1}{\color{blue}q_2^{-1}} + {\color{blue}t_2}, {\color{blue}q_2}{\color{red}q_1}, {\color{blue}s_2}{\color{red}s_1}) \cdot v \\
\end{multline}
\begin{equation}
{\color{green}q} = {\color{blue}q_2}{\color{red}q_1}
\end{equation}
\begin{equation}
{\color{green}s} = {\color{blue}s_2}{\color{red}s_1}
\end{equation}
\begin{equation}
{\color{green}t} = {\color{blue}s_2}{\color{blue}q_2}{\color{red}t_1}{\color{blue}q_2^{-1}} + {\color{blue}t_2}
\end{equation}
\begin{itemize}
\item Работает только для \underline{изотропных} масштабирований!
\end{itemize}
\end{frame}

\begin{frame}[fragile]
\frametitle{Анимация трёхмерных моделей}
\begin{itemize}
\item Анимация положения объекта в пространстве -- неплохо, но скучно
\pause
\item Хочется анимировать саму модель, т.е. двигать её вершины
\pause
\item 2 способа:
\begin{itemize}
\item Покадровая анимация (\textit{morph-target animation})
\item Скелетная анимация (\textit{skeletal animation})
\end{itemize}
\end{itemize}
\end{frame}

\begin{frame}[fragile]
\frametitle{Покадровая анимация моделей}
\begin{itemize}
\item Анимируются в явном виде все вершины по отдельности
\pause
\item Анимация хранится в виде `кадров': фиксированных состояний модели (наборов координат вершин)
\pause
\item Вершинный шейдер принимает два набора атрибутов позиций вершин и интерполирует между ними
\begin{itemize}
\item Интерполяция может использовать easing functions
\end{itemize}
\pause
\item Фактически, это сплайн, значение которого -- набор координат всех вершин
\end{itemize}
\end{frame}

\begin{frame}[fragile]
\frametitle{Покадровая анимация моделей}
\slideimage{keyframes.jpg}
\end{frame}

\begin{frame}[fragile]
\frametitle{Покадровая анимация моделей}
\begin{itemize}
\item Много вариантов реализации:
\begin{itemize}
\item При смене кадра анимации загружать в VBO новые данные
\item При смене кадра менять VBO/VAO
\item Хранить кадры отдельно (например, в buffer textures), передавать в шейдер только номер кадра
\end{itemize}
\pause
\item Много проблем:
\pause
\begin{itemize}
\item {\color{red}---} Сложно модифицировать: нужно двигать \textit{все} вершины модели
\pause
\item {\color{red}---} Сложно добавлять процедурную анимацию (напр. поворачивать голову в нужную сторону)
\pause
\item {\color{red}---} Требует много памяти
\pause
\item {\color{red}---} Для хорошего качества нужно много кадров, иначе будут артефакты интерполяции (напр. модель начнёт пересекать саму себя)
\end{itemize}
\pause
\item Обычно \textbf{не} используется для 3D моделей
\end{itemize}
\end{frame}

\begin{frame}[fragile]
\frametitle{Скелетная анимация моделей}
\begin{itemize}
\item Модель привязывается к виртуальному \textit{скелету}
\pause
\item Скелет -- иерархия виртуальных \textit{костей}
\pause
\item Каждая вершина привязана к одной или (чаще) нескольким костям
\pause
\item Каждой паре вершина-кость соответствует некоторый вес: насколько эта кость влияет на эту вершину (сумма весов для одной вершины должна равняться 1)
\pause
\item Кадры анимации задаются только для скелета (исходная модель существует в одном экземпляре)
\pause
\item Интерполируются только преобразования костей (костей гораздо меньше, чем вершин \begin{math}\Longrightarrow\end{math} это не страшно делать даже на CPU)
\pause
\item В вершинном шейдере вычисляется итоговое преобразование для вершины как среднее между преобразованиями связанных с ней костей
\end{itemize}
\end{frame}

\begin{frame}[fragile]
\frametitle{Скелетная анимация моделей}
\slideimage{skeletal.jpg}
\end{frame}

\begin{frame}[fragile]
\frametitle{Скелетная анимация моделей}
\usemintedstyle{lightbulb}
\begin{minted}[bgcolor=codebg]{glsl}
uniform mat4x4 bones[16];

layout (location = 0) in vec4 in_position;
layout (location = 1) in ivec2 in_joints;
layout (location = 2) in vec2 in_weights;

void main()
{
  gl_Position =
      in_weights.x * bones[in_joints.x] * in_position
    + in_weights.y * bones[in_joints.y] * in_position;
}
\end{minted}
\end{frame}

\begin{frame}[fragile]
\frametitle{Скелетная анимация моделей}
\begin{itemize}
\item К нормалям тоже нужно применять преобразования (но не сдвиги!)
\pause
\item Преобразования костей часто заданы в локальной для кости системе координат для удобства \begin{math}\Longrightarrow\end{math} перед применением нужно умножить на преобразование, переводящее из системы координат модели в локальную систему координат кости (\textit{inverse bind matrix} в формате \mintinline{cpp}|glTF|)
\pause
\item Кости обычно тоже выстроены в иерархию \begin{math}\Longrightarrow\end{math} перед применением нужно вычислить суммарное преобразование (композицию)
\end{itemize}
\end{frame}

\begin{frame}[fragile]
\frametitle{Скелетная анимация моделей}
\begin{itemize}
\item Преобразования для кости и для родительской кости заданы в разных локальных системах координат \begin{math}\Longrightarrow\end{math} в композицию преобразований нужно вставить смену системы координат
\pause
\item В формате glTF считается, что смена системы координат от дочерней к родительской кости \textit{встроена в саму анимацию}
\pause
\item \begin{math}\Longrightarrow\end{math} нужно откуда-то взять преобразования для костей, если они не описаны в анимации (в glTF это исходные преобразования соответствующих нод)
\pause
\item Нужно быть внимательным к особенностям задания преобразований и системам координат в разных редакторах и форматах!
\end{itemize}
\end{frame}

\begin{frame}[fragile]
\frametitle{Скелетная анимация моделей}
\begin{itemize}
\item В формате glTF вычисление преобразования для кости выглядит как
\usemintedstyle{lightbulb}
\begin{minted}[bgcolor=codebg]{glsl}

    ...
    * bone.parent.parent.local_transform
    * bone.parent.local_transform *
    * bone.local_transform
    * bone.inverse_bind_matrix
\end{minted}
\end{itemize}
\end{frame}

\begin{frame}[fragile]
\frametitle{Скелетная анимация моделей}
\begin{itemize}
\item {\color{green}+} Удобно и интуитивно модифицировать (все 3D-редакторы имеют поддержку скелетных анимаций)
\pause
\item {\color{green}+} Небольшой расход памяти (модель не дублируется)
\pause
\item {\color{green}+} Легко добавлять процедурную анимацию
\pause
\item \underline{Самый распространённый} способ анимировать модели
\end{itemize}
\end{frame}

\begin{frame}[fragile]
\frametitle{Скелетная анимация}
\begin{itemize}
\item Откуда брать скелетную анимацию?
\pause
\begin{itemize}
\item Сплайны -- так обычно описываются анимации в 3D редакторах и форматах моделей (отдельный сплайн для вращения, масштаба и сдвига каждой кости)
\pause
\item Процедурная анимация -- код, генерирующий анимацию на лету
\end{itemize}
\end{itemize}
\end{frame}

\begin{frame}[fragile]
\frametitle{Inverse kinematics}
\begin{itemize}
\item Вычисление координат вершины по известным преобразованиям костей называется \textit{forward kinematics}
\pause
\item Часто хочется уметь решать обратную задачу: по финальному положению вершины вычислить подходящие преобразования костей
\pause
\begin{itemize}
\item Повернуть голову, чтобы она смотрела в нужном направлении
\pause
\item Поставить ногу на ландшафт с неизвестным заранее наклоном
\pause
\item Повернуть руку так, чтобы кисть схватила нужный объект
\end{itemize}
\pause
\item Удобно при формировании анимации в редакторе; необходимо для нетривиальных процедурных анимаций
\pause
\item Эта (обратная) задача называется \textit{inverse kinematics (IK)}
\end{itemize}
\end{frame}

\begin{frame}[fragile]
\frametitle{Inverse kinematics}
\begin{itemize}
\item Финальное положение точки -- функция от преобразований отдельных костей
\pause
\item Часто кости могут только вращаться (напр. тело человека), но не сдвигаться или масштабироваться
\pause
\item Тогда положение точки -- функция углов \begin{math}p = f(\theta_1, \theta_2, \dots \theta_n)\end{math}
\pause
\item \begin{math}p\end{math} известна, \begin{math}\{\theta_i\}\end{math} неизвестны \begin{math}\Longrightarrow\end{math} IK сводится к задаче решения нелинейной системы уравнений
\pause
\item В некоторых простых частных случаях можно решить явно (напр. поровот головы -- \mintinline{cpp}|atan2| от вектора направления)
\pause
\item В общем случае решается итеративными методами (напр. многомерным методом Ньютона)
\end{itemize}
\end{frame}

\begin{frame}[fragile]
\frametitle{Скелетная анимация}
\begin{itemize}
\item Если для модели задано несколько анимаций (ходьба, бег, прыжок, поворот, и т.д.) нужно уметь между ними переключаться
\pause
\item Обычно достаточно интерполировать \mintinline{cpp}|local_transform| каждой кости от первой ко второй анимации
\pause
\item Не все переходы имеют смысл (напр. люди обычно не переходят в лежачее положение сразу из бега или прыжка) \begin{math}\Longrightarrow\end{math} часто используют state-машины анимаций с настраиваемыми переходами между состояниями
\pause
\item В современности почти всегда используется комбинированный подход: state-машина для переходов между заранее подготовленными анимациями + специфичная для ситуации процедурная анимация
\end{itemize}
\end{frame}

\begin{frame}[fragile]
\frametitle{State-машина для анимации}
\slideimage{state-machine.png}
\end{frame}

\begin{frame}[fragile]
\frametitle{Скелетная анимация}
\begin{itemize}
\item Есть способы для некоторых ситуаций полностью процедурно генерировать анимацию
\pause
\item Анимация передвижения пауков: \href{https://www.youtube.com/watch?v=e6Gjhr1IP6w}{\texttt{youtube.com/watch?v=e6Gjhr1IP6w}}
\pause
\item Оффлайн генерация анимации движения для двуногих: \href{https://www.youtube.com/watch?v=pgaEE27nsQw}{\texttt{youtube.com/watch?v=pgaEE27nsQw}}
\end{itemize}
\end{frame}

\begin{frame}[fragile]
\frametitle{Скелетная анимация: ссылки}
\begin{itemize}
\item \href{https://en.wikipedia.org/wiki/Skeletal_animation}{\texttt{en.wikipedia.org/wiki/Skeletal\_animation}}
\item \href{https://en.wikipedia.org/wiki/Inverse_kinematics}{\texttt{en.wikipedia.org/wiki/Inverse\_kinematics}}
\item \href{https://learnopengl.com/Guest-Articles/2020/Skeletal-Animation}{\texttt{learnopengl.com/Guest-Articles/2020/Skeletal-Animation}}
\item \href{https://ogldev.org/www/tutorial38/tutorial38.html}{\texttt{ogldev.org/www/tutorial38/tutorial38.html}}
\item \href{https://lisyarus.github.io/blog/graphics/2023/07/03/gltf-animation.html}{\texttt{Skeletal animation in glTF}}
\item \href{https://www.youtube.com/watch?v=f3Cr8Yx3GGA}{\texttt{Видео-туториал по скелетной анимации}}
\item \href{https://www.youtube.com/watch?v=R-T3Mk5oDHI}{\texttt{Видео про графы анимаций в крупных движках}}
\end{itemize}
\end{frame}

\begin{frame}[fragile]
\frametitle{Форматы 3D моделей}
\begin{itemize}
\item Существует колоссально много форматов для 3D-моделей
\pause
\item Они отличаются
\pause
\begin{itemize}
\item Набором атрибутов вершин (вторые текстурные координаты, весы для анимации)
\pause
\item Поддержкой материалов и их сложностью
\pause
\item Поддержкой иерархии объектов и сцен
\pause
\item Поддержкой анимаций
\end{itemize}
\end{itemize}
\end{frame}

\begin{frame}[fragile]
\frametitle{Форматы 3D моделей}
\fontsize{8pt}{8pt}
\selectfont
\begin{itemize}
\item \mintinline{cpp}|.x| -- устаревший формат, часто использовавшийся вместе с DirectX
\pause
\item \mintinline{cpp}|.3ds| -- устаревший формат, использовавшийся в девяностых и нулевых, не поддерживал анимацию и нормали
\pause
\item \mintinline{cpp}|.obj| -- текстовый формат, использующийся из-за своей простоты, не поддерживает нестандартных атрибутов и анимацию
\pause
\item \mintinline{cpp}|.dae (COLLADA)| -- распространённый современный XML-формат, поддерживает почти всё
\pause
\begin{itemize}
\fontsize{8pt}{8pt}
\selectfont
\item Из-за многословности XML формат занимает очень много памяти; обычно используется как промежуточный перед конвертацией в специфичный для движка бинарный формат
\end{itemize}
\pause
\item \mintinline{cpp}|.gltf| -- современный JSON-формат, разработанный Khronos Group; поддерживает нетривиальные атрибуты, материалы и анимацию; в JSON хранится логическое описание данных, а сами бинарные данные (вершины, индексы, анимации) могут храниться в отдельных файлых; очень удобен для загрузки и использования в OpenGL
\end{itemize}
\end{frame}

\begin{frame}[fragile]
\frametitle{Форматы 3D моделей: ссылки}
\begin{itemize}
\item \href{https://www.khronos.org/collada}{\texttt{Документация по COLLADA}}
\item \href{https://registry.khronos.org/glTF/specs/2.0/glTF-2.0.html}{\texttt{Документация по glTF}}
\item \href{https://en.wikipedia.org/wiki/List_of_file_formats#3D_graphics}{\texttt{Список 3D форматов}}
\end{itemize}
\end{frame}

\end{document}