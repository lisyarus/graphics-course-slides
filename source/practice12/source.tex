% (c) Nikita Lisitsa, lisyarus@gmail.com, 2023

\documentclass[10pt]{beamer}

\usepackage[T2A]{fontenc}
\usepackage[russian]{babel}
\usepackage{minted}

\usepackage{graphicx}
\graphicspath{ {./images/} }

\usepackage{adjustbox}

\usetheme{metropolis}

\usepackage{color}
\usepackage{soul}

\usepackage{hyperref}

\definecolor{blue}{rgb}{0,0,1}
\definecolor{red}{rgb}{1,0,0}

\makeatletter
\newcommand{\slideimage}[1]{
  \begin{figure}
    \begin{adjustbox}{width=\textwidth, totalheight=\textheight-2\baselineskip-2\baselineskip,keepaspectratio}
      \includegraphics{#1}
    \end{adjustbox}
  \end{figure}
}
\makeatother

\title{Компьютерная графика}
\subtitle{Практика 12: Volume rendering}
\date{2023}

\usemintedstyle{solarized-light}

\setbeamertemplate{footline}[frame number]

\begin{document}

\frame{\titlepage}

\begin{frame}[fragile]
\slideimage{0.png}
\end{frame}

\begin{frame}[fragile]
\frametitle{Задание 1}
Находим пересечение с AABB объекта (во фрагментном шейдере)
\begin{itemize}
\item Вычисляем \textbf{нормированный} вектор направления из камеры в текущую точку поверхности -- это вектор направления луча
\item Вычисляем пересечение этого луча с AABB: интервал \begin{math}[t_{min}, t_{max}]\end{math} для которых \begin{math}p + t \cdot d\end{math} содержится в AABB (в коде уже есть функция \mintinline{glsl}|intersect_bbox|, возвращает \mintinline{glsl}|vec2(tmin, tmax)|)
\item Делаем \mintinline{glsl}|tmin = max(tmin, 0.0)|, чтобы не включать часть пересечения сзади камеры
\item В качестве цвета пикселя выводим \mintinline{glsl}|vec3(tmax - tmin)| (это значение часто будет больше единицы, так что можно разделить, например, на 4.0)
\end{itemize}
\end{frame}

\begin{frame}[fragile]
\slideimage{1.png}
\end{frame}

\begin{frame}[fragile]
\frametitle{Задание 2}
Вычисляем optical depth куба (во фрагментном шейдере)
\begin{itemize}
\item Заводим константу для коэффициента поглощения: \mintinline{glsl}|absorption = 1.0|
\item Вычисляем optical depth: \mintinline{glsl}|optical_depth = (tmax - tmin) * absorption|
\item Вычисляем непрозрачность пикселя: \mintinline{glsl}|opacity = 1.0 - exp(-optical_depth)|
\item Записываем значение opacity в альфа-канал результирующего цвета (RGB-каналы заполните вашим любимым цветом)
\item Можно поиграться со значением \mintinline{glsl}|absorption| чтобы понять, как оно влияет на результат
\end{itemize}
\end{frame}

\begin{frame}[fragile]
\slideimage{2.png}
\end{frame}

\begin{frame}[fragile]
\frametitle{Задание 3}
Загружаем 3D текстуру
\begin{itemize}
\item Создаём текстуру типа \mintinline{glsl}|GL_TEXTURE_3D|, min/mag фильтры -- \mintinline{glsl}|GL_LINEAR|
\item Устанавливаем параметры \mintinline{glsl}|WRAP_R|, \mintinline{glsl}|WRAP_S|, \mintinline{glsl}|WRAP_T| в \mintinline{glsl}|GL_CLAMP_TO_EDGE|
\item Считываем данные из файла \mintinline{glsl}|cloud_data_path| (128x64x64, одноканальная, 1 байт на пиксель):
\begin{itemize}
\item Заводим \mintinline{cpp}|std::vector<char> pixels(...)| нужного размера
\item Открываем файл \mintinline{cpp}|std::ifstream input(path, std::ios::binary)|
\item Читаем данные \mintinline{cpp}|input.read(pixels.data(), pixels.size())|
\end{itemize}
\item Загружаем в текстуру с помощью \mintinline{glsl}|glTexImage3D| (internal format -- \mintinline{glsl}|GL_R8|, format -- \mintinline{glsl}|GL_RED|, type -- \mintinline{glsl}|GL_UNSIGNED_BYTE|)
\end{itemize}
\end{frame}

\begin{frame}[fragile]
\frametitle{Задание 3}
Загружаем 3D текстуру
\begin{itemize}
\item Добавляем текстуру в шейдер (\mintinline{glsl}|uniform sampler3D|), выводим в качестве цвета значение из текстуры в точке \mintinline{glsl}|p = camera_position +|
\mintinline{glsl}|+ direction * (tmin + tmax) / 2.0|
\begin{itemize}
\item Нужно перевести пространственные координаты в текстурные: \mintinline{glsl}|(p - bbox_min) / (bbox_max - bbox_min)|
\item Удобно завести функцию, возвращающую значение из текстуры по точке в пространстве
\end{itemize}
\item В качестве альфа-канала возьмём 1, иначе ничего не увидим
\end{itemize}
\end{frame}

\begin{frame}[fragile]
\frametitle{Задание 3}
N.B. можно взять другую текстуру: \mintinline{glsl}|bunny.data| (есть в репозитории с заданием)
\begin{itemize}
\item Размер -- 64x64x64
\item \mintinline{glsl}|bbox_min = vec3(-1, -1, -1)|
\item \mintinline{glsl}|bbox_max = vec3( 1,  1,  1)|
\end{itemize}
\end{frame}

\begin{frame}[fragile]
\slideimage{3.png}
\end{frame}

\begin{frame}[fragile]
\frametitle{Задание 4}
Вычисляем optical depth с помощью front-to-back алгоритма (во фрагментном шейдере)
\begin{itemize}
\item Инициализируем \mintinline{glsl}|optical_depth = 0|
\item Делаем цикл, например, в 64 шага; один шаг цикла соответствует 1/64 части отрезка \mintinline{glsl}|dt = (tmax - tmin) / 64|
\begin{itemize}
\item Вместо 64 можно взять любое другое число; чем больше, тем красивее и медленнее
\item Каждой итерации \mintinline{glsl}|i| цикла соответвует значение \mintinline{glsl}|t = tmin + (i + 0.5) * dt|
\item Каждому значению \mintinline{glsl}|t| соответствует точка луча \mintinline{glsl}|p = camera_position + t * direction|
\item Берём плотность из текстуры в текущей точке \mintinline{glsl}|p|
\item Обновляем optical depth: \mintinline{glsl}|optical_depth += absorption * density * dt|
\end{itemize}
\item Вычисляем opacity как в задании 2
\end{itemize}
\end{frame}

\begin{frame}[fragile]
\slideimage{4.png}
\end{frame}

\begin{frame}[fragile]
\frametitle{Задание 5}
Вычисляем рассеяние (во фрагментном шейдере), считаем что фазовая функция не зависит от угла рассеяния (тогда \begin{math}f(p,\theta) = \frac{1}{4\pi}\end{math})
\begin{itemize}
\item Коэффициент поглощения можно сделать поменьше (или даже нулём)
\item Заводим коэффициенты рассеяния \mintinline{glsl}|scattering = 4.0| и вымирания \mintinline{glsl}|extinction = absorption + scattering|
\item Заводим интенсивность света \mintinline{glsl}|light_color = vec3(16.0)|
\item Инициализируем рассеянный свет \mintinline{glsl}|color = vec3(0.0)|
\item В цикле аккумулируем и optical depth, и рассеянный свет
\item \mintinline{glsl}|optical_depth += extinction * density * dt|
\end{itemize}
\end{frame}

\begin{frame}[fragile]
\frametitle{Задание 5}
Вычисляем рассеяние
\begin{itemize}
\item Для рассеяния нужно посчитать \mintinline{glsl}|light_optical_depth| аналогичным вложенным циклом (число итераций может быть другое, например 16) вдоль луча из текущей точки в направлении света \mintinline{glsl}|light_direction|
\begin{itemize}
\item Придётся вызвать \mintinline{glsl}|intersect_bbox| на каждую итерацию внешнего цикла
\item Придётся читать из текстуры на каждую итерацию внутреннего цикла
\end{itemize}
\item Обновляем рассеянный свет как \mintinline{glsl}|color += light_color * exp(-light_optical_depth) *|
\mintinline{glsl}|* exp(-optical_depth) * dt * density *|
\mintinline{glsl}|* scattering / 4.0 / PI|
\item В качестве результата шейдера выводим \mintinline{glsl}|vec4(color, alpha)|
\end{itemize}
\end{frame}

\begin{frame}[fragile]
\slideimage{5.png}
\end{frame}

\begin{frame}[fragile]
\frametitle{Задание 6*}
Разные коэффициенты рассеяния для разных цветов
\begin{itemize}
\item Обычный блендинг не умеет делать альфа-канал для каждого цвета по отдельности, так что заменим цвет фона (\mintinline{glsl}|glClearColor|) на чёрный
\item Коэффициенты \mintinline{glsl}|absorption|, \mintinline{glsl}|scattering|, \mintinline{glsl}|extinction|, а также величины \mintinline{glsl}|optical_depth| и \mintinline{glsl}|light_optical_depth| должны стать \mintinline{glsl}|vec3|
\item В координаты \mintinline{glsl}|scattering| нужно записать три разных числа (подберите что-нибудь сами в районе 1..10)
\item Поиграйтесь со значением \mintinline{glsl}|scattering|, чтобы посмотреть, как оно влияет на результат
\end{itemize}
\end{frame}

\begin{frame}[fragile]
\slideimage{6.png}
\end{frame}

\end{document}