% (c) Nikita Lisitsa, lisyarus@gmail.com, 2025

\documentclass[10pt]{beamer}

\usepackage[T2A]{fontenc}
\usepackage[russian]{babel}
\usepackage{minted}

\usepackage{graphicx}
\graphicspath{ {./images/} }

\usetheme{metropolis}

\usepackage{adjustbox}

\usepackage{color}
\usepackage{soul}

\usepackage{hyperref}

\definecolor{codebg}{RGB}{29,35,49}
\definecolor{red}{RGB}{255,0,0}
\definecolor{green}{RGB}{0,191,0}
\definecolor{blue}{RGB}{0,0,255}

\makeatletter
\newcommand{\slideimage}[1]{
  \begin{figure}
    \begin{adjustbox}{width=\textwidth, totalheight=\textheight-2\baselineskip-2\baselineskip,keepaspectratio}
      \includegraphics{#1}
    \end{adjustbox}
  \end{figure}
}
\makeatother

\title{Компьютерная графика}
\subtitle{Практика 5: Текстуры}
\date{2025}

\setbeamertemplate{footline}[frame number]

\begin{document}

\frame{\titlepage}

\begin{frame}[fragile]
\slideimage{0.png}
\end{frame}

\usemintedstyle{solarized-light}

\begin{frame}[fragile]
\frametitle{Задание 1}
Загружаем модель \verb|cow|, аналогично предыдущей практике:
\begin{itemize}
\item Создаём \textbf{VAO}, \textbf{VBO}, \textbf{EBO}
\item Загружаем вершины в \textbf{VBO}
\item Загружаем индексы в \textbf{EBO}
\item Настраиваем атрибуты в \textbf{VAO} (пока только первые два)
\item Связываем индексы с \textbf{VAO}
\item Рисуем с помощью \mintinline{cpp}|glDrawElements|
\item N.B. модель можно крутить и двигать стрелочками
\end{itemize}
\end{frame}

\begin{frame}[fragile]
\slideimage{1.png}
\end{frame}

\begin{frame}[fragile]
\frametitle{Задание 2}
Добавляем текстурные координаты
\begin{itemize}
\item Описываем новый (\mintinline{glsl}|index = 2|) атрибут для VAO, соответствующий полю \mintinline{cpp}|obj_data::vertex::texcoord|
\item Добавляем входной атрибут типа \mintinline{glsl}|vec2| в вершинном шейдере, и передаём его во фрагментный
\item Во фрагментном шейдере используем текстурные координаты в качестве цвета (альбедо), например \mintinline{glsl}|albedo = vec3(texcoord, 0.0)|
\end{itemize}
\end{frame}

\begin{frame}[fragile]
\slideimage{2.png}
\end{frame}

\begin{frame}[fragile]
\frametitle{Задание 3}
Создаём текстуру шахматной раскраски
\begin{itemize}
\item Создаём объект текстуры типа \mintinline{cpp}|GL_TEXTURE_2D|
\item Выставляем для него \mintinline{cpp}|MIN| и \mintinline{cpp}|MAG| фильтры в \mintinline{cpp}|GL_NEAREST| (функция \mintinline{cpp}|glTexParameteri|)
\item Выбираем размер текстуры (лучше довольно большой, в духе \begin{math}512\times 512\end{math} или \begin{math}1024\times 1024\end{math})
\item В качестве данных загружаем пиксели шахматной раскраски
\item Пиксели можно хранить, например, как \mintinline{cpp}|std::uint32_t|
\item Создать массив пикселей можно как \mintinline{cpp}|std::vector<std::uint32_t> pixels(size * size);|
\item Заполнить его чёрными \mintinline{cpp}|0xFF000000u| и белыми \mintinline{cpp}|0xFFFFFFFFu| пикселями в порядке шахматной раскраски
\item Для \mintinline{cpp}|glTexImage2D| параметры \mintinline{cpp}|internalFormat = GL_RGBA8|, \mintinline{cpp}|format = GL_RGBA|, \mintinline{cpp}|type = GL_UNSIGNED_BYTE|
\end{itemize}
\end{frame}

\begin{frame}[fragile]
\frametitle{Задание 3}
Используем созданную текстуру
\begin{itemize}
\item Во фрагментном шейдере добавляем uniform-переменную типа \mintinline{glsl}|sampler2D| и берём из неё цвет (\mintinline{glsl}|albedo|) функцией \mintinline{glsl}|texture|
\item В значение uniform-переменной записываем 0 (\mintinline{cpp}|glUniform1i|)
\item Перед рисованием делаем нашу текстуру текущей для texture unit'а номер 0 (\mintinline{cpp}|glActiveTexture| + \mintinline{cpp}|glBindTexture|)
\end{itemize}
\end{frame}

\begin{frame}[fragile]
\slideimage{3.png}
\end{frame}

\begin{frame}[fragile]
\begin{itemize}
\item Если подвигать и покрутить модель, будет виден сильный \textit{муар} -- эффект, возникающий, когда значения сигнала (в нашем случае -- текстура) читаются (\textit{семплируются}) с частотой меньшей, чем частота самого сигнала (в нашем случае частоту определяют пиксели экрана)
\end{itemize}
\end{frame}

\begin{frame}[fragile]
\frametitle{Задание 4}
Вручную добавляем mipmap-уровни
\begin{itemize}
\item Меняем \mintinline{cpp}|MIN|-фильтр текстуры на \mintinline{cpp}|GL_NEAREST_MIPMAP_NEAREST|
\item После загрузки данных текстуры вызываем \mintinline{cpp}|glGenerateMipmap|, чтобы сгенирировались mipmap-уровни
\item Перезаписываем \textbf{1-ый} mipmap-уровень монохромной картинкой \textbf{\color{red}красного} цвета
\item Перезаписываем \textbf{2-ый} mipmap-уровень монохромной картинкой \textbf{\color{green}зелёного} цвета
\item Перезаписываем \textbf{3-ый} mipmap-уровень монохромной картинкой \textbf{\color{blue}синего} цвета
\begin{itemize}
\item Размер \textbf{1-ого} уровня должен быть \textbf{ровно в 2 раза меньше} исходной текстуры \textbf{по обеим осям}, \textbf{2-ого} уровня -- \textbf{ровно в 4 раза меньше}, и т.д.
\end{itemize}
\item Подвигайте модель, чтобы увидеть, как меняется выбранный mipmap-уровень
\end{itemize}
\end{frame}

\begin{frame}[fragile]
\slideimage{4.png}
\end{frame}

\begin{frame}[fragile]
\frametitle{Задание 5}
Загружаем настоящую текстуру
\begin{itemize}
\item Создаём \textbf{новую текстуру}
\item Настраиваем для неё \mintinline{cpp}|MIN| и \mintinline{cpp}{MAG} фильтры в \mintinline{cpp}|GL_LINEAR_MIPMAP_LINEAR| и \mintinline{cpp}|GL_LINEAR| соответственно
\item Загружаем её данные из файла с помощью функции \mintinline{cpp}|stbi_load|, путь до изображения есть в переменной \mintinline{cpp}|cow_texture_path|, число каналов \mintinline{cpp}|desired_channels| указывать равным 4
\item Загружаем эти данные на GPU с помощью \mintinline{cpp}|glTexImage2D| и \textbf{не забываем сгенерировать mipmaps} (\mintinline{cpp}|glGenerateMipmap|)
\item Очищаем данные на CPU с помощью \mintinline{cpp}|stbi_image_free|
\item Используем для этой текстуры texture unit \textbf{с номером 1}, шейдер на этом этапе меняться не должен
\item Старая текстура всё ещё должна остаться в texture unit'е \textbf{с номером 0}, за выбор текстуры отвечает только uniform-переменная
\end{itemize}
\end{frame}

\begin{frame}[fragile]
\slideimage{5.png}
\end{frame}

\begin{frame}[fragile]
\frametitle{Задание 6*}
Крутим текстуру по модели
\begin{itemize}
\item Передаём текущее время в шейдер в качестве uniform-переменной
\item Сдвигаем текстурные координаты на какое-то зависящее от времени значение
\end{itemize}
\end{frame}

\begin{frame}[fragile]
\slideimage{6.png}
\end{frame}

\end{document}