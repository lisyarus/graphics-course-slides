% (c) Nikita Lisitsa, lisyarus@gmail.com, 2021

\documentclass{beamer}

\usepackage[T2A]{fontenc}
\usepackage[utf8]{inputenc}
\usepackage[russian]{babel}

\usepackage{graphicx}
\graphicspath{ {./images/} }

\usepackage{adjustbox}

\usepackage{color}
\usepackage{soul}

\usepackage{hyperref}

\definecolor{blue}{rgb}{0,0,1}
\definecolor{red}{rgb}{1,0,0}

\makeatletter
\newcommand{\slideimage}[1]{
  \begin{figure}
    \begin{adjustbox}{width=\textwidth, totalheight=\textheight-2\baselineskip-2\baselineskip,keepaspectratio}
      \includegraphics{#1}
    \end{adjustbox}
  \end{figure}
}
\makeatother

\title{Компьютерная графика}
\subtitle{Практика 8: Shadow mapping}
\date{2021}

\setbeamertemplate{footline}[frame number]

\begin{document}

\frame{\titlepage}

\begin{frame}[fragile]
\slideimage{0.png}
\end{frame}

\begin{frame}[fragile]
\frametitle{Задание 1}
\fontsize{10pt}{10pt}
При инициализации создаём и настраиваем shadow map и framebuffer
\begin{itemize}
\item Выбираем размер shadow map: например, \verb|shadow_map_size = 1024|
\item Создаём текстуру для shadow map: min/mag фильтры -- \verb|GL_NEAREST|, размеры -- \verb|shadow_map_size x shadow_map_size|, internal format -- \verb|GL_DEPTH_COMPONENT24|, format -- \verb|GL_DEPTH_COMPONENT|, type -- \verb|GL_FLOAT|, в данных -- \verb|nullptr|
\item Настраиваем ей параметры \verb|GL_TEXTURE_WRAP_S| и \verb|GL_TEXTURE_WRAP_T| в значение \verb|GL_CLAMP_TO_EDGE|
\item Создаём framebuffer, присоединяем к нему нашу текстуру в качестве глубины (\verb|glFramebufferTexture|, \verb|GL_DEPTH_ATTACHMENT|), \verb|target| лучше использовать \verb|GL_DRAW_FRAMEBUFFER|
\item Проверяем, что фреймбуффер настроен правильно (\verb|glCheckFramebufferStatus|)
\item N.B. Экран будет чёрный, так как мы не сделали дефолтный фреймбуфер текущим :)
\end{itemize}
\end{frame}

\begin{frame}[fragile]
\frametitle{Задание 1}
\slideimage{1.png}
\end{frame}

\begin{frame}[fragile]
\frametitle{Задание 2}
\begin{tiny}
Добавляем дебажный прямоугольник с собственной шейдерной программой, чтобы видеть содержимое нашей shadow map
\begin{itemize}
\item В начале рендеринга (перед \verb|glClear|, сразу после обработки событий) делаем текущим дефолтный (ID = 0) фреймбуфер, чтобы снова увидеть сцену
\item Создаём новый вершинный шейдер: выдаёт (в \verb|gl_Position| захардкоженные координаты вершин, используя \verb|gl_VertexID| (как в первой практике), и передаёт (через \verb|out vec2 texcoord|) во фрагментный шейдер текстурные координаты (без каких-либо матриц)
\item Должно быть 6 вершин -- два треугольника, образующих прямоугольник
\item Координаты вершин должны быть где-то в нижнем левом углу экрана (например, [-1.0 .. -0.5] по обеим осям)
\item Текстурные координаты должны быть [0.0 .. 1.0] по обеим осям, чтобы они покрыли всю текстуру, т.е. (0, 0) у левого нижнего угла, (1, 0) у правого нижнего, и т.д.
\item Фрагментный шейдер: читает цвет из переданной текстуры (\verb|uniform sampler2D|) и выводит в \verb|out_color|, можно только красный канал: \verb|vec4(texture(...).r)| (другие каналы содержат нули -- так себя ведёт тип пикселя \verb|GL_DEPTH_COMPONENT|)
\item При инициализации создаём фиктивный VAO (без настройки атрибутов вершин)
\item После рисования основной модели, перед \verb|SDL_GL_SwapBuffers| рисуем прямоугольник с помощью \verb|glDrawArrays(GL_TRIANGLES, 0, 6)| (не забываем сделать текущими созданный VAO, новую шейдерную программу и текстуру shadow map, а также выключить тест глубины, чтобы прямоугольник не оказался `за' основной сценой)
\item По-хорошему для связи \verb|sampler2D| и текстуры нужен texture unit; для простоты можем воспользоваться тем, что по умолчанию активный texture unit -- нулевой, и значение uniform-переменных по умолчанию -- тоже ноль
\item N.B. прямоугольник будет белым (или чёрным, зависит от драйвера), так как shadow map пока пустой
\end{itemize}
\end{tiny}
\end{frame}

\begin{frame}[fragile]
\frametitle{Задание 2}
Теперь весь код рисования кадра должен выглядеть как-то так:
\begin{itemize}
\item Делаем текущим дефолтный (ID = 0) фреймбуфер, настраиваем viewport, очищаем color и depth буферы, настраиваем depth test и culling
\item Включаем основную шейдерную программу, рисуем сцену
\item Включаем новую шейдерную программу, рисуем прямоугольник
\end{itemize}
\end{frame}

\begin{frame}[fragile]
\frametitle{Задание 2}
\slideimage{2.png}
\end{frame}

\begin{frame}[fragile]
\frametitle{Задание 3}
\begin{tiny}
Генерируем shadow map
\begin{itemize}
\item Выбираем проекцию для shadow map: для начала сгодится проекция `снизу-вверх' (как будто камера смотрит сверху)
\item При рисовании кадра вычисляем оси проекции:
\item \verb|light_Z = glm::vec3(0, -1, 0)| -- направление, противоположное вгзляду камеры
\item \verb|light_X = glm::vec3(1,  0, 0)|
\item \verb|light_Y = glm::cross(light_X, light_Z)|
\item Матрица проекции: \begin{verbatim}glm::mat4(glm::transpose(
    glm::mat3(light_X, light_Y, light_Z)))\end{verbatim} (пользуемся тем, что матрица из этих трёх векторов -- ортогональная; в общем случае \verb|transpose| надо заменить на \verb|inverse|: см. лекцию про камеры и проекции)
\item Пишем новую шейдерную программу для вычисления shadow map:
\item Вершинный шейдер преобразует вершины \verb|gl_Position = shadow_projection * model * vec4(in_position, 1.0)| (матрица view здесь не нужна -- мы не настраиваем реальную камеру, а просто вычисляем тени)
\item Фрагментный шейдер ничего не делает (пустая функция \verb|main|)
\item \textbf{Перед} рисованием основной сцены и прямоугольника: используем созданный ранее фреймбуфер для рисования, настраиваем viewport (размер -- \verb|shadow_map_size x shadow_map_sizeu|, очищаем буфер глубины, включаем front-face culling (чтобы избавиться от shadow acne), включаем depth test, рисуем нашу модель созданной шейдерной программой
\item После этого не забываем вернуть back-face culling
\item Модель должна появиться в нашем дебажном прямоугольнике
\end{itemize}
\end{tiny}
\end{frame}

\begin{frame}[fragile]
\frametitle{Задание 3}
Теперь весь код рисования кадра должен выглядеть как-то так:
\begin{itemize}
\item Делаем текущим созданный в задании 1 фреймбуфер, настраиваем viewport, очищаем depth буфер, настраиваем depth test и front-face culling
\item Включаем шейдерную программу для рисования shadow map, рисуем сцену
\item Делаем текущим дефолтный (ID = 0) фреймбуфер, настраиваем viewport, очищаем color и depth буферы, настраиваем depth test и back-face culling
\item Включаем основную шейдерную программу, рисуем сцену
\item Включаем шейдерную программу для прямоугольника, рисуем прямоугольник
\end{itemize}
\end{frame}

\begin{frame}[fragile]
\frametitle{Задание 3}
\slideimage{3.png}
\end{frame}

\begin{frame}[fragile]
\frametitle{Задание 4}
\fontsize{10pt}{10pt}
Используем shadow map
\begin{itemize}
\item Передаём текстуру shadow map (\verb|uniform sampler2D|) и проекцию для неё (\verb|uniform mat4 projection|) в основную шейдерную программу
\item Во фрагментном шейдере:
\begin{itemize}
\item Вычисляем ndc-координаты текущей точки после применения проекции: \begin{verbatim}vec4 ndc = shadow_projection
* model * vec4(position, 1.0)\end{verbatim}
\item Проверяем точку на попадание в видимую область shadow map (XY-координаты \verb|ndc| должны быть в диапазоне [-1..1])
\item Если точка попала в shadow map, вычисляем её текстурные координаты для shadow map \verb|shadow_texcoord = ndc.xy * 0.5 + 0.5| и глубину \verb|shadow_depth = ndc.z * 0.5 + 0.5|
\item Если значение в shadow map \verb|texture(shadow_map, shadow_texcoord)| меньше глубины нашей точки, она в тени (к ней не нужно применять прямое освещение, но ambient остаётся)
\end{itemize}
\item Тень будет выглядеть так, будто свет падает сверху
\end{itemize}
\end{frame}

\begin{frame}[fragile]
\frametitle{Задание 4}
\slideimage{4.png}
\end{frame}

\begin{frame}[fragile]
\frametitle{Задание 5}
\fontsize{10pt}{10pt}
Вычисляем настоящую проекцию
\begin{itemize}
\item \verb|light_Z = -light_direction|
\item \verb|light_X| -- любой вектор, ортогональный \verb|light_Z|
\item \verb|light_Y = glm::cross(light_X, light_Z)|
\end{itemize}
\end{frame}

\begin{frame}[fragile]
\frametitle{Задание 5}
\slideimage{5.png}
\end{frame}

\begin{frame}[fragile]
\frametitle{Задание 6}
\fontsize{10pt}{10pt}
Включаем PCF
\begin{itemize}
\item Меняем min/mag фильтры shadow map на \verb|GL_LINEAR|
\item Настраиваем текстуре shadow map опции \verb|GL_TEXTURE_COMPARE_MODE = GL_COMPARE_REF_TO_TEXTURE| и \verb|GL_TEXTURE_COMPARE_FUNC = GL_LEQUAL| (тоже через \verb|glTexParameteri|)
\item Заменяем в основном фрагментном шейдере \verb|sampler2D shadow_map| на \verb|sampler2DShadow|
\item Сравнение \verb|texture(shadow_map, shadow_texcoord) < shadow_depth| заменяется на один вызов \verb|texture(shadow_map, shadow_texcoord)| -- вернёт 0 или 1 (если в тени или не в тени, соответственно)
\item N.B. дебажный прямоугольник перестанет работать :(
\end{itemize}
\end{frame}

\begin{frame}[fragile]
\frametitle{Задание 6}
\slideimage{6.png}
\end{frame}

\begin{frame}[fragile]
\frametitle{Задание 7*}
\fontsize{10pt}{10pt}
Добавляем размытие к PCF
\begin{itemize}
\item Во фрагментном шейдере, вместо однократного чтения shadow map \verb|texture(shadow_map, shadow_texcoord)| читаем значения из соседних пикселей (надо будет что-то прибавить к \verb|shadow_texcoord|) и усредняем по Гауссу
\item Тени должны получиться более размытыми
\end{itemize}
\end{frame}

\begin{frame}[fragile]
\frametitle{Задание 7*}
\slideimage{7.png}
\end{frame}

\end{document}