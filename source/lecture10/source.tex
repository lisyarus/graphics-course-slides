% (c) Nikita Lisitsa, lisyarus@gmail.com, 2025

\documentclass[10pt]{beamer}

\usepackage[T2A]{fontenc}
\usepackage[russian]{babel}
\usepackage{minted}

\usepackage{graphicx}
\graphicspath{ {./images/} }

\usepackage{adjustbox}

\usepackage{color}
\usepackage{soul}

\usepackage{hyperref}

\usepackage{amsmath}

\usepackage{tikz}
\usetikzlibrary{decorations}
\usetikzlibrary{decorations.pathreplacing}
\usepackage{xifthen}

\usetheme{metropolis}
\setminted{fontsize=\footnotesize}

\definecolor{red}{rgb}{1,0,0}
\definecolor{green}{rgb}{0,0.5,0}
\definecolor{blue}{rgb}{0,0,1}
\definecolor{magenta}{rgb}{0.75,0,0.75}
\definecolor{codebg}{RGB}{29,35,49}

\makeatletter
\newcommand{\slideimage}[1]{
  \begin{figure}
    \begin{adjustbox}{width=\textwidth, totalheight=\textheight-2\baselineskip-2\baselineskip,keepaspectratio}
      \includegraphics{#1}
    \end{adjustbox}
  \end{figure}
}
\makeatother

\title{Компьютерная графика}
\subtitle{Лекция 10: Normal mapping, material mapping, environment mapping, отражения, несколько источников света}
\date{2025}

\setbeamertemplate{footline}[frame number]

\usemintedstyle{solarized-light}

\begin{document}

\frame{\titlepage}

\begin{frame}[fragile]
\frametitle{Normal mapping}
\begin{itemize}
\item Хотим высокую детализацию поверхности объекта
\end{itemize}
\slideimage{orange.png}
\end{frame}

\begin{frame}[fragile]
\frametitle{Normal mapping}
\begin{itemize}
\item Можно взять очень много вершин
\pause
\item Очень дорого, особенно когда объектов много
\end{itemize}
\slideimage{orange-mesh.png}
\end{frame}

\begin{frame}[fragile]
\frametitle{Normal mapping}
\begin{itemize}
\item Наблюдение: высокая детализация даёт 2 вещи
\begin{itemize}
\item Детализацию \textit{геометрии} -- слабо заметна
\item Детализацию \textit{нормалей} -- сильно заметна, так как влияет на освещение
\end{itemize}
\pause
\item Давайте откажемся от детализации координат!
\end{itemize}
\end{frame}

\begin{frame}[fragile]
\frametitle{Normal mapping}
\slideimage{orange-normal-map.png}
\end{frame}

\begin{frame}[fragile]
\frametitle{Normal mapping}
\begin{itemize}
\item \textit{Normal mapping}: используем текстуру, в которой закодированы нормали в точках объекта
\pause
\item Вместо нормалей, приходящих в вершинах, достаём нормали из текстуры во фрагментном шейдере (используя те же текстурные координаты, что и для цвета)
\pause
\item При этом можно использовать слабо детализированную геометрию
\end{itemize}
\end{frame}

\begin{frame}[fragile]
\frametitle{Normal mapping}
\slideimage{orange-normals.jpg}
\end{frame}

\begin{frame}[fragile]
\frametitle{Normal mapping}
\slideimage{bricks-normal-map.png}
\end{frame}

\begin{frame}[fragile]
\frametitle{Normal mapping}
\slideimage{bricks-normals.png}
\end{frame}

\begin{frame}[fragile]
\frametitle{Normal mapping}
\slideimage{pupil-normals.jpg}
\end{frame}

\begin{frame}[fragile]
\frametitle{Normal mapping: система координат}
\begin{itemize}
\item В какой системе координат хранить нормали в текстуре?
\pause
\item В системе координат объекта
\begin{itemize}
\item {\color{green}+} Нет лишних операций, просто читаем нормаль из текстуры (и, возможно, применяем model матрицу)
\item {\color{red}—} Неудобно при подготовке данных: любое изменение модели (напр. слегка повернули часть объекта) влечёт пересчёт текстуры нормалей
\end{itemize}
\pause
\item В локальной системе координат пикселя, учитывая геометрию объекта
\begin{itemize}
\item {\color{red}—} Нужно восстанавливать локальную систему координат точки в шейдере
\item {\color{green}+} Можно сэкономить и хранить только две координаты нормали
\item {\color{green}+} Такая текстура легче воспринимается на глаз
\item Наиболее  распространённый способ
\item Такие текстуры часто выглядят синими, потому что направление Z считается нормалью вершины
\end{itemize}
\end{itemize}
\end{frame}

\begin{frame}[fragile]
\frametitle{Normal mapping: TBN}
\begin{itemize}
\item Для второго способа (локальная система координат пикселя) нужно восстановить эту систему координат
\pause
\item Она задаётся ортонормированным базисом из трёх векторов -- tangent, bitangent, normal (TBN):
\pause
\begin{itemize}
\item \textit{Tangent} -- смотрит в направлении роста первой текстурной координаты вдоль поверхности
\item \textit{Bitangent} -- смотрит в направлении роста второй текстурной координаты вдоль поверхности
\item \textit{Normal} -- перпендикуляр к поверхности
\end{itemize}
\pause
\item Обычно их хранят в атрибутах вершин
\item Достаточно любых двух векторов (обычно -- normal + tangent), так как третий восстанавливается через векторное произведение
\end{itemize}
\end{frame}

\begin{frame}[fragile]
\frametitle{Normal mapping: TBN и shader derivatives}
\begin{itemize}
\item Есть способ восстановить TBN используя \mintinline{glsl}|dFdx & dFdy|:
\pause
\item Нас интересует производная первой текстурной координаты по мировым координатам точки (т.е. в каком направлении растёт эта координата)
\pause
\item Через \mintinline{glsl}|dFdy & dFdy| мы можем получить производную текстурной координаты по пикселям экрана и производную мировых координат по пикселям экрана
\pause
\item Мы знаем \begin{math}\frac{\partial A}{\partial C}\end{math} и \begin{math}\frac{\partial B}{\partial C}\end{math}, можем вычислить \begin{math}\frac{\partial A}{\partial B} = \frac{\partial A}{\partial C} \cdot \left(\frac{\partial B}{\partial C}\right)^{-1}\end{math}
\pause
\item Сводится к операциям над матрицами \begin{math}2\times 2\end{math} в шейдере
\pause
\item Метод менее точен, но позволяет избежать дополнительных данных в вершинах
\end{itemize}
\end{frame}

\begin{frame}[fragile]
\frametitle{Normal mapping: формат хранения}
\fontsize{10pt}{10pt}
\begin{itemize}
\item Как хранить нормали в текстуре?
\pause
\item Нормаль -- 3х-компонентный нормированный вектор
\pause
\item Floating-point текстуры (\mintinline{cpp}|GL_RGB32F| или \mintinline{cpp}|GL_RGB16F|)
\begin{itemize}
\item {\color{red}—} Занимают много места (3x16-bit = 6 байт на пиксель)
\item {\color{red}—} Излишняя точность
\item {\color{red}—} Значительная часть возможных значений (координаты больше 1 по модулю) вообще не используется
\end{itemize}
\pause
\item Signed normalized текстуры (\mintinline{cpp}|GL_RGB8_SNORM|)
\begin{itemize}
\item {\color{green}+} Компактные
\item {\color{green}+} Достаточно точности
\item {\color{red}—} Форматы изображений обычно не умеют работать с signed пикселями
\end{itemize}
\pause
\item Обычные unsigned normalized текстуры (\mintinline{cpp}|GL_RGB8|, \mintinline{cpp}|GL_RGB10_A2|, \mintinline{cpp}|GL_RGB565|, \mintinline{cpp}|GL_RGB5_A1|, \mintinline{cpp}|GL_R3_G3_B2|)
\begin{itemize}
\item {\color{green}+} Компактные
\item {\color{green}+} Достаточно точности
\item {\color{green}+} Можно использовать обычные форматы изображений
\item Координаты хранятся в виде, отмасштабированном в диапазон [0, 1], т.е. \mintinline{cpp}|0.5 * (x + 1)|
\end{itemize}
\end{itemize}
\end{frame}

\begin{frame}[fragile]
\frametitle{Normal mapping: формат хранения}
\fontsize{10pt}{10pt}
\begin{itemize}
\item Если нормаль всегда ориентирована локально `вверх' (как в варианте с TBN), можно не хранить Z-компоненту нормали: её можно восстановить по XY как \begin{math}\sqrt{1-x^2-y^2}\end{math}
\pause
\item Для текстуры нормалей обычно имеет смысл включать линейную интерполяцию и мипмапы, но мипмапы могут работать неправильно на сложных поверхностях (brdf с усреднённой нормалью -- не то же самое, что усреднённые значения brdf)
\pause
\item Нормаль, прочитанную из текстуры, лучше явно нормировать в шейдере (функцией \mintinline{cpp}|normalize|), так как она может быть не нормированной из-за неточностей формата хранения и линейной интерполяции, что приведёт к артефактам при расчёте specular освещения
\end{itemize}
\end{frame}

\begin{frame}[fragile]
\frametitle{Normal mapping: ссылки}
\begin{itemize}
\item \href{https://learnopengl.com/Advanced-Lighting/Normal-Mapping}{\texttt{learnopengl.com/Advanced-Lighting/Normal-Mapping}}
\item \href{http://www.opengl-tutorial.org/intermediate-tutorials/tutorial-13-normal-mapping}{\texttt{opengl-tutorial.org/intermediate-tutorials/tutorial-13-normal-mapping}}
\item \href{https://sudonull.com/post/14500-Learn-OpenGL-Lesson-55-Normal-Mapping}{\texttt{sudonull.com/post/14500-Learn-OpenGL-Lesson-55-Normal-Mapping}}
\item \href{https://habr.com/ru/post/415579}{\texttt{habr.com/ru/post/415579}}
\end{itemize}
\end{frame}

\begin{frame}[fragile]
\frametitle{Material mapping}
\begin{itemize}
\item В обработке освещения часто встречаются настраиваемые параметры, помимо цвета (альбедо) объекта (например, glossiness и roughness для specular освещения)
\pause
\item \textit{Material mapping}: варьировать их, читая из текстуры, так же как альбедо или нормали
\end{itemize}
\end{frame}

\begin{frame}[fragile]
\frametitle{Material mapping}
\slideimage{material_map.png}
\end{frame}

\begin{frame}[fragile]
\frametitle{Environment mapping}
\begin{itemize}
\item Одноцветный фон сцены -- скучно, хочется нарисовать что-нибудь `в далеке'
\pause
\item Тратить ресурсы на то, чтобы рисовать что-то очень далёкое, не хочется
\pause
\item \textit{Environment mapping}: рисовать статическую (обычно), заранее подготовленную картинку `environment map' (облака, горы, etc.)
\end{itemize}
\end{frame}

\begin{frame}[fragile]
\frametitle{Environment mapping: как хранить?}
\begin{itemize}
\item Environment map задаёт цвет `фона' в каждом возможном направлении (формально -- количество света, приходящего в сцену из этого направления)
\pause
\item \textbf{\alert{N.B.}}: лучше, если ambient освещение совпадает с усреднённым значением environment map, иначе картинка будет выглядеть неестественно
\pause
\item Варианты хранения:
\begin{itemize}
\item Cubemap-текстура
\pause
\item Обычная 2D текстура с равнопромежуточной проекцией (текстуртные координаты соответствуют `долготе' и `широте' направления взгляда)
\pause
\item Две 2D текстуры с параболической/fisheye проекцией
\end{itemize}
\end{itemize}
\end{frame}

\begin{frame}[fragile]
\frametitle{Environment map}
\slideimage{env-map.jpg}
\end{frame}

\begin{frame}[fragile]
\frametitle{Environment mapping: как рисовать?}
\begin{itemize}
\item Первый вариант:
\pause
\begin{itemize}
\item Нарисовать куб вокруг камеры
\pause
\item По координате точки и координате камеры получить направление взгляда
\pause
\item Использовать направление для обращения к environment map
\end{itemize}
\pause
\item Второй вариант:
\begin{itemize}
\item Нарисовать полноэкранный прямоугольник
\pause
\item Используя обратную view+projection матрицу получить координаты этого прямоугольника в пространстве
\pause
\item По координате точки и координате камеры получить направление взгляда
\pause
\item Использовать направление для обращения к environment map
\end{itemize}
\pause
\item Environment map нужно рисовать первым объектом сцены, \textbf{без} теста глубины
\end{itemize}
\end{frame}

\begin{frame}[fragile]
\frametitle{Environment/reflection mapping}
\begin{itemize}
\item Рисовать честные отражения в real-time очень сложно
\pause
\item Дешёвая аппроксимация: будем учитывать в отражениях только environment map
\pause
\item Такая техника известна под названиями \textit{reflection mapping} или \textit{environment mapping}
\pause
\item {\color{green}+} Просто в реализации
\item {\color{green}+} Быстро работает
\item {\color{red}—} Не позволяет учесть динамические объекты
\pause
\item Хорошо работает для плохо заметных отражений: дверные ручки, гильзы от пуль, ржавые трубы, etc
\end{itemize}
\end{frame}

\begin{frame}[fragile]
\frametitle{Reflection mapping: реализация}
\begin{itemize}
\item Во фрагментном шейдере вычисляем направление луча из камеры в точку
\pause
\item Вычисляем направление отражённого луча, используя нормаль к поверхности (возможно, с normal mapping'ом)
\pause
\item Используем вычисленное направление для чтения из environment map
\pause
\item Можно явно использовать mipmap'ы environment map'а чтобы аппроксимировать размытое отражение (roughness)
\end{itemize}
\end{frame}

\begin{frame}[fragile]
\frametitle{Reflection mapping}
\slideimage{reflection.png}
\end{frame}

\begin{frame}[fragile]
\frametitle{Environment/reflection mapping: ссылки}
\begin{itemize}
\item \href{https://learnopengl.com/Advanced-OpenGL/Cubemaps}{\texttt{learnopengl.com/Advanced-OpenGL/Cubemaps}}
\end{itemize}
\end{frame}

\begin{frame}[fragile]
\frametitle{Отражения}
\begin{itemize}
\item Расчёт отражений в real-time -- очень сложная задача
\pause
\item Для плоских зеркал можно нарисовать сцену целиком, применив отражающее преобразование и нарисовав зеркало предварительно в stencil буфер, чтобы перевёрнутая сцена нарисовалась только там, где находится зеркало
\pause
\begin{itemize}
\item {\color{red}—} Рисовать заново всю сцену -- дорого
\item {\color{red}—} Не работает для неплоских зеркал
\end{itemize}
\pause
\item Можно нарисовать сцену целиком, вычисляя проекцию объекта на зеркало в вершинном шейдере
\begin{itemize}
\item {\color{red}—} Дорого
\item {\color{red}—} Будут артефакты с растягиванием вершин (зависит от формы зеркала)
\end{itemize}
\end{itemize}
\end{frame}

\begin{frame}[fragile]
\frametitle{Отражения}
\begin{itemize}
\item Аппроксимация: использовать ту же идею, что и в reflection mapping'е, но вместо заранее подготовленного environment map нарисуем сцену в текстуру
\pause
\begin{itemize}
\item {\color{red}—} Дорого (рисуем всю сцену ещё раз)
\item {\color{red}—} Отражения не зависят от точки, в которой происходит отражение
\item {\color{green}+} Обычно артефакты не очень заметны
\end{itemize}
\end{itemize}
\end{frame}

\begin{frame}[fragile]
\frametitle{Cubemap отражения}
\begin{itemize}
\item Удобнее всего использовать cubemap-текстуру
\pause
\item Алгоритм:
\begin{itemize}
\item Выбираем точку, из которой будет строиться текстура для отражений (обычно -- позиция камеры или игрока)
\pause
\item Рисуем сцену в cubemap-текстуру камерой, находящейся в этой точке (как в shadow mapping от точечного источника света)
\pause
\item При рисовании сцены на экран вычисляем отражённый луч из камеры (как в reflection mapping)
\pause
\item Читаем значение из cubemap-текстуры по этому направлению -- это отражённый цвет
\end{itemize}
\pause
\item Как в reflection mapping, можно использовать mipmap'ы или размыть cubemap чтобы имитировать нечёткие отражения
\pause
\item Можно сэкономить, взяв cubemap низкого качества, и рисуя в него только крупные объекты
\pause
\item Можно сэкономить, расставляя зеркала только внутри зданий/комнат, где заранее известен набор отражаемых объектов
\end{itemize}
\end{frame}

\begin{frame}[fragile]
\frametitle{Cubemap отражения (GTA 5)}
\slideimage{gta_reflections.jpg}
\end{frame}

\begin{frame}[fragile]
\frametitle{Много источников света}
\slideimage{many-lights.png}
\end{frame}

\begin{frame}[fragile]
\frametitle{Как обрабатывать много источников света?}
\begin{itemize}
\item Простейший способ -- использовать по отдельному набору uniform-переменных для каждого источника света, и обрабатывать их все сразу в шейдере
\pause
\item Можно использовать массивы uniform-переменных
\pause
\begin{itemize}
\item Задаются как \mintinline{glsl}|uniform vec3 light_position[N]| (\mintinline{glsl}|N| -- константа)
\item Каждый элемент массива -- отдельная uniform-переменная, для неё нужно отдельно вызвать \mintinline{cpp}|glGetUniformLocation|
\item Имя этой переменной -- имя массива + индекс, например \mintinline{glsl}|light_position[0]|
\item Альтернативно, можно загрузить весь массив сразу функциями \mintinline{cpp}|glUniform*v|
\end{itemize}
\end{itemize}
\end{frame}

\begin{frame}[fragile]
\frametitle{Как обрабатывать много источников света?}
\begin{itemize}
\item Сколько источников света можно так обработать?
\pause
\item OpenGL гарантирует, что под uniform-переменные есть как минимум 1024 выделенных компоненты (\mintinline{glsl}|vec3| занимает 3 компоненты, и т.п.)
\pause
\item В варианте 3 \mintinline{glsl}|vec3| на источник (position, color, attenuation) -- около 113 источников света
\pause
\item Чем больше источников, тем больше вычислений в шейдере, тем больше нагрузка на GPU
\end{itemize}
\end{frame}

\begin{frame}[fragile]
\frametitle{Как обрабатывать много источников света?}
\begin{itemize}
\item Вместо массивов uniform-переменных можно использовать:
\begin{itemize}
\item Uniform buffer objects (UBO) -- обычно 16K
\item Buffer textures -- от 64K
\item Shader storage buffer objects (SSBO, OpenGL 4.3) -- от 128M
\end{itemize}
\end{itemize}
\end{frame}

\begin{frame}[fragile]
\frametitle{Как обрабатывать много источников света?}
\begin{itemize}
\item Другой способ: рисовать сцену по одному разу на каждый источник света с аддитивным blending'ом, `добавляя' вклад каждого источника света
\pause
\item Очень дорого:
\begin{itemize}
\item Многократно повторяются одни и те же вычисления в вершинных шейдерах
\item Многократно читаются одни и те же данные атрибутов вершин, текстур, и т.д.
\end{itemize}
\end{itemize}
\end{frame}

\begin{frame}[fragile]
\frametitle{Несколько источников света}
\begin{itemize}
\item Оба подхода выполняют много лишних операций:
\begin{itemize}
\item Вычисляется полное освещение для объекта, который может быть скрыт другим объектом
\pause
\item Вычисляется вклад в освещение для объекта, находящегося слишком далеко от источника света
\end{itemize}
\pause
\item Можно частично решить, вычисляя для каждого объекта список источников, которыми он освещён \begin{math}\Longrightarrow\end{math} больше изменений uniform-переменных при рендеринге, меньше возможность для batching'а (объединения объектов в группы или в общие буферы)
\pause
\item Более современные решения: \textit{deferred shading}, \textit{tiled/clustered shading}
\end{itemize}
\end{frame}

\begin{frame}[fragile]
\frametitle{Deferred shading}
\begin{itemize}
\item Идея: вместо рисования сцены на экран, сначала нарисуем её (с точки зрения камеры) в набор буферов, вместе называющийся \textit{G-buffer}, содержащих
\begin{itemize}
\item Цвет пикселей (альбедо)
\item Нормали пикселей
\item Материалы пикселей
\item Позиции пикселей в мировой системе координат
\item И т.п.
\end{itemize}
\pause
\item Шейдер читает из G-buffer'а параметры материала в этом пикселе и вычисляет вклад освещения для всех источников света
\end{itemize}
\end{frame}

\begin{frame}[fragile]
\frametitle{Deferred shading}
\begin{itemize}
\item Можно применять освещение одним шейдером на все источники, а можно аддитивным blending'ом по одному
\pause
\item В любом случае нужно рисовать полноэкранный прямоугольник
\pause
\item В варианте с blending'ом можно вычислять освещение только для пикселей, расположенных рядом с источником света, вместо полноэкранного прямоугольника рисуя некую геометрию, ограничивающую область, освещаемую этим источником света (\textit{proxy geometry})
\end{itemize}
\end{frame}

\begin{frame}[fragile]
\frametitle{Deferred shading: G-buffer}
\slideimage{g_buffer.png}
\end{frame}

\begin{frame}[fragile]
\frametitle{Deferred shading: G-buffer (GTA V)}
\slideimage{gta.jpg}
\end{frame}

\begin{frame}[fragile]
\frametitle{Deferred shading: G-buffer (RDR 2)}
\slideimage{rdr2.jpg}
\end{frame}

\begin{frame}[fragile]
\frametitle{Deferred shading}
\begin{itemize}
\item {\color{green}+} Позволяет учесть тысячи источников света
\item {\color{green}+} После применения освещения остаётся много информации (G-buffer), которую можно использовать для других эффектов (SSAO, screen-space reflections, ...)
\item {\color{green}+} Не вычисляет освещение для объектов, скрытых другими объектами
\item {\color{green}+} Есть много вариантов оптимизации хранения G-buffer'а (сжимать нормали, восстанавливать позицию по глубине)
\item {\color{red}—} Не позволяет учесть освещённые полупрозрачные объекты (на каждый слой прозрачных объектов нужен свой G-buffer)
\item {\color{red}—} Очень много операций записи и чтения памяти (G-buffer) на каждый кадр
\item Всё ещё один из самых популярных алгоритмов
\end{itemize}
\end{frame}

\begin{frame}[fragile]
\frametitle{Tiled/clustered shading}
\begin{itemize}
\item Идея: разобьём видимую область на кусочки
\begin{itemize}
\item \textit{Tiled}: экран разбивается на маленькие `тайлы', например 8x8 пикселей
\item \textit{Clustered}: трёхмерная видимая область (view frustum) разбивается на `кластеры', например 16x8x24 (всего 3072 кластера) по X, Y и Z соответственно
\end{itemize}
\pause
\item Для каждого тайла/кластера вычисляем список источников света, влияющих на объекты в этом тайле/кластере
\pause
\item Рисуем сцену как обычно; при расчёте освещения вычисляем, в каком мы находимся тайле/кластере, и читаем оттуда список источников света
\end{itemize}
\end{frame}

\begin{frame}[fragile]
\frametitle{Clustered shading}
\slideimage{clustered.png}
\end{frame}

\begin{frame}[fragile]
\frametitle{Tiled/clustered shading}
\begin{itemize}
\item Хранить список источников можно в buffer textures или в shader storage buffer objects (OpenGL 4.3)
\pause
\item Вычислять список источников для каждого тайла/кластера можно на CPU или на GPU (compute shaders)
\pause
\item На GPU быстрее, но сложнее реализация и формат хранения
\pause
\item Даже на CPU позволяет обрабатывать тысячи источников света
\end{itemize}
\end{frame}

\begin{frame}[fragile]
\frametitle{Clustered shading}
\slideimage{clustered_mars.png}
\end{frame}

\begin{frame}[fragile]
\frametitle{Tiled/clustered shading}
\begin{itemize}
\item {\color{green}+} Позволяет учесть тысячи источников света
\item {\color{green}+} Позволяет рисовать прозрачные освещённые объекты
\item {\color{red}—} Вычисляет освещение и для объектов, скрытых другими объектами
\item Из-за меньшего количества операций с памятью обычно работает быстрее, чем deferred shading
\end{itemize}
\end{frame}

\begin{frame}[fragile]
\frametitle{Тени от многих источников света}
\begin{itemize}
\item Для каждого источника придётся рисовать shadow map
\pause
\item Как хранить много shadow map?
\pause
\item Array texture
\begin{itemize}
\item {\color{green}+} Удобный доступ (слой = номер источника)
\item {\color{red}—} Все shadow map будут одинакового размера \begin{math}\Rightarrow\end{math} нельзя сэкономить на тенях далёких источников
\end{itemize}
\pause
\item \textit{Shadow atlas}: использовать одну большую текстуру, в кусках которой хранятся shadow maps всех источников
\begin{itemize}
\item {\color{green}+} Можно варьировать размер shadow map для каждого источника
\item {\color{red}—} Сложнее доступ к shadow map
\item {\color{red}—} Нужно аккуратно работать с текстурными координатами, чтобы не залезть в чужой shadow map
\item {\color{red}—} Нужен алгоритм упаковки многих текстур в одну
\item Самый распространённый способ
\end{itemize}
\end{itemize}
\end{frame}

\begin{frame}[fragile]
\frametitle{Shadow atlas}
\slideimage{shadow_atlas.png}
\end{frame}

\begin{frame}[fragile]
\frametitle{Несколько источников света}
\begin{itemize}
\item \href{https://learnopengl.com/Lighting/Multiple-lights}{\texttt{learnopengl.com/Lighting/Multiple-lights}}
\item \href{https://en.wikibooks.org/wiki/GLSL_Programming/GLUT/Multiple_Lights}{\texttt{en.wikibooks.org/wiki/GLSL\_Programming/GLUT/Multiple\_Lights}}
\item \href{https://learnopengl.com/Advanced-Lighting/Deferred-Shading}{\texttt{learnopengl.com/Advanced-Lighting/Deferred-Shading}}
\item \href{http://www.aortiz.me/2018/12/21/CG.html}{\texttt{www.aortiz.me/2018/12/21/CG.html}}
\end{itemize}
\end{frame}

\end{document}