% (c) Nikita Lisitsa, lisyarus@gmail.com, 2021

\documentclass{beamer}

\usepackage[T2A]{fontenc}
\usepackage[utf8]{inputenc}
\usepackage[russian]{babel}

\usepackage{graphicx}
\graphicspath{ {./images/} }

\usepackage{adjustbox}

\usepackage{color}
\usepackage{soul}

\usepackage{hyperref}

\usepackage{amsmath}

\usepackage{tikz}
\usetikzlibrary{decorations}
\usetikzlibrary{decorations.pathreplacing}
\usepackage{xifthen}

\definecolor{red}{rgb}{1,0,0}
\definecolor{green}{rgb}{0,0.5,0}
\definecolor{blue}{rgb}{0,0,1}
\definecolor{magenta}{rgb}{0.75,0,0.75}

\makeatletter
\newcommand{\slideimage}[1]{
  \begin{figure}
    \begin{adjustbox}{width=\textwidth, totalheight=\textheight-2\baselineskip-2\baselineskip,keepaspectratio}
      \includegraphics{#1}
    \end{adjustbox}
  \end{figure}
}
\makeatother

\title{Компьютерная графика}
\subtitle{Лекция 15: оптимизация рендеринга, timer queries, batching, instancing, uniform buffers, frustum culling, occlusion culling}
\date{2021}

\setbeamertemplate{footline}[frame number]

\begin{document}

\frame{\titlepage}

\begin{frame}[fragile]
\frametitle{Оптимизация \textendash{} это сложно}
На производительность (CPU) влияют:
\pause
\begin{itemize}
\item Общая загруженность системы
\pause
\item Throttling
\pause
\item Количество и паттерн доступов к памяти (cache-friendliness)
\pause
\item Помещаются ли данные в кэш
\pause
\item Branch prediction
\pause
\item Как функции программы лежат в памяти (опять кэш)
\pause
\item Оптимизации компилятора
\pause
\item Многое другое
\end{itemize}
\end{frame}

\begin{frame}[fragile]
\frametitle{Оптимизация на GPU \textendash{} это очень сложно}
\begin{itemize}
\item Асинхронность
\pause
\item Параллельность
\pause
\item Много встроенных операций (fixed-function pipeline)
\pause
\item Сложные операции с памятью (доступ к текстуре: mipmaps + фильтрация)
\pause
\item Многое другое
\end{itemize}
\end{frame}

\begin{frame}[fragile]
\frametitle{Измерение времени работы \textendash{} неправильный способ}
\begin{verbatim}
while (true) {
  auto frame_start = clock::now();

  // нарисовали сцену
  ...

  auto frame_end = clock::now();

  SwapBuffers();
}
\end{verbatim}
\pause
\begin{itemize}
\item \verb|frame_end - frame_start| \textendash{} сколько времени ушло на то, чтобы \textbf{вызвать OpenGL-команды}
\pause
\item В реальности драйвер поставил их в очередь, и скорее всего GPU ещё не начала их выполнять
\end{itemize}
\end{frame}

\begin{frame}[fragile]
\frametitle{Измерение времени работы \textendash{} простой способ}
\begin{verbatim}
disableVsync();
auto last_frame_start = clock::now();
while (true) {
  auto frame_start = clock::now();
  auto frame_time = frame_start - last_frame_start;
  last_frame_start = frame_start;

  // нарисовали сцену
  ...

  SwapBuffers();
}
\end{verbatim}
\pause
\begin{itemize}
\item Из-за выключенного vsync видеокарта будет работать \begin{math}\pm\end{math} постоянно
\pause
\item В итоге мы получим примерное время, тратящееся на рисование одного кадра
\end{itemize}
\end{frame}

\begin{frame}[fragile]
\frametitle{Измерение времени работы: glFlush и glFinish}
\begin{itemize}
\item Многие (старые) туториалы по измерению времени кадра советуют вызывать \verb|glFlush| или \verb|glFinish| в конце кадра
\pause
\item \verb|glFlush| сбрасывает буфер команд (хранящийся внутри драйвера) с CPU на GPU
\pause
\item \verb|glFinish| ждёт, пока GPU не завершит обрабатывать все посланные команды
\pause
\item \verb|SwapBuffers| сама вызывает \verb|glFlush|
\pause
\item \verb|glFinish| ухудшает производительность: половину времени вы отправляете команды на GPU, а GPU (скорее всего) ничего не делает; половину времени вы ждёте, пока GPU закончит выполнять команды
\end{itemize}
\end{frame}

\begin{frame}[fragile]
\frametitle{Измерение времени работы: FPS vs frame duration}
\begin{itemize}
\item FPS (frames per second, количество кадров в секунду) \textendash{} очень неудобная метрика:
\pause
\begin{itemize}
\item Нелинейна: если кадр рисовался 10 мс, и мы добавили что-то рисующееся 1 мс, и ещё что-то рисующееся 1 мс, то FPS изменялся от 100 до 90.9 до 83.3
\end{itemize}
\pause
\item Обычно используют время, тратящееся на рисование кадра или конкретного объекта/эффекта (миллисекунды/микросекунды)
\end{itemize}
\end{frame}

\begin{frame}[fragile]
\frametitle{Измерение времени работы \textendash{} правильный способ: timer queries}
\begin{itemize}
\item Query objects \textendash{} объекты OpenGL, позволяющие узнать некоторую статистику с GPU:
\pause
\begin{itemize}
\item Сколько было нарисовано пикселей
\pause
\item Сколько сгенерировано примитивов (геометрическим шейдером)
\pause
\item Сколько прошло времени
\end{itemize}
\pause
\item \verb|glGenQueries/glDeleteQueries|
\pause
\item \textbf{Нет} \verb|glBindQuery|!
\end{itemize}
\end{frame}

\begin{frame}[fragile]
\frametitle{Измерение времени работы \textendash{} правильный способ: timer queries}
\begin{itemize}
\item \verb|glBeginQuery/glEndQuery| \textendash{} статистика будет собрана для команд между этими вызовами
\pause
\item \textbf{Не могут} быть вложенными
\end{itemize}
\pause
\begin{verbatim}
GLuint query_id;
glGenQueries(1, &query_id);

...

glBegin(GL_TIME_ELAPSED, query_id);

// что-нибудь рисуем

glEnd(GL_TIME_ELAPSED);
\end{verbatim}
\end{frame}

\begin{frame}[fragile]
\frametitle{Измерение времени работы \textendash{} правильный способ: timer queries}
\begin{itemize}
\item GPU работает асинхронно \begin{math}\Rightarrow\end{math} результат query будет готов не сразу
\pause
\item Узнать, готов ли результат:
\begin{verbatim}
glGetQueryObjectiv(query_id,
  GL_QUERY_RESULT_AVAILABLE, &result);
\end{verbatim}
\pause
\item Получить результат (блокирует поток, если результат ещё не готов; неявно вызывает \verb|glFlush|)
\begin{verbatim}
glGetQueryObjectiv(query_id,
  GL_QUERY_RESULT, &result);
\end{verbatim}
\pause
\item Время возвращается в \textbf{наносекундах}, т.е. знаковый 32-битный тип может представить 2 секунды
\pause
\item Если 64-битные и беззнаковые версии этих функций
\end{itemize}
\end{frame}

\begin{frame}[fragile]
\frametitle{Измерение времени работы \textendash{} правильный способ: пул timer queries}
\begin{itemize}
\item Хотим мерять время рисования каждого кадра, но результат для предыдущего кадра может быть не готов к началу следующего кадра
\pause
\item \begin{math}\Rightarrow\end{math} Заводим пул (pool) query-объектов:
\pause
\begin{itemize}
\item Храним расширяемый массив (\verb|std::vector|) query-объектов: ID + свободен или нет
\pause
\item Когда нам нужен новый query, ищем в массиве свободный объект, если такого нет - добавляем новый
\pause
\item В конце рисования кадра проходим по всем несвободным объектам и проверяем: если результат уже готов, обрабатываем его и помечаем объект свободным
\end{itemize}
\pause
\item Средний размер пула \textendash{} на сколько кадров отстаёт GPU от CPU
\end{itemize}
\end{frame}

\begin{frame}[fragile]
\frametitle{Timer queries: ссылки}
\begin{itemize}
\item \href{https://www.khronos.org/opengl/wiki/Query_Object}{khronos.org/opengl/wiki/Query\_Object}
\item \href{https://www.lighthouse3d.com/tutorials/opengl-timer-query}{Туториал по использованию timer queries}
\end{itemize}
\end{frame}

\begin{frame}[fragile]
\frametitle{Поиск bottleneck'а}
\begin{itemize}
\item Мы знаем, что что-то тормозит
\pause
\item OpenGL pipeline включает много компонентов, какой именно тормозит?
\pause
\item Обычно компоненты конвейера влияют на следующие за ними компоненты
\pause
\begin{itemize}
\item Больше вершин \begin{math}\Rightarrow\end{math} больше вызовов вершинного шейдера
\item Больше примитивов \begin{math}\Rightarrow\end{math} больше пикселей
\item Больше пикселей \begin{math}\Rightarrow\end{math} больше вызовов фрагментного шейдера
\item Больше пикселей \begin{math}\Rightarrow\end{math} больше операций записи в память
\end{itemize}
\pause
\item Удобно искать bottleneck с конца конвейера
\end{itemize}
\end{frame}

\begin{frame}[fragile]
\fontsize{10pt}{10pt}
\frametitle{Поиск bottleneck'а}
\begin{itemize}
\item Упростим до предела фрагментный шейдер (напр. выведем фиксированный цвет)
\pause
\begin{itemize}
\item Стало лучше? \begin{math}\Rightarrow\end{math} Слишком тяжёлый фрагментный шейдер
\end{itemize}
\pause
\item Уменьшим размер окна до чего-нибудь в духе 50x50 пикселей
\pause
\begin{itemize}
\item Стало лучше? \begin{math}\Rightarrow\end{math} Слишком много операций записи в память
\end{itemize}
\pause
\item Упростим до предела вершинный шейдер (напр. вернём фиксированные координаты)
\pause
\begin{itemize}
\item Стало лучше? \begin{math}\Rightarrow\end{math} Слишком тяжёлый фрагментный шейдер
\end{itemize}
\pause
\item Уменьшим число вершин (параметр \verb|count| в \verb|glDraw*|)
\pause
\begin{itemize}
\item Стало лучше? \begin{math}\Rightarrow\end{math} Слишком много вершин
\end{itemize}
\pause
\item Ничего не помогло \begin{math}\Rightarrow\end{math} CPU-bound
\pause
\begin{itemize}
\item Слишком много OpenGL-вызовов
\pause
\item Слишком много других операций на CPU
\end{itemize}
\end{itemize}
\end{frame}

\begin{frame}[fragile]
\fontsize{10pt}{10pt}
\frametitle{Оптимизация шейдеров}
\begin{itemize}
\item Выполняем меньше операций
\pause
\item Избегаем вызова сложных функций (\verb|sin|, \verb|exp|, \verb|pow|)
\pause
\item Реорганизуем вычисления (напр. \verb|exp(a+b+c+d)| вместо \verb|exp(a)*exp(b)*exp(c)*exp(d)|)
\pause
\item Предпосчитываем что-нибудь (в константный массив в шейдере или в текстуру)
\pause
\item Меньше читаем из текстур
\pause
\item Читаем из текстур меньшего размера (лучше утилизируется текстурный кэш)
\pause
\item Близкие пиксели читают близкие части текстуры (лучше утилизируется текстурный кэш)
\pause
\item Используем mipmap'ы
\end{itemize}
\end{frame}

\begin{frame}[fragile]
\frametitle{Оптимизация числа вершин}
\begin{itemize}
\item Используем индексированный рендеринг (меньше данных нужно прочитать из памяти; лучше используется вершинный кэш)
\pause
\item Используем примитивы, группирующие вершины \textendash{} line strip, triangle strip, triangle fan, etc (те же причины)
\pause
\item Используем LOD (level of detail)
\end{itemize}
\end{frame}

\begin{frame}[fragile]
\frametitle{Оптимизация количества OpenGL-вызовов}
\begin{itemize}
\item Batching: группируем объекты по используемому шейдеру, текстуре, другим настройкам (меньше переключения состояния OpenGL \begin{math}\Rightarrow\end{math} меньше OpenGL-вызовов)
\pause
\item Instancing: рисуем много объектов одним OpenGL-вызовом
\pause
\item Uniform buffers: передаём uniform-переменные не по одной, а записываем их в буффер (вместо большого количества вызовов \verb|glUniform*| один вызов \verb|glBufferData|)
\pause
\item Indirect rendering: переносим вычисления того, что нужно нарисовать, на GPU (OpenGL 4.0 + compute shaders)
\end{itemize}
\end{frame}

\begin{frame}[fragile]
\frametitle{Оптимизация чего угодно}
\begin{itemize}
\item Рисуем поменьше
\begin{itemize}
\item Frustum culling: не рисуем то, что не попадёт в камеру
\pause
\item Occlusion culling: не рисуем то, чего не видно (закрыто другими объектами)
\end{itemize}
\pause
\item Рисуем с уменьшенным разрешением
\begin{itemize}
\item Обычно используется для эффектов (blur и т.п.)
\end{itemize}
\pause
\item Переводим рисование в отдельный поток
\begin{itemize}
\item Освобождает основной (UI) поток
\pause
\item Позволяет делать полезную работу, пока render-поток ждёт VSync
\pause
\item Сильно усложняет код
\pause
\item Все OpenGL-вызовы нужно делать из render-потока
\pause
\item Применяется только в крайних случаях
\end{itemize}
\end{itemize}
\end{frame}

\begin{frame}[fragile]
\frametitle{Оптимизация: ссылки}
\begin{itemize}
\item \href{https://www.khronos.org/opengl/wiki/Performance}{khronos.org/opengl/wiki/Performance}
\item \href{https://www.opengl.org/pipeline/article/vol003_8}{opengl.org/pipeline/article/vol003\_8}
\item \href{https://www.nvidia.com/docs/IO/8230/GDC2003_OGL_Performance.pdf}{Доклад с GDC 2003, всё ещё актуальный}
\end{itemize}
\end{frame}

\begin{frame}[fragile]
\fontsize{10pt}{10pt}
\frametitle{LOD (level of detail)}
\begin{itemize}
\item Когда модель далеко от камеры, рисуем упрощённую модель вместо детализированной
\pause
\item Таких уровней детализации может быть несколько, вплоть до почти непрерывного изменения детализации (Unreal 5 Nanite)
\pause
\item Упрощённая модель может иметь отдельные VAO/VBO/EBO, а может лежать вместе с основной моделью (рисование конкретного LOD'а сводится к передаче правильных \verb|first| и \verb|count| в \verb|glDrawArrays| и т.п.)
\pause
\item Автоматическая генерация упрощённой модели \textendash{} предмет активных исследований
\begin{itemize}
\item Большинство современных подходов используют edge collapse: пара вершин, соединённых ребром, схлопывается в одну вершину
\end{itemize}
\pause
\item Конкретный уровень детализации выбирается на основе желаемого видимого размера треугольников
\pause
\item Выбор уровня детализации можно перенести на GPU (indirect rendering)
\end{itemize}
\end{frame}

\begin{frame}[fragile]
\frametitle{Batching}
\begin{itemize}
\item Может быть неявным: объекты, использующие один шейдер/материал/etc сами по себе лежат в одном месте
\pause
\item Может быть явным: движок рендеринга получает список объектов и сам их сортирует
\end{itemize}
\end{frame}

\begin{frame}[fragile]
\fontsize{10pt}{10pt}
\frametitle{Instanced rendering (instancing)}
\begin{itemize}
\item Позволяет рисовать несколько копий (instances) объекта одним OpenGL-вызовом
\pause
\item Обычно атрибуты вершин берутся в соответствии с индексом вершины: \verb|offset + stride * vertexID|
\pause
\item При использовании instancing конкретный атрибут вычисляется из номера instance: \verb|offset + stride * (instanceID / divisor)| (целочисленное деление)
\pause
\item Включить instancing для конкретного атрибута: \verb|glVertexAttribDivisor(index, divisor)|:
\begin{itemize}
\item \verb|divisor = 0|: атрибут не использует instancing и использует номер вершины
\item \verb|divisor != 0|: атрибут использует instancing и использует номер instance (по формуле выше)
\end{itemize}
\pause
\item Вызвать instanced rendering: \verb|glDrawArraysInstanced|
\begin{itemize}
\item Параметры \textendash{} те же, что у \verb|glDrawArrays|, плюс последний параметр \textendash{} количество instance'ов (значения \verb|instanceID| будут в диапазоне \verb|0 .. isntance_count-1|)
\end{itemize}
\pause
\item Аналогично есть \verb|glDrawElementsInstanced|
\end{itemize}
\end{frame}

\begin{frame}[fragile]
\frametitle{Instancing: ссылки}
\begin{itemize}
\item \href{https://www.khronos.org/opengl/wiki/Vertex_Rendering#Instancing}{khronos.org/opengl/wiki/Vertex\_Rendering\#Instancing}
\item \href{https://learnopengl.com/Advanced-OpenGL/Instancing}{learnopengl.com/Advanced-OpenGL/Instancing}
\item \href{https://ogldev.org/www/tutorial33/tutorial33.html}{ogldev.org/www/tutorial33/tutorial33.html}
\item \href{https://habr.com/ru/post/352962}{habr.com/ru/post/352962}
\end{itemize}
\end{frame}

\begin{frame}[fragile]
\frametitle{Uniform buffers}
\begin{itemize}
\item Позволяет использовать в качестве uniform-переменных данные из буфера
\pause
\item Специальный тип (target) для буферов: \verb|GL_UNIFORM_BUFFER| (создание и загрузка данных \textendash{} так же, как для других буферов)
\pause
\item В шейдере \textendash{} т.н. \textit{buffer-backed interface block}
\pause
\item Нужно быть внимательным с memory layout данных в буфере (конкретные правила описаны в спецификации)
\pause
\item Каждый interface block нужно привязать к \textit{binding index}: \verb|glGetUniformBlockIndex + glUniformBlockBinding| (в OpenGL 4.2 можно задать прямо в шейдере)
\item Uniform buffer нужно привизать к тому же \textit{binding index}: \verb|glBindBufferBase| или \verb|glBindBufferRange|
\end{itemize}
\end{frame}

\begin{frame}[fragile]
\fontsize{10pt}{10pt}
\frametitle{Uniform buffers: ссылки}
\begin{itemize}
\item \href{https://www.khronos.org/opengl/wiki/Uniform_Buffer_Object}{khronos.org/opengl/wiki/Uniform\_Buffer\_Object}
\item \href{https://www.khronos.org/opengl/wiki/Interface_Block_(GLSL)#Buffer_backed}{khronos.org/opengl/wiki/Interface\_Block\_(GLSL)\#Buffer\_backed}
\item \href{https://learnopengl.com/Advanced-OpenGL/Advanced-GLSL}{learnopengl.com/Advanced-OpenGL/Advanced-GLSL}
\end{itemize}
\end{frame}

\begin{frame}<1-2>[fragile,label=frustum_culling]
\fontsize{10pt}{10pt}
\frametitle{Frustum culling}
\begin{itemize}
\item Давайте не рисовать то, что заведомо не попадёт в камеру
\pause
\item Видимая область камеры \textendash{} параллелепипед (ортографическая проекция) или усечённая пирамида (перспективная проекция)
\pause
\item Для каждого объекта нужно посчитать какую-нибудь оболочку (bounding volume) и пересечь её с видимой областью
\pause
\item Оболочка должна:
\begin{itemize}
\item Как можно плотнее прилегать к объекту
\pause
\item Быть дешёвой для вычисления
\pause
\item Иметь поменьше вершин (упрощает поиск пересечений)
\pause
\item Быть выпуклой (упрощает поиск пересечений)
\end{itemize}
\pause
\item Варианты оболочки:
\begin{itemize}
\item Выпуклая оболочка \textendash{} плотно прилегает, но дорого вычислять и много вершин
\pause
\item Ограничивающая сфера/эллипсоид \textendash{} легко вычислять, довольно нетривиальный алгоритм пересечения
\pause
\item AABB (axis-aligned bounding box) \textendash{} легко вычислять (но нужно пересчитывать каждый кадр), не всегда плотно прилегает, легко искать пересечения
\pause
\item OBB (oriented bounding box) \textendash{} легко вычислять (предподсчитали для модели, поворачиваем вместе с моделью), плотно прилегает, легко искать пересечения
\end{itemize}
\end{itemize}
\end{frame}

\begin{frame}[fragile]
\frametitle{Усечённая пирамида (frustum)}
\slideimage{frustum.png}
\end{frame}

\againframe<3->{frustum_culling}

\begin{frame}[fragile]
\frametitle{Frustum culling: SAT}
\begin{itemize}
\item Для детектирования пересечения выпуклых многогранников (frustum, AABB, OBB, etc.) обычно используется алгоритм, основанный на SAT \textendash{} Separating Axis Theorem (в математике известна как HST \textendash{} Hyperplane Separation Theorem)
\pause
\item Пусть A и B \textendash{} два непересекающихся замкнутых выпуклых подмножества Евклидова пространства (есть версия для Банаховых пространств). Тогда:
\pause
\begin{itemize}
\item HST: существует гиперплоскость (separating hyperplane), проходящая между этими подмножествами
\pause
\item SAT: существует прямая (separating axis), такая, что проекции A и B на эту прямую не пересекаются
\pause
\item HST \begin{math}\Leftrightarrow\end{math} SAT: гиперплоскость \begin{math}\Leftrightarrow\end{math} перпендикулярная ей прямая
\end{itemize}
\pause
\item N.B.: на этой же теореме основывается метод SVM \textendash{} Support Vector Machine
\item N.B.: тот же алгоритм используется для детектирования столкновений в физических движках
\end{itemize}
\end{frame}

\begin{frame}[fragile]
\frametitle{Frustum culling: SAT}
\slideimage{sat.png}
\end{frame}

\begin{frame}[fragile]
\frametitle{Frustum culling: вычисление проекции на прямую}
\begin{itemize}
\item Проекция выпуклого замкнутого ограниченного множества на прямую \textendash{} отрезок
\pause
\item Нужно выбрать некоторую точку \begin{math}o\end{math} на прямой, тогда проекция точки \begin{math}p\end{math} на прямую вычисляется как \begin{math}(p - o) \cdot n\end{math} (где \begin{math}n\end{math} \textendash{} вектор направления прямой)
\pause
\item Выбор другой точки \begin{math}o\end{math} или замена вектора \begin{math}n\end{math} на коллинеарный приведёт к сдвигу и масштабированию проекций \begin{math}\Rightarrow\end{math} непересекающиеся проекции останутся непересекающимися
\pause
\item \begin{math}\Rightarrow\end{math} нам не важна начальная точка, можно вычислять \begin{math}p \cdot n\end{math} (интерпретируя точку \begin{math}p\end{math} как радиус-вектор из начала координат)
\end{itemize}
\end{frame}

\begin{frame}[fragile]
\fontsize{10pt}{10pt}
\frametitle{Frustum culling: вычисление проекции на прямую}
Псевдокод вычисления проекции выпуклого множества на прямую:
\begin{verbatim}
float vmin = inf, vmax = -inf;
for (p : vertices) {
  float v = dot(p, n);
  vmin = min(vmin, v);
  vmax = max(vmax, v);
}
\end{verbatim}
\end{frame}

\begin{frame}[fragile]
\fontsize{10pt}{10pt}
\frametitle{Frustum culling: вычисление пересечения проекций}
\begin{itemize}
\item Число принадлежит отрезку \verb|[vmin, vmax]|, если \begin{math}v_{min} \leq v \leq v_{max}\end{math}
\pause
\item Число принадлежит пересечению двух отрезков, если выполняются два уравнения
\begin{equation}
\begin{cases}
v^1_{min} \leq v \leq v^1_{max} \\
v^2_{min} \leq v \leq v^2_{max}
\end{cases}
\end{equation}
\pause
\item Чтобы эта система имела решения (т.е. отрезки пересекались), нужно
\begin{equation}
\begin{cases}
v^1_{min} \leq v^2_{max} \\
v^2_{min} \leq v^1_{max}
\end{cases}
\end{equation}
\end{itemize}
\end{frame}

\begin{frame}[fragile]
\fontsize{10pt}{10pt}
\frametitle{Frustum culling: вычисление пересечения проекций}
Псевдокод детектирования пересечения двух отрезков:
\begin{verbatim}
if (v1min <= v2max && v2min <= v1max)
  return true;
else
  return false;
\end{verbatim}
\end{frame}

\begin{frame}[fragile]
\frametitle{Frustum culling: SAT}
\begin{itemize}
\item Если проекции объектов на любые прямые пересекаются, то объекты пересекаются, иначе \textendash{} не пересекаются
\pause
\item Возможных прямых бесконечно много, мы можем проверить только конечное их число
\end{itemize}
\end{frame}

\begin{frame}[fragile]
\frametitle{Frustum culling: SAT, 2D}
\begin{itemize}
\item Будем мысленно сдвигать объекты вдоль separating axis друг к другу до первого пересечения
\pause
\item Три варианта пересечения:
\begin{itemize}
\item Ребро-ребро
\item Ребро-вершина
\item Вершина-вершина
\end{itemize}
\pause
\item Во всех трёх случаях в качестве направления separating axis можно взять нормаль к какому-нибудь ребру
\pause
\item SAT для выпуклых многоугольников: в качестве множества направлений для separating axis берём множество нормалей к рёбрам обоих объектов
\end{itemize}
\end{frame}

\begin{frame}[fragile]
\frametitle{Frustum culling: SAT, 3D}
\begin{itemize}
\item Можно рассмотреть ту же идею, но в 3D больше случаев
\pause
\item SAT для выпуклых многогранников: в качестве множества направлений для separating axis берём
\begin{itemize}
\item Множество нормалей к граням обоих объектов
\item + множество попарных векторных произведений \begin{math}e_1 \times e_2\end{math}, где \begin{math}e_1\end{math} и \begin{math}e_2\end{math} \textendash{} рёбра первого и второго многогранников, соответственно (для всех пар рёбер)
\end{itemize}
\pause
\item N.B.: нас интересуют только направления с точностью до умножения на константу, так что во многих случаях необязательно рассматривать все грани и рёбра
\pause
\begin{itemize}
\item Например, у куба/параллелепипеда только три неколлинеарных нормали и три неколлинеарных ребра
\end{itemize}
\end{itemize}
\end{frame}

\begin{frame}[fragile]
\fontsize{8pt}{8pt}
\frametitle{Frustum culling: SAT, 3D}
\begin{verbatim}
bool intersect_along(Body b1, Body b2, vec3 n) {
  auto p1 = project(b1.vertices, n);
  auto p2 = project(b2.vertices, n);
  return intersect(p1, p2);
}

bool intersect(Body b1, Body b2) {
  for (n : b1.face_normals) {
    if (intersect_along(b1, b2, n))
      return true;
  }

  for (n : b2.face_normals) {
    if (intersect_along(b1, b2, n))
      return true;
  }

  for (e1 : b1.edges) {
    for (e2 : b2.edges) {
      vec3 n = cross(e1, e2);
      if (intersect_along(b1, b2, n))
        return true;
    }
  }

  return false;
}
\end{verbatim}
\end{frame}

\begin{frame}[fragile]
\frametitle{Frustum culling: вычисление camera frustum}
\begin{itemize}
\item В независимости от проекции, комбинаторно видимая область \textendash{} куб, т.е. имеет 8 вершин, 12 рёбер, и 6 граней, соединённых как в кубе
\pause
\item Достаточно вычислить координаты 8-ми вершин
\pause
\item Можно вычислить напрямую из параметров и свойств камеры \begin{math}\Rightarrow\end{math} сложный ad-hoc алгоритм
\pause
\item Можно вычислить из матрицы \begin{math}T\end{math} (view + projection):
\pause
\begin{itemize}
\item Координаты вершин видимой области в NDC (normalized device coordinates): \begin{math}(\pm 1, \pm 1, \pm 1)\end{math}
\pause
\item Координаты вершин видимой области в сцене: \begin{math}\operatorname{Proj}\left(T^{-1} \cdot (\pm 1, \pm 1, \pm 1, 1)\right)\end{math}
\end{itemize}
\end{itemize}
\end{frame}

\begin{frame}[fragile]
\frametitle{Frustum culling}
\begin{itemize}
\item Объекты можно как-то групировать, чтобы отсекать сразу большие группы объектов, не попадающие в камеру:
\pause
\begin{itemize}
\item Деревья (BVH \textendash{} bounding volume hierarchy, octree, R-tree, ...)
\pause
\item Сетки/bins: группируем объекты в ячейки квадратной/кубической сетки, отсекаем сразу целые ячейки
\end{itemize}
\pause
\item Можно перевести отсечение на GPU (indirect rendering)
\end{itemize}
\end{frame}

\begin{frame}[fragile]
\frametitle{Frustum culling: ссылки}
\begin{itemize}
\item Есть много способов написать этот алгоритм; есть вариации, соптимизированные для параллелограммов/кубов
\item Часто в туториалах описывают неправильный/неполный алгоритм, выдающий false positive пересечения \textendash{} не страшно для frustum culling, но неэффективно
\item \href{https://en.wikipedia.org/wiki/Hyperplane_separation_theorem}{en.wikipedia.org/wiki/Hyperplane\_separation\_theorem}
\item \href{https://www.geometrictools.com/Documentation/MethodOfSeparatingAxes.pdf}{Статья с разбором SAT и выводом алгоритмов для 2D и 3D}
\item \href{https://learnopengl.com/Guest-Articles/2021/Scene/Frustum-Culling}{Большая статья с разбором разных вариантов bounding volume}
\end{itemize}
\end{frame}

\begin{frame}[fragile]
\fontsize{10pt}{10pt}
\frametitle{Occlusion culling}
\begin{itemize}
\item Не будем рисовать объекты (ocludees), закрытые другими объектами (occluders)
\pause
\item \textbf{Очень} много вариаций
\pause
\item Если есть заранее известные occluders (здания, ландшафт)
\pause
\begin{itemize}
\item Можно растеризовать их на CPU в буфер глубины низкого разрешения, и использовать его для проверки видимости
\pause
\item Можно нарисовать их обычным способом, и затем полагаться на early depth test (z pre-pass)
\end{itemize}
\pause
\item Можно использовать буфер глубины с предыдущего кадра
\pause
\begin{itemize}
\item Нужно запомнить матрицу проекции предыдущего кадра
\pause
\item Может давать неточный результат
\end{itemize}
\pause
\item Лучше построить max-mipmaps по буферу глубины (HiZ \textendash{} hierarchical Z): специальным шейдером строить mipmaps, вычисляя максимум среди группы 2x2 пикселей
\pause
\begin{itemize}
\item По размеру объекта (ocludee) вычисляем mipmap-уровень: если в Z-буфере значение меньше, чем минимальная глубина нашего объекта, то объект не будет видно
\end{itemize}
\pause
\item Можно сгенерировать HiZ по списку известных occluders
\pause
\item Можно запомнить список видимых объектов с прошлого кадра и использовать их как occluders
\end{itemize}
\end{frame}

\begin{frame}[fragile]
\frametitle{Occlusion culling}
\begin{itemize}
\item Главное \textendash{} избежать \textit{pipeline stall}, т.е. блокирующего ожидания окончания работы GPU на CPU
\pause
\item Если Z-буфер для occlusion culling (возможно, в виде HiZ) построен на CPU, проверку тоже можно осуществлять на CPU
\pause
\item Если Z-буфер построен на GPU (взят с прошлого кадра, или построен по известным occluders), проверку хочется делать на GPU
\pause
\begin{itemize}
\item Можно использовать compute shaders + indirect rendering (OpenGL 4.0)
\pause
\item Можно использовать occlusion queries + conditional rendering
\end{itemize}
\end{itemize}
\end{frame}

\begin{frame}[fragile]
\frametitle{Occlusion culling: conditional rendering}
\begin{itemize}
\item Выключаем рисование в z-буфер и в цветовой буфер (нас интересует только то, \textbf{будет ли нарисован} объект)
\pause
\item Внутри \verb|glBeginQuery(GL_ANY_SAMPLES_PASSED, query_id)| рисуем дешёвую аппроксимацию объекта (напр. bounding box)
\pause
\item Обратно включаем рисование в z-буфер и в цветовой буфер
\pause
\item Внутри пары \verb|glBeginConditionalRender(query_id, GL_QUERY_NO_WAIT)| и \verb|glEndConditionalRender| рисуем сам объект
\end{itemize}
\end{frame}

\begin{frame}[fragile]
\frametitle{Occlusion culling: ссылки}
\begin{itemize}
\item \href{https://www.khronos.org/opengl/wiki/Vertex_Rendering#Conditional_rendering}{Статья про conditional rendering на OpenGL wiki}
\item \href{https://arm-software.github.io/opengl-es-sdk-for-android/occlusion_culling.html}{Статья про много вариантов реализации occlusion culling}
\item \href{https://developer.nvidia.com/gpugems/gpugems2/part-i-geometric-complexity/chapter-6-hardware-occlusion-queries-made-useful}{Ещё одна статья}
\item \href{https://interplayoflight.wordpress.com/2017/11/15/experiments-in-gpu-based-occlusion-culling}{И ещё одна статья}
\item \href{http://www.diva-portal.org/smash/get/diva2:934562/FULLTEXT02.pdf}{И ещё одна, очень большая, статья}
\end{itemize}
\end{frame}

\end{document}