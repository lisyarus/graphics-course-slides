% (c) Nikita Lisitsa, lisyarus@gmail.com, 2023

\documentclass{beamer}

\usepackage[T2A]{fontenc}
\usepackage[russian]{babel}
\usepackage{minted}

\usepackage{graphicx}
\graphicspath{ {./images/} }

\usepackage{adjustbox}

\usepackage{tikz}

\usepackage{color}
\usepackage{soul}

\usepackage{hyperref}

\definecolor{blue}{rgb}{0,0,1}
\definecolor{red}{rgb}{1,0,0}

\usetheme{metropolis}
\setminted{fontsize=\footnotesize}

\makeatletter
\newcommand{\slideimage}[1]{
  \begin{figure}
    \begin{adjustbox}{width=\textwidth, totalheight=\textheight-2\baselineskip-2\baselineskip,keepaspectratio}
      \includegraphics{#1}
    \end{adjustbox}
  \end{figure}
}
\makeatother

\title{Компьютерная графика}
\subtitle{Финальный проект}
\date{2024}

\setbeamertemplate{footline}[frame number]

\begin{document}

\frame{\titlepage}

\begin{frame}[fragile]
\frametitle{Задание}
\begin{itemize}
\item Можно сделать любое количество заданий из описанных в данных слайдах
\pause
\item Каждое задание -- отдельный мини-проект с некоторым количеством самостоятельного изучения
\pause
\item Как обычно, можно сделать задание не полностью
\pause
\item Во всех заданиях должны двигаться камера и источники света (если они есть), если не написано обратное
\pause
\item Все альбедо-текстуры должны загружаться как sRGB
\pause
\item Во всех заданиях должны быть гамма-коррекция и какой-нибудь tone mapping (лучше Uncharted или ACES)
\end{itemize}
\end{frame}

\begin{frame}[fragile]
\frametitle{Что можно использовать}
\begin{itemize}
\item Можно пользоваться вспомогательными библиотеками (например, для загрузки текстур, моделей и сцен)
\pause
\item Весь OpenGL-код должен быть написан вами, т.е. библиотека не должна загружать данные на GPU, создавать текстуры, и т.п.
\pause
\item Можно брать код из практик и домашних заданий, в т.ч. загрузчики моделей, анимаций и шрифтов (при необходимости их можно доработать)
\pause
\item Можно брать сцены и текстуры из практик и домашних заданий
\end{itemize}
\end{frame}

\begin{frame}[fragile]
\frametitle{Список заданий}
\begin{itemize}
\item Планета Земля с volumetric атмосферой
\item Большой лес с impostor-деревьями
\item Metaballs
\item Бассейн с водой
\end{itemize}
\end{frame}

\begin{frame}
\frametitle{Планета}
\slideimage{planet.jpg}
\end{frame}

\begin{frame}[fragile]
\frametitle{Планета: задание}
\fontsize{8pt}{8pt}
\selectfont
\begin{itemize}
\item Сфера с текстурой нашей планеты
\item Освещение по Фонгу движущимся солнцем
\item На неосвещённой стороне используем другую текстуру ``ночной Земли''
\item Карта glossines для specular-освещения (это будет, в основном, вода)
\item Карта высот планеты, сдвигающая вершины вверх
\begin{itemize}
\fontsize{8pt}{8pt}
\selectfont
\item Лучше высоты искуственно на что-нибудь домножить, чтобы горы было лучше видно
\item Здесь, возможно, придётся добавить геометрический шейдер для расчёта нормалей
\end{itemize}
\item Тени обычным shadow-mapping'ом
\item Volumetric слой атмосферы (сфера чуть большего радиуса вокруг Земли) с рассеянием и объёмными тенями
\item Облака -- ещё одна сфера где-то посередине между уровнем моря и самыми высокими горами
\begin{itemize}
\fontsize{8pt}{8pt}
\selectfont
\item На них тоже должно быть освещение (можно только диффузное) и тени
\item Они сами должны отбрасывать тень на Землю
\end{itemize}
\end{itemize}
\end{frame}

\begin{frame}[fragile]
\frametitle{Планета: советы}
\fontsize{8pt}{8pt}
\selectfont
\begin{itemize}
\item Вершины сферы можно сгенерировать самим, проще всего через сферические координаты, равномерным шагом по широте и долготе
\item Можно не хранить сами вершины, и вычислять их из номера вершины в шейдере
\item Все текстуры легко гуглятся, главное -- чтобы они использовали географическую проекцию и не были обрезаны по полюсам
\item В карте высот надо разобраться, в каком формате хранятся высоты
\item В карте высот могут быть глубины океанов, их лучше заменить нулём (уровнем моря)
\item В вычислении проекции для shadow mapping'а нужно учесть, что вы поднимаете точки сферы на значение высоты
\item Накладывать атмосферу можно в шейдере, рисующем Землю/облака, но тогда нужно нарисовать ещё одну сферу с front-face culling, чтобы нарисовалась часть атмосферы в космосе
\item Можно рисовать атмосферу отдельной сферой (с обычным back-face culling) вокруг планеты со своим шейдером (и с прозрачностью!), но тогда вам нужен свой буфер глубины, чтобы прочитать из него расстояние до нарисованной точки на поверхности/облаках
\item Коэффициенты рассеяния Релея берём из реальных данных для красного, зелёного и синего цвета (тоже легко гуглятся)
\begin{itemize}
\fontsize{8pt}{8pt}
\selectfont
\item Они зависят от высоты точки над поверхностью Земли, это важно учесть
\item Нужно быть внимательными с единицами измерения!
\end{itemize}
\item Проще всего будет работать с реальными величинами: радиус Земли \begin{math}\approx\end{math} 6400 км, толщина атмосферы \begin{math}\approx\end{math} 100 км
\end{itemize}
\end{frame}

\begin{frame}[fragile]
\frametitle{Планета: баллы}
\fontsize{8pt}{8pt}
\selectfont
\begin{itemize}
\item \textbf{1 балл}: есть сфера, освещённая по Фонгу
\item \textbf{1 балл}: текстура альбедо Земли + ночной стороны Земли
\item \textbf{1 балл}: текстура glossiness
\item \textbf{1 балл}: вершины сдвигаются картой высот
\item \textbf{2 балла}: по карте высот посчитаны нормали
\item \textbf{2 балла}: shadow mapping
\item \textbf{2 балла}: объёмная атмосфера с рассеянием
\item \textbf{2 балла}: рассеяние учитывает тени
\item \textbf{2 балла}: облака с тенями и отбрасывающие тень
\end{itemize}
Максимум: \textbf{14 баллов}
\end{frame}

\begin{frame}
\frametitle{Лес}
\slideimage{impostors-forest.jpg}
\end{frame}

\begin{frame}[fragile]
\frametitle{Лес: задание}
\fontsize{8pt}{8pt}
\selectfont
\begin{itemize}
\item Плоскость с какой-нибудь текстурой (например, травы)
\item На плоскости много (напр. 10000) деревьев, нарисованных техникой \textit{impostors}:
\begin{itemize}
\fontsize{8pt}{8pt}
\selectfont
\item Вместо объекта рисуется billboard -- прямоугольник, повёрнутый в сторону камеры
\item На прямоугольник наносится одна из заранее подготовленных текстур объекта в зависимости от направления взгляда
\end{itemize}
\item На старте программы по некой сетке направлений рендерим наше дерево (без освещения!) отдельно в текстуры альбедо, нормалей, материала, и глубины (а-ля gbuffer)
\item Генерируем набор случайных позиций, поворотов (в горизонтальной плоскости), масштабов, и цветов деревьев, равномерно распределённых на плоскости
\item В процессе работы программы instancing'ом или геометрическим шейдером рисуем все деревья одной командой рисования
\item Конкретные текстуры для одного инстанса выбираются по направлению на камеру
\item Вместо одного направления можно брать 4 соседних и интерполировать по ним
\item На деревьях обычное освещение по Фонгу + тени (обычным shadow mapping'ом)
\end{itemize}
\end{frame}

\begin{frame}
\frametitle{Лес: impostor текстуры}
\slideimage{impostor-textures.jpg}
\end{frame}

\begin{frame}[fragile]
\frametitle{Лес: советы}
\fontsize{8pt}{8pt}
\selectfont
\begin{itemize}
\item В качестве объекта берём любую 3D-модель с деревом
\item В качестве сетки направлений можно взять полусферу с фиксированными шагами по высоте и широте
\item Для impostor текстур хорошо использовать array texture (один слой -- одно направление из сетки)
\item Вершинный/геометрический шейдер поворачивает прямоугольник в сторону камеры и выбирает номер слоя array-текстуры на основании направления на камеру и поворота самого инстанса дерева
\item Фрагментный шейдер читает эти текстуры, вычисляет освещение, и выставляет \mintinline{glsl}|gl_FragDepth| используя буфер глубины impostor'а
\item Цвет инстанса дерева можно просто смешать с текстурой альбедо перед вычислением освещения 
\item Не вся текстура impostor'а будет содержать дерево; для полупрозрачности можно использовать \mintinline{glsl}|discard|
\item Текстура альбедо должна быть в sRGB!
\item Нужно, чтобы корень дерева всегда оставался в одном и том же месте!
\item Удобно расположить корень дерева (т.е. позицию инстанса) в фиксированном месте прямоугольника impostor'а: например, в центре по X, и в центре или в самом внизу по Y
\item Для теней нужно использовать impostor, смотрящий в сторону источника света, а не в сторону камеры!
\end{itemize}
\end{frame}

\begin{frame}[fragile]
\frametitle{Лес: баллы}
\fontsize{8pt}{8pt}
\selectfont
\begin{itemize}
\item \textbf{1 балл}: есть плоскость с текстурой, освещённая по Фонгу
\item \textbf{2 балла}: инстансингом или геометрическим шейдером рисуются impostor-прямоугольники
\item \textbf{1 балл}: у инстансов есть случайные повороты, масштабы, и цвета
\item \textbf{2 балла}: генерируются и используются текстуры альбедо impostor'ов
\item \textbf{2 балла}: генерируются и используются текстуры нормалей impostor'ов
\item \textbf{1 балл}: генерируются и используются текстуры материала impostor'ов
\item \textbf{1 балл}: генерируются и используются текстуры глубины impostor'ов
\item \textbf{2 балла}: shadow mapping
\item \textbf{2 балла}: интерполяция между 4 соседними направлениями для всех текстур impostor'ов
\end{itemize}
Максимум: \textbf{14 баллов}
\end{frame}

\begin{frame}
\frametitle{Metaballs}
\slideimage{metaballs.jpg}
\end{frame}

\begin{frame}[fragile]
\frametitle{Metaballs: задание}
\fontsize{8pt}{8pt}
\selectfont
\begin{itemize}
\item Что-то в духе первого домашнего задания, но в 3D
\item Описываем набор трёхмерных шариков, как-то движущихся в 3D, каждый со своей позицией \begin{math}p_i\end{math}, радиусом \begin{math}r_i\end{math}, и цветом \begin{math}c_i\end{math}
\item Рисуем изоповерхность функции \begin{math}f(p) = K\end{math} с настраиваемым значением \begin{math}K\end{math}
\item В качестве функции берём \begin{math}f(p) = \sum w_i = \sum \exp\left(-\frac{|p-p_i|}{r_i}\right)\end{math}
\item Цвета интерполируем с softmax-весами: итоговый цвет это \begin{math}\sum \lambda_i c_i\end{math} где \begin{math}\lambda_i = \frac{w_i}{\sum w_j} = \frac{w_i}{f(p)}\end{math}
\item Функцию и цвета нужно хранить как 3D текстуру и обновлять на каждый кадр (руками, или с помощью рендеринга по слоям, или compute shader'ом)
\item Изоповерхность рисуем по квадратной сетке алгоритмом marching cubes или marching tetrahedra в геометрическом шейдере
\item У поверхности должны быть нормали и освещение по Фонгу (со specular составляющей)
\end{itemize}
\end{frame}

\begin{frame}[fragile]
\frametitle{Metaballs: советы}
\fontsize{8pt}{8pt}
\selectfont
\begin{itemize}
\item Для генерации 3D текстуры с функцией на GPU можно передать все шарики в шейдер массивом uniform-переменных, uniform-блоком, или buffer texture
\item Можно генерировать её аддитивным блендингом, так как функция это сумма вкладов каждого шара по отдельности (тогда не нужно передавать все шарики в шейдер -- они рисуются по одному)
\item В текстуре можно хранить цвет, умноженный на значение функции (т.е. \begin{math}\sum w_i c_i\end{math}), и делить его на сумму весов уже при чтении этой текстуры, -- тогда его тоже можно рисовать аддитивным блендингом
\item Можно хранить и цвет, и значение функции в одной текстуре с форматом \mintinline{cpp}|RGBA32F| (RGB -- цвет, A -- значение функции)
\item Кубическую сетку для marching cubes можно не хранить явно, а вычислять на основе координат вершин
\item В marching cubes очень много случаев, вместо него можно разбить каждый квадрат сетки на тетраэдры (вдоль диагонали) и делать на них marching tetrahedra, в котором всего 4 различных случая
\item Нормали можно взять плоские, а можно попробовать восстановить гладкие нормали по градиенту функции \begin{math}\nabla f(p)\end{math} (который можно посчитать конечными разностями)
\end{itemize}
\end{frame}

\begin{frame}[fragile]
\frametitle{Metaballs: баллы}
\fontsize{8pt}{8pt}
\selectfont
\begin{itemize}
\item \textbf{1 балл}: как-нибудь симулируются летающие шарики
\item \textbf{2 балла}: на CPU генерируется 3D текстура со значениями функции и цветами
\item либо \textbf{3 балла}: текстура генерируется на GPU любым способом
\begin{itemize}
\fontsize{8pt}{8pt}
\selectfont
\item \textbf{+2 балла}: текстура генерируется аддитивным блендингом
\item либо \textbf{+2 балла}: текстура генерируется compute shader'ом
\end{itemize}
\item \textbf{3 балла}: строятся изоповерхности функции
\item \textbf{2 балла}: изоповерхности раскрашены в правильные цвета
\item \textbf{2 балла}: изоповерхности используют нормали и освещение по Фонгу
\begin{itemize}
\fontsize{8pt}{8pt}
\selectfont
\item \textbf{+1 балл}: нормали гладкие, восстановлены по значениям функции
\end{itemize}
\end{itemize}
Максимум: \textbf{14 баллов}
\end{frame}

\begin{frame}
\frametitle{Вода}
\slideimage{water.jpg}
\end{frame}

\begin{frame}[fragile]
\frametitle{Вода: задание}
\fontsize{8pt}{8pt}
\selectfont
\begin{itemize}
\item Плоская поверхность ``дна'' с какой-нибудь простой повторяющейся текстурой и диффузным освещением от солнца (стенки ``бассейна'' рисовать не нужно)
\item Динамическая поверхность воды на какой-то высоте над дном, высота и нормали вершин вычисляются на основе суммы каких-нибудь синусоид
\item Вокруг -- какой-нибудь environment map
\item На воде должны быть отражения и преломления
\item На дне должны быть каустики от солнца
\item В этом задании солнце может быть статическим
\end{itemize}
\end{frame}

\begin{frame}[fragile]
\frametitle{Вода: советы}
\fontsize{8pt}{8pt}
\selectfont
\begin{itemize}
\item Меш воды лучше взять довольно детализированный; координаты вершин можно вычислять на основе номера вершины
\item Нормаль вычисляется явно через производную функции высоты \begin{math}z = f(x,y)\end{math} как \begin{math}\left(-\frac{\partial f}{\partial x},-\frac{\partial f}{\partial y}, 1\right)\end{math}
\item Вода частично отражает, а частично преломляет свет; коэффициенты нужно взять из аппроксимации Шлика (Schlick) уравнений Френеля; для воды коэффициент преломления берём \begin{math}n = 1.333\end{math}
\item Угол преломлённого луча вычисляем через закон Снеллиуса (Snell's law)
\item Для отражённого луча берём свет из environment map
\item Преломлённый луч пересекаем руками с прямоугольником дна: если пересёкся, возвращаем цвет дна в этой точке (с учётом текстуры и освещения), иначе -- тоже свет из environment map
\item Каустики рисуем отдельным проходом в особую текстуру аддитивным блендингом: вершинный шейдер превращает каждую вершину в точку на дне, в которую попадёт преломлённый луч от солнца
\item Наложенные каустики должны быть видны на самом дне, и через воду
\end{itemize}
\end{frame}

\begin{frame}[fragile]
\frametitle{Вода: баллы}
\fontsize{8pt}{8pt}
\selectfont
\begin{itemize}
\item \textbf{1 балл}: есть плоскость дна с текстурой, освещённая по Фонгу
\item \textbf{1 балл}: есть environment map
\item \textbf{2 балла}: есть динамическая поверхность воды с плавными нормалями
\item \textbf{2 балла}: есть отражение environment map
\item \textbf{2 балла}: есть преломление
\item \textbf{1 балла}: отражение + преломление по закону Френеля
\item \textbf{3 балла}: генерируется текстура каустик
\item \textbf{1 балл}: каустики видны на дне
\item \textbf{1 балл}: каустики видны через воду
\end{itemize}
Максимум: \textbf{14 баллов}
\end{frame}

\begin{frame}[fragile]
\frametitle{Алгоритмы}
\begin{itemize}
\item Любое задание, сделанное хотя бы на 5 баллов, можно выложить на GitHub, приложить красивый скриншот, и сделать Readme, в котором будет показан этот скриншот, -- за это даётся ещё \textbf{1 балл}
\pause
\item Если у вас есть другая идея, которую очень хочется реализовать -- напишите мне, обсудим
\pause
\item Сдача в день зачёта или экзамена
\end{itemize}
\end{frame}

\end{document}