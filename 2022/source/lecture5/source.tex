% (c) Nikita Lisitsa, lisyarus@gmail.com, 2021

\documentclass{beamer}

\usepackage[T2A]{fontenc}
\usepackage[utf8]{inputenc}
\usepackage[russian]{babel}

\usepackage{graphicx}
\graphicspath{ {./images/} }

\usepackage{adjustbox}

\usepackage{color}
\usepackage{soul}

\usepackage{hyperref}

\usepackage{tikz}
\usetikzlibrary{decorations}
\usetikzlibrary{decorations.pathreplacing}
\usepackage{xifthen}

\usepackage{alltt}

\definecolor{red}{rgb}{1,0,0}
\definecolor{green}{rgb}{0,0.5,0}
\definecolor{blue}{rgb}{0,0,1}
\definecolor{magenta}{rgb}{0.75,0,0.75}

\makeatletter
\newcommand{\slideimage}[1]{
  \begin{figure}
    \begin{adjustbox}{width=\textwidth, totalheight=\textheight-2\baselineskip-2\baselineskip,keepaspectratio}
      \includegraphics{#1}
    \end{adjustbox}
  \end{figure}
}
\makeatother

\title{Компьютерная графика}
\subtitle{Лекция 5: Текстуры}
\date{2021}

\setbeamertemplate{footline}[frame number]

\begin{document}

\frame{\titlepage}

\begin{frame}[fragile]
\frametitle{Что такое текстуры?}
\begin{itemize}
\item Изображения в памяти GPU (чаще -- несколько изображений)
\pause
\item Их можно читать из шейдеров
\pause
\item В них можно загружать пиксели
\pause
\item В них можно рисовать (вместо рисования в окно)
\pause
\item Поддерживают много специфичных для изображений операций
\end{itemize}
\end{frame}

\begin{frame}[fragile]
\frametitle{Что нужно знать про текстуры?}
\begin{itemize}
\item Типы текстур в OpenGL
\pause
\item Как работать с текстурами в шейдере
\pause
\item Какие есть настройки чтения из текстур
\pause
\item Как загрузить данные в текстуру
\pause
\item Как привязать текстуру к шейдеру
\pause
\item Как использовать текстуру для 3D моделей
\end{itemize}
\end{frame}

\begin{frame}[fragile]
\frametitle{Что такое текстуры в OpenGL?}
\begin{itemize}
\item Объекты OpenGL
\pause
\item \verb|glGenTextures|, \verb|glDeleteTextures|, \verb|glBindTexture|
\pause
\item \verb|target| -- тип текстуры (об этом чуть позже)
\pause
\item Если текстура сделана текущей для какого-то \verb|target|, её никогда нельзя сделать текущей для другого \verb|target|
\begin{itemize}
\item Таким образом, тип текстуры определяется первым вызовом \verb|glBindTexture| для неё
\end{itemize}
\pause
\item ID текущей для конкретного \verb|target| текстуры хранится не в контексте OpenGL, а в т.н. texture unit'ах (о них позже)
\end{itemize}
\end{frame}

\begin{frame}<1>[fragile,label=types]
\frametitle{Типы текстур}
\begin{itemize}
\item \verb|GL_TEXTURE_1D| -- одномерная текстура (полоска пикселей)
\pause
\item \verb|GL_TEXTURE_2D| -- двумерная текстура
\pause
\item \verb|GL_TEXTURE_3D| -- трёхмерная текстура
\pause
\item \verb|GL_TEXTURE_1D_ARRAY| -- массив одномерных текстур одинакового размера
\pause
\item \verb|GL_TEXTURE_2D_ARRAY| -- массив двумерных текстур одинакового размера
\pause
\item \verb|GL_TEXTURE_CUBE_MAP| -- текстуры, хранящиеся как 6 граней куба
\pause
\item \verb|GL_TEXTURE_CUBE_MAP_ARRAY| -- массив cubemap'ов (OpenGL 4.0)
\pause
\item \verb|GL_TEXTURE_RECTANGLE| -- прямоугольные текстуры \pause (плохое название, другие текстуры тоже могут быть прямоугольными; в современном OpenGL бесполезны)
\pause
\item \verb|GL_TEXTURE_BUFFER| -- текстуры, берущие данные из buffer object'а
\end{itemize}
\end{frame}

\begin{frame}[fragile]
\frametitle{Одномерная текстура (GL\_TEXTURE\_1D)}
\begin{itemize}
\item Градиенты цветов
\item Предподсчитанные значения функций (напр. для освещения)
\end{itemize}
\slideimage{texture_1d.png}
\end{frame}

\againframe<2>{types}

\begin{frame}[fragile]
\frametitle{Двумерная текстура (GL\_TEXTURE\_2D)}
\slideimage{texture_2d.jpg}
\end{frame}

\begin{frame}[fragile]
\frametitle{Двумерная текстура (GL\_TEXTURE\_2D)}
\begin{itemize}
\item Самый часто используемый тип
\item Любые числовые свойства поверхностей: цвет (альбедо), нормаль, тип материала
\item Карта высот ландшафта
\item Элементы UI
\item Предподсчитанные значения 2D функции
\item etc.
\end{itemize}
\end{frame}

\againframe<3>{types}

\begin{frame}[fragile]
\frametitle{Трёхмерная текстура (GL\_TEXTURE\_3D)}
\slideimage{texture_3d.png}
\end{frame}

\begin{frame}[fragile]
\frametitle{Трёхмерная текстура (GL\_TEXTURE\_3D)}
\begin{itemize}
\item Любые числовые свойства пространства: плотность, цвет, коэффициент рассеяния
\item Объёмные (volumetric) данные: дым, туман, и т.п.
\item Воксельные данные
\item Предподсчитанные значения 3D функции
\end{itemize}
\end{frame}

\againframe<4>{types}

\begin{frame}[fragile]
\frametitle{Массив одномерных текстур (GL\_TEXTURE\_1D\_ARRAY)}
\begin{itemize}
\item Набор градиентов цветов
\end{itemize}
\slideimage{texture_1d_array.png}
\end{frame}

\againframe<5>{types}

\begin{frame}[fragile]
\frametitle{Массив двумерных текстур (GL\_TEXTURE\_2D\_ARRAY)}
\begin{itemize}
\item Массив текстур для однотипных объектов, например, вокселей
\end{itemize}
\slideimage{texture_2d_array.png}
\end{frame}

\againframe<6>{types}

\begin{frame}[fragile]
\frametitle{Cubemap-текстура (GL\_TEXTURE\_CUBEMAP)}
\begin{itemize}
\item Небо (skybox), отражения (environment map, reflection map), предпосчитанные значения функции на сфере
\end{itemize}
\slideimage{texture_cubemap.png}
\end{frame}

\againframe<7->{types}

\begin{frame}[fragile]
\frametitle{Текстуры в шейдере - samplers}
\begin{itemize}
\item В шейдере текстура представлена uniform-переменной с особым типом, соответствующим типу текстуры и формату пикселей (floating-point / integer / unsigned integer)
\pause
\item \verb|sampler1D/isampler1D/usampler1D|
\pause
\item \verb|sampler2D/isampler2D/usampler2D|
\pause
\item \verb|sampler3D/isampler3D/usampler3D|
\pause
\item \verb|sampler1DArray/isampler1DArray/usampler1DArray|
\pause
\item \verb|sampler2DArray/isampler2DArray/usampler2DArray|
\pause
\item \verb|samplerCube/isamplerCube/usamplerCube|
\pause
\item \verb|samplerBuffer/isamplerBuffer/usamplerBuffer|
\end{itemize}
\end{frame}

\begin{frame}[fragile]
\frametitle{Чтение из текстур}
\begin{itemize}
\item Как читать из текстуры в шейдере?
\pause
\item \verb|texture(sampler, coords)| -- возвращает \verb|vec4/ivec4/uvec4| в зависимости от типа текстуры (даже если в текстуре на самом деле меньше компонент)
\pause
\item \verb|coords| -- текстурные координаты, их тип и смысл зависят от типа текстуры
\pause
\begin{itemize}
\item Для 1D/2D/3D текстур \verb|float/vec2/vec3|: нормированные координаты ([0..1] по каждой оси)
\pause
\item Для 1D/2D array текстур \verb|vec2/vec3|: первые одна/две координаты -- нормированные, последняя -- номер "слоя" (не нормированный)
\pause
\item Для cubemap текстур \verb|vec3|: вектор направления из центра куба, возвращается значение из пересечения луча с поверхностью куба
\pause
\item Частое название в случае 2D текстур: UV-координаты
\end{itemize}
\end{itemize}
\end{frame}

\begin{frame}[fragile]
\frametitle{Чтение из текстур}
\begin{itemize}
\item Как читать из текстуры в шейдере? Примеры:
\pause
\begin{itemize}
\item \verb|sampler2D|: \verb|texture(sampler, vec2(0.6, 0.5))| -- пиксель чуть правее центра текстуры
\pause
\item \verb|sampler2DArray|: \verb|texture(sampler, vec3(0.1, 0.9, 12.0))| -- пиксель в левом нижнем углу на 12ом слое
\pause
\item \verb|samplerCube|: \verb|texture(sampler, vec3(0.1, 1.0, -0.1))| -- пиксель на +Y грани куба, около центра грани
\end{itemize}
\end{itemize}
\end{frame}

\begin{frame}[fragile]
\frametitle{Интерполяция атрибутов}
\begin{itemize}
\item Текстурные координаты обычно хранятся как атрибут вершин и интерполируются из вершинного шейдера во фрагментный (хотя могут и вычисляться)
\pause
\item Если интерполировать атрибуты вершин, не учитывая перспективную проекцию, результат будет выглядеть неправильно
\pause
\item На самом деле атрибуты по умолчанию интерполируются чуть сложнее (perspective-correct interpolation)
\pause
\item Три варианта интерполяции атрибутов:
\begin{itemize}
\item \verb|flat| -- значение пикселя совпадает со значением первой вершины
\item \verb|noperspective| -- значение пикселя получается линейной интерполяцией в экранных координатах
\item \verb|smooth| -- значение пикселя получается линейной интерполяцией с учётом барицентрической проекции (по умолчанию)
\end{itemize}
\end{itemize}
\end{frame}

\begin{frame}[fragile]
\frametitle{noperspective/smooth}
\slideimage{perspective.png}
\end{frame}

\begin{frame}[fragile]
\frametitle{Чтение из текстур}
\begin{itemize}
\item Текстура хранит набор дискретных пикселей, но обращение к ней происходит по floating-point координатам
\pause
\item \begin{math}\Rightarrow\end{math} режимы фильтрации текстур
\begin{itemize}
\item \verb|GL_NEAREST| -- использовать ближайший к указанным координатам пиксель
\item \verb|GL_LINEAR| -- использовать линейную/билинейную/трилинейную интерполяцию между соседними 2/4/8 пикселями (для 1D/2D/3D соответственно)
\item Для array-текстур номер слоя всегда ближайший!
\end{itemize}
\pause
\item Настраиваются отдельно для случая, когда
\begin{itemize}
\item Пиксель текстуры больше пикселя на экране (magnification) -- \verb|GL_TEXTURE_MAG_FILTER|
\item Пиксель текстуры меньше пикселя на экране (minification) -- \verb|GL_TEXTURE_MIN_FILTER|
\end{itemize}
\end{itemize}
\pause
\begin{verbatim}
glTexParameteri(GL_TEXTURE_2D, GL_TEXTURE_MAG_FILTER,
    GL_NEAREST)
\end{verbatim}
\end{frame}

\begin{frame}
\frametitle{Nearest vs Linear}
\slideimage{pikachu.png}
\end{frame}

\begin{frame}
\frametitle{Nearest vs Linear}
\slideimage{nearest_linear.png}
\end{frame}

\begin{frame}
\frametitle{Minification vs Magnification}
\slideimage{earth.png}
\end{frame}

\begin{frame}[fragile]
\frametitle{Mipmaps}
\begin{itemize}
\item Когда пиксель текстуры меньше пикселя на экране (minification), текстура выглядит плохо:
\begin{itemize}
\item \verb|GL_NEAREST| -- часть пикселей текстуры не попадают на экран
\item \verb|GL_LINEAR| -- нормально до x2 уменьшения, потом пиксели тоже не попадают на экран
\end{itemize}
\pause
\item Решение: mipmap-уровни
\end{itemize}
\end{frame}

\begin{frame}[fragile]
\frametitle{Mipmaps}
\begin{itemize}
\item Mipmaps -- уменьшенные копии текстуры для использования при минификации
\pause
\item Для текстуры \begin{math}W\times H\end{math} первый уровень имеет размер \begin{math}\lceil\frac{W}{2}\rceil\times \lceil\frac{H}{2}\rceil\end{math}, второй уровень имеет размер \begin{math}\lceil\frac{W}{4}\rceil\times \lceil\frac{H}{4}\rceil\end{math}, и т.д.
\item Последний mipmap-уровень имеет размер \begin{math}1\times 1\end{math}
\item Для array-текстур отдельный набор mipmap'ов для каждого слоя
\pause
\item Их можно
\begin{itemize}
\item Загрузить отдельно, так же, как саму текстуру
\item Попросить OpenGL сгенерировать на основе самой текстуры
\item Не использовать
\end{itemize}
\end{itemize}
\end{frame}

\begin{frame}
\frametitle{Mipmaps}
\slideimage{mipmaps.png}
\end{frame}

\begin{frame}[fragile]
\frametitle{Mipmaps}
\begin{itemize}
\item Опции для \verb|GL_TEXTURE_MIN_FILTER|:
\pause
\begin{itemize}
\item \begin{alltt}GL_\textcolor{magenta}{NEAREST}_MIPMAP_\textcolor{magenta}{NEAREST}\end{alltt} -- выбирается \textcolor{magenta}{ближайший} mipmap уровень, с него выбирается \textcolor{magenta}{ближайший} пиксель
\pause
\item \begin{alltt}GL_\textcolor{green}{LINEAR}_MIPMAP_\textcolor{magenta}{NEAREST}\end{alltt} -- выбирается \textcolor{magenta}{ближайший} mipmap уровень, с него делается \textcolor{green}{интерполяция} ближайших пикселей
\pause
\item \begin{alltt}GL_\textcolor{magenta}{NEAREST}_MIPMAP_\textcolor{green}{LINEAR}\end{alltt} -- выбираются два ближайших mipmap уровня, с них выбираются \textcolor{magenta}{ближайшие} пиксели, между ними происходит линейная \textcolor{green}{интерполяция} (включено по умолчанию)
\pause
\item \begin{alltt}GL_\textcolor{green}{LINEAR}_MIPMAP_\textcolor{green}{LINEAR}\end{alltt} -- выбираются два ближайших mipmap уровня, в них по отдельности делается \textcolor{green}{интерполяция} ближайших пикселей, между результатами снова происходит \textcolor{green}{интерполяция}
\end{itemize}
\end{itemize}
\end{frame}

\begin{frame}[fragile]
\frametitle{Mipmaps}
\begin{itemize}
\item По умолчанию включен \verb|GL_NEAREST_MIPMAP_LINEAR|
\item Если вы забыли сгенерировать/загрузить mipmap-уровни, текстура считается невалидной, и будет рисоваться чёрной
\item Не забывайте выставлять правильные min/mag filter и/или генерировать mipmap'ы!
\end{itemize}
\end{frame}

\begin{frame}[fragile]
\frametitle{Би/трилинейная фильтрация}
\begin{itemize}
\item Размерность текстуры (1D/2D/3D) и тип фильтрации влияют на общее число линейных интерполяций, отсюда названия \textit{билинейная} фильтрация и \textit{трилинейная} фильтрация
\pause
\item Часто под трилинейной фильтрацией имеют в виду конкретно 2D текстуру с \verb|GL_LINEAR_MIPMAP_LINEAR|
\end{itemize}
\end{frame}

\begin{frame}[fragile]
\frametitle{Анизотропная фильтрация}
\begin{itemize}
\item Текстура может быть нормального размера по одной оси, и маленькая по другой
\pause
\item Большой mipmap уровень хорош для одной оси, маленький -- для другой
\pause
\item Решение: анизотропная фильтрация
\pause
\item Есть в OpenGL 4.6, но почти везде доступна через расширение \verb|EXT_texture_filter_anisotropic|
\begin{itemize}
\item (подробнее про расширения OpenGL поговорим позже)
\end{itemize}
\end{itemize}
\end{frame}

\begin{frame}
\frametitle{Анизотропная фильтрация}
\slideimage{anisotropy.png}
\end{frame}

\begin{frame}[fragile]
\frametitle{Wrapping mode}
\begin{itemize}
\item При чтении из текстуры координаты могут выходить за пределы диапазона
\pause
\item При \verb|GL_LINEAR| для точек у границы текстуры может не найтись достаточное количество соседних пикселей
\pause
\item \begin{math}\Rightarrow\end{math} Wrapping mode:
\begin{itemize}
\item \verb|GL_CLAMP_TO_EDGE| -- брать пиксель с границы
\item \verb|GL_CLAMP_TO_BORDER| -- брать значение фиксированного цвета
\item \verb|GL_REPEAT| -- повторять текстуру (включен по умолчанию)
\item \verb|GL_MIRRORED_REPEAT| -- повторять текстуру, отражая её на каждый повтор
\end{itemize}
\pause
\item Настраивается отдельно для каждой текстурной координаты: S, T, R
\end{itemize}
\begin{verbatim}
glTexParameteri(GL_TEXTURE_2D, GL_TEXTURE_WRAP_S,
    GL_CLAMP_TO_EDGE)
\end{verbatim}
\end{frame}

\begin{frame}
\frametitle{Wrapping}
\slideimage{wrapping.png}
\end{frame}

\begin{frame}[fragile]
\frametitle{Выбор mipmap уровня}
\begin{itemize}
\item Как GPU понимает, какой выбрать mipmap-уровень, и происходит ли minification/magnification? \pause По значениям в соседних пикселях
\pause
\item Фрагментный шейдер всегда запускается на группах 2x2 пикселя (даже если часть из них не были растеризованы)
\pause
\item В GLSL есть функции \verb|dFdx/dFdy| -- вычисляют разницу некой величины для пары соседних пикселей в группе 2x2
\pause
\begin{itemize}
\item Например, \verb|dFdx(value)| -- разница между значением \verb|value| в пикселе справа и значением в пикселе слева
\end{itemize}
\pause
\item По \verb|dFdx(coords)| и \verb|dFdy(coords)| вычисляется mipmap-уровень (где \verb|coords| -- текстурные координаты)
\end{itemize}
\end{frame}

\begin{frame}[fragile]
\frametitle{Чтение из текстур}
\begin{itemize}
\item \verb|texture| -- учитывает mipmaps и настройки фильтрации
\pause
\item \verb|textureLod| -- обратиться напрямую к указанному mipmap-уровню (LOD = level of detail)
\pause
\item \verb|texelFetch| -- обратиться напрямую к указанному пикселю, минуя всю фильтрацию
\end{itemize}
\end{frame}

\begin{frame}[fragile]
\frametitle{Формат текстуры}
\begin{itemize}
\item Текстура -- набор (1D/2D/3D массив или 6 2D массивов) пикселей
\pause
\item Пиксель -- 1, 2, 3 или 4 компоненты в определённом формате
\pause
\item Форматов очень много, вот часто встречающиеся:
\begin{itemize}
\item \verb|GL_RGB8| -- 3 нормированных 8-битных канала
\item \verb|GL_RGBA8| -- 4 нормированных 8-битных канала
\item \verb|GL_RGBA32F| -- 4 32-битных floating-point канала
\item \verb|GL_DEPTH_COMPONENT24| -- специальный формат для z-буфера
\end{itemize}
\end{itemize}
\end{frame}

\begin{frame}[fragile]
\frametitle{Загрузить данные в текстуру}
\begin{itemize}
\item 1D: \verb|glTexImage1D|
\pause
\item 2D, 1D array или грань cubemap: \verb|glTexImage2D|
\pause
\item 3D или 2D array: \verb|glTexImage3D|
\end{itemize}
\end{frame}

\begin{frame}[fragile]
\frametitle{Загрузить данные в текстуру}
\begin{verbatim}
glTexImage2D(GLenum target, GLint level,
  GLint internalFormat,
  GLsizei width, GLsizei height, GLint border,
  GLenum format, GLenum type, const GLvoid * data);
\end{verbatim}
\begin{itemize}
\item \verb|target| -- тип текстуры (например, \verb|GL_TEXTURE_2D|)
\item \verb|level| -- mipmap level, который мы хотим загрузить (0 для основной текстуры)
\item \verb|internalFormat| -- формат хранения пикселей (например, \verb|GL_RGBA8|)
\item \verb|width|, \verb|height| -- размер в пикселях
\item \verb|border| -- легаси, должно быть 0
\item \verb|format| -- какие компоненты есть в массиве \verb|data| (например, \verb|GL_RGB|)
\item \verb|type| -- какого типа компоненты в массиве \verb|data| (например, \verb|GL_UNSIGNED_BYTE|)
\item \verb|data| -- указатель на массив пикселей (сначала первая строка пикселей, потом вторая, и т.д.)
\end{itemize}
\end{frame}

\begin{frame}[fragile]
\frametitle{Загрузить данные в текстуру}
\begin{itemize}
\item Если \verb|data = nullptr|, под текстуру выделится память, но данные не будут записаны
\pause
\item Есть ограничения на возможные пары \verb|format| и \verb|type| (например, \verb|type = GL_UNSIGNED_SHORT_5_6_5| требует \verb|format = GL_RGB|)
\pause
\item \verb|format + type| может не совпадать с \verb|internalFormat|, тогда происходит преобразование в \verb|internalFormat|
\pause
\item По умолчанию ожидается, что начало каждой строки пикселей в массиве \verb|data| выровнено на 4 байта
\item Это может нарушиться, например, с \verb|format = GL_RGB| и \verb|type = GL_UNSIGNED_BYTE| с нечётной шириной текстуры
\item Настраивается с помощью \verb|glPixelStorei(GL_UNPACK_ALIGNMENT, 1)|
\end{itemize}
\end{frame}

\begin{frame}[fragile]
\frametitle{Генерация mipmap}
\begin{itemize}
\item После загрузки нулевого mipmap-уровня, можно сгенерировать все остальные с помощью \verb|glGenerateMipmap(target)|
\pause
\item Делает линейное усреднение пикселей \begin{math}\Rightarrow\end{math} не подходит, если в текстуре хранятся нелинейные данные (цвет в формате sRGB, нормали, параметры материала)
\end{itemize}
\end{frame}

\begin{frame}[fragile]
\frametitle{Как связать текстуру и sampler в шейдере?}
\begin{itemize}
\item ID текущей текстуры для каждого target хранится в т.н. texture unit'е
\item Texture unit'ы -- \textbf{не объекты OpenGL}, их нельзя создавать/удалять
\item Их фиксированное число -- \verb|glGet(GL_MAX_COMBINED_TEXTURE_IMAGE_UNITS)|, как минимум 48
\item Есть текущий texture unit (по умолчанию -- 0)
\item \verb|glActiveTexture(GL_TEXTURE0 + i)| -- сделать текущим texture unit с номером \verb|i|
\item \verb|glBindTexture(...)| делает текстуру текущей для данного \verb|target| в текущем \verb|texture unit|
\end{itemize}
\end{frame}

\begin{frame}[fragile]
\frametitle{Как связать текстуру и sampler в шейдере?}
\begin{itemize}
\item В качестве значения uniform-переменной типа \verb|sampler2D| (и т.п.) указывается \textit{номер texture unit'а}
\pause
\item Если для \verb|sampler2D| установлено значение \verb|i|, то в качестве текстуры будет взята \verb|GL_TEXTURE_2D| из texture unit'а номер \verb|i|
\pause
\item \verb|glUniform1i(color_texture_location, i)|
\pause
\item Если нужны несколько текстур одновременно в одном шейдере, нужно их сделать текущими для разных texture unit'ов, и передать номера этих texture unit'ов в uniform-переменные шейдера
\end{itemize}
\end{frame}

\begin{frame}[fragile]
\frametitle{Texture units}
\slideimage{texture_units.png}
\end{frame}

\begin{frame}[fragile]
\frametitle{Как использовать текстуру для 3D-модели?}
\begin{itemize}
\item Обычно каждая вершина хранит пару (нормированных) текстурных координат (UV-развёртка)
\pause
\item Они передаются из вершинного шейдера во фрагментный и интерполируются
\pause
\item Интерполированные значения текстурных координат используются во фрагментном шейдере для чтения из текстуры
\end{itemize}
\end{frame}

\begin{frame}[fragile]
\frametitle{Как использовать текстуру для 3D-модели?}
\slideimage{uvs.png}
\end{frame}

\begin{frame}[fragile]
\frametitle{Форматы изображений}
\begin{itemize}
\item \verb|glTexImage2D| ожидает на вход массив готовых пикселей
\pause
\item Обычно изображения хранятся в более сложных форматах: PNG, JPG, и т.д.
\pause
\item Есть формат Netpbm, хранящий сырые пиксели и простой для чтения/записи
\pause
\item Есть библиотеки для чтения разных форматов - libpng, libjpeg
\pause
\item \verb|stb_image.h| - single-header библиотека для чтения разных форматов
\pause
\item Boost GIL (Generic Image Library)
\end{itemize}
\end{frame}

\begin{frame}[fragile]
\frametitle{Псевдокод типичного использования текстуры}
\begin{verbatim}
// инициализация
image = loadImage('image.png')
texture = createTexture()
texture.setMinFilter(GL_LINEAR_MIPMAP_NEAREST)
texture.setMagFilter(GL_NEAREST)
texture.load(image)
texture.generateMipmap()

// рендеринг
program.use()
vao.bind()
program.uniform("color_texture") = 0;
glActiveTexture(GL_TEXTURE0);
texture.bind()
glDrawArrays(...)
\end{verbatim}
\end{frame}

\begin{frame}[fragile]
\frametitle{Ссылки}
\begin{itemize}
\item \href{https://www.khronos.org/opengl/wiki/Texture}{\nolinkurl{khronos.org/opengl/wiki/Texture}}
\item \href{https://www.khronos.org/opengl/wiki/Sampler_Object}{\nolinkurl{khronos.org/opengl/wiki/Sampler\_Object}}
\item \href{https://www.khronos.org/opengl/wiki/Image_Format}{\nolinkurl{khronos.org/opengl/wiki/Image\_Format}}
\item \href{https://learnopengl.com/Getting-started/Textures}{\nolinkurl{learnopengl.com/Getting-started/Textures}}
\item \href{http://www.opengl-tutorial.org/beginners-tutorials/tutorial-5-a-textured-cube}{\nolinkurl{opengl-tutorial.org/beginners-tutorials/tutorial-5-a-textured-cube}}
\item \href{https://open.gl/textures}{\nolinkurl{open.gl/textures}}
\end{itemize}
\end{frame}

\end{document}