% (c) Nikita Lisitsa, lisyarus@gmail.com, 2025

\documentclass[10pt]{beamer}

\usepackage[T2A]{fontenc}
\usepackage[russian]{babel}
\usepackage{minted}

\usepackage{graphicx}
\graphicspath{ {./images/} }

\usetheme{metropolis}

\usepackage{adjustbox}

\usepackage{color}
\usepackage{soul}

\usepackage{hyperref}

\definecolor{blue}{rgb}{0,0,1}
\definecolor{red}{rgb}{1,0,0}
\definecolor{codebg}{RGB}{29,35,49}

\makeatletter
\newcommand{\slideimage}[1]{
  \begin{figure}
    \begin{adjustbox}{width=\textwidth, totalheight=\textheight-2\baselineskip-2\baselineskip,keepaspectratio}
      \includegraphics{#1}
    \end{adjustbox}
  \end{figure}
}
\makeatother

\title{Компьютерная графика}
\subtitle{Практика 7: Рендеринг в текстуру, пост-обработка}
\date{2025}

\setbeamertemplate{footline}[frame number]

\begin{document}

\frame{\titlepage}

\begin{frame}[fragile]
\slideimage{0.png}
\end{frame}

\usemintedstyle{solarized-light}

\begin{frame}[fragile]
\frametitle{Задание 1}
Создаём и настраиваем framebuffer
\usemintedstyle{solarized-light}
\begin{itemize}
\item Создаём текстуру для цветового буфера: min/mag фильтры -- \mintinline{cpp}|GL_NEAREST|, размеры -- \mintinline{cpp}|width/2| на \mintinline{cpp}|height/2|, формат -- \mintinline{cpp}|GL_RGBA8|, в качестве данных можно передать \mintinline{cpp}|nullptr|
\item Создаём renderbuffer для буфера глубины: \mintinline{cpp}|glGenRenderbuffers|, \mintinline{cpp}|glBindRenderbuffer|, \mintinline{cpp}|glRenderbufferStorage|, формат -- \mintinline{cpp}|GL_DEPTH_COMPONENT24|, размеры -- как у текстуры
\end{itemize}
\end{frame}

\begin{frame}[fragile]
\frametitle{Задание 1}
Создаём и настраиваем framebuffer
\begin{itemize}
\item Создаём framebuffer: \mintinline{cpp}|glGenFramebuffers|
\item Присоединяем к нему цвет и глубину: \mintinline{cpp}|glBindFramebuffer| (target = \mintinline{cpp}|GL_DRAW_FRAMEBUFFER|), \mintinline{cpp}|glFramebufferTexture| с параметром attachment = \mintinline{cpp}|GL_COLOR_ATTACHMENT0|, \mintinline{cpp}|glFramebufferRenderbuffer| с параметром attachment = \mintinline{cpp}|GL_DEPTH_ATTACHMENT|
\item Проверяем, что фреймбуффер настроен правильно: \mintinline{cpp}|glCheckFramebufferStatus| должна вернуть \mintinline{cpp}|GL_FRAMEBUFFER_COMPLETE|
\end{itemize}
\end{frame}

\begin{frame}[fragile]
\frametitle{Задание 1}
Создаём и настраиваем framebuffer
\usemintedstyle{solarized-light}
\begin{itemize}
\item Текстуре и renderbuffer'у нужно обновлять размер при изменении размера экрана: там, где обрабатывается \mintinline{cpp}|SDL_WINDOWEVENT_RESIZED|, нужно добавить \mintinline{cpp}|glBindTexture|, \mintinline{cpp}|glTexImage2D|, \mintinline{cpp}|glBindRenderbuffer|, \mintinline{cpp}|glRenderbufferStorage|
\item Картинка будет чёрной, так как теперь мы ничего не рисуем в default framebuffer
\end{itemize}
\end{frame}

\begin{frame}[fragile]
\slideimage{1.png}
\end{frame}

\begin{frame}[fragile]
\frametitle{Задание 2}
Рисуем в текстуру
\begin{itemize}
\item Перед \mintinline{cpp}|glClear| устанавливаем наш framebuffer текущим для рисования (\mintinline{cpp}|glBindFramebuffer|) и настраиваем \mintinline{cpp}|glViewport|
\item После рисования основной модели, перед рисованием прямоугольника делаем текущим для рисования основной framebuffer (id = 0), настраиваем \mintinline{cpp}|glViewport| и очищаем цветовой буфер и буфер глубины (\mintinline{cpp}|glClear|)
\item Раскомментируем рисование прямоугольника: пока он всё ещё выведется просто с цветовым градиентом, так как шейдер мы не меняли
\item Дракон виден не будет, так как он нарисовался в другой фреймбуфер
\end{itemize}
\end{frame}

\begin{frame}[fragile]
\slideimage{2.png}
\end{frame}

\begin{frame}[fragile]
\frametitle{Задание 3}
Выводим нарисованную текстуру
\begin{itemize}
\item Добавляем в шейдер для прямоугольника текстуру, которую мы нарисовали с помощью framebuffer'а: \mintinline{glsl}|uniform sampler2D render_result;|
\item В \mintinline{glsl}|out_color| пишем значение пикселя из текстуры
\item Устанавливаем значение соответствующей uniform-переменной в 0 (\mintinline{glsl}|glGetUniformLocation|, \mintinline{glsl}|glUniform1i|)
\item Перед рисованием прямоугольника устанавливаем текстуру текущей для texture unit 0 (\mintinline{glsl}|glActiveTexture|, \mintinline{glsl}|glBindTexture|)
\end{itemize}
\end{frame}

\begin{frame}[fragile]
\slideimage{3.png}
\end{frame}

\begin{frame}[fragile]
\frametitle{Задание 4}
Выводим текстуру несколько раз
\begin{itemize}
\item Заносим весь рендеринг (\textbf{не} обработку событий SDL и \textbf{не} функцию \mintinline{cpp}|SDL_GL_SwapWindow|) в цикл, повторяющий рендеринг 4 раза
\item В зависимости от номера повторения меняем uniform-переменные \mintinline{cpp}|center| и \mintinline{cpp}|size|, чтобы картинка нарисовалась во всех четырёх углах экрана
\item В зависимости от номера повторения меняем \mintinline{cpp}|glClearColor| (цвета выберите сами)
\end{itemize}
\end{frame}

\begin{frame}[fragile]
\frametitle{Задание 4}
Выводим текстуру несколько раз
\begin{itemize}
\item В зависимости от номера повторения меняем проекцию:
\begin{itemize}
\item 0 -- перспективная проекция, ничего не меняем
\item 1 -- ортографическая проекция на плоскость XY
\item 2 -- ортографическая проекция на плоскость XZ
\item 3 -- ортографическая проекция на плоскость YZ
\end{itemize}
\item Конкретное соответствие номеров и проекций не так важно, главное -- увидеть три ортографических проекции на разные плоскости и одну перспективную
\item Ортографическую проекцию можно сделать с помощью \mintinline{cpp}|projection = glm::ortho(...)| (\textbf{учитывая aspect ratio экрана!}) и настроив \mintinline{cpp}|view| какими-нибудь вращениями
\end{itemize}
\end{frame}

\begin{frame}[fragile]
\slideimage{4.png}
\end{frame}

\begin{frame}[fragile]
\frametitle{Задание 5}
Делаем color grading
\begin{itemize}
\item Добавляем во фрагментный шейдер пост-обработки (с которым рисуется прямоугольник) uniform-переменную \mintinline{glsl}|uniform int mode;| и устанавливаем её в цикле рендеринга в номер картинки (0..3)
\item При \mintinline{glsl}|mode == 1| преобразуем цвет как \mintinline{glsl}|color = floor(color * 4.0) / 3.0|
\item При остальных значениях \mintinline{glsl}|mode| пока выводим просто текстуру без искажений
\end{itemize}
\end{frame}

\begin{frame}[fragile]
\slideimage{5.png}
\end{frame}

\begin{frame}[fragile]
\frametitle{Задание 6}
Искажаем изображение
\begin{itemize}
\item При \mintinline{glsl}|mode == 2| искажаем изображение: прибавляем к текстурной координате что-нибудь зависящее от времени и координаты, например 
\usemintedstyle{lightbulb}
\begin{minted}[bgcolor=codebg]{glsl}
texcoord + vec2(sin(texcoord.y * 50.0
    + time) * 0.01, 0.0)
\end{minted}
\end{itemize}
\end{frame}

\begin{frame}[fragile]
\slideimage{6.png}
\end{frame}

\begin{frame}[fragile]
\frametitle{Задание 7*}
Делаем гауссово размытие
\begin{itemize}
\item При \mintinline{glsl}|mode == 3| делаем гауссово размытие, например, с \mintinline{glsl}|N = 7| и \mintinline{glsl}|radius = 5.0| (код и описание есть в слайдах лекции)
\item N.B.: у вас получится \begin{math}(2N+1)^2\end{math} обращений к текстуре
\end{itemize}
\end{frame}

\begin{frame}[fragile]
\slideimage{7.png}
\end{frame}

\end{document}