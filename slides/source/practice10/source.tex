% (c) Nikita Lisitsa, lisyarus@gmail.com, 2021

\documentclass{beamer}

\usepackage[T2A]{fontenc}
\usepackage[utf8]{inputenc}
\usepackage[russian]{babel}

\usepackage{graphicx}
\graphicspath{ {./images/} }

\usepackage{adjustbox}

\usepackage{color}
\usepackage{soul}

\usepackage{hyperref}

\definecolor{blue}{rgb}{0,0,1}
\definecolor{red}{rgb}{1,0,0}

\makeatletter
\newcommand{\slideimage}[1]{
  \begin{figure}
    \begin{adjustbox}{width=\textwidth, totalheight=\textheight-2\baselineskip-2\baselineskip,keepaspectratio}
      \includegraphics{#1}
    \end{adjustbox}
  \end{figure}
}
\makeatother

\title{Компьютерная графика}
\subtitle{Практика 10: Скелетная анимация}
\date{2021}

\setbeamertemplate{footline}[frame number]

\begin{document}

\frame{\titlepage}

\begin{frame}[fragile]
\frametitle{Задание 1}
Вычисляем позу для модели и применяем её
\begin{itemize}
\item В массиве \verb|poses| 6 фиксированных поз, преобразования костей в них заданы относительно родительских костей!
\pause
\item Заводим массив объектов \verb|bone_pose| для хранения посчитанных преобразований скелета (по одному преобразованию на кость)
\pause
\item Вычисляем преобразования:
\begin{itemize}
\item Для корневой кости (\verb|parent == -1|) копируем преобразование из выбранной позы (например, \verb|poses[0]|)
\item Для некорневой кости берём преобразование из выбранной позы и умножаем \textit{слева} на уже посчитанное преобразование родительской кости
\item N.B.: в данных кости идут в порядке топологической сортировки, т.е. родительская кость всегда идёт раньше дочерней
\end{itemize}
\end{itemize}
\end{frame}

\begin{frame}[fragile]
\frametitle{Задание 1 (продолжение)}
Вычисляем позу для модели и применяем её
\begin{itemize}
\item Заводим по массиву uniform-переменных на повороты, сдвиги и масштабирования костей (всего 3 массива uniform'ов, размером по 64 элемента)
\pause
\item Заводим три массива под uniform location и заполняем их (\verb|glGetUniformLocation|, для генерации имён можно воспользоваться конструкцией \verb|"bone_rotation[" + std::to_string(i) + "]"|)
\pause
\item При рендеринге выставляем зачения всех uniform-переменных для костей из посчитанного (на предыдущем слайде) массива преобразований
\pause
\item N.B.: координаты кватернионов идут в порядке (w, x, y, z)
\end{itemize}
\end{frame}

\begin{frame}[fragile]
\frametitle{Задание 1 (продолжение)}
Вычисляем позу для модели и применяем её
\begin{itemize}
\item В вершинном шейдере применяем к позиции входной вершины преобразования двух костей (их индексы лежат в \verb|in_bone_id|) и суммируем с весами из \verb|in_bone_weight|
\pause
\item Делаем то же самое с нормалями (к ним не нужно применять translation!)
\pause
\item N.B.: код для вращения кватернионами в шейдере уже есть
\pause
\item N.B.: не забываем про матрицу model: её всё ещё нужно применить после скелетных преобразований
\end{itemize}
\end{frame}

\begin{frame}[fragile]
\frametitle{Задание 2}
Интерполируем между соседними позами
\begin{itemize}
\item Как-нибудь вычисляем номер текущей позы в зависимости от времени (например, \verb|floor(time)|)
\item Номер следующей позы = (номер текущей позы + 1) % 6
\pause
\item Вычисляем параметр интерполяции между текущей позой и следующей (например, \verb|t = time - floor(time)|)
\pause
\item При вычислении массива преобразований из задания 1 не просто берём преобразование из позы, а интерполируем две соседние позы
\pause
\item N.B.: для интерполяции кватернионов можно воспользоваться функцией \verb|glm::slerp|
\item N.B.: для интерполяции масштабов и сдвигов можно воспользоваться функцией \verb|glm::mix|
\end{itemize}
\end{frame}

\begin{frame}[fragile]
\frametitle{Задание 3}
Добавляем easing
\begin{itemize}
\item Применяем к параметру интерполяции какую-нибудь easing-функцию, например \begin{math}3t^2-2t^3\end{math}
\end{itemize}
\end{frame}

\end{document}