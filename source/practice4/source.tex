% (c) Nikita Lisitsa, lisyarus@gmail.com, 2023

\documentclass[10pt]{beamer}

\usepackage[T2A]{fontenc}
\usepackage[russian]{babel}
\usepackage{minted}

\usepackage{graphicx}
\graphicspath{ {./images/} }

\usetheme{metropolis}

\usepackage{adjustbox}

\usepackage{color}
\usepackage{soul}

\usepackage{hyperref}

\definecolor{codebg}{RGB}{29,35,49}

\makeatletter
\newcommand{\slideimage}[1]{
  \begin{figure}
    \begin{adjustbox}{width=\textwidth, totalheight=\textheight-2\baselineskip-2\baselineskip,keepaspectratio}
      \includegraphics{#1}
    \end{adjustbox}
  \end{figure}
}
\makeatother

\title{Компьютерная графика}
\subtitle{Практика 4: Индексы, перспективная проекция, буфер глубины}
\date{2023}

\setbeamertemplate{footline}[frame number]

\begin{document}

\frame{\titlepage}

\begin{frame}[fragile]
\frametitle{Напоминание про VBO и VAO}
\usemintedstyle{solarized-light}
\begin{itemize}
\item \textbf{VBO} -- хранит данные, ничего не знает о формате
\item \textbf{VAO} -- описывает \textit{формат} и \textit{расположение} атрибутов вершин
\item Концептуально, расположение = id буфера + сдвиг + \textit{stride}
\item \textbf{VAO} также хранит id текущего \textbf{EBO} (\mintinline{cpp}|GL_ELEMENT_ARRAY_BUFFER|) -- оттуда берутся индексы
\item Для рендеринга нужен \textit{только} \textbf{VAO}
\item Чтобы обновить данные, нужен \textit{только} \textbf{VBO}
\item Чтобы обновить индексы, нужен \textit{только} \textbf{EBO} (можно использовать \mintinline{cpp}|target = GL_ARRAY_BUFFER|)
\end{itemize}
\end{frame}

\begin{frame}[fragile]
\frametitle{Практика 4}
\usemintedstyle{solarized-light}
\begin{itemize}
\item В этой практике нельзя менять код шейдеров!
\item Все преобразования модели (сдвиги, повороты, и т.п.) и настройки камеры нужно выполнять через матрицы \mintinline{cpp}|model|, \mintinline{cpp}|view| и \mintinline{cpp}|projection|
\end{itemize}
\end{frame}

\begin{frame}[fragile]
\frametitle{Практика 4}
\slideimage{0.png}
\end{frame}

\begin{frame}[fragile]
\frametitle{Задание 1}
\usemintedstyle{solarized-light}
Рисуем зайца
\begin{itemize}
\item Создаём VAO, VBO, EBO
\item Загружаем данные (\mintinline{cpp}|bunny.vertices| и \mintinline{cpp}|bunny.indices|) в VBO и EBO
\item Настраиваем атрибуты для VAO (нужно понять правильные настройки по вершинному шейдеру и по описанию структуры \mintinline{cpp}|obj_data::vertex| в \mintinline{cpp}|obj_parser.hpp|)
\item Рисуем с помощью \mintinline{cpp}|glDrawElements|, \mintinline{cpp}|mode = GL_TRIANGLES|, обращаем внимание на тип индексов (\mintinline{cpp}|GL_UNSIGNED_INT|)
\item N.B. заяц будет рисоваться странно из-за отключенного теста глубины, так и задумано
\end{itemize}
\end{frame}

\begin{frame}[fragile]
\frametitle{Задание 1}
\slideimage{1.png}
\end{frame}

\begin{frame}[fragile]
\frametitle{Задание 2}
\usemintedstyle{solarized-light}
Вращаем зайца
\begin{itemize}
\item Меняем матрицу \mintinline{cpp}|model| чтобы заяц крутился в плоскости XZ
\item В качестве угла нужно взять что-нибудь зависящее от времени, например \mintinline{cpp}|float angle = time;|
\item Заяц будет обрезаться по \begin{math}z \in [-1, 1]\end{math} (особенно хорошо видно, что иногда обрезается хвост)
\item Отмасштабируем его \textit{по всем осям} \mintinline{cpp}|float scale = 0.5f|, тоже с помощью матрицы \mintinline{cpp}|model|
\end{itemize}
\end{frame}

\begin{frame}[fragile]
\frametitle{Задание 2}
\slideimage{2.png}
\end{frame}

\begin{frame}[fragile]
\frametitle{Задание 3}
\usemintedstyle{solarized-light}
Добавляем перспективу
\begin{itemize}
\item Выбираем значения \mintinline{cpp}|near|, \mintinline{cpp}|far|, \mintinline{cpp}|top|, \mintinline{cpp}|right|
\begin{itemize}
\item \mintinline{cpp}|near| -- маленький, но не слишком, в духе \begin{math}0.001 \dots 0.1\end{math}
\item \mintinline{cpp}|far| -- большой, но не слишком, в духе \begin{math}10.0 \dots 1000.0\end{math}
\item Отношение \mintinline{cpp}|right/near| -- тангенс половины угла обзора, например \mintinline{cpp}|right = near| соответствует углу обзора в \begin{math}90^\circ\end{math}
\item Отношение \mintinline{cpp}|right/top| -- aspect ratio экрана (\mintinline{cpp}|width/height|)
\end{itemize}
\item В матрицу \mintinline{cpp}|projection| записываем матрицу проекции с использованием выбранных значений (см. слайд с лекции)
\item Заяц будет виден изнутри
\end{itemize}
\end{frame}

\begin{frame}[fragile]
\frametitle{Задание 3}
\slideimage{3.png}
\end{frame}

\begin{frame}[fragile]
\frametitle{Задание 4}
\usemintedstyle{solarized-light}
Включаем тест глубины
\begin{itemize}
\item Сдвигаем камеру по оси Z на какое-то расстояние (например, на 2-3 единицы), меняя матрицу \mintinline{cpp}|view|
\begin{itemize}
\item N.B.: Камера смотрит в сторону -Z, поэтому \textit{сдвиг камеры} должен быть в сторону \textit{положительной} оси Z
\item N.B.: При этом матрица \mintinline{cpp}|view| содержит \textit{обратное преобразование!}
\end{itemize}
\item Заяц будет рисоваться неправильно: задние грани перекрывают передние
\item Включим тест глубины: \mintinline{cpp}|glEnable(GL_DEPTH_TEST)|
\item Не забываем очищать буфер глубины в начале каждого кадра: \mintinline{cpp}|glClear(GL_DEPTH_BUFFER_BIT)|
\begin{itemize}
\item Можно это делать одновременно с очисткой цветового буфера:  \mintinline{cpp}=glClear(GL_DEPTH_BUFFER_BIT | GL_COLOR_BUFFER_BIT)=
\end{itemize}
\end{itemize}
\end{frame}

\begin{frame}[fragile]
\frametitle{Задание 4}
\slideimage{4.png}
\end{frame}

\begin{frame}[fragile]
\frametitle{Задание 5}
\usemintedstyle{solarized-light}
Двигаем зайца
\begin{itemize}
\item Заведём переменные \mintinline{cpp}|bunny_x| и \mintinline{cpp}|bunny_y| с координатами центра зайца по X и Y
\item Изменим матрицу \mintinline{cpp}|model|, добавив соответствующие сдвиги по X и Y
\item Перед рисованием каждого кадра обновим положение зайца:
\begin{itemize}
\item Если нажата клавиша влево \mintinline{cpp}|SDLK_LEFT|, сдвинем зайца влево: \mintinline{cpp}|bunny_x -= speed * dt|
\item Аналогично \mintinline{cpp}|SDLK_RIGHT|, \mintinline{cpp}|SDLK_DOWN|, \mintinline{cpp}|SDLK_UP|
\item Состояние нажатости клавиш доступно в словаре \mintinline{cpp}|button_down|
\end{itemize}
\end{itemize}
\end{frame}

\begin{frame}[fragile]
\frametitle{Задание 5}
\slideimage{5.png}
\end{frame}

\begin{frame}[fragile]
\frametitle{Задание 6}
\usemintedstyle{solarized-light}
Играем с face culling
\begin{itemize}
\item Включим back-face culling: \mintinline{cpp}|glEnable(GL_CULL_FACE)|
\item Ничего не изменится: заяц сделан так, чтобы все треугольники были CCW
\item Изменим режим: \mintinline{cpp}|glCullFace(GL_FRONT)|
\item Должны быть видны задние грани куба и не видны передние
\item Выглядеть будет странно -- не пугайтесь, попробуйте увидеть и понять, что происходит :)
\end{itemize}
\end{frame}

\begin{frame}[fragile]
\frametitle{Задание 6}
\slideimage{6.png}
\end{frame}

\begin{frame}[fragile]
\frametitle{Задание 7*}
\usemintedstyle{solarized-light}
Три вращающихся зайца
\begin{itemize}
\item Рисуем три зайца в разных местах, вращающихся в плоскостях XY, XZ, YZ соответственно
\item Три раза настраиваем матрицу \mintinline{cpp}|model| + делаем \mintinline{cpp}|glUniformMatrix| + \mintinline{cpp}|glDrawElements|
\item Всё ещё можно двигать стрелочками, т.е. должны учитываться \verb|bunny_x| и \verb|bunny_y| (для одного зайца, или для всех -- как хотите)
\end{itemize}
\end{frame}

\begin{frame}[fragile]
\frametitle{Задание 7*}
\slideimage{7.png}
\end{frame}

\end{document}