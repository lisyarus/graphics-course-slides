% (c) Nikita Lisitsa, lisyarus@gmail.com, 2021

\documentclass{beamer}

\usepackage[T2A]{fontenc}
\usepackage[utf8]{inputenc}
\usepackage[russian]{babel}

\usepackage{graphicx}
\graphicspath{ {./images/} }

\usepackage{adjustbox}

\usepackage{color}
\usepackage{soul}

\usepackage{hyperref}

\usepackage{amsmath}

\usepackage{tikz}
\usetikzlibrary{decorations}
\usetikzlibrary{decorations.pathreplacing}
\usepackage{xifthen}

\definecolor{red}{rgb}{1,0,0}
\definecolor{green}{rgb}{0,0.5,0}
\definecolor{blue}{rgb}{0,0,1}
\definecolor{magenta}{rgb}{0.75,0,0.75}

\makeatletter
\newcommand{\slideimage}[1]{
  \begin{figure}
    \begin{adjustbox}{width=\textwidth, totalheight=\textheight-2\baselineskip-2\baselineskip,keepaspectratio}
      \includegraphics{#1}
    \end{adjustbox}
  \end{figure}
}
\makeatother

\title{Компьютерная графика}
\subtitle{Лекция 15: оптимизация рендеринга, timer queries, frustum culling, occlusion culling, instancing}
\date{2021}

\setbeamertemplate{footline}[frame number]

\begin{document}

\frame{\titlepage}

\begin{frame}[fragile]
\frametitle{Оптимизация \textendash{} это сложно}
На производительность (CPU) влияют:
\pause
\begin{itemize}
\item Общая загруженность системы
\pause
\item Количество и паттерн доступов к памяти (cache-friendliness)
\pause
\item Помещаются ли данные в кеш
\pause
\item Branch prediction
\pause
\item Как функции программы лежат в памяти (опять кеш)
\pause
\item Многое другое
\end{itemize}
\end{frame}

\begin{frame}[fragile]
\frametitle{Оптимизация на GPU \textendash{} это очень сложно}
\begin{itemize}
\item Асинхронность
\pause
\item Параллельность
\pause
\item Много встроенных операций (fixed-function pipeline)
\pause
\item Сложные операции с памятью (доступ к текстуре: mipmaps + фильтрация)
\pause
\item Многое другое
\end{itemize}
\end{frame}

\begin{frame}[fragile]
\frametitle{Измерение времени работы \textendash{} неправильный способ}
\begin{verbatim}
while (true) {
  auto frame_start = clock::now();

  // нарисовали сцену
  ...

  auto frame_end = clock::now();

  SwapBuffers();
}
\end{verbatim}
\pause
\begin{itemize}
\item \verb|frame_end - frame_start| \textendash{} сколько времени ушло на то, чтобы \textbf{вызвать OpenGL-команды}
\pause
\item В реальности драйвер поставил их в очередь, и скорее всего GPU ещё не начала их выполнять
\end{itemize}
\end{frame}

\begin{frame}[fragile]
\frametitle{Измерение времени работы \textendash{} простой способ}
\begin{verbatim}
disableVsync();
auto last_frame_start = clock::now();
while (true) {
  auto frame_start = clock::now();
  auto frame_time = frame_start - last_frame_start;
  last_frame_start = frame_start;

  // нарисовали сцену
  ...

  SwapBuffers();
}
\end{verbatim}
\pause
\begin{itemize}
\item Из-за выключенного vsync видеокарта будет работать \begin{math}\pm\end{math} постоянно
\pause
\item В итоге мы получим примерное время, тратящееся на рисование одного кадра
\end{itemize}
\end{frame}

\begin{frame}[fragile]
\frametitle{Измерение времени работы: glFlush и glFinish}
\begin{itemize}
\item Многие (старые) туториалы по измерению времени кадра советуют вызывать \verb|glFlush| или \verb|glFinish| в конце кадра
\pause
\item \verb|glFlush| сбрасывает буфер команд (хранящийся внутри драйвера) с CPU на GPU
\pause
\item \verb|glFinish| ждёт, пока GPU не завершит обрабатывать все посланные команды
\pause
\item \verb|SwapBuffers| сама вызывает \verb|glFlush|
\pause
\item \verb|glFinish| ухудшает производительность: половину времени вы отправляете команды на GPU, а GPU (скорее всего) ничего не делает; половину времени вы ждёте, пока GPU закончит выполнять команды
\end{itemize}
\end{frame}

\begin{frame}[fragile]
\frametitle{Измерение времени работы: FPS vs frame duration}
\begin{itemize}
\item FPS (frames per second, количество кадров в секунду) \textendash{} очень неудобная метрика:
\pause
\begin{itemize}
\item Нелинейна: если кадр рисовался 10 мс, и мы добавили что-то рисующееся 1 мс, и ещё что-то рисующееся 1 мс, то FPS изменялся от 100 до 90.9 до 83.3
\end{itemize}
\pause
\item Обычно используют время, тратящееся на рисование кадра или конкретного объекта/эффекта (миллисекунды/микросекунды)
\end{itemize}
\end{frame}

\begin{frame}[fragile]
\frametitle{Измерение времени работы \textendash{} правильный способ: timer queries}
\begin{itemize}
\item Query objects \textendash{} объекты OpenGL, позволяющие узнать некоторую статистику с GPU:
\pause
\begin{itemize}
\item Сколько было нарисовано пикселей
\pause
\item Сколько сгенерировано примитивов (геометрическим шейдером)
\pause
\item Сколько прошло времени
\end{itemize}
\pause
\item \verb|glGenQueries/glDeleteQueries|
\pause
\item \textbf{Нет} \verb|glBindQuery|!
\end{itemize}
\end{frame}

\begin{frame}[fragile]
\frametitle{Измерение времени работы \textendash{} правильный способ: timer queries}
\begin{itemize}
\item \verb|glBeginQuery/glEndQuery| \textendash{} статистика будет собрана для команд между этими вызовами
\pause
\item \textbf{Не могут} быть вложенными
\end{itemize}
\pause
\begin{verbatim}
GLuint query_id;
glGenQueries(1, &query_id);

...

glBegin(GL_TIME_ELAPSED, query_id);

// что-нибудь рисуем

glEnd(GL_TIME_ELAPSED);
\end{verbatim}
\end{frame}

\begin{frame}[fragile]
\frametitle{Измерение времени работы \textendash{} правильный способ: timer queries}
\begin{itemize}
\item GPU работает асинхронно \begin{math}\Rightarrow\end{math} результат query будет готов не сразу
\pause
\item Узнать, готов ли результат:
\begin{verbatim}
glGetQueryObjectiv(query_id,
  GL_QUERY_RESULT_AVAILABLE, &result);
\end{verbatim}
\pause
\item Получить результат (блокирует поток, если результат ещё не готов; неявно вызывает \verb|glFlush|)
\begin{verbatim}
glGetQueryObjectiv(query_id,
  GL_QUERY_RESULT, &result);
\end{verbatim}
\pause
\item Время возвращается в \textbf{наносекундах}, т.е. знаковый 32-битный тип может представить 2 секунды
\pause
\item Если 64-битные и беззнаковые версии этих функций
\end{itemize}
\end{frame}

\begin{frame}[fragile]
\frametitle{Измерение времени работы \textendash{} правильный способ: пул timer queries}
\begin{itemize}
\item Хотим мерять время рисования каждого кадра, но результат для предыдущего кадра может быть не готов к началу следующего кадра
\pause
\item \begin{math}\Rightarrow\end{math} Заводим пул (pool) query-объектов:
\pause
\begin{itemize}
\item Храним расширяемый массив (\verb|std::vector|) query-объектов: ID + свободен или нет
\pause
\item Когда нам нужен новый query, ищем в массиве свободный объект, если такого нет - добавляем новый
\pause
\item В конце рисования кадра проходим по всем несвободным объектам и проверяем: если результат уже готов, обрабатываем его и помечаем объект свободным
\end{itemize}
\pause
\item Средний размер пула \textendash{} на сколько кадров отстаёт GPU от CPU
\end{itemize}
\end{frame}

\begin{frame}[fragile]
\frametitle{Timer queries: ссылки}
\begin{itemize}
\item \href{https://www.khronos.org/opengl/wiki/Query_Object}{khronos.org/opengl/wiki/Query\_Object}
\item \href{https://www.lighthouse3d.com/tutorials/opengl-timer-query}{Туториал по использованию timer queries}
\end{itemize}
\end{frame}

\end{document}