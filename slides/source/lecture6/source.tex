% (c) Nikita Lisitsa, lisyarus@gmail.com, 2021

\documentclass{beamer}

\usepackage[T2A]{fontenc}
\usepackage[utf8]{inputenc}
\usepackage[russian]{babel}

\usepackage{graphicx}
\graphicspath{ {./images/} }

\usepackage{adjustbox}

\usepackage{color}
\usepackage{soul}

\usepackage{hyperref}

\usepackage{tikz}
\usetikzlibrary{decorations}
\usetikzlibrary{decorations.pathreplacing}
\usepackage{xifthen}

\definecolor{red}{rgb}{1,0,0}
\definecolor{green}{rgb}{0,0.5,0}
\definecolor{blue}{rgb}{0,0,1}
\definecolor{magenta}{rgb}{0.75,0,0.75}

\makeatletter
\newcommand{\slideimage}[1]{
  \begin{figure}
    \begin{adjustbox}{width=\textwidth, totalheight=\textheight-2\baselineskip-2\baselineskip,keepaspectratio}
      \includegraphics{#1}
    \end{adjustbox}
  \end{figure}
}
\makeatother

\title{Компьютерная графика}
\subtitle{Лекция 6: Ошибки OpenGL, расширения OpenGL, blending, освещение}
\date{2021}

\setbeamertemplate{footline}[frame number]

\begin{document}

\frame{\titlepage}

\begin{frame}[fragile]
\frametitle{Ошибки OpenGL}
\begin{itemize}
\item Часто драйвер может понять, что в определённом OpenGL-вызове содержится ошибка
\pause
\begin{itemize}
\item \verb|glTexImage2D| с параметром \verb|target|, не являющимся одним из типов текстур
\pause
\item \verb|glTexParameteri| устанавливающая mag filter в что-то отличное от \verb|GL_LINEAR| или \verb|GL_NEAREST|
\pause
\item \verb|glBindBuffer| с ID буффера, который не был получен через \verb|glGenBuffers|
\pause
\item etc.
\end{itemize}
\pause
\item В таких случаях генерируется ошибка как особое значение типа \verb|GLenum|
\pause
\item Как правило, вызвавшая ошибку операция не выполняется
\pause
\item \verb|glGetError()| - получить значение ошибки, если она была
\end{itemize}
\end{frame}

\begin{frame}[fragile]
\frametitle{Ошибки OpenGL}
Возможные значения ошибок:
\begin{itemize}
\item \verb|GL_NO_ERROR| - ошибки нет
\pause
\item \verb|GL_INVALID_ENUM| - недопустимое значения перечисления (например, \verb|GL_TEXTURE_2D| как первый параметр \verb|glBindBuffer|)
\pause
\item \verb|GL_INVALID_VALUE| - недопустимое значение числового аргумента (например, отрицательная ширина текстуры в \verb|glTexImage2D|)
\pause
\item \verb|GL_INVALID_OPERATION| - недопустимая операция (например, \verb|glUniform1i| при отсутствии текущей шейдерной программы)
\pause
\item \verb|GL_INVALID_FRAMEBUFFER_OPERATION| - операция рисования/чтения пикселей с невалидным framebuffer'ом (о них мы поговорим позже)
\pause
\item \verb|GL_OUT_OF_MEMORY| - закончилась память на GPU (может быть вызвана любой OpenGL-командой, обычно не обрабатывается)
\end{itemize}
\end{frame}

\begin{frame}[fragile]
\frametitle{Ошибки OpenGL}
\begin{itemize}
\item Документация к каждой функции описывает все возможные ошибки, которые эта функция может вызвать, и в каких случаях
\pause
\begin{itemize}
\item N.B. \verb|GL_OUT_OF_MEMORY| нигде не указан, потому что может появиться где угодно
\end{itemize}
\pause
\item Не все ошибки покрываются этим механизмом (например: \verb|glDrawArrays| с правильно настроенным VAO но с пустым VBO с вершинами)
\pause
\item Нет информации о том, какая именно функция вызвала ошибку (любая из тех, что были вызваны до \verb|glGetError|)
\pause
\begin{itemize}
\item \begin{math}\Rightarrow\end{math} Придётся расставлять проверку после каждого OpenGL-вызова
\end{itemize}
\end{itemize}
\end{frame}

\begin{frame}[fragile]
\frametitle{Ошибки OpenGL}
\begin{itemize}
\item Драйвер может хранить один флаг ошибки, и пропускать все последующие ошибки до вызова \verb|glGetError|
\pause
\item Драйвер может хранить несколько флагов, и \verb|glGetError| возвращает и очищает любой из них
\pause
\item \begin{math}\Rightarrow\end{math} Чтобы очистить ошибки, нужно вызывать \verb|glGetError| в цикле:
\end{itemize}
\begin{verbatim}
while (glGetError() != GL_NO_ERROR);
\end{verbatim}
\end{frame}

\begin{frame}[fragile]
\frametitle{Ошибки OpenGL}
\begin{itemize}
\item OpenGL 4.3+: более удобный механизм, \verb|glDebugMessageCallback|
\pause
\item \href{https://www.khronos.org/opengl/wiki/OpenGL_Error}{khronos.org/opengl/wiki/OpenGL\_Error}
\item \href{https://docs.gl/gl3/glGetError}{docs.gl/gl3/glGetError}
\end{itemize}
\end{frame}

\end{document}