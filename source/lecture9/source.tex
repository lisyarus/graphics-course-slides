% (c) Nikita Lisitsa, lisyarus@gmail.com, 2025

\documentclass[10pt]{beamer}

\usepackage[T2A]{fontenc}
\usepackage[russian]{babel}
\usepackage{minted}

\usepackage{graphicx}
\graphicspath{ {./images/} }

\usepackage{adjustbox}

\usepackage{color}
\usepackage{soul}

\usepackage{hyperref}

\usepackage{amsmath}

\usepackage{tikz}
\usetikzlibrary{decorations}
\usetikzlibrary{decorations.pathreplacing}
\usepackage{xifthen}

\usetheme{metropolis}
\setminted{fontsize=\footnotesize}

\definecolor{red}{rgb}{1,0,0}
\definecolor{green}{rgb}{0,0.5,0}
\definecolor{blue}{rgb}{0,0,1}
\definecolor{magenta}{rgb}{0.75,0,0.75}
\definecolor{codebg}{RGB}{29,35,49}

\makeatletter
\newcommand{\slideimage}[1]{
  \begin{figure}
    \begin{adjustbox}{width=\textwidth, totalheight=\textheight-2\baselineskip-2\baselineskip,keepaspectratio}
      \includegraphics{#1}
    \end{adjustbox}
  \end{figure}
}
\makeatother

\title{Компьютерная графика}
\subtitle{Лекция 9: Вычисление проекции для shadow mapping, ESM, VSM, PSM, CSM, Ambient Occlusion}
\date{2025}

\setbeamertemplate{footline}[frame number]

\begin{document}

\usemintedstyle{solarized-light}

\frame{\titlepage}

\begin{frame}[fragile]
\frametitle{Проекция для shadow map: точечный источник}
\begin{itemize}
\item 6 перспективных проекций для 6 граней cubemap-текстуры shadow map (одно рисование всей сцены на каждую из них)
\pause
\item Каждая проекция смотрит в направлении некоторой оси: \begin{math}\pm X, \pm Y, \pm Z\end{math}
\pause
\item Соответствующие грани cubemap-текстуры в OpenGL называются \mintinline{cpp}|GL_TEXTURE_CUBE_MAP_POSITIVE_X|, \mintinline{cpp}|GL_TEXTURE_CUBE_MAP_NEGATIVE_X| и т.д. (параметр \mintinline{cpp}|glTexImage2D|, \mintinline{cpp}|glFramebufferTexture2D|, etc)
\pause
\item Каждую проекцию нужно будет покрутить вокруг оси взгляда, чтобы она совпала с тем, как эта грань cubemap текстуры определена в OpenGL (проще всего -- подбором одного из четырёх вращений на \begin{math}90^\circ\end{math})
\pause
\item Угол видимой области -- \begin{math}90^\circ\end{math} по обеим осям (квадратная камера, aspect ratio = 1)
\end{itemize}
\end{frame}

\begin{frame}[fragile]
\frametitle{Проекция для shadow map: точечный источник}
\begin{itemize}
\item \mintinline{cpp}|near| и \mintinline{cpp}|far| -- максимальное (соответственно, минимальное), при котором не обрезается ничего лишнего
\pause
\begin{itemize}
\item \mintinline{cpp}|near| -- минимальное расстояние до вершин объектов сцены (часто просто хардкодится)
\item \mintinline{cpp}|far| -- максимальное расстояние до вершин объектов сцены
\end{itemize}
\pause
\item Например, \mintinline{cpp}|far| можно посчитать как \begin{math}\max\limits_V |(V - C) \cdot D|\end{math}, где
\begin{itemize}
\item \begin{math}C\end{math} -- координаты центра проекции (источника света)
\item \begin{math}D\end{math} -- единичный вектор направления взгляда проекции
\item \begin{math}V\end{math} -- вершины объектов сцены
\end{itemize}
\pause
\item Можно заранее вычислить bounding box сцены и вычислять максимум только по его вершинам
\end{itemize}
\end{frame}

\begin{frame}[fragile]
\frametitle{Проекция для shadow map: направленный источник}
\begin{itemize}
\item Одна ортографическая проекция: задаётся центром \begin{math}C\end{math} и осями \begin{math}X, Y, Z\end{math}
\pause
\item \begin{math}Z\end{math} параллельный направлению на источник света
\pause
\item \begin{math}X, Y\end{math} перпендикулярны \begin{math}Z\end{math} (и друг другу)
\begin{itemize}
\item Можно выбрать любые, перпендикулярные \begin{math}Z\end{math}
\item Можно попробовать выбрать их так, чтобы максимизировать эффективное разрешение shadow map (т.е. чтобы одному пикселю shadow map соответствовала как можно меньшая область пространства) -- сводится к задаче OBB (oriented bounding box)
\end{itemize}
\pause
\item Откуда взять \begin{math}C\end{math} и длины векторов \begin{math}X,Y,Z\end{math}?
\end{itemize}
\end{frame}

\begin{frame}[fragile]
\frametitle{Проекция для shadow map: направленный источник}
\begin{itemize}
\item Вычислим bounding box всей сцены
\pause
\item В качестве \begin{math}C\end{math} возьмём центр bounding box'а сцены
\pause
\item Мы знаем направление вектора \begin{math}X\end{math}, нужно вычислить его длину
\begin{itemize}
\item Пусть \begin{math}\hat X\end{math} -- вектор направления (нормированный)
\item Мы хотим, чтобы для всех вершин \begin{math}И\end{math} сцены выполнялось \begin{math}|(V - C) \cdot \hat X| \leq |X|\end{math}
\pause
\item Пройдёмся по всем восьми вершинам bounding box'а сцены и вычислим \begin{math}\max\limits_V |(V - C) \cdot \hat X|\end{math} -- это длина вектора \begin{math}X\end{math}
\end{itemize}
\pause
\item Аналогично для \begin{math}Y, Z\end{math}
\end{itemize}
\end{frame}

\begin{frame}[fragile]
\frametitle{Проекция для shadow map: направленный источник}
\begin{itemize}
\item Описанный алгоритм слишком консервативен: в shadow map попадает вообще вся сцена
\pause
\item В общем случае мы хотим, чтобы в shadow map попали
\pause
\begin{itemize}
\item Объекты, видимые из камеры
\pause
\item Объекты, бросающие на них тень
\end{itemize}
\pause
\item Можно взять bounding box области, видимой из основной камеры, и вычислить, что может бросать тень на эту область, и взять проекцию для shadow map, в которую попадают только эти объекты
\pause
\item Можно подобрать оси X, Y для проекции shadow map так, чтобы максимально плотно уместить видимую область
\end{itemize}
\end{frame}

\begin{frame}[fragile]
\frametitle{Проблемы shadow mapping: пикселизация}
\begin{itemize}
\item В базовом shadow mapping хорошо видны пиксели shadow map (квадратная лесенка на границе тени)
\pause
\begin{itemize}
\item Чем больше сцена, тем хуже (пиксели становятся больше)
\pause
\item Особенно плохо, если тень движется
\end{itemize}
\pause
\item Один из вариантов решения для направленных источников: при движении камеры сдвигать область проекции shadow map только на расстояния, кратные пикселям shadow map
\pause
\begin{itemize}
\item Пиксели тени всё равно видны, но по крайней мере они не будут двигаться и шуметь
\pause
\item Не подходит, если источник света движется
\end{itemize}
\end{itemize}
\end{frame}

\begin{frame}[fragile]
\frametitle{Проблемы shadow mapping: пикселизация}
\begin{itemize}
\item Второй вариант решения: сгладить тени, убрав пиксели и сделав аппроксимацию мягких теней
\pause
\item Саму shadow map сглаживать (размывать / использовать линейную фильтрацию / mipmaps) нет смысла: получатся интерполированные значения глубин, сравнение с ними даёт видимые артефакты и ничего не улучшает
\pause
\item Встроенный в hardware PCF (shadow samplers) усредняет только по соседним 4 пикселям
\pause
\item Ручное размытие в финальном шейдере приходится делать \begin{math}O(N^2)\end{math} несепарабельным фильтром
\end{itemize}
\end{frame}

\begin{frame}[fragile]
\frametitle{Пиксели shadow mapping}
\slideimage{shadow_map_nearest.png}
\end{frame}

\begin{frame}[fragile]
\frametitle{Пиксели shadow mapping (с линейной фильтрацией)}
\slideimage{shadow_map_linear.png}
\end{frame}

\begin{frame}[fragile]
\frametitle{PCF}
\slideimage{pcf.png}
\end{frame}

\begin{frame}[fragile]
\frametitle{PCF + 7x7 размытие}
\slideimage{pcf_gauss.png}
\end{frame}

\begin{frame}[fragile]
\frametitle{Filtered shadow maps}
\begin{itemize}
\item Хочется, чтобы shadow map содержал значения, которые имеет смысл интерполировать (\textit{фильтровать})
\pause
\item Тогда можно применить весь арсенал работы с текстурами: линейную фильтрацию, mipmaps, размытие, etc.
\pause
\item В частности, размытие можно будет сделать предварительно (после генерации shadow map, но до её использования), а не прямо в финальном шейдере
\end{itemize}
\end{frame}

\begin{frame}[fragile]
\frametitle{Filtered shadow maps}
\begin{itemize}
\item Идея: рассматривать значение в одном пикселе shadow map как описание некого распределения вероятности
\pause
\item График уровня освещённости в зависимости от расстояния до источника света (\begin{math}\color{blue}Z_0\end{math} -- глубина объекта, бросающего тень)
\end{itemize}
\begin{center}
\begin{tikzpicture}
\draw[black,thick] (0.0, 0.0) -- (10.0, 0.0);
\draw[red,thick] (0.0, 0.0) -- (5.0, 0.0) -- (5.0, 1.0) -- (10.0, 1.0);

\node[black] at (10.5, 0.0) {0};
\node[red] at (10.5, 1.0) {1};
\node[blue] at (5.0, -0.5) {\begin{math}Z_0\end{math}};
\end{tikzpicture}
\end{center}
\end{frame}

\begin{frame}[fragile]
\frametitle{Filtered shadow maps}
\begin{itemize}
\item Распределения вероятности можно усреднять: если \begin{math}F_1(z), F_2(z)\end{math} -- распределения вероятности, то \begin{math}F(z) = (1 - t) \cdot F_1(z) + t \cdot F_2(z)\end{math} -- тоже распределение вероятности
\pause
\item Есть много вариантов того, как представить распределение одним пикселем shadow map
\end{itemize}
\end{frame}

\begin{frame}<1>[fragile,label=esm]
\frametitle{Exponential shadow maps (ESM)}
\begin{itemize}
\item Идея: функцию распределения можно аппроксимировать экспонентой \begin{math}F(z) = \min(1, \exp(C \cdot (z_0 - z))) = \min(1, \exp(-Cz)\exp(Cz_0))\end{math} (\begin{math}z\end{math} -- расстояние до освещаемого пикселя, \begin{math}z_0\end{math} -- расстояние до источника тени)
\pause
\item Интерполяция таких функций сводится к интерполяции \begin{math}\exp(Cz_0)\end{math}
\pause
\item Будем записывать в shadow map значение \begin{math}\exp(Cz_0)\end{math}
\pause
\item В шейдере будем умножать это значение на \begin{math}\exp(-Cz)\end{math}
\pause
\item Если результат больше единицы (\begin{math}z < z_0\end{math}), принимаем освещённость за единицу
\end{itemize}
\end{frame}

\begin{frame}[fragile]
\frametitle{Exponential shadow maps (ESM)}
\slideimage{exp_plot.png}
\end{frame}

\againframe<1->{esm}

\begin{frame}[fragile]
\frametitle{ESM с линейной фильтрацией}
\slideimage{exp.png}
\end{frame}

\begin{frame}[fragile]
\frametitle{ESM с линейной фильтрацией и 7x7 размытием по Гауссу}
\slideimage{exp_gauss.png}
\end{frame}

\begin{frame}[fragile]
\frametitle{ESM}
\begin{itemize}
\item {\color{green}+} Shadow map можно фильтровать (размывать, mipmaps, etc)
\item {\color{green}+} Прост в реализации
\item {\color{red}—} Требует floating-point буфер глубины или рендеринг в floating-point текстуру
\item {\color{red}—} Константу \verb|C| приходится делать очень большой \begin{math}\Rightarrow\end{math} могут быть проблемы с точностью
\end{itemize}
\end{frame}

\begin{frame}[fragile]
\frametitle{Variance shadow maps (VSM)}
\begin{itemize}
\item Идея: запишем в shadow map мат. ожидание глубины и её квадрата
\pause
\item Их можно усреднять:
\begin{equation}
\int z \left[(1-t)p_1(z) + tp_2(z)\right] dz = (1-t)\int z p_1(z) dz + t\int z p_2(z) dz
\end{equation}
\item (аналогично для \begin{math}z^2\end{math})
\pause
\item Моделируем пиксель как дискретное распределение с единственным значением \begin{math}\Rightarrow\end{math} мат.ожидание глубины совпадает с её значением (и аналогично для квадрата глубины)
\end{itemize}
\end{frame}

\begin{frame}[fragile]
\frametitle{Variance shadow maps (VSM)}
\begin{itemize}
\item По мат.ожиданию глубины (\begin{math}\mu\end{math}) и её квадрата можно вычислить дисперсию
\begin{equation}
\sigma^2 = \mathbb{E}[z^2] - \mathbb{E}[z]^2
\end{equation}
\pause
\item По мат.ожиданию глубины и дисперсии можно оценить освещённость через \href{https://en.wikipedia.org/wiki/Cantelli%27s_inequality}{одностороннее неравенство Чебышёва (оно же -- неравенство Кантелли)}:
\begin{equation}
P(z) = \frac{\sigma^2}{\sigma^2 + (z - \mu)^2}
\end{equation}
\end{itemize}
\end{frame}

\begin{frame}[fragile]
\frametitle{Variance shadow maps (VSM): алгоритм}
\begin{itemize}
\item В shadow map записываем глубину и её квадрат (нужна текстура с 2 компонентами)
\pause
\item По желанию с результирующей текстурой shadow map производим фильтрацию (например, размытие)
\pause
\item В шейдере читаем shadow map, вычисляем \begin{math}\mu\end{math} и \begin{math}\sigma^2\end{math}
\pause
\item Если глубина нашего пикселя меньше \begin{math}\mu\end{math}, освещённость равна 1
\pause
\item Иначе, вычисляем освещённость через неравенство Чебышёва
\end{itemize}
\end{frame}

\begin{frame}[fragile]
\frametitle{Variance shadow maps (VSM): shadow bias}
\begin{itemize}
\item VSM позволяет элегантно включить shadow bias
\pause
\item Моделируем пиксель как параллелограмм с линейно меняющейся глубиной
\pause
\item Градиент глубины можно аппроксимировать с помощью \verb|dFdx| и \verb|dFdy| в шейдере
\pause
\item Итоговый квадрат глубины вычисляется как
\begin{equation}
z^2 + \frac{1}{4}\left[\left(\frac{\partial z}{\partial x}\right)^2 + \left(\frac{\partial z}{\partial y}\right)^2\right]
\end{equation}
\pause
\item Убирает \textit{почти} все артефакты shadow acne
\end{itemize}
\end{frame}

\begin{frame}[fragile]
\frametitle{VSM с линейной фильтрацией}
\slideimage{vsm.png}
\end{frame}

\begin{frame}[fragile]
\frametitle{VSM с линейной фильтрацией и 7x7 размытием по Гауссу}
\slideimage{vsm_gauss.png}
\end{frame}

\begin{frame}[fragile]
\frametitle{VSM}
\begin{itemize}
\item {\color{green}+} Shadow map можно фильтровать (размывать, mipmaps, etc)
\item {\color{green}+} Достаточно прост в реализации
\item {\color{red}—} Требует большего объёма памяти
\item {\color{red}—} Возникает артефакт light bleeding
\end{itemize}
\end{frame}

\begin{frame}[fragile]
\frametitle{Light bleeding}
\slideimage{vsm_light_bleeding.png}
\end{frame}

\begin{frame}[fragile]
\frametitle{Light bleeding}
\begin{itemize}
\item Обычно возникает, когда \begin{math}\sigma^2\end{math} оказывается слишком большим (в близкие пиксели попали очень далёкие объекты)
\pause
\item Можно частично исправить, трактуя маленькие значения коэффициента освещённости как нулевые, и преобразовав остальные значения в диапазон \begin{math}[0, 1]\end{math}
\pause
\item Конкретные значения нужно подбирать в зависимости от сцены
\end{itemize}
\end{frame}

\begin{frame}[fragile]
\frametitle{Light bleeding}
\slideimage{light_bleeding_scheme.jpg}
\end{frame}

\begin{frame}[fragile]
\frametitle{Convolution shadow maps (CSM)}
\begin{itemize}
\item Записывает в shadow map коэффициенты преобразования Фурье от функции распределения
\pause
\item {\color{green}+} Можно фильтровать (размывать, mipmaps, etc)
\item {\color{red}—} Сложнее в реализации
\item {\color{red}—} Требует многокомпонентных буферов (часто до 16 компонент на пиксель)
\item {\color{red}—} Возникают ringing-артефакты (см. \href{https://en.wikipedia.org/wiki/Gibbs_phenomenon}{Gibbs phenomenon})
\end{itemize}
\end{frame}

\begin{frame}[fragile]
\frametitle{Ссылки}
\begin{itemize}
\item \href{https://ogldev.org/www/tutorial42/tutorial42.html}{\texttt{ogldev.org/www/tutorial42/tutorial42.html}}
\item \href{https://learnopengl.com/Advanced-Lighting/Shadows/Shadow-Mapping}{\texttt{learnopengl.com/Advanced-Lighting/Shadows/Shadow-Mapping}}
\item \href{https://docs.microsoft.com/en-us/windows/win32/dxtecharts/common-techniques-to-improve-shadow-depth-maps}{\texttt{docs.microsoft.com/en-us/windows/win32/dxtecharts/common-techniques-to-improve-shadow-depth-maps}}
\item \href{https://jankautz.com/publications/esm_gi08.pdf}{\texttt{Оригинальная статья про ESM}}
\item \href{https://www.intel.com/content/dam/develop/external/us/en/documents/vsm-paper-182631.pdf}{\texttt{Оригинальная статья про VSM}}
\item \href{https://developer.nvidia.com/gpugems/gpugems3/part-ii-light-and-shadows/chapter-8-summed-area-variance-shadow-maps}{\texttt{Улучшения VSM, включая более реалистичные мягкие тени}}
\end{itemize}
\end{frame}

\begin{frame}[fragile]
\frametitle{Тени на больших сценах}
\begin{itemize}
\item Если сцена очень большая, даже огромных (8k) shadow map текстур с хорошим сглаживанием может не хватить
\pause
\item Классически есть два решения:
\begin{itemize}
\item Использовать неравномерное соответствие пикселей shadow map и объектов в мире
\item Использовать более одной shadow map
\end{itemize}
\end{itemize}
\end{frame}

\begin{frame}[fragile]
\frametitle{Perspective shadow maps (PSM)}
\begin{itemize}
\item При генерации shadow map помимо обычного преобразования вершин применяет ещё перспективное преобразование так, чтобы больше пикселей shadow map приходилось на близкие к камере объекты
\end{itemize}
\slideimage{psm.jpeg}
\end{frame}

\begin{frame}[fragile]
\frametitle{Perspective shadow maps (PSM)}
\begin{itemize}
\item {\color{green}+} Часто решает проблему точности shadow map
\item {\color{red}—} Сложен в реализации
\item {\color{red}—} Очень много худших случаев (например, камера смотрит на источник света), в которых алгоритм почти не работает
\end{itemize}
\end{frame}

\begin{frame}[fragile]
\frametitle{Cascaded shadow maps (CSM)}
\begin{itemize}
\item Будем использовать несколько shadow maps (\textit{cascades}): одну -- для ближайших объектов, другую -- для объектов на среднем расстоянии, третью -- для далёких объектов, и т.п.
\end{itemize}
\slideimage{csm.jpg}
\end{frame}

\begin{frame}[fragile]
\frametitle{Cascaded shadow maps (CSM)}
\begin{itemize}
\item {\color{green}+} Прост в реализации
\item {\color{green}+} Легко комбинировать с VSM/ESM/etc
\item {\color{red}—} Большой расход памяти
\item {\color{red}—} Нужно аккуратно обработать переход между разными каскадами
\item Самый \textit{\underline{часто использующийся}} сегодня алгоритм, поддерживается большинством движков
\end{itemize}
\end{frame}

\begin{frame}[fragile]
\frametitle{Shadow mapping на больших сценах: ссылки}
\begin{itemize}
\item \href{https://www-sop.inria.fr/reves/Basilic/2002/SD02/PerspectiveShadowMaps.pdf}{\texttt{Оригинальная статья про PSM}}
\item \href{https://developer.download.nvidia.com/books/HTML/gpugems/gpugems_ch14.html}{\texttt{PSM + улучшения}}
\item \href{https://developer.download.nvidia.com/SDK/10.5/opengl/src/cascaded_shadow_maps/doc/cascaded_shadow_maps.pdf}{\texttt{Статья про CSM}}
\item \href{https://ogldev.org/www/tutorial49/tutorial49.html}{\texttt{ogldev.org/www/tutorial49/tutorial49.html}}
\item \href{https://learnopengl.com/Guest-Articles/2021/CSM}{\texttt{learnopengl.com/Guest-Articles/2021/CSM}}
\end{itemize}
\end{frame}

\begin{frame}[fragile]
\frametitle{Ambient occlusion}
\begin{itemize}
\item Ambient освещение светит `отовсюду' -- эвристическая аппроксимация многократных отражений в уравнении рендеринга
\pause
\item Это свет, значит от него может быть тень
\pause
\item Какие-то части сцены получают больше ambient света, какие-то -- меньше
\pause
\item Обычно затеняются углы, углубления, трещины, стыки объектов, и т.п.
\pause
\item Очень сильно увеличивает реализм изображения
\end{itemize}
\end{frame}

\begin{frame}[fragile]
\frametitle{Ambient occlusion (Crysis, 2007)}
\slideimage{crysis-ao.jpg}
\end{frame}

\begin{frame}[fragile]
\frametitle{Ambient occlusion}
\slideimage{unity-ao.jpg}
\end{frame}

\begin{frame}[fragile]
\frametitle{Ambient occlusion}
\slideimage{artstation-ao.jpg}
\end{frame}

\begin{frame}[fragile]
\frametitle{Ambient occlusion}
\slideimage{birch-combined.png}
\end{frame}

\begin{frame}[fragile]
\frametitle{Ambient occlusion: алгоритмы}
\begin{itemize}
\item Baking: предподсчитать ambient occlusion для модели, записать её в одноканальную текстуру и использовать в шейдере как множитель при ambient освещении
\pause
\begin{itemize}
\item {\color{green}+} Быстро работает
\item {\color{red}—} Не учитывает occlusion между разными объектами
\item {\color{red}—} Не работает для анимированных объектов
\end{itemize}
\end{itemize}
\end{frame}

\begin{frame}[fragile]
\frametitle{Ambient occlusion}
\slideimage{baked-ao.png}
\end{frame}

\begin{frame}[fragile]
\frametitle{Baking}
\begin{itemize}
\item Записывание предподсчитанных свойств объектов в текстуры в общем случае называется \textit{baking} (\textit{запеканием})
\pause
\item Все 3D-редакторы умеют запекать нормали, ambient occlusion, и многое другое
\end{itemize}
\end{frame}

\begin{frame}[fragile]
\frametitle{Ambient occlusion: алгоритмы}
\begin{itemize}
\item Вычислять ambient occlusion на лету, используя какие-нибудь эвристики (алгоритмы SSAO, HBAO, HBAO+, HDAO, VXAO, ...)
\pause
\begin{itemize}
\item {\color{green}+} Работает с любыми сценами
\item {\color{red}—} Сложнее в реализации
\item {\color{red}—} Медленее работает
\item {\color{red}—} Обычно требуют подгона параметров под сцену
\end{itemize}
\end{itemize}
\end{frame}

\begin{frame}[fragile]
\frametitle{SSAO (Crytek, 2007)}
\begin{itemize}
\item Screen-space ambient occlusion (SSAO) -- один из первых real-time алгоритмов для генерации ambient occlusion
\pause
\item Термин \textit{screen-space} означает, что алгоритм использует только те данные, которые попадают на экран в процессе рендеринга (т.е. не использует саму исходную геометрию, не использует рендеринг в текстуру с другого ракурса, и т.п.)
\pause
\item Идея: вычислим пересечение полусферы некоторого радиуса в освещаемой точке с окружающей геометрией, объём этого пересечения -- аппроксимация ambient occlusion (чем больше вокруг пикселя других объектов, тем слабее он освещён)
\end{itemize}
\end{frame}

\begin{frame}[fragile]
\frametitle{SSAO (Crytek, 2007)}
\slideimage{crysis-ao.jpg}
\end{frame}

\begin{frame}[fragile]
\frametitle{SSAO}
\begin{itemize}
\item Вычислять настоящее пересечение -- дорого, вместо этого аппроксимируем его монте-карло интегрированием: возьмём набор случайных точек в полусфере вокруг освещаемой точки и сравним их глубину со значением из Z-буффера
\pause
\item Обычно набор случайных точек фиксирован, передаётся в виде небольшой текстуры, где в цветах пикселей записаны координаты точек (относительно исходной точки)
\pause
\item Использование везде одного и того же набора точек приводит к \textit{banding}-у
\end{itemize}
\end{frame}

\begin{frame}[fragile]
\frametitle{SSAO banding}
\begin{itemize}
\item Banding -- ситуация, когда хорошо видна дискретизация цветов
\pause
\item В SSAO источник banding'а -- монте-карло интегрирование: отношение \begin{math}\frac{K}{N}\end{math} (K -- количество видимых точек, N -- всех точек) может принимать небольшой набор (N+1) дискретных значений
\pause
\item Можно вращать набор случайных точек вокруг нормали освещаемой точки на угол, случайный для каждого освещаемого пикселя (угол тоже передаётся текстурой или вычисляется псевдо-случайным на основе координат пикселя)
\pause
\item Приводит к `шуму' в изображении, который менее заметен и который можно размыть
\pause
\item Разменять banding на шум -- \textit{классический приём} real-time графики
\end{itemize}
\end{frame}

\begin{frame}[fragile]
\frametitle{SSAO banding}
\slideimage{ssao.jpg}
\end{frame}

\begin{frame}[fragile]
\frametitle{SSAO: ссылки}
\begin{itemize}
\item \href{https://artis.inrialpes.fr/Membres/Olivier.Hoel/ssao/nVidiaHSAO/ScreenSpaceAO.pdf}{\texttt{Оригинальная статья про SSAO}}
\item \href{https://learnopengl.com/Advanced-Lighting/SSAO}{\texttt{learnopengl.com/Advanced-Lighting/SSAO}}
\end{itemize}
\end{frame}

\end{document}