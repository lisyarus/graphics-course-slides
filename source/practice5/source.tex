% (c) Nikita Lisitsa, lisyarus@gmail.com, 2021

\documentclass{beamer}

\usepackage[T2A]{fontenc}
\usepackage[utf8]{inputenc}
\usepackage[russian]{babel}

\usepackage{graphicx}
\graphicspath{ {./images/} }

\usepackage{adjustbox}

\usepackage{color}
\usepackage{soul}

\usepackage{hyperref}

\definecolor{blue}{rgb}{0,0,1}
\definecolor{red}{rgb}{1,0,0}

\makeatletter
\newcommand{\slideimage}[1]{
  \begin{figure}
    \begin{adjustbox}{width=\textwidth, totalheight=\textheight-2\baselineskip-2\baselineskip,keepaspectratio}
      \includegraphics{#1}
    \end{adjustbox}
  \end{figure}
}
\makeatother

\title{Компьютерная графика}
\subtitle{Практика 5: Текстуры}
\date{2021}

\setbeamertemplate{footline}[frame number]

\begin{document}

\frame{\titlepage}

\begin{frame}[fragile]
\slideimage{0.png}
\end{frame}

\begin{frame}[fragile]
\frametitle{Задание 1}
Загружаем модель \verb|cow|, как в предыдущей практике:
\begin{itemize}
\item Создаём VAO, VBO, EBO
\item Загружаем вершины в VBO
\item Загружаем индексы в EBO
\item Настраиваем атрибуты в VAO (пока только первые два)
\item Связываем индексы с VAO
\item Рисуем с помощью \verb|glDrawElements|
\item N.B. модель можно крутить и двигать стрелочками
\end{itemize}
\end{frame}

\begin{frame}[fragile]
\slideimage{1.png}
\end{frame}

\begin{frame}[fragile]
\frametitle{Задание 2}
Добавляем текстурные координаты
\begin{itemize}
\item Описываем новый (index=2) атрибут для VAO
\item Добавляем входной атрибут типа \verb|vec2| в вершинном шейдере, и передаём его без изменений во фрагментный
\item Во фрагментном шейдере используем текстурные координаты в качестве цвета (альбедо), например \verb|albedo = vec3(texcoord, 0.0)|
\end{itemize}
\end{frame}

\begin{frame}[fragile]
\slideimage{2.png}
\end{frame}

\begin{frame}[fragile]
\frametitle{Задание 3}
\fontsize{8pt}{8pt}\selectfont
Создаём и используем текстуру шахматной раскраски
\begin{itemize}
\item Создаём объект текстуры типа \verb|GL_TEXTURE_2D|
\item Выставляем для него min и mag фильтры в \verb|GL_NEAREST| (\verb|glTexParameteri|)
\item Выбираем размер текстуры (лучше довольно большой, в духе 512 или 1024)
\item В качестве данных загружаем пиксели шахматной раскраски
\begin{itemize}
\item Пиксели удобнее всего хранить как \verb|std::uint32_t|
\item Создать массив пикселей можно как \verb|std::vector<std::uint32_t> pixels(size * size);|
\item Заполнить его чёрными \verb|0xFF000000u| и белыми \verb|0xFFFFFFFFu| пикселями в порядке шахматной раскраски
\item Для \verb|glTexImage2D| параметры \verb|internalFormat = GL_RGBA8|, \verb|format = GL_RGBA|, \verb|type = GL_UNSIGNED_BYTE|
\end{itemize}
\item Во фрагментном шейдере добавляем uniform-переменную типа \verb|sampler2D| и берём из неё цвет (\verb|albedo|) функцией \verb|texture|
\item В значение uniform-переменной записываем 0 (\verb|glUniform1i|)
\item Перед рисованием делаем нашу текстуру текущей для texture unit'а номер 0 (\verb|glActiveTexture| + \verb|glBindTexture|)
\end{itemize}
\end{frame}

\begin{frame}[fragile]
\slideimage{3.png}
\end{frame}

\begin{frame}[fragile]
\begin{itemize}
\item Если подвигать и покрутить модель, будет видный сильный \textit{муар} -- эффект, возникающий, когда значения сигнала (в нашем случае -- текстура) читаются с частотой меньшей, чем частота самого сигнала (в нашем случае частоту определяют пиксели экрана)
\end{itemize}
\end{frame}

\begin{frame}[fragile]
\frametitle{Задание 4}
Вручную добавляем mipmap-уровни
\begin{itemize}
\item Меняем min-фильтр текстуры на \verb|GL_NEAREST_MIPMAP_NEAREST|
\item После загрузки данных текстуры вызываем \verb|glGenerateMipmap|, чтобы сгенирировались mipmap-уровни
\item Перезаписываем 1-ый mipmap-уровень монохромной картинкой красного цвета
\begin{itemize}
\item Размер по обеим осям должен быть ровно в 2 раза меньше, чем размер исходного изображения
\end{itemize}
\item Перезаписываем 2-ый mipmap-уровень монохромной картинкой зелёного цвета
\begin{itemize}
\item Размер по обеим осям должен быть ровно в 4 раза меньше, чем размер исходного изображения
\end{itemize}
\item Перезаписываем 3-ый mipmap-уровень монохромной картинкой синего цвета
\begin{itemize}
\item Размер по обеим осям должен быть ровно в 8 раз меньше, чем размер исходного изображения
\end{itemize}
\item Подвигайте модель, чтобы увидеть, как меняется выбранный mipmap-уровень
\end{itemize}
\end{frame}

\begin{frame}[fragile]
\slideimage{4.png}
\end{frame}

\begin{frame}[fragile]
\frametitle{Задание 5}
Загружаем настоящую текстуру
\begin{itemize}
\item Создаём \textit{новую} текстуру
\item Настраиваем для неё mag и min фильтры в \verb|GL_LINEAR| и \verb|GL_LINEAR_MIPMAP_LINEAR| соответственно
\item Загружаем её данные из файла с помощью функции \verb|stbi_load|, путь до изображения есть в переменной \verb|cow_texture_path|, число каналов \verb|desired_channels| указывать равным 4
\item Загружаем эти данные на GPU с помощью \verb|glTexImage2D| и не забываем сгенерировать mipmaps
\item Очищаем данные на CPU с помощью \verb|stbi_image_free|
\item Используем для этой текстуры texture unit с номером 1, шейдер на этом этапе меняться не должен
\end{itemize}
\end{frame}

\begin{frame}[fragile]
\slideimage{5.png}
\end{frame}

\begin{frame}[fragile]
\frametitle{Задание 6*}
Крутим текстуру по модели
\begin{itemize}
\item Передаём текущее время в шейдер в качестве uniform-переменной
\item Сдвигаем текстурные координаты на какое-то зависящее от времени значение
\end{itemize}
\end{frame}

\begin{frame}[fragile]
\slideimage{6.png}
\end{frame}

\end{document}