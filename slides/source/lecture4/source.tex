% (c) Nikita Lisitsa, lisyarus@gmail.com, 2021

\documentclass{beamer}

\usepackage[T2A]{fontenc}
\usepackage[utf8]{inputenc}
\usepackage[russian]{babel}

\usepackage{graphicx}
\graphicspath{ {./images/} }

\usepackage{adjustbox}

\usepackage{color}
\usepackage{soul}

\usepackage{hyperref}

\usepackage{tikz}
\usetikzlibrary{decorations}
\usetikzlibrary{decorations.pathreplacing}
\usepackage{xifthen}

\definecolor{red}{rgb}{1,0,0}
\definecolor{green}{rgb}{0,0.5,0}
\definecolor{blue}{rgb}{0,0,1}
\definecolor{magenta}{rgb}{0.75,0,0.75}

\makeatletter
\newcommand{\slideimage}[1]{
  \begin{figure}
    \begin{adjustbox}{width=\textwidth, totalheight=\textheight-2\baselineskip-2\baselineskip,keepaspectratio}
      \includegraphics{#1}
    \end{adjustbox}
  \end{figure}
}
\makeatother

\title{Компьютерная графика}
\subtitle{Лекция 4: Камера, проекции, буфер глубины}
\date{2021}

\setbeamertemplate{footline}[frame number]

\begin{document}

\frame{\titlepage}

\begin{frame}[fragile]
\frametitle{Проекции}
\begin{itemize}
\item Мир - двухмерный/трёхмерный, произвольного размера, смотрим с произвольного ракурса
\pause
\item Экран - двухмерный, \begin{math}[-1, 1] \times [-1, 1]\end{math}
\pause
\begin{itemize}
\item Координата Z тоже \begin{math}[-1, 1]\end{math}
\item Про неё поговорим подробнее чуть позже
\end{itemize}
\pause
\item За преобразование отвечает проекция
\end{itemize}
\end{frame}

\begin{frame}[fragile]
\frametitle{Ортографическая проекция}
\begin{itemize}
\item Самый простой способ: \pause игнорировать третью координату
\begin{center}Концептуально: \begin{math}(x, y, z) \mapsto (x, y)\end{math}\end{center}
\begin{center}
\begin{math}
\begin{pmatrix}
1 & 0 & 0 & 0 \\
0 & 1 & 0 & 0 \\
0 & 0 & 1 & 0 \\
0 & 0 & 0 & 1
\end{pmatrix}
\end{math}
\end{center}
\pause
\item Именно это делает OpenGL (если \verb|gl_Position.w = 1|)
\pause
\item X и Y всё ещё \begin{math}[-1, 1]\end{math}
\pause
\item Как сделать \begin{math}X \in [-W, W]\end{math} и \begin{math}Y \in [-H, H]\end{math}?
\pause
\begin{center}
\begin{math}
\begin{pmatrix}
\frac{1}{W} & 0 & 0 & 0 \\
0 & \frac{1}{H} & 0 & 0 \\
0 & 0 & 1 & 0 \\
0 & 0 & 0 & 1
\end{pmatrix}
\end{math}
\end{center}
\end{itemize}
\end{frame}

\begin{frame}[fragile]
\frametitle{Ортографическая проекция}
\begin{itemize}
\item Как сделать \begin{math}X \in [X_0 - W, X_0 + W]\end{math} и \begin{math}Y \in [Y_0 - H, Y_0 + H]\end{math}?
\pause
\begin{itemize}
\item Сдвинуть на \begin{math}(-X_0, -Y_0)\end{math}, а затем применить масштабирование:
\end{itemize}
\begin{center}
\begin{math}
\begin{pmatrix}
\frac{1}{W} & 0 & 0 & 0 \\
0 & \frac{1}{H} & 0 & 0 \\
0 & 0 & 1 & 0 \\
0 & 0 & 0 & 1
\end{pmatrix}
\cdot
\begin{pmatrix}
1 & 0 & 0 & -X_0 \\
0 & 1 & 0 & -Y_0 \\
0 & 0 & 1 & 0 \\
0 & 0 & 0 & 1
\end{pmatrix}
\end{math}
\end{center}
\pause
\item Если размер области \begin{math}W\end{math}, нужно разделить на \begin{math}W\end{math}
\item Если центр в \begin{math}X_0\end{math}, нужно сдвинуть на \begin{math}-X_0\end{math}
\pause
\item Общая идея: если камера получена каким-то преобразованием, нужно применить обратное преобразование
\end{itemize}
\end{frame}

\begin{frame}[fragile]
\frametitle{Ортографическая проекция: обратное преобразование}
\begin{itemize}
\item Можно считать, что по умолчанию (если \verb|gl_Position.w = 1|) OpenGL делает ортографическую проекцию на квадрат \begin{math}[-1, 1]^2\end{math} в плоскости XY параллельно оси Z
\item Камера находится в начале координат
\pause
\item Если ко всему виртуальному миру (включая камеру!) применить аффинное преобразование, изображение не изменится
\pause
\item \begin{math}\Rightarrow\end{math} Если наша камера получена аффинным преобразованием из стандартной камеры OpenGL, к объектам мира нужно применить обратное преобразование
\end{itemize}
\end{frame}

\begin{frame}[fragile]
\frametitle{Ортографическая проекция}
\slideimage{orthographic.png}
\end{frame}

\begin{frame}[fragile]
\frametitle{Ортографическая проекция: общий случай}
\begin{itemize}
\item В общем случае ортографическую камеру можно задать
\pause
\begin{itemize}
\item Положением камеры \begin{math}C = (C_x, C_y, C_z)\end{math}
\pause
\item Осями координат камеры \begin{math}X = (X_x, X_y, X_z), Y = (Y_x, Y_y, Y_z), Z = (Z_x, Z_y, Z_z)\end{math}
\end{itemize}
\pause
\item Делает проекцию параллельно вектору \begin{math}Z\end{math} на параллелограмм \begin{math}C \pm X \pm Y\end{math}
\item Параллелограмм отождествляется с экраном
\pause
\item Обычно \begin{math}X, Y, Z\end{math} взаимно ортогональны
\pause
\item Как выразить эту проекцию матрицей?
\pause
\item Преобразование из стандартной системы координат OpenGL \begin{math}[-1, 1]^3\end{math} в координаты области, видимой через эту камеру: смена системы координат + параллельный перенос
\begin{center}
\begin{math}
\begin{pmatrix}
X_x & Y_x & Z_x & C_x \\
X_y & Y_y & Z_y & C_y \\
X_z & Y_z & Z_z & C_z \\
0 & 0 & 0 & 1
\end{pmatrix}
\end{math}
\end{center}
\pause
\item Обратное преобразование (проекция) - обратная матрица
\end{itemize}
\end{frame}

\begin{frame}[fragile]
\frametitle{Ортографическая проекция: общий случай}
\begin{itemize}
\item Можно представить \begin{math}X, Y, Z\end{math} как произведение длины и нормированного вектора:
\begin{center}
\begin{math}X = W \cdot \hat X \quad Y = H \cdot \hat Y \quad Z = D \cdot \hat Z\end{math}
\end{center}
\pause
\item Матрицу можно разбить на перенос, масштабирование и поворот
\begin{center}
\begin{math}
\begin{pmatrix}
X_x & Y_x & Z_x & C_x \\
X_y & Y_y & Z_y & C_y \\
X_z & Y_z & Z_z & C_z \\
0 & 0 & 0 & 1
\end{pmatrix}
=
\begin{pmatrix}
1 & 0 & 0 & C_x \\
0 & 1 & 0 & C_y \\
0 & 0 & 1 & C_z \\
0 & 0 & 0 & 1
\end{pmatrix}
\cdot
\begin{pmatrix}
\hat X_x & \hat Y_x & \hat Z_x & 0 \\
\hat X_y & \hat Y_y & \hat Z_y & 0 \\
\hat X_z & \hat Y_z & \hat Z_z & 0 \\
0 & 0 & 0 & 1
\end{pmatrix}
\cdot
\begin{pmatrix}
W & 0 & 0 & 0 \\
0 & H & 0 & 0 \\
0 & 0 & D & 0 \\
0 & 0 & 0 & 1
\end{pmatrix}
\end{math}
\end{center}
\end{itemize}
\end{frame}

\begin{frame}[fragile]
\frametitle{Ортографическая проекция: общий случай}
\begin{itemize}
\item Тогда обратная матрица (матрица проекции):
\begin{center}
\begin{math}
\begin{pmatrix}
X_x & Y_x & Z_x & C_x \\
X_y & Y_y & Z_y & C_y \\
X_z & Y_z & Z_z & C_z \\
0 & 0 & 0 & 1
\end{pmatrix}^{-1}
=
\begin{pmatrix}
\frac{1}{W} & 0 & 0 & 0 \\
0 & \frac{1}{H} & 0 & 0 \\
0 & 0 & \frac{1}{D} & 0 \\
0 & 0 & 0 & 1
\end{pmatrix}
\cdot
\begin{pmatrix}
\hat X_x & \hat Y_x & \hat Z_x & 0 \\
\hat X_y & \hat Y_y & \hat Z_y & 0 \\
\hat X_z & \hat Y_z & \hat Z_z & 0 \\
0 & 0 & 0 & 1
\end{pmatrix}^{-1}
\cdot
\begin{pmatrix}
1 & 0 & 0 & -C_x \\
0 & 1 & 0 & -C_y \\
0 & 0 & 1 & -C_z \\
0 & 0 & 0 & 1
\end{pmatrix}
\end{math}
\end{center}
\end{itemize}
\end{frame}

\begin{frame}[fragile]
\frametitle{Ортографическая проекция: ортогональный случай}
\begin{itemize}
\item Если \begin{math}X, Y, Z\end{math} ортогональны, то 
\begin{center}
\begin{math}
\begin{pmatrix}
\hat X_x & \hat Y_x & \hat Z_x & 0 \\
\hat X_y & \hat Y_y & \hat Z_y & 0 \\
\hat X_z & \hat Y_z & \hat Z_z & 0 \\
0 & 0 & 0 & 1
\end{pmatrix}^{-1}
=
\begin{pmatrix}
\hat X_x & \hat Y_x & \hat Z_x & 0 \\
\hat X_y & \hat Y_y & \hat Z_y & 0 \\
\hat X_z & \hat Y_z & \hat Z_z & 0 \\
0 & 0 & 0 & 1
\end{pmatrix}^T
=
\begin{pmatrix}
\hat X_x & \hat X_y & \hat X_z & 0 \\
\hat Y_x & \hat Y_y & \hat Y_z & 0 \\
\hat Z_x & \hat Z_y & \hat Z_z & 0 \\
0 & 0 & 0 & 1
\end{pmatrix}
\end{math}
\end{center}
\end{itemize}
\end{frame}

\begin{frame}[fragile]
\frametitle{Ортографическая проекция: применение}
\begin{itemize}
\pause
\item 2D рендеринг: 2D игры, UI, карты
\pause
\item Проектировние моделей, зданий, деталей (вид сверху, вид сбоку, вид спереди)
\pause
\item Стилизация: 3D мир с ортографической проекцией
\end{itemize}
\end{frame}

\begin{frame}[fragile]
\frametitle{Ортографическая проекция: проблемы}
\begin{itemize}
\pause
\item Объекты на разном расстоянии от камеры выглядят одинаково
\pause
\item Нельзя оценить расстояние до объекта по его изображению
\pause
\item Реальные камеры и глаза работают не так
\end{itemize}
\end{frame}

\begin{frame}[fragile]
\frametitle{Перспективная проекция}
\slideimage{perspective.png}
\end{frame}

\begin{frame}[fragile]
\frametitle{Перспективная проекция}
\begin{itemize}
\item Есть {\color{blue}центр проекции} и {\color{blue}плоскость проекции}
\item {\color{red}Проекция} {\color{green}точки} - пересечение {\color{magenta}прямой}, проходящей через эту {\color{green}точку} и {\color{blue}центр проекции}, с {\color{blue}плоскостью проекции}
\end{itemize}
\pause
\begin{center}
\begin{tikzpicture}
\draw[blue,thick] (2.0, -2.0) -- (2.0, 2.0);
\draw[magenta,thick,dashed] (0.0, 0.0) -- (6.0, 1.5);
\node at (0.0, 0.0) {\color{blue}\textbullet};
\node at (2.0, 0.5) {\color{red}\textbullet};
\node at (6.0, 1.5) {\color{green}\textbullet};
\end{tikzpicture}
\end{center}
\end{frame}

\begin{frame}[fragile]
\frametitle{Перспективная проекция}
\begin{center}
\begin{tikzpicture}
\draw[blue,thick] (2.0, -2.0) -- (2.0, 2.0);
\draw[magenta,thick,dashed] (0.0, 0.0) -- (6.0, 1.5);
\draw[black,dashed,-stealth] (-2.0, 0.0) -- (8.0, 0.0);
\draw[black,dashed,-stealth] (0.0, -3.0) -- (0.0, 3.0);
\draw[black,dashed] (6.0, 1.5) -- (6.0, -1.0);
\node at (0.0, 0.0) {\color{blue}\textbullet};
\node at (2.0, 0.5) {\color{red}\textbullet};
\node at (6.0, 1.5) {\color{green}\textbullet};

\node at (8.0, -0.25) {Z};
\node at (-0.25, 3.0) {Y};

\draw[thick,decorate,decoration={brace,amplitude=10pt,mirror}] (0.125, -0.125) -- (1.875, -0.125);
\draw[thick,decorate,decoration={brace,amplitude=10pt,mirror}] (0.125, -1.125) -- (5.875, -1.125);

\node at (1.0, -0.75) {\begin{math}z_P\end{math}};
\node at (3.0, -1.75) {\begin{math}z\end{math}};

\draw[thick,decorate,decoration={brace,amplitude=5pt,mirror}] (6.125, 0.125) -- (6.125, 1.375);
\draw[thick,decorate,decoration={brace,amplitude=2pt,mirror}] (2.125, 0.125) -- (2.125, 0.375);

\node at (2.5, 0.25) {\begin{math}y_P\end{math}};
\node at (6.5, 0.75) {\begin{math}y\end{math}};

\node at (3.0, 3.0) {\begin{math}y_P = y\frac{z_P}{z} = z_P\frac{y}{z}\end{math}};
\end{tikzpicture}
\end{center}
\end{frame}

\begin{frame}[fragile]
\frametitle{Перспективная проекция}
\begin{itemize}
\item Чтобы вычислить перспективную проекцию с центром в начале координат и плоскостью проеции \begin{math}Z=z_P\end{math}, нужно разделить на Z-координату:
\begin{center}
\begin{math}
(x,y,z) \mapsto \left(z_P\frac{x}{z}, z_P\frac{y}{z}\right)
\end{math}
\end{center}
\pause
\item Как выразить эту проекцию матрицей? \pause Никак: матрицы не умеют делить одни координаты на другие
\end{itemize}
\end{frame}

\begin{frame}[fragile]
\frametitle{Perspective divide}
\begin{itemize}
\item Чтобы поддержать перспективную проекцию, в графическом конвейере (после вершинного шейдера, перед переводом в экранные координаты) есть специальный обязательный шаг: perspective divide
\pause
\item \verb|gl_Position| делится на последнюю координату \verb|gl_Position.w|
\begin{center}
\begin{math}
(x, y, z, w) \mapsto \left(\frac{x}{w}, \frac{y}{w}, \frac{z}{w}\right)
\end{math}
\end{center}
\pause
\item Если \begin{math}w=1\end{math}, ничего не меняется (ортографическая проекция)
\pause
\item Если \begin{math}w=\end{math} расстояние до камеры, получается перспективная проекция
\end{itemize}
\end{frame}

\begin{frame}[fragile]
\frametitle{Перспективная проекция}
\begin{itemize}
\item Как выразить перспективную проекцию матрицей с последующим perspective divide?
\pause
\begin{center}
\begin{math}
\begin{pmatrix}
1 & 0 & 0 & 0 \\
0 & 1 & 0 & 0 \\
0 & 0 & 1 & 0 \\
0 & 0 & 1 & 0 \\
\end{pmatrix}
\cdot
\begin{pmatrix}
x \\ y \\ z \\ 1
\end{pmatrix}
=
\begin{pmatrix}
x \\ y \\ z \\ z
\end{pmatrix}
\mapsto
\begin{pmatrix}
x/z \\ y/z \\ 1
\end{pmatrix}
\end{math}
\end{center}
\pause
\item Обычно матрица перспективной проекции выглядит не так
\end{itemize}
\end{frame}

\begin{frame}[fragile]
\frametitle{Наложение объектов}
\begin{itemize}
\item По умолчанию примитивы, рисующиеся позже, накладываются поверх уже нарисованных
\pause
\item Обычно не проблема в 2D рендеринге: рисуем слои в нужном порядке
\pause
\begin{itemize}
\item 2D игра-платформер: фон, потом ландшафт, потом персонажи, потом эффекты
\pause
\item Карта: ландшафт, потом дороги и здания, потом иконки заведений
\pause
\item UI: фон виджета, потом рисунок кнопки, потом текст кнопки
\end{itemize}
\pause
\item В 3D можно сортировать треугольники в порядке убывания расстояния до камеры (painter's algorithm), но это очень медленно:
\pause
\begin{itemize}
\item Нужно отсортировать миллионы треугольников каждый кадр
\pause
\item Нужно загрузить их на GPU
\pause
\item Треугольники одного объекта не идут непрерывно \begin{math}\Rightarrow\end{math} больше изменений состояния (переключений шейдеров, etc) при рендеринге
\end{itemize}
\end{itemize}
\end{frame}

\begin{frame}[fragile]
\frametitle{Z-буфер}
\begin{itemize}
\item Решение: буфер глубины (z-buffer)
\pause
\item Хранит для каждого пикселя экрана расстояние до камеры
\pause
\item Тест глубины: при рисовании очередного пикселя (после фрагментного шейдера) проверим: если уже нарисованный пиксель ближе к камере, то новый рисовать не будем
\pause
\item Включить: \verb|glEnable(GL_DEPTH_TEST)|
\pause
\item Настроить поведение теста глубины: \verb|glDepthFunc(GL_LEQUAL)|
\begin{itemize}
\item \verb|LEQUAL| = less or equal: пиксель рисуется, если его расстояние до камеры меньше или равно расстояния, записанного в z-буфер
\end{itemize}
\end{itemize}
\end{frame}

\begin{frame}[fragile]
\frametitle{Z-буфер}
\begin{itemize}
\item К координате Z после perspective divide применяется преобразование \begin{math}[-1, 1] \mapsto [0, 1]\end{math} (т.е. \begin{math}z \mapsto \frac{z+1}{2}\end{math}), и это значение записывается в z-буфер
\pause
\begin{itemize}
\item Доступно во фрагментном шейдере: \verb|gl_FragDepth|
\pause
\item Преобразование можно настроить с помощью \verb|glDepthRange|
\end{itemize}
\pause
\item Z-буфер хранит или в формате unsigned normalized (16/24/32-bit), или floating-point (32-bit)
\begin{itemize}
\item Обычно unsigned normalized 24-bit
\end{itemize}
\pause
\item Нужно очищать перед каждым кадром: \verb|glClear(GL_DEPTH_BUFFER_BIT)|
\pause
\item Default framebufer (тот, что используется по умолчанию, для рисования в окно) может иметь свой z-буфер - это нужно настраивать перед созданием контекста
\end{itemize}
\end{frame}

\begin{frame}[fragile]
\frametitle{Depth clipping/clamping}
\begin{itemize}
\item Z-координата (после perspective divide) должна быть в диапазоне \begin{math}[-1, 1]\end{math}
\pause
\item Что делать в вершинами, выпадающими за диапазон? Два варианта:
\pause
\begin{itemize}
\item Depth clipping (отсечение по глубине): примитив (точка/линия/треугольник) обрезается по плоскостям \begin{math}Z = \pm 1\end{math} (как будто вне области \begin{math}|Z|\leq 1\end{math} ничего нет)
\pause
\item Depth clamping: отсечения не происходит, но после растеризации глубина всех пикселей сжимается до диапазона \begin{math}[-1, 1]\end{math}
\pause
\item По умолчанию - depth clipping
\item Включить depth clamping: \verb|glEnable(GL_DEPTH_CLAMP)|
\end{itemize}
\end{itemize}
\end{frame}

\begin{frame}[fragile]
\frametitle{Clipping}
\begin{itemize}
\item На самом деле, с координатами X и Y тоже происходит отсечение
\pause
\item Как будто пиксели, не попадающие на экран, просто игнорируются
\pause
\item Не настраивается
\pause

\begin{center}
\begin{math}
\begin{matrix}
-1 \leq X \leq 1 \\
-1 \leq Y \leq 1 \\
-1 \leq Z \leq 1
\end{matrix}
\end{math}
\end{center}

\pause
\item С учётом perspective divide (если \begin{math}W>0\end{math}):

\begin{center}
\begin{math}
\begin{matrix}
-1 \leq X/W \leq 1 \\
-1 \leq Y/W \leq 1 \\
-1 \leq Z/W \leq 1
\end{matrix}
\Leftrightarrow
\color{red}
\begin{matrix}
-W \leq X \leq W \\
-W \leq Y \leq W \\
-W \leq Z \leq W
\end{matrix}
\end{math}
\end{center}

\pause
\item Если \begin{math}W<0\end{math}, точка не попадает на экран
\end{itemize}

\end{frame}

\begin{frame}[fragile]
\frametitle{Полный путь координат}
\begin{itemize}
\item Входные данные: точка \begin{math}q_{object}\end{math} в системе координат объекта (object space)
\pause
\item Позиционируем объект: \begin{math}q_{world} = \operatorname{Transform} \cdot q_{object}\end{math} в мировых координатах (world space)
\pause
\item Учитываем расположение камеры: \begin{math}q_{camera} = \operatorname{View} \cdot q_{world}\end{math} в координатах камеры (view/camera space)
\pause
\item Проекция камеры: \verb|gl_Position = |\begin{math}q_{clip} = \operatorname{Projection} \cdot q_{camera}\end{math} в clip space
\pause
\item Perspective divide: \begin{math}q_{ndc} = xyz_{clip} / w_{clip}\end{math} в normalized device coordinates
\pause
\begin{itemize}
\item \begin{math}x_{ndc}\end{math} и \begin{math}y_{ndc}\end{math} определяют координату на экране
\item \begin{math}z_{ndc}\end{math} определяет глубину
\end{itemize}
\end{itemize}
\end{frame}

\begin{frame}[fragile]
\frametitle{Перспективная проекция: near \& far clip planes}
\begin{itemize}
\item Если записывать в w координату z, то после perspective divide z=1, т.е. у всех точек одинаковая глубина \begin{math}\Rightarrow\end{math} нам нужна другая формула проекции!
\pause
\item Итоговый диапазон глубин ограничен \begin{math}\Rightarrow\end{math} нужно как-то ограничить диапазон возможных расстояний до камеры
\pause
\item \begin{math}0 < near < far < \infty\end{math}
\pause
\item Хотим, чтобы диапазон \begin{math}z\in [near,far]\end{math} после применения матрицы проекции и perspective divide превратился в \begin{math}[-1,1]\end{math}
\end{itemize}
\end{frame}

\begin{frame}[fragile]
\frametitle{Перспективная проекция: near \& far clip planes}
\begin{itemize}
\item Хотим, чтобы диапазон \begin{math}z\in [near,far]\end{math} после применения матрицы проекции и perspective divide превратился в \begin{math}[-1,1]\end{math}
\pause
\begin{center}
\begin{math}
\begin{matrix}
z \mapsto \frac{Az + B}{z} \\ \\
near \mapsto \frac{A\cdot near + B}{near} = -1 \\ \\
far \mapsto \frac{A\cdot far + B}{far} = 1
\end{matrix}
\end{math}
\end{center}
\pause
\item Линейная система на A и B:
\begin{center}
\begin{math}
\begin{matrix}
A \cdot near + B = -near \\ \\
A \cdot far + B = far
\end{matrix}
\Rightarrow
\begin{matrix}
A = \frac{far + near}{far - near} \\ \\
B = \frac{-2 far \cdot near}{far - near}
\end{matrix}
\end{math}
\end{center}
\end{itemize}
\end{frame}

\begin{frame}[fragile]
\frametitle{Перспективная проекция: матрица проекции}
\begin{center}
\begin{math}
\begin{pmatrix}
1 & 0 & 0 & 0 \\
0 & 1 & 0 & 0 \\
0 & 0 & A & B \\
0 & 0 & 1 & 0
\end{pmatrix}
=
\begin{pmatrix}
1 & 0 & 0 & 0 \\
0 & 1 & 0 & 0 \\
0 & 0 & \frac{far + near}{far - near} & -\frac{2 far \cdot near}{far - near} \\
0 & 0 & 1 & 0
\end{pmatrix}
\end{math}
\end{center}
\end{frame}

\begin{frame}[fragile]
\frametitle{Перспективная проекция: правая система координат}
\begin{itemize}
\item Обычно направление вгзляда камеры выбирают не в сторону положительной оси Z, а в сторону отрицательной оси Z, чтобы получилась правая система координат
\pause
\item Отсечение по \begin{math}z=-near\end{math} и \begin{math}z=-far\end{math}
\pause
\item Расстояние до камеры - не Z, а -Z
\end{itemize}
\pause
\begin{center}
\begin{math}
\begin{matrix}
\frac{A(-near)+B}{near} = -1 \\ \\
\frac{A(-far)+B}{far} = 1
\end{matrix}
\end{math}
\end{center}
\pause
\begin{center}
\begin{math}
\begin{pmatrix}
1 & 0 & 0 & 0 \\
0 & 1 & 0 & 0 \\
0 & 0 & A & B \\
0 & 0 & -1 & 0
\end{pmatrix}
=
\begin{pmatrix}
1 & 0 & 0 & 0 \\
0 & 1 & 0 & 0 \\
0 & 0 & -\frac{far + near}{far - near} & -\frac{2 far \cdot near}{far - near} \\
0 & 0 & -1 & 0
\end{pmatrix}
\end{math}
\end{center}
\end{frame}

\begin{frame}[fragile]
\frametitle{Перспективная проекция: view frustum}
Область, видимая перспективной камерой - усечённая пирамида (frustum)
\slideimage{frustum.png}
\end{frame}

\begin{frame}[fragile]
\frametitle{Перспективная проекция: view frustum}
Как управлять шириной и высотой видимой области?
\pause
\begin{center}
\begin{math}
\begin{pmatrix}
\only<2>{1}\only<3->{C} & 0 & \only<2>{0}\only<3->{D} & 0 \\
0 & \only<2>{1}\only<3->{E} & \only<2>{0}\only<3->{F} & 0 \\
0 & 0 & A & B \\
0 & 0 & -1 & 0
\end{pmatrix}
\end{math}
\end{center}
\end{frame}

\begin{frame}[fragile]
\frametitle{Перспективная проекция: view frustum}
Как управлять шириной и высотой видимой области?
\begin{tikzpicture}
\draw[black,dashed,-stealth] (-2.0, 0.0) -- (8.0, 0.0);
\draw[black,dashed,stealth-] (0.0, -3.0) -- (0.0, 3.0);
\draw[blue,thick] (3.0, -2.0) -- (3.0, 2.0);
\draw[black,thick] (0.0, 0.0) -- (4.5, -3.0);
\draw[black,thick] (0.0, 0.0) -- (4.5, 3.0);
\draw[black,dashed] (0.0, 2.0) -- (4.0, 2.0);
\draw[black,dashed] (0.0, -2.0) -- (4.0, -2.0);
\node at (0.0, 0.0) {\textbullet};

\node at (8.0, -0.25) {Z};
\node at (-0.25, -3.0) {X};
\node at (3.0, -2.5) {\color{blue}z=near};
\node at (4.5, 2.0) {x=left};
\node at (4.5, -2.0) {x=right};
\end{tikzpicture}
\end{frame}

\begin{frame}[fragile]
\frametitle{Перспективная проекция: view frustum}
Как управлять шириной и высотой видимой области?
\begin{itemize}
\item Обычно left=-right
\pause
\item \begin{math}\frac{-left}{near}\end{math} - тангенс угла между осью (Z) проекции и левой границей видимой области
\item \begin{math}\frac{right}{near}\end{math} - тангенс угла между осью (Z) проекции и правой границей видимой области
\pause
\item Аналогично top и bottom для Y
\pause
\item Хотим, чтобы точка с координатами X=left и Z=-near перешла (после применения матрица проекции и perspective divide) в точку с X=-1 и Z=-1
\item Хотим, чтобы точка с координатами X=right и Z=-near перешла (после применения матрица проекции и perspective divide) в точку с X=1 и Z=-1
\end{itemize}
\end{frame}

\begin{frame}[fragile]
\frametitle{Перспективная проекция: view frustum}
\begin{center}
\begin{math}
\begin{matrix}
\frac{C\cdot left + D(-near)}{near} = -1 \\ \\
\frac{C\cdot right + D(-near)}{near} = 1
\end{matrix}
\Rightarrow
\begin{matrix}
C = \frac{2\cdot near}{right - left} \\ \\
D = \frac{right + left}{right - left}
\end{matrix}
\end{math}
\end{center}
\pause
Аналогично для Y
\begin{center}
\begin{math}
\begin{matrix}
\frac{E\cdot bottom + F(-near)}{near} = -1 \\ \\
\frac{E\cdot top + F(-near)}{near} = 1
\end{matrix}
\Rightarrow
\begin{matrix}
E = \frac{2\cdot near}{top - bottom} \\ \\
F = \frac{top + bottom}{top - bottom}
\end{matrix}
\end{math}
\end{center}
\end{frame}

\begin{frame}[fragile]
\frametitle{Перспективная проекция: матрица проекции}
\begin{center}
\begin{math}
\begin{pmatrix}
\frac{2\cdot near}{right - left} & 0 & \frac{right + left}{right - left} & 0 \\
0 & \frac{2\cdot near}{top - bottom} & \frac{top + bottom}{top - bottom} & 0 \\
0 & 0 & -\frac{far + near}{far - near} & -\frac{2 far \cdot near}{far - near} \\
0 & 0 & -1 & 0
\end{pmatrix}
\end{math}
\end{center}

\pause
Часто left=-right и bottom=-top, тогда

\begin{center}
\begin{math}
\begin{pmatrix}
\frac{near}{right} & 0 & 0 & 0 \\
0 & \frac{near}{top} & 0 & 0 \\
0 & 0 & -\frac{far + near}{far - near} & -\frac{2 far \cdot near}{far - near} \\
0 & 0 & -1 & 0
\end{pmatrix}
\end{math}
\end{center}
Это самый часто встречающийся вид матрицы перспективной проекции
\end{frame}

\begin{frame}[fragile]
\frametitle{Перспективная проекция}
\begin{itemize}
\item Как выбирать параметры камеры?
\pause
\item Пусть хотим камеру с шириной обзора по X равной \begin{math}\theta\end{math} радиан, и с aspect ratio (отношение ширины к высоте) равным \begin{math}R\end{math}
\pause
\item \begin{math}-left=right=near \cdot \tan\left(\frac{\theta}{2}\right)\end{math}
\item \begin{math}-bottom=top=\frac{1}{R} \cdot right = \frac{near}{R} \cdot \tan\left(\frac{\theta}{2}\right)\end{math}
\end{itemize}
\end{frame}

\begin{frame}[fragile]
\frametitle{Перспективная проекция}
\begin{itemize}
\item Как выбирать параметры near и far?
\pause
\item Всё ближе near или дальше far обрезается - почему не взять near\begin{math}\approx 0\end{math} и far=\begin{math}\infty\end{math}?
\pause
\item Если near = 0, третья строка матрицы проекции вырождается в \begin{math}\begin{pmatrix}0 & 0 & -1 & 0\end{pmatrix}\end{math}, и у всех точек будет одинаковая глубина, т.е. не получится использовать z-буфер
\pause
\item Если far \begin{math}\rightarrow\infty\end{math}, третья строка стремится к \begin{math}\begin{pmatrix}0 & 0 & -1 & -2near\end{pmatrix}\end{math}
\end{itemize}
\end{frame}

\begin{frame}[fragile]
\frametitle{Перспективная проекция}
\begin{itemize}
\item Z-буфер имеет ограниченную точность
\pause
\item Чем больше отношение \begin{math}\frac{far}{near}\end{math}, тем меньше точности приходится на единицу расстояния
\pause
\item Как выбирать параметры near и far?
\pause
\item near: максимально возможный, при котором не обрезается что-то существенное
\begin{itemize}
\item При этом камера искуственно держится на некотором расстоянии от любых объектов
\end{itemize}
\pause
\item far: максимально возможный, при котором не возникает Z-fighting'а
\end{itemize}
\end{frame}

\begin{frame}[fragile]
\frametitle{Z-fighting}
Ситуация, когда две очень близких почти (или полностью) параллельных плоскости рисуются с артефактом из-за недостаточной точности Z-буфера:
\slideimage{z-fighting.png}
\end{frame}

\begin{frame}[fragile]
\frametitle{Перспективная проекция: матрица проекции}
\begin{itemize}
\item Что, если мы хотим переместить и повернуть камеру?
\pause
\item Так же, как с ортографической проекцией: сначала применим обратное преобразование, а потом матрицу проекции
\pause
\begin{center}
\begin{math}
\begin{pmatrix}
C & 0 & D & 0 \\
0 & E & F & 0 \\
0 & 0 & A & B \\
0 & 0 & -1 & 0
\end{pmatrix}
\cdot
\begin{pmatrix}
X_x & Y_x & Z_x & C_x \\
X_y & Y_y & Z_y & C_y \\
X_z & Y_z & Z_z & C_z \\
0 & 0 & 0 & 1
\end{pmatrix}^{-1}
\end{math}
\end{center}
\pause
\item Обычно именно эту матрицу передают в шейдер
\end{itemize}
\end{frame}

\begin{frame}[fragile]
\frametitle{Проекции: ссылки}
\begin{itemize}
\item \href{http://songho.ca/opengl/gl_transform.html}{songho.ca/opengl/gl\_transform.html}
\item \href{http://songho.ca/opengl/gl_projectionmatrix.html}{songho.ca/opengl/gl\_projectionmatrix.html}
\end{itemize}
\end{frame}

\begin{frame}[fragile]
\frametitle{Бонус: как перевести их экранных координат в мировые?}
\begin{itemize}
\item Есть пиксель на экране с координатами \begin{math}(P_X, P_Y) \in [-1, 1]^2\end{math} (например, координаты указателя мыши) и матрица \begin{math}T\end{math} камеры (view + projection)
\pause
\item В пространстве ему соответствует луч из камеры - как его найти?
\pause
\item Найдём две точки, пересекающие этот луч, на near и far плоскостях
\pause
\item Пусть \begin{math}\operatorname{Proj}\end{math} - оператор, выполняющий perspective divide:
\begin{center}
\begin{math}
\operatorname{Proj}\begin{pmatrix}x \\ y \\ z \\ w\end{pmatrix} = \begin{pmatrix}x/w \\ y/w \\ z/w\end{pmatrix}
\end{math}
\end{center}
\end{itemize}
\end{frame}

\begin{frame}[fragile]
\frametitle{Бонус: как перевести их экранных координат в мировые?}
\begin{itemize}
\item Мы хотим решить уравнение
\begin{center}
\begin{math}
\operatorname{Proj}\cdot T \begin{pmatrix}x \\ y \\ z \\ 1\end{pmatrix} = \begin{pmatrix}M_X \\ M_Y \\ \pm 1\end{pmatrix}
\end{math}
\end{center}
\pause
\item Как выглядит множество \begin{math}\operatorname{Proj}^{-1} \begin{pmatrix}x \\ y \\ z\end{pmatrix}\end{math}\only<2>{?} \pause \begin{math}=\begin{pmatrix}\lambda x \\ \lambda y \\ \lambda z \\ \lambda \end{pmatrix} \forall \lambda  \neq 0\end{math}
\pause
\begin{center}
\begin{math}
\Rightarrow T \begin{pmatrix}x \\ y \\ z \\ 1\end{pmatrix} = \begin{pmatrix}\lambda M_X \\ \lambda M_Y \\ \pm \lambda \\ \lambda\end{pmatrix}
\end{math}
, но мы не знаем \begin{math}\lambda\end{math}
\end{center}
\end{itemize}
\end{frame}

\begin{frame}[fragile]
\frametitle{Бонус: как перевести их экранных координат в мировые?}
\begin{center}
\begin{math}
\begin{pmatrix}x \\ y \\ z \\ 1\end{pmatrix} = T^{-1} \begin{pmatrix}\lambda M_X \\ \lambda M_Y \\ \pm \lambda \\ \lambda\end{pmatrix}
\end{math}
\end{center}
\pause
\begin{itemize}
\item С чего мы взяли, что последняя координата будет равна 1? \pause Воспользуемся \begin{math}\lambda\end{math}:
\begin{center}
\begin{math}
\begin{pmatrix}x/\lambda \\ y/\lambda \\ z/\lambda \\ 1/\lambda \end{pmatrix} =
T^{-1} \begin{pmatrix}M_X \\ M_Y \\ \pm 1 \\ 1\end{pmatrix}
\end{math}
\end{center}
\end{itemize}
\end{frame}

\begin{frame}[fragile]
\frametitle{Бонус: как перевести их экранных координат в мировые?}
\begin{center}
\begin{math}
\begin{pmatrix}x/\lambda \\ y/\lambda \\ z/\lambda \\ 1/\lambda \end{pmatrix} =
T^{-1} \begin{pmatrix}M_X \\ M_Y \\ \pm 1 \\ 1\end{pmatrix}
\end{math}
\end{center}
\pause
\begin{center}
\begin{math}
\begin{pmatrix}x \\ y \\ z\end{pmatrix} =
\operatorname{Proj} \begin{pmatrix}x/\lambda \\ y/\lambda \\ z/\lambda \\ 1/\lambda \end{pmatrix} =
\operatorname{Proj} T^{-1} \begin{pmatrix}M_X \\ M_Y \\ \pm 1 \\ 1\end{pmatrix}
\end{math}
\end{center}
\end{frame}

\end{document}