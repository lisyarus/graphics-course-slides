% (c) Nikita Lisitsa, lisyarus@gmail.com, 2021

\documentclass{beamer}

\usepackage[T2A]{fontenc}
\usepackage[utf8]{inputenc}
\usepackage[russian]{babel}

\usepackage{graphicx}
\graphicspath{ {./images/} }

\usepackage{adjustbox}

\usepackage{color}
\usepackage{soul}

\usepackage{hyperref}

\definecolor{blue}{rgb}{0,0,1}
\definecolor{red}{rgb}{1,0,0}

\makeatletter
\newcommand{\slideimage}[1]{
  \begin{figure}
    \begin{adjustbox}{width=\textwidth, totalheight=\textheight-2\baselineskip-2\baselineskip,keepaspectratio}
      \includegraphics{#1}
    \end{adjustbox}
  \end{figure}
}
\makeatother

\title{Компьютерная графика}
\subtitle{Практика 3: VAO, VBO, кривые Безье}
\date{2021}

\setbeamertemplate{footline}[frame number]

\begin{document}

\frame{\titlepage}

\begin{frame}[fragile]
\frametitle{Кривые Безье}
\begin{itemize}
\item Используются для генерации плавных кривых
\item Пусть даны точки \begin{math}p_0, p_1, \dots, p_n\end{math}
\item Кривая Безье с параметром \begin{math}t \in [0, 1]\end{math} определяется как аффинная комбинация

\begin{math}
b(t) = \sum\limits_{i=0}^n b_{i,n}(t) \cdot p_i
\end{math}

\item \begin{math}b_{i,n}(t)\end{math} - полиномы Бернштейна
\pause
\item Кривая первого порядка (\begin{math}n=1\end{math}): отрезок \begin{math}p_0 p_1\end{math}

\begin{math}b(t) = (1-t)\cdot p_0 + t \cdot p_1\end{math}
\pause
\item Кривая второго порядка (\begin{math}n=2\end{math}):

\begin{math}b(t) = (1-t)^2 \cdot p_0 + 2t(1-t) \cdot p_1 + t^2 \cdot p_2\end{math}
\end{itemize}
\end{frame}

\begin{frame}[fragile]
\frametitle{Задание 1}
Загружаем данные в VBO
\begin{itemize}
\item Создайте и заполните массив из трёх вершин типа \verb|vertex|
\pause
\item Создайте VBO и загрузите в него данные: \verb|glGenBuffers|, \verb|glBindBuffer|, \verb|glBufferData|
\pause
\item Проверьте, что данные загрузились, создав временную переменную, и считав в неё координату какой-нибудь вершины (\verb|glGetBufferSubData|) и выведя результат в \verb|std::cout|
\end{itemize}
\end{frame}

\begin{frame}[fragile]
\frametitle{Задание 2}
Рисуем с помощью VAO
\begin{itemize}
\item Создайте VAO и настройте атрибуты вершин: \verb|glGenVertexArrays|, \verb|glBindVertexArray|, \verb|glEnableVertexAttribArray|, \verb|glVertexAttribPointer|
\pause
\item Нарисуйте треугольник с помощью этого VAO: \verb|glDrawArrays|
\end{itemize}
\end{frame}

\begin{frame}[fragile]
\frametitle{Задание 3}
Переходим к оконным координатам
\begin{itemize}
\item Заполните view-матрицу преобразованием, которое переводит X из \verb|[0, width]| в \verb|[-1, 1]|, и Y из \verb|[height, 0]| в \verb|[-1, 1]|
\pause
\item Измените координаты треугольника, чтобы он был заметен на экране
\end{itemize}
\end{frame}

\begin{frame}[fragile]
\frametitle{Задание 4}
Динамически добавляем/удаляем точки мышкой
\begin{itemize}
\item Замените статический массив с вершинами на контейнер \verb|std::vector| (изначально пустой)
\pause
\item При нажатии левой кнопки мыши (\verb|SDL_BUTTON_LEFT|) добавьте новую вершину в контейнер с координатами мыши (цвета выбирайте как угодно)
\pause
\item При нажатии правой кнопки мыши (\verb|SDL_BUTTON_RIGHT|) удалите последнюю вершину из контейнера, если он не пустой
\pause
\item Если контейнер с вершинами изменился, обновите данные в соответствующем VBO
\pause
\item Рисуем линию из всех точек: \verb|GL_LINE_STRIP|
\end{itemize}
\end{frame}

\begin{frame}[fragile]
\frametitle{Задание 5}
Нарисуем сами точки
\begin{itemize}
\item \verb|glPointSize(10)| чтобы точки были заметны
\item Ещё один вызов \verb|glDrawArrays| чтобы нарисовать точки (\verb|GL_POINTS|)
\end{itemize}
\end{frame}

\begin{frame}[fragile]
\frametitle{Задание 6}
Генерируем и рисуем кривые Безье
\begin{itemize}
\item Создаёте ещё один VBO, \verb|std::vector| и VAO для точек кривой Безье (не забудьте настроить атрибуты в VAO)
\pause
\item Заведите параметр, управляющий уровнем детализации кривой: \verb|int quality = 4;|
\pause
\item При добавлении/удалении точек пересчитываем заново всю кривую Безье
\item Если в исходной ломаной \verb|N| отрезков, то в кривой Безье должно быть \verb|N * quality| отрезков
\item Параметр \verb|t| должен равномерно меняться от 0 до 1 вдоль всей кривой
\item Цвет - любой, но отличающийся от цвета исходной ломаной
\item Данные в VBO должны обновляться только при их изменении
\pause
\item Рисуем кривую, используя ту же шейдерную программу и новый VAO
\end{itemize}
\end{frame}

\begin{frame}[fragile]
\frametitle{Задание 7}
Динамически меняем уровень детализации
\begin{itemize}
\item При нажатии на стрелку влево (\verb|SDL_LEFT|) уровень детализации \verb|quality| должен уменьшаться (но не может быть меньше 1)
\pause
\item При нажатии на стрелку вправо (\verb|SDL_RIGHT|) уровень детализации \verb|quality| должен увеличиваться
\end{itemize}
\end{frame}

\begin{frame}[fragile]
\frametitle{Задание 8*}
\fontsize{8pt}{8pt}\selectfont
Добавляем ползающий пунктир к кривой Безье
\begin{itemize}
\item Пунктира можно добиться, не рисуя половину пикселей кривой в зависимости от расстояния до начала
\pause
\item Например, пиксели кривой на расстоянии \verb|[0, 20]| от начала рисуются, на расстоянии \verb|[20, 40]| от начала не рисуются, \verb|[40, 60]| рисуются, и т.д.
\pause
\item Для этого нужно добавить в вершины кривой расстояние вдоль кривой от её начала
\item Расстояние до первой вершины = 0
\item Расстояние до  вершины \verb|i| = расстояние до  вершины \verb|(i-1)| + расстояние от вершины \verb|(i-1)| до вершины \verb|i|
\item Для расстояния между вершинами пригодится функция C++ \verb|std::hypot|
\pause
\item Расстояние нужно интерполировать между вершинами и передать во фрагментный шейдер
\item Чтобы понять, нужно ли рисовать пиксель, пригодится функция GLSL \verb|fmod| (например, \verb|fmod(distance, 40.0) < 20.0)|
\item Чтобы отменить рисование пикселя, нужно в функции \verb|main| фрагментного шейдера вызвать команду \verb|discard;|
\pause
\item Само расстояние нужно добавить как поле вершины и как атрибут в VAO и вершинный шейдер
\pause
\item Чтобы пунктир двигался, можно прибавить к расстоянию что-то, зависящее от времени (придётся передать время в шейдер как \verb|uniform|)
\pause
\item Два варианта организации шейдеров:
\begin{itemize}
\item Два отдельных шейдера: один для ломаной, один для пунктирной кривой Безье
\item Один общий шейдер: \verb|uniform|-настройка, нужно ли рисовать пунктир (например, \verb|uniform int dash;|, \verb|dash == 1| означает рисовать пунктир)
\end{itemize}
\end{itemize}
\end{frame}

\end{document}