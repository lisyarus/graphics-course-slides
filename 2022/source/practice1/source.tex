% (c) Nikita Lisitsa, lisyarus@gmail.com, 2021

\documentclass{beamer}

\usepackage[T2A]{fontenc}
\usepackage[utf8]{inputenc}
\usepackage[russian]{babel}

\usepackage{graphicx}
\graphicspath{ {./images/} }

\usepackage{adjustbox}

\usepackage{color}
\usepackage{soul}

\usepackage{hyperref}

\definecolor{blue}{rgb}{0,0,1}
\definecolor{red}{rgb}{1,0,0}

\makeatletter
\newcommand{\slideimage}[1]{
  \begin{figure}
    \begin{adjustbox}{width=\textwidth, totalheight=\textheight-2\baselineskip-2\baselineskip,keepaspectratio}
      \includegraphics{#1}
    \end{adjustbox}
  \end{figure}
}
\makeatother

\title{Компьютерная графика}
\subtitle{Практика 1: Рисуем треугольник}
\date{2021}

\setbeamertemplate{footline}[frame number]

\begin{document}

\frame{\titlepage}

\begin{frame}[fragile]
\frametitle{Задание 1}
\fontsize{8pt}{8pt}\selectfont
\begin{itemize}
\item Пишем вспомогательную функцию для создания шейдера
\begin{verbatim}
GLuint create_shader(GLenum shader_type,
    const char * shader_source)
\end{verbatim}
\pause
\item Нужно использовать:
\begin{itemize}
\item \verb|glCreateShader| -- создаёт и возвращает ID шейдера
\item \verb|glShaderSource| -- задаёт исходный код шейдера (функция умеет принимать массив строк, но у нас она будет одна; в качестве длины можно указать \verb|NULL|)
\item \verb|glCompileShader| -- компилирует шейдер
\end{itemize}
\pause
\item Функция должна возвращать ID созданного шейдера (то, что вернула \verb|glCreateShader|)
\pause
\item Функция должна бросать исключение (например, \verb|std::runtime_error|), если компиляция шейдера провалилась
\begin{itemize}
\item \verb|glGetShaderiv| -- позволяет узнать параметры шейдера, в т.ч. статус компиляции (\verb|GL_COMPILE_STATUS|)
\end{itemize}
\pause
\item Заведите строковую константу с любым значением, создайте фрагментный шейдер (\verb|shader_type = GL_FRAGMENT_SHADER|) с помощью вашей функции и убедитесь, что бросается исключение
\item Это нужно делать где-то после \verb|glewInit()| и до начала основного цикла \verb|while (running)|
\end{itemize}
\end{frame}

\begin{frame}[fragile]
\frametitle{Задание 2}
\begin{itemize}
\item Бросаемое исключение должно содержать текст ошибки компиляции шейдера
\pause
\begin{itemize}
\item \verb|glGetShaderiv| -- позволяет узнать длину лога компиляции (\verb|GL_INFO_LOG_LENGTH|)
\item \verb|glGetShaderInfoLog| -- позволяет получить сам текст компиляции (записывает его в указанное вами место в памяти!)
\item Можно завести статический массив \verb|char info_log[1024];| и записать туда лог
\item Можно сначала узнать длину с помощью \verb|glGetShaderiv|, создать переменную \verb|std::string info_log(info_log_length, '\0');|, и записать лог в неё (через \verb|info_log.data()|)
\end{itemize}
\end{itemize}
\end{frame}

\begin{frame}[fragile]
\frametitle{Задание 3}
\begin{itemize}
\item Пишем настоящий фрагментный (пиксельный) шейдер
\pause
\item Удобно использовать raw string literal из C++11:
\begin{verbatim}
const char fragment_source[] = R"(#version 330 core

layout (location = 0) out vec4 out_color;

void main()
{
    // vec4(R, G, B, A)
    out_color = vec4(1.0, 0.0, 0.0, 1.0);
}
)";
\end{verbatim}
\pause
\item (цвет можно выбрать любой)
\end{itemize}
\end{frame}

\begin{frame}[fragile]
\frametitle{Задание 4}
\fontsize{8pt}{8pt}\selectfont
\begin{itemize}
\item Пишем вершинный шейдер
\pause
\begin{verbatim}
R"(#version 330 core

const vec2 VERTICES[3] = vec2[3](
    vec2(0.0, 0.0),
    vec2(1.0, 0.0),
    vec2(0.0, 1.0)
);

void main()
{
    gl_Position = vec4(VERTICES[gl_VertexID], 0.0, 1.0);
}
)";
\end{verbatim}
\pause
\item Создаём вершинный шейдер (\verb|shader_type = GL_VERTEX_SHADER|) с этим кодом
\pause
\item Теперь у нас создано два шейдера: вершинный и фрагментный, каждый со своим кодом
\end{itemize}
\end{frame}

\begin{frame}[fragile]
\frametitle{Задание 5}
\fontsize{8pt}{8pt}\selectfont
\begin{itemize}
\item Пишем функцию создания шейдерной программы
\begin{verbatim}
GLuint create_program(GLuint vertex_shader,
    GLuint fragment_shader)
\end{verbatim}
\pause
\item Программа = несколько скомпилированных шейдеров, слинкованных вместе
\item Нужно использовать:
\begin{itemize}
\item \verb|glCreateProgram| -- создаёт программу и возвращает её ID
\item \verb|glAttachShader| -- присоединяет шейдер к программе
\item \verb|glLinkProgram| -- линкует шейдеры (собирает программу из присоединённых шейдеров)
\item \verb|glGetProgramiv| -- 2 раза: получить статус (\verb|GL_LINK_STATUS|) и длину лога линковки (\verb|GL_INFO_LOG_LENGTH|)
\item \verb|glGetProgramInfoLog| -- получить текст лога линковки, если линковка не удалась
\end{itemize}
\pause
\item Функция возвращает ID созданной программы (результат \verb|glCreateProgram|)
\pause
\item Аналогично \verb|create_shader|, функция должна бросать исключение с текстом ошибки, если линковка программы не удалась
\pause
\item Вызываем созданную функцию, чтобы создать программу, используя два созданных ранее шейдера
\end{itemize}
\end{frame}

\begin{frame}[fragile]
\frametitle{Задание 6}
\begin{itemize}
\item Создаём Vertex Array Object
\item VAO содержит информацию о входных данных - расположение данных о вершинах, типы атттрибутов
\item В нашем случае нужен только номинально, но без него ничего не будет рисоваться
\pause
\begin{itemize}
\item \verb|glGenVertexArrays| -- создаёт Vertex Array (функция умеет создавать сразу несколько VAO, поэтому принимает количество и указатель на массив; нам хватит одного VAO)
\end{itemize}
\pause
\item Рисуем треугольник, используя созданные программу и VAO: где-то в теле основного цикла, после \verb|glClear| и до \verb|SDL_GL_SwapBuffers|:
\begin{itemize}
\item \verb|glUseProgram| -- включаем использование созданной шейдерной программы
\item \verb|glBindVertexArray| -- включаем VAO
\item \verb|glDrawArrays| -- рисуем треугольники (\verb|GL_TRIANGLES|), номер стартовой вершины -- 0, количество вершин -- 3
\end{itemize}
\end{itemize}
\end{frame}

\begin{frame}[fragile]
\frametitle{Задание 6}
\slideimage{task_6.png}
\end{frame}

\begin{frame}[fragile]
\frametitle{Задание 7}
\fontsize{8pt}{8pt}\selectfont
\begin{itemize}
\item Добавляем градиентное закрашивание
\item Из вершинного шейдера во фрагментный можно передавать данные: они будут интерполироваться между вершинами
\end{itemize}
\pause
\begin{verbatim}
В вершинном шейдере:
out vec3 color;
void main()
{
    gl_Position = vec4(VERTICES[gl_VertexID], 0.0, 1.0);
    color = ...; // что-нибудь, зависящее от gl_VertexID 
                 // или от gl_Position
}

Во фрагментном шейдере:
in vec3 color;
void main()
{
    out_color = vec4(color, 1.0);
}
\end{verbatim}
\end{frame}

\begin{frame}[fragile]
\frametitle{Задание 7}
\slideimage{task_7.png}
\end{frame}

\begin{frame}[fragile]
\frametitle{Задание 8}
\begin{itemize}
\item Запрещаем интерполяцию переменных
\pause
\begin{verbatim}
flat out vec3 color;

flat in vec3 color;
\end{verbatim}
\pause
\item Будет использоваться значение в последней вершине
\item Можно настроить с помощью \verb|glProvokingVertex|
\end{itemize}
\end{frame}

\begin{frame}[fragile]
\frametitle{Задание 8}
\slideimage{task_8.png}
\end{frame}

\begin{frame}[fragile]
\frametitle{Задание 9*}
\begin{itemize}
\item Раскрашиваем треугольник в шахматном порядке
\pause
\item Делается во фрагментном шейдере
\pause
\item Нужно проинтерполировать какое-нибудь двумерное значение между вершинами
\pause
\item Функция \verb|floor| будет полезной
\end{itemize}
\end{frame}

\begin{frame}[fragile]
\frametitle{Задание 9}
\slideimage{task_9.png}
\end{frame}

\end{document}