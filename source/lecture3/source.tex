% (c) Nikita Lisitsa, lisyarus@gmail.com, 2023

\documentclass[10pt]{beamer}

\usepackage[T2A]{fontenc}
\usepackage[russian]{babel}
\usepackage{minted}

\usepackage{graphicx}
\graphicspath{ {./images/} }

\usepackage{tikz}
\usepackage{xifthen}

\usepackage{adjustbox}

\usepackage{color}
\usepackage{soul}

\usepackage{hyperref}

\usetheme{metropolis}

\definecolor{red}{rgb}{1,0,0}
\definecolor{green}{rgb}{0,0.5,0}
\definecolor{blue}{rgb}{0,0,1}
\definecolor{codebg}{RGB}{29,35,49}
\setminted{fontsize=\footnotesize}

\makeatletter
\newcommand{\slideimage}[1]{
  \begin{figure}
    \begin{adjustbox}{width=\textwidth, totalheight=\textheight-2\baselineskip-2\baselineskip,keepaspectratio}
      \includegraphics{#1}
    \end{adjustbox}
  \end{figure}
}
\makeatother

\title{Компьютерная графика}
\subtitle{Лекция 3: Объекты OpenGL, буферы, аттрибуты вершин, индексированный рендеринг}
\date{2023}

\setbeamertemplate{footline}[frame number]

\begin{document}

\frame{\titlepage}

\begin{frame}[fragile]
\frametitle{Объекты OpenGL}
\begin{itemize}
\item Основные строительные блоки, через которые мы работаем с OpenGL
\pause
\item Хранят настройки рендеринга, запоминают состояние
\pause
\item Принадлежат текущему контексту OpenGL
\pause
\begin{itemize}
\item Часть объектов можно делить между разными контекстами, см. \textit{OpenGL context sharing}
\end{itemize}
\end{itemize}
\end{frame}

\begin{frame}[fragile]
\frametitle{Примеры объектов OpenGL}
\begin{itemize}
\item \textbf{Шейдеры, шейдерные программы} -- программируемая часть конвейера
\pause
\item \textbf{Буферы} -- хранят данные на GPU
\pause
\item \textbf{Vertex Array} -- описывают формат и хранение вершин
\pause
\item \textbf{Текстуры} -- изображения в памяти GPU, которые можно читать из шейдера, и в которые можно рисовать
\pause
\item \textbf{Renderbuffer} -- буферы, в которые можно рисовать
\pause
\item \textbf{Framebuffer} -- содержит настройки рисования в текстуры и renderbuffer'ы
\pause
\item И другие
\end{itemize}
\end{frame}

\begin{frame}[fragile]
\frametitle{Создание/удаление объектов OpenGL}
\usemintedstyle{solarized-light}
\begin{itemize}
\item В коде объект представляется идентификатором типа \mintinline{cpp}|GLuint|
\begin{itemize}
\item Id уникален среди объектов одного типа (шейдеры, программы, ...)
\end{itemize}
\pause
\item Шейдеры и программы:
\begin{itemize}
\item \mintinline{cpp}|glCreateShader () / glDeleteShader (shader)|
\item \mintinline{cpp}|glCreateProgram() / glDeleteProgram(program)|
\end{itemize}
\pause
\item Остальные объекты:
\begin{itemize}
\item \mintinline{cpp}|glGenBuffers(count, ptr) / glDeleteBuffers(count, ptr)|
\item \mintinline{cpp}|glGenVertexArrays        / glDeleteVertexArrays|
\item \mintinline{cpp}|glGenTextures            / glDeleteTextures|
\item \mintinline{cpp}|glGenRenderbuffers       / glDeleteRenderbuffers|
\item \mintinline{cpp}|glGenFramebuffers        / glDeleteFramebuffers|
\end{itemize}
\end{itemize}
\end{frame}

\begin{frame}[fragile]
\frametitle{Создание/удаление объектов OpenGL}
\usemintedstyle{solarized-light}
\begin{itemize}
\item Можно создать/удалить один объект:
\usemintedstyle{lightbulb}
\begin{minted}[bgcolor=codebg]{cpp}
GLuint texture;
glGenTextures(1, &texture);
...
glDeleteTexture(1, &texture);
\end{minted}
\end{itemize}
\end{frame}

\begin{frame}[fragile]
\frametitle{Объекты OpenGL}
\begin{itemize}
\item Подразумевается, что объекты переиспользуются по максимуму
\item Не нужно создавать новую текстуру каждый кадр -- создайте один раз и переиспользуйте её
\item Не нужно создавать новый буфер при каждом обновлении данных -- создайте один раз и переиспользуйте его
\end{itemize}
\end{frame}

\begin{frame}[fragile]
\frametitle{Объекты OpenGL с нулевым id}
\begin{itemize}
\item Как правило, объект с нулевым id считается несуществующим (как нулевой указатель)
\pause
\item Исключения:
\begin{itemize}
\item \textbf{Framebuffer} с нулевым id -- \textit{default framebuffer}, рисует в окно, привязанное к контексту OpenGL, его \textit{нельзя настраивать}
\pause
\item В \textbf{OpenGL ES}: \textbf{vertex array} с нулевым id -- ничем не отличается от других vertex array'ев, но существует (создан) по умолчанию
\end{itemize}
\pause
\item Функции создания объектов \textit{никогда} не возвращают нулевой id
\item Функции удаления объектов \textit{игнорируют} нулевой id (т.е. удалить нулевой объект -- не ошибка)
\pause
\item Нулевой id можно использовать как маркер отсутствия объекта
\end{itemize}
\end{frame}

\begin{frame}[fragile]
\frametitle{Удаление объектов}
\begin{itemize}
\item По-хорошему, объекты нужно удалять, когда они перестают быть нужны (все движки и серьёзные обёртки над OpenGL это делают)
\pause
\item На практике, они удаляются автоматически в конце работы программы (при удалении контекста OpenGL), и мы будем этим пользоваться в течение курса
\end{itemize}
\end{frame}

\begin{frame}[fragile]
\frametitle{Работа с объектами OpenGL}
\usemintedstyle{solarized-light}
\begin{itemize}
\item Почти всегда чтобы работать с объектом, нужно сделать его `текущим'
\begin{itemize}
\item Текущий объект запоминается в контексте OpenGL
\item Если вы работаете с одним контекстом, можно считать, что id текущего объекта -- глобальная константа
\end{itemize}
\pause
\item Некоторые функции не требуют выставления объекта текущим, а принимают объект напрямую: \mintinline{cpp}|glShaderSource|, \mintinline{cpp}|glCompileShader|, \mintinline{cpp}|glLinkProgram|, \mintinline{cpp}|glGetUniformLocation|, ...
\pause
\item Некоторые объекты нельзя сделать текущими: шейдеры
\end{itemize}
\end{frame}

\begin{frame}[fragile]
\frametitle{Текущий объект}
\usemintedstyle{solarized-light}
\begin{itemize}
\item Сделать программу текущей: \mintinline{cpp}|glUseProgram(program)|
\begin{itemize}
\item Функции \mintinline{cpp}|glUniform1f|, ... работают с \textit{текущей программой}
\end{itemize}
\pause
\item Сделать vertex array текущим: \mintinline{cpp}|glBindVertexArray(vao)|
\begin{itemize}
\item Функции работы с vertex array используют \textit{текущий vertex array}
\end{itemize}
\pause
\item Функции рисования (\mintinline{cpp}|glDrawArrays| и др.) используют \textit{текущую шейдерную программу} и \textit{текущий vertex array}
\end{itemize}
\end{frame}

\begin{frame}[fragile]
\frametitle{Текущий объект}
\usemintedstyle{solarized-light}
\begin{itemize}
\item Сделать текущим буфер, текстуру, и т.д.:
\begin{itemize}
\item \mintinline{cpp}|glBindBuffer      (target, id)|
\item \mintinline{cpp}|glBindTexture     (target, id)|
\item \mintinline{cpp}|glBindRenderbuffer(target, id)|
\item \mintinline{cpp}|glBindFramebuffer (target, id)|
\end{itemize}
\pause
\item Что такое \mintinline{cpp}|target|?
\end{itemize}
\end{frame}

\begin{frame}[fragile]
\frametitle{Текущий объект: target}
\usemintedstyle{solarized-light}
\begin{itemize}
\item Для большинства видов объектов (буферы, текстуры, ...) есть отдельный набор возможных значений \mintinline{cpp}|target|
\pause
\item Для таких объектов нет одного глобального текущего объекта, но есть текущий объект для каждого конкретного значения \mintinline{cpp}|target|
\pause
\item Можно считать, что есть глобальный словарь \begin{math}\operatorname{target} \rightarrow \operatorname{id}\end{math} текущих объектов (отдельный для буферов, текстур, и т.д.)
\pause
\item Смысл и особенности разных значений \mintinline{cpp}|target| зависят от типа объекта
\end{itemize}
\end{frame}

\begin{frame}[fragile]
\frametitle{Буферы}
\usemintedstyle{solarized-light}
\begin{itemize}
\item Могут хранить произвольные данные на GPU
\pause
\item \mintinline{cpp}|glGenBuffers / glDeleteBuffers|
\pause
\item Возможные значения \mintinline{cpp}|target| для \mintinline{cpp}|glBindBuffer|:
\begin{itemize}
\item \mintinline{cpp}|GL_ARRAY_BUFFER| (VBO) -- массив вершин
\pause
\item \mintinline{cpp}|GL_ELEMENT_ARRAY_BUFFER| (EBO/IBO) -- массив индексов вершин
\pause
\item \mintinline{cpp}|GL_UNIFORM_BUFFER| (UBO) -- массив для uniform-блоков
\pause
\item ...и другие
\end{itemize}
\pause
\item Один и тот же буфер \textit{можно} использовать с разными значениями \mintinline{cpp}|target|
\pause
\item Текущий \mintinline{cpp}|GL_ELEMENT_ARRAY_BUFFER| хранится не глобально, а \alert{\textbf{в текущем VAO!}}
\end{itemize}
\end{frame}

\begin{frame}[fragile]
\frametitle{Буферы: запись данных}
\begin{itemize}
\item Выделить память на GPU и загрузить данные:
\usemintedstyle{lightbulb}
\begin{minted}[bgcolor=codebg]{cpp}
glBufferData(GLenum target, GLsizeiptr size,
    const GLvoid * data, GLenum usage)
\end{minted}
\vspace*{-0.5cm}
\pause
\usemintedstyle{solarized-light}
\begin{itemize}
\item \mintinline{cpp}|target| -- \mintinline{cpp}|GL_ARRAY_BUFFER| и т.п.
\item \mintinline{cpp}|size| -- размер данных в байтах
\item \mintinline{cpp}|data| -- указатель на данные
\item \mintinline{cpp}|usage| -- подсказка драйверу о том, как данные будут использоваться
\end{itemize}
\pause
\item Если \mintinline{cpp}|data = nullptr|, буфер будет выделен, но данные будут не инициализированы
\pause
\item Если буфер уже содержал данные, они заменяются новыми (и происходит реаллокация памяти)
\end{itemize}
\end{frame}

\begin{frame}[fragile]
\frametitle{Буферы: запись данных}
\begin{itemize}
\usemintedstyle{solarized-light}
\item \mintinline{cpp}|target| буфера нужен для взаимодействия с другими объектами OpenGL; при загрузке данных в буфер этот параметр не очень важен и используется только чтобы обратиться к \textit{текущему буферу}
\end{itemize}
\end{frame}

\begin{frame}[fragile]
\frametitle{Буферы: запись данных}
\usemintedstyle{solarized-light}
\begin{itemize}
\item После вызова \mintinline{cpp}|glBufferData| с данными по указателю \mintinline{cpp}|data| можно делать всё, что угодно (в т.ч. удалить)
\pause
\item Копирование данных в память GPU тоже происходит асинхронно
\pause
\item \begin{math}\Longrightarrow\end{math} Драйвер, скорее всего, сначала копирует данные в собственную память, а потом инициирует отложенную загрузку на GPU
\end{itemize}
\end{frame}

\begin{frame}[fragile]
\frametitle{Буферы: параметр usage}
\usemintedstyle{solarized-light}
\begin{table}
\centering
\begin{tabular}{ c c c }
 \mintinline{cpp}|GL_STATIC_DRAW| & \mintinline{cpp}|GL_STATIC_READ| & \mintinline{cpp}|GL_STATIC_COPY| \\ 
 \mintinline{cpp}|GL_DYNAMIC_DRAW| & \mintinline{cpp}|GL_DYNAMIC_READ| & \mintinline{cpp}|GL_DYNAMIC_COPY| \\
 \mintinline{cpp}|GL_STREAM_DRAW| & \mintinline{cpp}|GL_STREAM_READ| & \mintinline{cpp}|GL_STREAM_COPY|
\end{tabular}
\end{table}
\pause
\begin{itemize}
\item Данные будут обновляться
\begin{itemize}
\item \mintinline{cpp}|STATIC| -- один раз
\item \mintinline{cpp}|DYNAMIC| -- иногда
\item \mintinline{cpp}|STREAM| -- почти каждый кадр
\end{itemize}
\pause
\item Буфер будет использоваться для:
\begin{itemize}
\item \mintinline{cpp}|DRAW| -- записи данных в него
\item \mintinline{cpp}|READ| -- чтения данных из него
\item \mintinline{cpp}|COPY| -- и записи, и чтения
\end{itemize}
\pause
\item Это только \textit{подсказка} драйверу и не влияет на корректность работы
\end{itemize}
\end{frame}

\begin{frame}[fragile]
\frametitle{Буферы: запись и чтение данных}
\usemintedstyle{lightbulb}
\begin{itemize}
\item Загрузить данные в часть буфера:
\begin{minted}[bgcolor=codebg]{cpp}
glBufferSubData(GLenum target, GLintptr offset,
    GLsizeiptr size, const GLvoid * data)
\end{minted}
\vspace*{-0.5cm}
\pause
\usemintedstyle{solarized-light}
\begin{itemize}
\item Гарантированно не реаллоцирует память GPU
\item Память уже должна быть выделена с помощью \mintinline{cpp}|glBufferData|
\end{itemize}
\pause
\item Прочитать данные из буфера:
\usemintedstyle{lightbulb}
\begin{minted}[bgcolor=codebg]{cpp}
glGetBufferSubData(GLenum target, GLintptr offset,
    GLsizeiptr size, GLvoid * data)
\end{minted}
\vspace*{-0.5cm}
\pause
\begin{itemize}
\item К моменту выхода из этой функции данные уже прочитаны
\item \begin{math}\Rightarrow\end{math} Синхронная функция, \textbf{блокирующая исполнение}
\end{itemize}
\end{itemize}
\end{frame}

\begin{frame}[fragile]
\frametitle{Mapped buffer}
\usemintedstyle{solarized-light}
\begin{itemize}
\item Можно получить виртуальный указатель на буфер или его часть и пользоваться им для чтения/записи
\pause
\item \mintinline{cpp}|glMapBuffer(target, access)| -- возвращает особый \textit{mapped} указатель
\item \mintinline{cpp}|access| может принимать значения
\begin{itemize}
\item \mintinline{cpp}|GL_READ_ONLY| -- по указателю можно читать
\item \mintinline{cpp}|GL_WRITE_ONLY| -- по указателю можно писать
\item \mintinline{cpp}|GL_READ_WRITE| -- по указателю можно читать и писать
\end{itemize}
\pause
\item \mintinline{cpp}|glUnmapBuffer(target)|
\pause
\item Между \mintinline{cpp}|glMapBuffer| и \mintinline{cpp}|glUnmapBuffer| работать с буфером (загружать данные другими методами, использовать данные для рисования) нельзя
\pause
\item После \mintinline{cpp}|glUnmapBuffer| mapped указатель использовать нельзя
\pause
\item \mintinline{cpp}|glUnmapBuffer| может вернуть \mintinline{cpp}|GL_FALSE| по зависящим от системы причинам, и память буфера при этом остаётся непроинициализированной
\end{itemize}
\end{frame}

\begin{frame}[fragile]
\frametitle{Буферы: типичный пример использования}
\usemintedstyle{lightbulb}
\begin{minted}[bgcolor=codebg]{cpp}
GLuint vbo;
glGenBuffers(1, &vbo);

std::vector<vertex> vertices;
...

glBindBuffer(GL_ARRAY_BUFFER, vbo);
glBufferData(GL_ARRAY_BUFFER,
    vertices.size() * sizeof(vertices[0]),
    vertices.data(), GL_STATIC_DRAW);
\end{minted}
\end{frame}

\begin{frame}[fragile]
\frametitle{Буферы: ссылки}
\begin{itemize}
\item \href{https://www.khronos.org/opengl/wiki/Buffer_Object}{\nolinkurl{khronos.org/opengl/wiki/Buffer\_Object}}
\item \href{https://www.songho.ca/opengl/gl_vbo.html}{\nolinkurl{songho.ca/opengl/gl\_vbo.html}}
\end{itemize}
\end{frame}

\begin{frame}[fragile]
\frametitle{Атрибуты вершин}
\usemintedstyle{solarized-light}
\begin{itemize}
\item Вершины -- основные входные данные графического конвейера
\pause
\item С точки зрения OpenGL, вершины характеризуются набором \textit{атрибутов}: позиция, цвет, нормаль, текстурные координаты, и т.п.
\pause
\item Каждый атрибут имеет свой тип и размерность
\pause
\item Все вершины в рамках одного вызова команды рисования (\mintinline{cpp}|glDrawArrays| и др.) имеет один набор атрибутов
\pause
\item Все настройки атрибутов конкретного набора вершин хранятся в VAO
\end{itemize}
\end{frame}

\begin{frame}[fragile]
\frametitle{Атрибуты вершин: шейдер}
\usemintedstyle{lightbulb}
\begin{minted}[bgcolor=codebg]{cpp}
#version 330 core

//             index     type  name
layout (location = 0) in vec3 position;
layout (location = 1) in vec3 normal;
layout (location = 2) in vec4 color;
layout (location = 3) in uint materialID;

void main() {
    ...
}
\end{minted}
\end{frame}

\begin{frame}[fragile]
\frametitle{Атрибуты вершин}
\usemintedstyle{solarized-light}
\begin{itemize}
\item Индекс: \mintinline{cpp}|0, 1, ... glGet(GL_MAX_VERTEX_ATTRIBS) - 1|
\begin{itemize}
\item \mintinline{cpp}|glGet(GL_MAX_VERTEX_ATTRIBS)| \begin{math}\geq\end{math} 16
\end{itemize}
\pause
\item Включен/выключен
\begin{itemize}
\item \mintinline{cpp}|glEnableVertexAttribArray(index)| -- включить аттрибут
\item Выключены по умолчанию
\item Состояние включенности хранится в \textbf{vertex array}
\end{itemize}
\pause
\item Если у вершины есть атрибут, не соответствующий атрибуту в шейдере, он просто игнорируется
\pause
\item Если в шейдере есть атрибут, не соответствующий атрибуту у вершин, он принимает значение по умолчанию:
\begin{itemize}
\item \mintinline{glsl}|float: 0.0|
\item \mintinline{glsl}| vec2: 0.0, 0.0|
\item \mintinline{glsl}| vec3: 0.0, 0.0, 0.0|
\item \mintinline{glsl}| vec4: 0.0, 0.0, 0.0, 1.0|
\end{itemize}
\end{itemize}
\end{frame}

\begin{frame}[fragile]
\frametitle{Атрибуты вершин: хранение}
\usemintedstyle{solarized-light}
Параметры хранения данных атрибута:
\begin{itemize}
\pause
\item Формат:
\begin{itemize}
\item Количество компонент -- 1, 2, 3, 4
\pause
\item Тип -- \mintinline{cpp}|GL_BYTE|, \mintinline{cpp}|GL_UNSIGNED_BYTE|, \mintinline{cpp}|GL_SHORT|, \mintinline{cpp}|GL_UNSIGNED_SHORT|, \mintinline{cpp}|GL_INT|, \mintinline{cpp}|GL_UNSIGNED_INT|, \mintinline{cpp}|GL_HALF_FLOAT|, \mintinline{cpp}|GL_FLOAT|, \mintinline{cpp}|GL_DOUBLE|, \mintinline{cpp}|GL_INT_2_10_10_10_REV|, \mintinline{cpp}|GL_UNSIGNED_INT_2_10_10_10_REV|
\pause
\item Нормированный / не нормированный
\end{itemize}
\pause
\item Расположение данных:
\begin{itemize}
\item Адрес начала данных (на GPU)
\item Расстояние (в байтах) между значениями этого атрибута, соответствующими соседним вершинам
\end{itemize}
\pause
\item Всё это запоминается в \textbf{vertex array}
\end{itemize}
\end{frame}

\begin{frame}[fragile]
\frametitle{Атрибуты вершин: нормированность}
\usemintedstyle{solarized-light}
\begin{itemize}
\item Как интерпретировать значение атрибута, который в шейдере объявлен как floating-point (\mintinline{glsl}|float|, \mintinline{glsl}|vec3|, и т.п.), но в данных он хранится как целочисленный?
\pause
\item 2 варианта:
\begin{itemize}
\item \textbf{Не нормированный}: передавать, как есть, делая преобразование \mintinline{glsl}|int -> float|
\pause
\item \textbf{Нормированный}: преобразовывать диапазон целочисленных значений в \begin{math}[-1, 1]\end{math} (для знаковых) или \begin{math}[0, 1]\end{math} (для беззнаковых)
\newline
\mintinline{cpp}|unsigned short: x -> x / 65535|
\newline
\mintinline{cpp}|short: x -> (2 x + 1) / 65535|
\end{itemize}
\pause
\item Полезно для передачи цвета: \mintinline{cpp}|0 .. 255| превращаются в стандартные для OpenGL \mintinline{cpp}|0.0 .. 1.0|
\pause
\item Полезно для компактного хранения атрибутов (например, нормалей)
\pause
\item Имеет смысл \textit{только} для floating-point аттрибутов
\end{itemize}
\end{frame}

\begin{frame}[fragile]
\frametitle{Атрибуты вершин}
\usemintedstyle{solarized-light}
\begin{itemize}
\item Перед настройкой конкретного атрибута нужно сделать текущим VAO, который вы хотите настроить
\pause
\item Если атрибут объявлен в шейдере как floating-point (\mintinline{glsl}|vec3| и т.п.), нужно использовать функцию \mintinline{cpp}|glVertexAttribPointer|, но храниться он \textit{может} как целочисленный
\pause
\item Если атрибут объявлен в шейдере как целочисленный (\mintinline{glsl}|ivec3,uvec4| и т.п.), нужно использовать функцию \mintinline{cpp}|glVertexAttribIPointer|, и храниться он \textit{должен} как целочисленный
\end{itemize}
\end{frame}

\begin{frame}[fragile]
\frametitle{Атрибуты вершин: floating-point}
\usemintedstyle{lightbulb}
\begin{minted}[bgcolor=codebg]{cpp}
glVertexAttribPointer(GLuint index, GLint size,
    GLenum type, GLboolean normalized,
    GLsizei stride, const GLvoid * pointer)
\end{minted}
\vspace*{-0.5cm}
\pause
\usemintedstyle{solarized-light}
\begin{itemize}
\item \mintinline{cpp}|index| -- индекс аттрибута (\mintinline{glsl}|layout (location = index)| в шейдере)
\pause
\item \mintinline{cpp}|size| -- размерность (число компонент)
\pause
\item \mintinline{cpp}|type| -- тип каждой компоненты
\pause
\item \mintinline{cpp}|normalized| -- нормированный или нет
\pause
\item \mintinline{cpp}|stride| -- расстояние между соседними значениями (в байтах)
\pause
\item \mintinline{cpp}|pointer| -- указатель на данные
\pause
\begin{itemize}
\item На самом деле, \mintinline{cpp}|pointer| -- сдвиг в байтах относительно начала памяти текущего \mintinline{cpp}|GL_ARRAY_BUFFER|
\item Например, если данные этого атрибута начинаются на 12ом байте текущего \mintinline{cpp}|GL_ARRAY_BUFFER|, нужно передать \mintinline{cpp}|(void *)(12)|
\pause
\item Разные атрибуты могут лежать в разных буферах
\end{itemize}
\end{itemize}
\end{frame}

\begin{frame}[fragile]
\frametitle{Атрибуты вершин: целочисленные}
\usemintedstyle{lightbulb}
\begin{minted}[bgcolor=codebg]{cpp}
glVertexAttribIPointer(GLuint index, GLint size,
    GLenum type,
    GLsizei stride, const GLvoid * pointer)
\end{minted}
\vspace*{-0.5cm}
\usemintedstyle{solarized-light}
\begin{itemize}
\item Всё то же самое, но 
\begin{itemize}
\item Нет параметра \mintinline{cpp}|normalized|
\item \mintinline{cpp}|type| может быть только целочисленным
\end{itemize}
\end{itemize}
\end{frame}

\begin{frame}[fragile]
\frametitle{Атрибуты вершин: stride}
\usemintedstyle{solarized-light}
\begin{itemize}
\item Пусть \mintinline{cpp}|start| -- смещение, указанное в параметре \mintinline{cpp}|pointer| функции \mintinline{cpp}|glVertexAttribPointer|
\pause
\item Тогда соответстующий аттрибут вершины с номером \mintinline{cpp}|i| будет взят по смещению \mintinline{cpp}|start + stride * i|
\pause
\begin{itemize}
\item Нулевая вершина: \mintinline{cpp}|start|
\item Первая вершина: \mintinline{cpp}|start + stride|
\item Вторая вершина: \mintinline{cpp}|start + 2 * stride|
\item И т.д.
\end{itemize}
\pause
\item Если \mintinline{cpp}|stride = 0|, \mintinline{cpp}|stride| считается равным размеру аттрибута \mintinline{cpp}|stride = size * sizeof(type)|
\pause
\begin{itemize}
\item Например, для \mintinline{cpp}|size = 3| и \mintinline{cpp}|type = GL_UNSIGNED_INT|, \mintinline{cpp}|stride| будет вычислен как \mintinline{cpp}|stride = 3 * sizeof(unsigned int) = 3 * 4 = 12|
\pause
\item Можно использовать, если значения конкретного атрибута идут в памяти подряд, друг за другом, без пропусков
\end{itemize}
\pause
\item Нужен для гибкости хранения аттрибутов
\end{itemize}
\end{frame}

\begin{frame}[fragile]
\frametitle{Атрибуты вершин: размерность}
\usemintedstyle{solarized-light}
\begin{itemize}
\item Размерность атрибута, указанная в \mintinline{cpp}|glVertexAttribPointer| (т.е. реальное количество хранимых в памяти компонент), не обязана совпадать с размерностью, указанной в шейдере
\pause
\item Если компонент больше, чем надо, они отбрасываются
\pause
\item Если компонент меньше, чем надо, они дополняются нулём (первые 3 компоненты) или единицей (4ая компонента)
\end{itemize}
\end{frame}

\begin{frame}[fragile]
\frametitle{Атрибуты вершин: пример array-of-structs, один буфер}
\usemintedstyle{lightbulb}
\begin{minted}[bgcolor=codebg,fontsize=\tiny]{cpp}
struct vertex {
    float position[3];
    float normal[3];
    unsigned char color[4];
};
vertex vertices[N];

glBindBuffer(GL_ARRAY_BUFFER, vbo);
glBufferData(GL_ARRAY_BUFFER, N * sizeof(vertices), vertices, GL_STATIC_DRAW);

glBindVertexArray(vao);

glEnableVertexAttribArray(0);
glVertexAttribPointer(0, 3, GL_FLOAT, GL_FALSE, sizeof(vertex), (void*)(0));

glEnableVertexAttribArray(1);
glVertexAttribPointer(1, 3, GL_FLOAT, GL_FALSE, sizeof(vertex), (void*)(12));

glEnableVertexAttribArray(2);
glVertexAttribPointer(2, 4, GL_UNSIGNED_BYTE, GL_TRUE, sizeof(vertex), (void*)(24));
\end{minted}
\vspace*{-1.125cm}
\begin{center}
\begin{tikzpicture}
\foreach \x
[evaluate=\x as \ta using int(\x * 28)]
[evaluate=\x as \tb using int(\x * 28 + 12)]
[evaluate=\x as \tc using int(\x * 28 + 24)]
[evaluate=\x as \td using int(\x * 28 + 28)]
in {0, 1, 2}
{
\filldraw[fill=red!40!white, draw=black,thick] (3.0 * \x + 0.0, 0) rectangle (3.0 * \x + 1.0, 0.5);
\filldraw[fill=green!40!white, draw=black,thick] (3.0 * \x + 1.0, 0) rectangle (3.0 * \x + 2.0, 0.5);
\filldraw[fill=blue!40!white, draw=black,thick] (3.0 * \x + 2.0, 0) rectangle (3.0 * \x + 3.0, 0.5);
\node[] at (3.0 * \x + 0.5, 0.25) {P\x};
\node[] at (3.0 * \x + 1.5, 0.25) {N\x};
\node[] at (3.0 * \x + 2.5, 0.25) {C\x};
\node[] at (3.0 * \x + 0.0, 0.75) {\ta};
\node[] at (3.0 * \x + 1.0, 0.75) {\tb};
\node[] at (3.0 * \x + 2.0, 0.75) {\tc};
\node[] at (3.0 * \x + 3.0, 0.75) {\td};
\draw[black,thick] (3.0 * \x, -0.125) -- (3.0 * \x, -0.5);
\draw[black,thick] (3.0 * \x + 3.0, -0.125) -- (3.0 * \x + 3.0, -0.5);
\node[] at (3.0 * \x + 1.5, -0.25) {v\x};
}
\end{tikzpicture}
\end{center}
\end{frame}

\begin{frame}[fragile]
\frametitle{Аттрибуты вершин: пример struct-of-arrays, один буфер}
\usemintedstyle{lightbulb}
\begin{minted}[bgcolor=codebg,fontsize=\tiny]{cpp}
float positions[3 * N];
float normals[3 * N];
unsigned char colors[4 * N];

glBindBuffer(GL_ARRAY_BUFFER, vbo);
glBufferData(GL_ARRAY_BUFFER, N * 28, nullptr, GL_STATIC_DRAW);
glBufferSubData(GL_ARRAY_BUFFER, 0, N * 12, positions);
glBufferSubData(GL_ARRAY_BUFFER, N * 12, N * 12, normals);
glBufferSubData(GL_ARRAY_BUFFER, N * 24, N * 4, colors);

glBindVertexArray(vao);

glEnableVertexAttribArray(0);
glVertexAttribPointer(0, 3, GL_FLOAT, GL_FALSE, 0, (void*)(0));

glEnableVertexAttribArray(1);
glVertexAttribPointer(1, 3, GL_FLOAT, GL_FALSE, 0, (void*)(12 * N));
    
glEnableVertexAttribArray(2);
glVertexAttribPointer(2, 4, GL_UNSIGNED_BYTE, GL_TRUE, 0, (void*)(24 * N));
\end{minted}
\vspace*{-1cm}
\begin{center}
\begin{tikzpicture}
\node[] at (0.0, 0.75) {0};
\foreach \x
[evaluate=\x as \ta using int(\x * 12 + 12)]
[evaluate=\x as \tb using int(\x * 12 + 48)]
[evaluate=\x as \tc using int(\x * 4 + 76)]
in {0, 1, 2}
{
\filldraw[fill=red!40!white, draw=black,thick] (\x + 0.0, 0) rectangle (\x + 1.0, 0.5);
\filldraw[fill=green!40!white, draw=black,thick] (\x + 3.0, 0) rectangle (\x + 4.0, 0.5);
\filldraw[fill=blue!40!white, draw=black,thick] (\x + 6.0, 0) rectangle (\x + 7.0, 0.5);
\node[] at (\x + 0.5, 0.25) {P\x};
\node[] at (\x + 3.5, 0.25) {N\x};
\node[] at (\x + 6.5, 0.25) {C\x};
\node[] at (\x + 1.0, 0.75) {\ta};
\node[] at (\x + 4.0, 0.75) {\tb};
\node[] at (\x + 7.0, 0.75) {\tc};
}
\end{tikzpicture}
\end{center}
\end{frame}

\begin{frame}[fragile]
\frametitle{Аттрибуты вершин: пример struct-of-arrays, три буфера}
\usemintedstyle{lightbulb}
\begin{minted}[bgcolor=codebg,fontsize=\tiny]{cpp}
float positions[3 * N];
float normals[3 * N];
unsigned char colors[4 * N];

glBindVertexArray(vao);

glBindBuffer(GL_ARRAY_BUFFER, position_vbo);
glBufferData(GL_ARRAY_BUFFER, 3 * N * sizeof(float), positions, GL_STATIC_DRAW);
glEnableVertexAttribArray(0);
glVertexAttribPointer(0, 3, GL_FLOAT, GL_FALSE, 0, (void*)(0));

glBindBuffer(GL_ARRAY_BUFFER, normal_vbo);
glBufferData(GL_ARRAY_BUFFER, 3 * N * sizeof(float), normals, GL_STATIC_DRAW);
glEnableVertexAttribArray(1);
glVertexAttribPointer(1, 3, GL_FLOAT, GL_FALSE, 0, (void*)(0));

glBindBuffer(GL_ARRAY_BUFFER, color_vbo);
glBufferData(GL_ARRAY_BUFFER, 4 * N * sizeof(unsigned char), colors, GL_STATIC_DRAW);
glEnableVertexAttribArray(2);
glVertexAttribPointer(2, 4, GL_UNSIGNED_BYTE, GL_TRUE, 0, (void*)(0));
\end{minted}
\vspace*{-1.5cm}
\begin{center}
\begin{tikzpicture}
\node[] at (0.0, 0.75) {0};
\node[] at (4.0, 0.75) {0};
\node[] at (8.0, 0.75) {0};
\foreach \x
[evaluate=\x as \ta using int(\x * 12 + 12)]
[evaluate=\x as \tb using int(\x * 12 + 12)]
[evaluate=\x as \tc using int(\x * 4 + 4)]
in {0, 1, 2}
{
\filldraw[fill=red!40!white, draw=black,thick] (\x + 0.0, 0) rectangle (\x + 1.0, 0.5);
\filldraw[fill=green!40!white, draw=black,thick] (\x + 4.0, 0) rectangle (\x + 5.0, 0.5);
\filldraw[fill=blue!40!white, draw=black,thick] (\x + 8.0, 0) rectangle (\x + 9.0, 0.5);
\node[] at (\x + 0.5, 0.25) {P\x};
\node[] at (\x + 4.5, 0.25) {N\x};
\node[] at (\x + 8.5, 0.25) {C\x};
\node[] at (\x + 1.0, 0.75) {\ta};
\node[] at (\x + 5.0, 0.75) {\tb};
\node[] at (\x + 9.0, 0.75) {\tc};
}
\end{tikzpicture}
\end{center}
\end{frame}

\begin{frame}[fragile]
\frametitle{Аттрибуты вершин: как хранить?}
\usemintedstyle{solarized-light}
\begin{itemize}
\item Если не можете выбрать, используйте array-of-structs (например, если вершины хранятся непрерывано в памяти, e.g. \mintinline{cpp}|std::vector<vertex>|)
\pause
\item Если часть аттрибутов постоянны, а другую часть нужно иногда обновлять -- разделите эти части по двум разным VBO
\pause
\item Некоторые форматы 3D сцен (напр. \href{https://registry.khronos.org/glTF/specs/2.0/glTF-2.0.html}{\texttt{glTF}}) хранят вершины в бинарном формате, удобном для загрузки в OpenGL: поддерживают несколько буферов и разный \mintinline{cpp}|stride|
\end{itemize}
\end{frame}

\begin{frame}[fragile]
\frametitle{Аттрибуты вершин: пример полностью}
\usemintedstyle{lightbulb}
\begin{minted}[bgcolor=codebg,fontsize=\scriptsize]{cpp}
// Инициализация:
program = createProgram()
vertices = createVertices()
vbo = createVBO(vertices)
vao = createVAO()
setupAttributes(vao, vbo) // glVertexAttribPointer, ...

// Рендеринг:
glUseProgram(program)
glBindVertexArray(vao)
glDrawArrays(GL_TRIANGLES, 0, count)
\end{minted}
\pause
\begin{itemize}
\item \href{https://www.khronos.org/opengl/wiki/Vertex_Specification}{\nolinkurl{khronos.org/opengl/wiki/Vertex\_Specification}}
\end{itemize}
\end{frame}

\begin{frame}[fragile]
\frametitle{Индексированный рендеринг}
\usemintedstyle{solarized-light}
\slideimage{spheres.png}
\begin{itemize}
\item В моделях одна вершина часто входит в несколько треугольников (в среднем, 6)
\pause
\item Если использовать \mintinline{cpp}|GL_TRIANGLES|, придётся копировать каждую вершину несколько раз
\pause
\item \mintinline{cpp}|GL_TRIANGLE_STRIP| и т.п. только частично решают проблему
\pause
\item \begin{math}\Rightarrow\end{math} Индексированный рендеринг
\end{itemize}
\end{frame}

\begin{frame}[fragile]
\frametitle{Индексированный рендеринг}
\usemintedstyle{solarized-light}
\begin{itemize}
\item \mintinline{cpp}|glDrawArrays(mode, 3, 5)|:
\begin{center}
\begin{tikzpicture}
\foreach \x in {0, ..., 9}
{
\ifthenelse{\x > 2 \AND \x < 8}
{\filldraw[fill=red!40!white, draw=black,thick] (\x, 0) rectangle (\x + 1.0, 0.5);}
{\draw[black,thick] (\x, 0) rectangle (\x + 1.0, 0.5);}

\node[] at (\x + 0.5, 0.25) {v\x};
}
\end{tikzpicture}
\end{center}
\pause

\vspace{1cm}

\item \mintinline{cpp}|glDrawElements(mode, ...)|:
\begin{center}
\begin{tikzpicture}
\foreach \x in {0, ..., 9}
{
\ifthenelse{\x = 6 \OR \x = 3 \OR \x = 4 \OR \x = 0 \OR \x = 8}
{\filldraw[fill=red!40!white, draw=black,thick] (\x, 2) rectangle (\x + 1.0, 2.5);}
{\draw[black,thick] (\x, 2) rectangle (\x + 1.0, 2.5);}
\node[] at (\x + 0.5, 2.25) {v\x};
}

\foreach \x in {0, ..., 9}
{
\ifthenelse{\x > 2 \AND \x < 8}
{\filldraw[fill=red!40!white, draw=black,thick] (\x, 0) rectangle (\x + 1.0, 0.5);}
{\draw[black,thick] (\x, 0) rectangle (\x + 1.0, 0.5);}
}
\node[] at (0.5, 0.25) {3};
\node[] at (1.5, 0.25) {0};
\node[] at (2.5, 0.25) {5};
\node[] at (3.5, 0.25) {6};
\node[] at (4.5, 0.25) {3};
\node[] at (5.5, 0.25) {4};
\node[] at (6.5, 0.25) {0};
\node[] at (7.5, 0.25) {8};
\node[] at (8.5, 0.25) {0};
\node[] at (9.5, 0.25) {1};

\draw[thick,-stealth] (3.5, 0.5) -- (6.5, 1.9);
\draw[thick,-stealth] (4.5, 0.5) -- (3.5, 1.9);
\draw[thick,-stealth] (5.5, 0.5) -- (4.5, 1.9);
\draw[thick,-stealth] (6.5, 0.5) -- (0.5, 1.9);
\draw[thick,-stealth] (7.5, 0.5) -- (8.5, 1.9);

% 0 1 2 3 4 5 6 7 8 9
% 3 0 5[6 3 4 0 8]0 1
\end{tikzpicture}
\end{center}
\end{itemize}
\end{frame}

\begin{frame}[fragile]
\frametitle{Индексированный рендеринг: квадрат}
\usemintedstyle{solarized-light}
\begin{center}
\begin{tikzpicture}
\draw[black,thick] (0, 0) rectangle (3, 3);
\draw[black,thick] (0, 3) -- (3, 0);
\node[] at (-0.25, -0.25) {0};
\node[] at (3.25, -0.25) {1};
\node[] at (-0.25, 3.25) {2};
\node[] at (3.25, 3.25) {3};
\end{tikzpicture}
\end{center}
\pause
\begin{itemize}
\item Без индексов, \mintinline{cpp}|GL_TRIANGLES|:
\begin{itemize}
\item Вершины: \mintinline{cpp}|v0, v1, v2, v2, v1, v3|
\end{itemize}
\pause
\item Без индексов, \mintinline{cpp}|GL_TRIANGLE_STRIP|:
\begin{itemize}
\item Вершины: \mintinline{cpp}|v0, v1, v2, v3|
\end{itemize}
\pause
\item С индексами, \mintinline{cpp}|GL_TRIANGLES|:
\begin{itemize}
\item Вершины: \mintinline{cpp}|v0, v1, v2, v3|
\item Индексы: \mintinline{cpp}|0, 1, 2, 2, 1, 3|
\end{itemize}
\end{itemize}
\end{frame}

\begin{frame}[fragile]
\frametitle{Индексированный рендеринг}
\usemintedstyle{solarized-light}
\begin{itemize}
\item Индексы хранятся в \mintinline{cpp}|GL_ELEMENT_ARRAY_BUFFER|
\pause
\item id текущего \mintinline{cpp}|GL_ELEMENT_ARRAY_BUFFER| хранится \alert{\textbf{в текущем VAO!}}
\pause
\item \begin{math}\Longrightarrow\end{math} \mintinline{cpp}|glBindVertexArray| автоматически делает текущим нужный \mintinline{cpp}|GL_ELEMENT_ARRAY_BUFFER| с индексами
\pause
\item Если вы хотите только \textit{загрузить} индексы, но пока не \textit{использовать} их, лучше использовать \mintinline{cpp}|target = GL_ARRAY_BUFFER|, чтобы случайно не испортить текущий VAO
\end{itemize}
\end{frame}

\begin{frame}[fragile]
\frametitle{Индексированный рендеринг}
\usemintedstyle{solarized-light}
\usemintedstyle{lightbulb}
\begin{minted}[bgcolor=codebg]{cpp}
glDrawElements(GLenum mode, GLsizei count,
    GLenum type, const GLvoid * indices)
\end{minted}
\vspace*{-0.5cm}
\pause
\begin{itemize}
\usemintedstyle{solarized-light}
\item \mintinline{cpp}|mode| -- как в \mintinline{cpp}|glDrawArrays|: \mintinline{cpp}|GL_TRIANGLES|, ...
\pause
\item \mintinline{cpp}|count| -- количество индексов
\pause
\item \mintinline{cpp}|type| -- тип индексов: \mintinline{cpp}|GL_UNSIGNED_BYTE|, \mintinline{cpp}|GL_UNSIGNED_SHORT|, \mintinline{cpp}|GL_UNSIGNED_INT|
\pause
\item \mintinline{cpp}|indices| -- смещение в байтах в текущем \mintinline{cpp}|GL_ELEMENT_ARRAY_BUFFER| до места, где начинаются индексы
\pause
\begin{itemize}
\item id текущего \mintinline{cpp}|GL_ELEMENT_ARRAY_BUFFER| хранится \alert{\textbf{в текущем VAO!}}
\pause
\item \begin{math}\Longrightarrow\end{math} Для рендеринга \alert{\textbf{не нужно}} делать текущим буфер с индексами, достаточно сделать текущим VAO
\end{itemize}
\end{itemize}
\end{frame}

\begin{frame}[fragile]
\frametitle{Индексированный рендеринг: пример}
\usemintedstyle{lightbulb}
\begin{minted}[bgcolor=codebg]{cpp}
// Инициализация:
program = createProgram()
vertices, indices = generateMesh()
vbo = createBuffer(vertices)
ebo = createBuffer(indices)
vao = createVAO()

glBindVertexArray(vao)
glBindBuffer(GL_ARRAY_BUFFER, vbo)
// настраиваем аттрибуты...
// загружаем вершины в vbo...
glBindBuffer(GL_ELEMENT_ARRAY_BUFFER, ebo)
// загружаем индексы в ebo...

// Рендеринг:
glUseProgram(program)
glBindVertexArray(vao)
glDrawElements(GL_TRIANGLES, count, GL_UNSIGNED_SHORT,
    (void*)(0))
\end{minted}
\end{frame}

\begin{frame}[fragile]
\frametitle{Primitive restart}
\usemintedstyle{solarized-light}
\begin{itemize}
\item При использовании \mintinline{cpp}|GL_LINE_STRIP|, \mintinline{cpp}|GL_LINE_LOOP|, \mintinline{cpp}|GL_TRIANGLE_STRIP|, \mintinline{cpp}|GL_TRIANGLE_FAN| все входные вершины образуют один line strip, line loop, triangle strip, triangle fan соответственно
\pause
\item Что, если мы хотим одну последовательность вершин разбить на несколько примитивов?
\pause
\item Можно сделать несколько вызовов \mintinline{cpp}|glDraw*|, но это медленно
\pause
\item Специальное значение индекса вершины означает что со следующей вершины нужно начать новый примитив:
\mintinline{python}|[0, 1, 2, 3, 65535, 4, 5, 6, 7] эквивалентно|
\mintinline{python}|[0, 1, 2, 3], [4, 5, 6, 7]|
\pause
\item Включить: \mintinline{cpp}|glEnable(GL_PRIMITIVE_RESTART)|
\pause
\item Настроить специальное значение: \mintinline{cpp}|glPrimitiveRestartIndex(index)|
\end{itemize}
\end{frame}

\begin{frame}[fragile]
\frametitle{Индексированный рендеринг: ссылки}
\begin{itemize}
\item \href{https://www.khronos.org/opengl/wiki/Vertex_Rendering#Basic_Drawing}{\nolinkurl{khronos.org/opengl/wiki/Vertex\_Rendering\#Basic\_Drawing}}
\item \href{http://www.opengl-tutorial.org/intermediate-tutorials/tutorial-9-vbo-indexing}{\nolinkurl{opengl-tutorial.org/intermediate-tutorials/tutorial-9-vbo-indexing}}
\end{itemize}
\end{frame}

\begin{frame}[fragile]
\frametitle{VAO, VBO, EBO: схема}
\begin{center}
\begin{tikzpicture}
\filldraw[fill=red!40!white, draw=black,thick] (3.0, 0) rectangle (4.0, 0.5);
\filldraw[fill=green!40!white, draw=black,thick] (4.0, 0) rectangle (5.0, 0.5);
\filldraw[fill=red!40!white, draw=black,thick] (5.0, 0) rectangle (6.0, 0.5);
\filldraw[fill=green!40!white, draw=black,thick] (6.0, 0) rectangle (7.0, 0.5);

\node[] at (1.5, 0.25) {VBO1:};
\node[] at (3.5, 0.25) {P0};
\node[] at (4.5, 0.25) {N0};
\node[] at (5.5, 0.25) {P1};
\node[] at (6.5, 0.25) {N1};
\node[] at (7.5, 0.25) {\begin{math}\cdots\end{math}};

\filldraw[fill=blue!40!white, draw=black,thick] (3.0, -1.0) rectangle (4.0, -0.5);
\filldraw[fill=blue!40!white, draw=black,thick] (4.0, -1.0) rectangle (5.0, -0.5);

\node[] at (1.5, -0.75) {VBO2:};
\node[] at (3.5, -0.75) {C0};
\node[] at (4.5, -0.75) {C1};
\node[] at (5.5, -0.75) {\begin{math}\cdots\end{math}};

\draw[draw=black,thick] (3.0, -2.0) rectangle (3.5, -1.5);
\draw[draw=black,thick] (3.5, -2.0) rectangle (4.0, -1.5);
\draw[draw=black,thick] (4.0, -2.0) rectangle (4.5, -1.5);
\draw[draw=black,thick] (4.5, -2.0) rectangle (5.0, -1.5);
\draw[draw=black,thick] (5.0, -2.0) rectangle (5.5, -1.5);
\draw[draw=black,thick] (5.5, -2.0) rectangle (6.0, -1.5);

\node[] at (1.5, -1.75) {EBO:};
\node[] at (3.25, -1.75) {0};
\node[] at (3.75, -1.75) {1};
\node[] at (4.25, -1.75) {2};
\node[] at (4.75, -1.75) {2};
\node[] at (5.25, -1.75) {1};
\node[] at (5.75, -1.75) {3};
\node[] at (6.5, -1.75) {\begin{math}\cdots\end{math}};

\draw[draw=black,thick] (3.0, -3.0) rectangle (5.0, -2.5);
\draw[draw=black,thick] (5.0, -3.0) rectangle (9.0, -2.5);
\filldraw[fill=red!40!white, draw=black,thick] (3.0, -3.5) rectangle (5.0, -3.0);
\draw[draw=black,thick] (5.0, -3.5) rectangle (9.0, -3.0);
\filldraw[fill=green!40!white, draw=black,thick] (3.0, -4.0) rectangle (5.0, -3.5);
\draw[draw=black,thick] (5.0, -4.0) rectangle (9.0, -3.5);
\filldraw[fill=blue!40!white, draw=black,thick] (3.0, -4.5) rectangle (5.0, -4.0);
\draw[draw=black,thick] (5.0, -4.5) rectangle (9.0, -4.0);

\node[] at (1.5, -2.75) {VAO:};
\node[] at (4.0, -2.75) {Indices:};
\node[] at (7.0, -2.75) {EBO};
\node[] at (4.0, -3.25) {Attrib 0:};
\node[] at (7.0, -3.25) {VBO1, offset, stride};
\node[] at (4.0, -3.75) {Attrib 1:};
\node[] at (7.0, -3.75) {VBO1, offset, stride};
\node[] at (4.0, -4.25) {Attrib 2:};
\node[] at (7.0, -4.25) {VBO2, offset, stride};
\end{tikzpicture}
\end{center}
\end{frame}

\begin{frame}[fragile]
\frametitle{Рендеринг нескольких объектов}
\begin{itemize}
\item Обычно: для каждого объекта свой набор VAO+VBO(+EBO)
\item Для обновления данных нужны только VBO (и, возможно, EBO)
\item Для рисования нужен только VAO
\end{itemize}
\end{frame}

\begin{frame}[fragile]
\frametitle{Рендеринг нескольких объектов}
Инициализация:
\usemintedstyle{lightbulb}
\begin{minted}[bgcolor=codebg]{cpp}
for (o in objects) {
    glGenVertexArrays(1, &o.vao)
    glGenBuffers(1, &o.vbo)
    glGenBuffers(1, &o.ebo)

    glBindVertexArray(o.vao)
    glBindBuffer(GL_ARRAY_BUFFER, o.vbo)
    for (i in attribs) {
        glEnableVertexAttribArray(i)
        glVertexAttribPointer(i, ...)
    }

    // запомнится в o.vao
    glBindBuffer(GL_ELEMENT_ARRAY_BUFFER, o.ebo)
}
\end{minted}
\end{frame}

\begin{frame}[fragile]
\frametitle{Рендеринг нескольких объектов}
Загрузка данных:
\usemintedstyle{lightbulb}
\begin{minted}[bgcolor=codebg]{cpp}
for (o in objects) {
    glBindBuffer(GL_ARRAY_BUFFER, o.vbo)
    glBufferData(GL_ARRAY_BUFFER, o.vertices)

    glBindBuffer(GL_ARRAY_BUFFER, o.ebo)
    glBufferData(GL_ARRAY_BUFFER, o.indices)
}
\end{minted}
\end{frame}

\begin{frame}[fragile]
\frametitle{Рендеринг нескольких объектов}
Рендеринг:
\usemintedstyle{lightbulb}
\begin{minted}[bgcolor=codebg]{cpp}
glUseProgram(program)
glUniform(...)
for (o in objects) {
    glBindVertexArray(o.vao)
    glDrawElements(o.indexCount)
}
\end{minted}
\end{frame}

\end{document}
