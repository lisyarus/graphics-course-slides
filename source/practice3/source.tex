% (c) Nikita Lisitsa, lisyarus@gmail.com, 2025

\documentclass[10pt]{beamer}

\usepackage[T2A]{fontenc}
\usepackage[russian]{babel}
\usepackage{minted}

\usepackage{graphicx}
\graphicspath{ {./images/} }

\usetheme{metropolis}

\usepackage{adjustbox}

\usepackage{color}
\usepackage{soul}

\usepackage{hyperref}

\definecolor{codebg}{RGB}{29,35,49}

\setminted{fontsize=\footnotesize}

\setlength\partopsep{-\topsep}

\definecolor{blue}{rgb}{0,0,1}
\definecolor{red}{rgb}{1,0,0}

\makeatletter
\newcommand{\slideimage}[1]{
  \begin{figure}
    \begin{adjustbox}{width=\textwidth, totalheight=\textheight-2\baselineskip-2\baselineskip,keepaspectratio}
      \includegraphics{#1}
    \end{adjustbox}
  \end{figure}
}
\makeatother

\title{Компьютерная графика}
\subtitle{Практика 3: VAO, VBO, кривые Безье}
\date{2025}

\setbeamertemplate{footline}[frame number]

\begin{document}

\frame{\titlepage}

\begin{frame}[fragile]
\frametitle{Кривые Безье}
\begin{itemize}
\item Используются для генерации плавных кривых
\pause
\item Строются по набору точек \begin{math}p_0, p_1, \dots, p_n\end{math}
\pause
\item Кривая Безье с параметром \begin{math}t \in [0, 1]\end{math} определяется как \textit{аффинная комбинация}

\begin{center}
\begin{math}
b(t) = \sum\limits_{k=0}^n b_{k,n}(t) \cdot p_k
\end{math}
\end{center}

\pause
\item Коэффициенты -- полиномы Бернштейна:

\begin{center}
\begin{math}
b_{k,n}(t) = \binom{n}{k}t^k(1-t)^{n-k}
\end{math}
\end{center}

\pause
\item Кривая первого порядка (\begin{math}n=1\end{math}): отрезок \begin{math}p_0 p_1\end{math}

\begin{center}
\begin{math}
b(t) = (1-t)\cdot p_0 + t \cdot p_1
\end{math}
\end{center}

\pause
\item Кривая второго порядка (\begin{math}n=2\end{math}):

\begin{center}
\begin{math}
b(t) = (1-t)^2 \cdot p_0 + 2t(1-t) \cdot p_1 + t^2 \cdot p_2
\end{math}
\end{center}

\end{itemize}
\end{frame}

\begin{frame}[fragile]
\frametitle{Задание 1}
Загружаем данные в VBO
\usemintedstyle{solarized-light}
\begin{itemize}
\item Создайте и заполните массив из трёх вершин типа \mintinline{cpp}|vertex|
\item Создайте VBO и загрузите в него данные: \mintinline{cpp}|glGenBuffers|, \mintinline{cpp}|glBindBuffer|, \mintinline{cpp}|glBufferData|
\item Проверьте, что данные загрузились, создав временную переменную, считав в неё координату какой-нибудь вершины (\mintinline{cpp}|glGetBufferSubData|) и выведя результат в \mintinline{cpp}|std::cout|
\end{itemize}
\end{frame}

\begin{frame}[fragile]
\frametitle{Задание 2}
Рисуем с помощью VAO
\usemintedstyle{solarized-light}
\begin{itemize}
\item Создайте VAO и настройте атрибуты вершин при инициализации программы: \mintinline{cpp}|glGenVertexArrays|, \mintinline{cpp}|glBindVertexArray|, \mintinline{cpp}|glEnableVertexAttribArray|, \mintinline{cpp}|glVertexAttribPointer|
\item Нарисуйте треугольник с помощью этого VAO: \mintinline{cpp}|glDrawArrays|
\item В цикле рисования \textbf{\alert{не}} должно быть настройки атрибутов или загрузки данных в буферы!
\end{itemize}
\end{frame}

\begin{frame}
\frametitle{Задание 2}
\slideimage{2.png}
\end{frame}

\begin{frame}[fragile]
\frametitle{Задание 3}
Переходим к оконным координатам
\usemintedstyle{solarized-light}
\begin{itemize}
\item Заполните view-матрицу преобразованием, которое совершает проеобразование

\begin{center}
\begin{math}
X: [0, width] \rightarrow [-1, 1]
\end{math}

\begin{math}
Y: [height, 0] \rightarrow [-1, 1]
\end{math}
\end{center}

\item Измените координаты треугольника, чтобы он был заметен на экране
\end{itemize}
\end{frame}

\begin{frame}[fragile]
\frametitle{Задание 4}
Динамически добавляем/удаляем точки мышкой
\usemintedstyle{solarized-light}
\begin{itemize}
\item Замените статический массив с вершинами на контейнер \mintinline{cpp}|std::vector| (изначально пустой)
\item При нажатии левой кнопки мыши (\mintinline{cpp}|SDL_BUTTON_LEFT|) добавьте новую вершину в контейнер с координатами мыши (цвета выбирайте как угодно)
\item При нажатии правой кнопки мыши (\mintinline{cpp}|SDL_BUTTON_RIGHT|) удалите последнюю вершину из контейнера, если он не пустой
\item Если контейнер с вершинами изменился, обновите данные в соответствующем VBO
\item Рисуем линию из всех точек: \mintinline{cpp}|GL_LINE_STRIP|
\item Делаем линию потолще: \mintinline{cpp}|glLineWidth(5.f)|
\end{itemize}
\end{frame}

\begin{frame}[fragile]
\frametitle{Задание 4}
\usemintedstyle{solarized-light}
\begin{itemize}
\item Данные должны обновляться (\mintinline{cpp}|glBindBuffer| + \mintinline{cpp}|glBufferData|) \textbf{\alert{только}} когда ломаная изменилась!
\item В цикле рисования всё ещё не должно быть настройки атрибутов
\end{itemize}
\end{frame}

\begin{frame}
\frametitle{Задание 4}
\slideimage{4.png}
\end{frame}

\begin{frame}[fragile]
\frametitle{Задание 5}
Нарисуем сами точки
\usemintedstyle{solarized-light}
\begin{itemize}
\item \mintinline{cpp}|glPointSize(10)| чтобы точки были заметны
\item Ещё один вызов \mintinline{cpp}|glDrawArrays| чтобы нарисовать точки (\mintinline{cpp}|GL_POINTS|)
\end{itemize}
\end{frame}

\begin{frame}
\frametitle{Задание 5}
\slideimage{5.png}
\end{frame}

\begin{frame}[fragile]
\frametitle{Задание 6}
Генерируем и рисуем кривые Безье
\usemintedstyle{solarized-light}
\begin{itemize}
\item Создаёте ещё один VBO, \mintinline{cpp}|std::vector| и VAO для точек кривой Безье (не забудьте настроить атрибуты в VAO)
\item Заведите параметр, управляющий уровнем детализации кривой: \mintinline{cpp}|int quality = 4;|
\item При добавлении/удалении точек пересчитываем \textit{заново} всю кривую Безье
\item Если в исходной ломаной \mintinline{cpp}|N| \textit{отрезков}, то в кривой Безье должно быть \mintinline{cpp}|N * quality| \textit{отрезков}
\item Параметр \mintinline{cpp}|t| должен равномерно меняться от 0 до 1 вдоль всей кривой
\item Цвет -- любой, но отличающийся от цвета исходной ломаной (чтобы её было лучше видно)
\item Данные в VBO должны обновляться \textbf{\alert{только}} при их изменении
\item Рисуем кривую, используя ту же шейдерную программу и новый VAO
\end{itemize}
\end{frame}

\begin{frame}
\frametitle{Задание 6}
\slideimage{6.png}
\end{frame}

\begin{frame}[fragile]
\frametitle{Задание 7}
Динамически меняем уровень детализации
\usemintedstyle{solarized-light}
\begin{itemize}
\item При нажатии на стрелку влево (\mintinline{cpp}|SDL_LEFT|) уровень детализации \mintinline{cpp}|quality| должен уменьшаться (но не может быть меньше 1)
\item При нажатии на стрелку вправо (\mintinline{cpp}|SDL_RIGHT|) уровень детализации \mintinline{cpp}|quality| должен увеличиваться
\end{itemize}
\end{frame}

\begin{frame}
\frametitle{Задание 7}
\slideimage{7.png}
\end{frame}

\begin{frame}[fragile]
\frametitle{Задание 8*}
\usemintedstyle{solarized-light}
Добавляем ползающий пунктир к кривой Безье
\begin{itemize}
\item Пунктира можно добиться, не рисуя половину пикселей кривой в зависимости от расстояния до начала
\item Например, пиксели кривой на расстоянии \mintinline{cpp}|[0, 20]| от начала рисуются, на расстоянии \mintinline{cpp}|[20, 40]| от начала не рисуются, \mintinline{cpp}|[40, 60]| рисуются, и т.д.
\item Для этого нужно добавить в вершины кривой расстояние вдоль кривой от её начала
\item Расстояние до первой вершины = 0
\item Расстояние до  вершины \mintinline{cpp}|i| = расстояние до  вершины \mintinline{cpp}|(i-1)| + расстояние от вершины \mintinline{cpp}|(i-1)| до вершины \mintinline{cpp}|i|
\item Для расстояния между вершинами пригодится функция C++ \mintinline{cpp}|std::hypot|
\item Расстояние нужно интерполировать между вершинами и передать во фрагментный шейдер
\end{itemize}
\end{frame}

\begin{frame}[fragile]
\frametitle{Задание 8*}
\usemintedstyle{solarized-light}
\begin{itemize}
\item Чтобы понять, нужно ли рисовать пиксель, пригодится функция GLSL \mintinline{glsl}|mod| (например, \mintinline{glsl}|mod(distance, 40.0) < 20.0)|
\item Чтобы отменить рисование пикселя, нужно в функции \mintinline{glsl}|main| фрагментного шейдера вызвать команду \mintinline{glsl}|discard;|
\item Само расстояние нужно добавить как поле вершины и как атрибут в VAO и вершинный шейдер
\item Чтобы пунктир двигался, можно прибавить к расстоянию что-то, зависящее от времени (придётся передать время в шейдер как \mintinline{cpp}|uniform|)
\item Два варианта организации шейдеров:
\begin{itemize}
\item Два отдельных шейдера: один для ломаной, один для пунктирной кривой Безье
\item Один общий шейдер: \mintinline{cpp}|uniform|-настройка, нужно ли рисовать пунктир (например, \mintinline{cpp}|uniform int dash;|, \mintinline{cpp}|dash == 1| означает рисовать пунктир)
\end{itemize}
\end{itemize}
\end{frame}

\begin{frame}
\frametitle{Задание 8}
\slideimage{8.png}
\end{frame}

\end{document}
