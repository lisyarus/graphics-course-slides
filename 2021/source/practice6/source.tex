% (c) Nikita Lisitsa, lisyarus@gmail.com, 2021

\documentclass{beamer}

\usepackage[T2A]{fontenc}
\usepackage[utf8]{inputenc}
\usepackage[russian]{babel}

\usepackage{graphicx}
\graphicspath{ {./images/} }

\usepackage{adjustbox}

\usepackage{color}
\usepackage{soul}

\usepackage{hyperref}

\definecolor{blue}{rgb}{0,0,1}
\definecolor{red}{rgb}{1,0,0}

\makeatletter
\newcommand{\slideimage}[1]{
  \begin{figure}
    \begin{adjustbox}{width=\textwidth, totalheight=\textheight-2\baselineskip-2\baselineskip,keepaspectratio}
      \includegraphics{#1}
    \end{adjustbox}
  \end{figure}
}
\makeatother

\title{Компьютерная графика}
\subtitle{Практика 6: Освещение}
\date{2021}

\setbeamertemplate{footline}[frame number]

\begin{document}

\frame{\titlepage}

\begin{frame}[fragile]
\frametitle{Задание 1}
Добавляем ambient освещение
\begin{itemize}
\item Добавляем uniform-переменную \verb|vec3 ambient| во фрагментный шейдер
\pause
\item Используем её для вычисления отражённого ambient освещения по формуле \verb|color = ambient * albedo| (альбедо читается из текстуры)
\pause
\item Устанавливаем значение uniform-переменной (\verb|glGetUniformLocation|, \verb|glUniform3f|)
\end{itemize}
\end{frame}

\begin{frame}[fragile]
\frametitle{Задание 2}
Добавляем точечный источник света
\begin{itemize}
\item Добавляем uniform-переменные, описывающие источник света
\begin{itemize}
\item \verb|vec3 light_position| - координаты источника
\item \verb|vec3 light_color| - цвет источника
\item \verb|vec3 light_attenuation| - параметры затухания источника с расстоянием
\end{itemize}
\pause
\item Для вычисления освещённости нам нужны расстояние от рисуемого пикселя до источника света, и направление на него \begin{math}\Rightarrow\end{math} нужно передать во фрагментный шейдер интерполированную позицию пикселя (\textbf{не} \verb|gl_Position|, а что-то в духе \verb|model * in_position|)
\pause
\item Вычисляем освещённость: \verb|albedo * light_color * cosine * intensity|
\pause
\item Устанавливаем какие-нибудь значения для uniform-переменных (в качестве \verb|attenuation| можно взять, например, \verb|vec3(1.0, 0.0, 0.1)|)
\begin{itemize}
\item Если источник плохо видно, можно сделать его поярче - компоненты \verb|light_color| могут быть и больше 1
\end{itemize}
\end{itemize}
\end{frame}

\begin{frame}[fragile]
\frametitle{Задание 3}
Три источника света
\begin{itemize}
\item Заменяем uniform-переменные, описывающие источник, на массивы uniform-переменных: \verb|vec3 light_position[3]| и т.п.
\pause
\item Во фрагментном шейдере суммируем отраженный свет каждого источника
\pause
\item Для \verb|glGetUniformLocation| в качестве имени переменной указываем \verb|light_position[0]|, \verb|light_position[1]|, и т.д.
\pause
\item В сумме будет 9 uniform-переменных (3 массива по 3), нужно задать им всем значения
\pause
\item Лучше, если источники света будут двигаться (например, крутиться вокруг общего центра)
\end{itemize}
\end{frame}

\begin{frame}[fragile]
\frametitle{Задание 4}
Добавляем стены
\begin{itemize}
\item Меняя матрицу \verb|model| и вызывая рисование ещё раз (\verb|glm::rotate|, \verb|glm::translate|, \verb|glUniformMatrix|, \verb|glDrawArrays|) добавляем три стены: слева, справа, и сзади
\pause
\begin{itemize}
\item N.B. Исходная геометрия - плоскость \begin{math}[-10, 10]\times[-10, 10]\end{math} параллельная XY, с Z = 0
\item Задняя стена - плоскость, параллельная XY, с Z = -10
\item Левая стена - плоскость, параллельная YZ, с X = -10
\item Правая стена - плоскость, параллельная YZ, с X = 10
\pause
\item Не забываем очищать матрицу перед записью в неё нового преобразования: \verb|model = glm::mat4(1.f)|
\end{itemize}
\pause
\item Если стена не освещается источниками света, значит она обращена в другую сторону - нужно повернуть её на 180 градусов 
\begin{itemize}
\item N.B. функции GLM принимают радианы
\end{itemize}
\end{itemize}
\end{frame}

\begin{frame}[fragile]
\frametitle{Задание 5}
\fontsize{10pt}{10pt}
Добавляем normal mapping
\begin{itemize}
\item Нормаль в вершинном шейдере нам больше не нужна
\pause
\item Добавляем во фрагментный шейдер uniform-текстуру \verb|sampler2D normal_map|
\pause
\item Читаем нормали из текстуры и переводим из диапазона \begin{math}[0, 1]\end{math} в диапазон \begin{math}[-1, 1]\end{math}
\pause
\item Нормали заданы в исходной системе координат объекта, но мы сдвигаем/поворачиваем объект \begin{math}\Rightarrow\end{math} к нормалям надо применить матрицу \verb|model| (её нужно объявить во фрагментном шейдере)
\pause
\item В качестве значения uniform-переменной снаружи записываем 0 (чтобы использовать текстуру из нулевого texture unit'а)
\pause
\item Создаём текстуру типа \verb|GL_TEXTURE_2D|, загружаем данные из \verb|brick_normal|, настраиваем mip-mapping (\verb|glGenTextures|, \verb|glBindTexture|, \verb|glTexImage2D|, \verb|glTexParameteri|, опционально \verb|glGenerateMipmap|)
\pause
\item При рендеринге делаем активным нулевой texture unit и делаем для него текущей нашу текстуру (\verb|glActiveTexture|, \verb|glBindTexture|)
\end{itemize}
\end{frame}

\begin{frame}[fragile]
\frametitle{Задание 6}
Добавляем ambient occlusion mapping
\begin{itemize}
\item Ещё одна текстура: \verb|sampler2D ao_map|
\pause
\item Читаем из неё значение ambient occlusion, умножаем на него ambient освещение в данной точке
\begin{itemize}
\item Эффект можно сделать заметнее, возведя значение из \verb|ao_map| в какую-нибудь степень (скажем, 4)
\end{itemize}
\pause
\item Данные для текстуры - \verb|brick_ao|, texture unit = 1
\end{itemize}
\end{frame}

\begin{frame}[fragile]
\frametitle{Задание 7}
Добавляем specular light и material mapping
\begin{itemize}
\item Ещё одна текстура: \verb|sampler2D roughness_map|
\pause
\item Читаем из неё значение roughness, яркость specular компоненты = (1 - roughness)
\pause
\item Для вычисления specular нужно направление на камеру \begin{math}\Rightarrow\end{math} нужно передать позицию камеры как uniform в шейдер
\pause
\item Вычисляем интенсивность specular как
\begin{verbatim}
pow(max(0.0, dot(reflected_direction, 
    camera_direction)), 4.0) * (1.0 - roughness)
\end{verbatim}
\begin{itemize}
\item 4 - настраиваемый параметр: чем больше, тем более точечным получается отражение
\end{itemize}
\pause
\item Добавляем specular-компоненту к результирующему освещению
\pause
\item Данные для текстуры - \verb|brick_roughness|, texture unit = 2
\end{itemize}
\end{frame}

\begin{frame}[fragile]
\frametitle{Задание 8}
Добавляем tone mapping
\begin{itemize}
\item Применяем к результирующему цвету пикселя простейший Reinhard operator: \verb|color = color / (vec3(1.0) + color)|
\end{itemize}
\end{frame}

\end{document}